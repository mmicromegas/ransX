%\documentclass[     12pt,                   %fontsize
%                    paper=a4,               %papersize
%                    pagesize,               %area
%                    DIV=calc,               %relation colls and rows see koma manual
%                    liststotocnumbered]     %numbered toc (best solution report sucks over here
%                    headsepline,            %header with line
%                    footsepline]            %foot with line
%                    {scrreprt}              %equals to report

\documentclass[10pt,paper=a4]{report}

%pagemargins
%\usepackage[left=2.cm,right=2.cm,top=2.cm,bottom=3.cm,landscape]{geometry}

%\usepackage[a4paper,landscape]{geometry}
\usepackage[left=1.8cm,right=2.cm,top=2.cm,bottom=3.cm,landscape]{geometry}

%Chapter caption is somewhere in the center of the page, so move the shit ;)
%\renewcommand*\chapterheadstartvskip{\vspace*{-1cm}}

%If you have a new chapter, the pagestyle will be plain (only pagenumber) according to koma
%To use ur setting redefine the chaperstyle
%\renewcommand*{\chapterpagestyle}{scrheadings} 

%Settings for head and foot

%\usepackage[myheadings]{fullpage}
%\pagestyle{myheadings}

\usepackage{fancyhdr}
%\fancyhead{}
%\fancyhead[CO,CE]{---Draft---}
\pagestyle{fancy}
\rhead{}


%\usepackage{scrpage2} 
%\clearscrheadfoot
%\pagestyle{scrheadings}

%\ihead{top left}
%\chead{top center}
%\ohead{top right}       

%\ifoot{bottom left} 
%\cfoot{bottom center}
%\ofoot{bottom right}

\usepackage{amsmath}
\usepackage{color}
\usepackage{graphicx}
\usepackage{cancel}
\usepackage[usenames,dvipsnames]{xcolor}
\usepackage{chngcntr}
\usepackage{natbib}

\usepackage{hyperref}
\hypersetup{
        colorlinks = true,
        linkcolor = blue,
        anchorcolor = red,
        citecolor = blue,
        filecolor = red,
        urlcolor = red
} 


\newcommand{\Msun}{\mbox{M$_\odot$\,}}         % M_sun 

\newcommand{\eht}{\overline}    
\newcommand{\fht}{\widetilde}    
\newcommand{\dr}{\frac{\partial}{\partial r}}
\newcommand{\dt}{\frac{\partial}{\partial t}}
\newcommand{\dth}{\frac{\partial}{\partial \theta}}
\newcommand{\dph}{\frac{\partial}{\partial \phi}}

\newcommand{\fav}{\widetilde}    
\newcommand{\av}{\overline}  

\def\ef#1{#1'}
\def\ff#1{#1''}
\def\fhtc#1{\left\{#1\right\}}
\def\erho{\eht{\rho}}

\newcommand{\dgr}{\mbox{$^\circ$}}           % degrees 

\counterwithout{section}{chapter}

\usepackage{titling}
\newcommand{\subtitle}[1]{%
  \posttitle{%
    \par\end{center}
    \begin{center}\large#1\end{center}
    \vskip0.5em}%
}


\title{{\bf ransX framework}}
\subtitle{Theory Guide}
\author{
        Miroslav Moc\'ak, Casey Meakin, Maxime Viallet \& David Arnett          
}

\date{\today}

\begin{document}
\bibliographystyle{plainnat}
%\begin{landscape}
\maketitle

\tableofcontents

\newpage

\section{Introduction}
\label{sect:intro}

\par We present a statistical analysis of turbulent convection in stars within our Reynolds-Averaged Navier Stokes (RANS) framework in spherical geometry which we derived from first principles (see Sect.\ref{sect:mean-field-derivation} and further sections for details). The analysed data include {\bf core convection during oxygen burning in a 15 $\Msun$ supernova progenitor, envelope convection in a 5 $\Msun$ red giant and core convection during core helium flash and hydrogen injection flash of a 1.25 $\Msun$ star}. These simulations have been partially already described by \citet{MeakinArnett2007,ArnettMeakin2010,VialletMeakin2013} and \citet{Mocak2009,Mocak2011}. 

\par We obtain our 1D RANS equations by introducing two types of averaging:  statistical averaging and  horizontal averaging \citep{Besnard1992,VialletMeakin2013}. In practice, statistical averages are computed by performing a time average (the ergodic hypothesis). Therefore, the combined average of a quantity $q$ is defined as

\begin{align}\label{eq:eht}
\eht{q}(r,t) = \frac{1}{T\Delta\Omega}\int_{t- T/2}^{t+T/2} q(r,\theta,\phi,t')~d\Omega~dt'
\end{align}

\noindent where $d \Omega = \sin \theta d \theta d \phi$ is the solid angle in spherical coordinates, $T$ is the averaging time period, and $\Delta\Omega$ is total solid angle being averaged over.

\par The flow variables are then decomposed into mean and fluctuation $q = \eht{q} + q'$, noting that $\eht{q'} = 0$ by construction. Similarly, we introduce Favre (or density weighted) averaged quantities by 

\begin{align}
\fht{q} = \frac{\eht{\rho q}}{\eht{\rho}}
\end{align}

\noindent which defines a complimentary decomposition of the flow into mean and fluctuations according to $q = \fht{q} + \ff{q}$. Here, $\ff{q}$ is the Favrian fluctuation and its mean is zero when Favre averaged $\fht{\ff{q}} = 0$. For a more complete elaboration on the algebra of these averaging procedures we refer the reader to \citet{Chassaing2010}. All calculated mean fields shown for a given equation in next sections were additionally multiplied by $4 \pi r^2$ . It gives us advantage to reflect changing volume of our computational domain as a function of radius and visualize better volume integral budgets of the individual mean fields, which are then equivalent to the area below corresponding mean field profile. We explore not only general properties of mean fields but also their resolution dependency, wedge-size dependency, convection zone depth dependency, position of driving source dependency and validity of few turbulence models.

\newpage

\section{Summary of Reynolds-averaged Navier Stokes equations in spherical geometry}

\subsection{Various mean fields equations (first-order moments)}

\begin{table}[!h]
%\caption{1D RANS equations (first-order moments) in Lagrangian form.}
\label{tab:rans}
\begin{align}
% Continuity equation
\fav{D}_t \av{\rho} =& -\av{\rho} \fav{d} + {\mathcal N_\rho}  \label{eq:rans_density}\\
% Momentum equation
\av{\rho}\fav{D}_t\fav{u}_r = & -\nabla_r \fav{R}_{rr} -\av{G^{M}_r} - \partial_r \av{P} + \av{\rho}\fav{g_r} + {\mathcal N_{ur}} \\ 
\av{\rho}\fav{D}_t\fav{u}_\theta = & -\nabla_r \fav{R}_{\theta r} -\av{G^{M}_\theta} - (1/r)\av{\partial_\theta P} + {\mathcal N_{u \theta}}  \\
\av{\rho}\fav{D}_t\fav{u}_\phi = & -\nabla_r \fav{R}_{\phi r} -\av{G^{M}_\phi} + {\mathcal N_{u \phi}} \\
% Internal energy
\av{\rho} \fav{D}_t \fav{\epsilon}_I = & -\nabla_r  ( f_I + f_T ) - \av{P} \ \av{d} - W_P  + {\mathcal S} + {\mathcal N_{\epsilon I}} \\
% KE equation
\av{\rho} \fav{D}_t \fav{\epsilon}_k = & -\nabla_r  ( f_k +  f_P ) - \fht{R}_{ir}\partial_r \fht{u}_i + W_b + W_P +\av{\rho}\fav{D}_t (\fav{u}_i \fav{u}_i / 2) + {\mathcal N_{\epsilon k}} \label{eq:rans_mke} \\
% Total energy equation
\av{\rho} \fav{D}_t \fav{\epsilon}_t = &  -\nabla_r ( f_I + f_T + f_k + f_P ) - \fht{R}_{ir}\partial_r \fht{u}_i - \av{P} \ \av{d} + W_b + {\mathcal S} + \av{\rho}\fav{D}_t (\fav{u}_i \fav{u}_i / 2) + {\mathcal N_{\epsilon t}}  \label{eq:rans_etot} \\
% Enthalpy
\erho\fav{D}_t \fav{h} = & -\nabla_r f_h - \Gamma_1\eht{P} \ \eht{d} - \Gamma_1 W_P + \Gamma_3 {\mathcal S} + \Gamma_3 \nabla_r f_T + {\mathcal N_h} \label{eq:rans_h} \\
% Entropy
\av{\rho} \fav{D}_t \fav{s} = &  -\nabla_r  f_s    + \av{(\nabla \cdot F_T)/T}+ \av{{\mathcal S}/T} + {\mathcal N_s}  \label{eq:rans_entropy} \\
% Pressure
\av{D}_t \av{P} = &  -\nabla_r f_P - \Gamma_1 \eht{P} \ \eht{d} + (1 -\Gamma_1) W_P + (\Gamma_3 -1){\mathcal S} + (\Gamma_3 - 1)\nabla_r f_T + {\mathcal N_P} \\
% Temperature
\av{D}_t \av{T} = & -\nabla_r f_T + (1-\Gamma_3)\eht{T}\ \eht{d} + (2-\Gamma_3)\eht{T'd'} + \eht{(\nabla \cdot F_T) / \rho c_v} + \eht{(\tau_{ij}\partial_i u_j)/\rho c_v} + \eht{\epsilon_{\rm nuc} / c_v} + {\mathcal N_T} \\
% Composition
\erho\fav{D}_t \fav{X}_\alpha = & -\nabla_r f_\alpha + \av{\rho}\fav{\dot{X}}_\alpha^{\rm nuc} + {\mathcal N_\alpha} \label{eq:rans_comp} \\
\erho\fav{D}_t \fav{A} = & -\nabla_r f_A - \av{\rho A^2\Sigma_\alpha (\dot{X}_\alpha^{\rm nuc} / A_\alpha)} + {\mathcal N_A}  \label{eq:rans_abar}\\
\erho\fav{D}_t \fav{Z} = & -\nabla_r f_Z  - \av{\rho Z A \Sigma_\alpha (\dot{X}_\alpha^{\rm nuc} / A_\alpha)}  + \overline{\rho A \Sigma_\alpha (Z_\alpha \dot{X}_\alpha^{\rm nuc} / A_\alpha)} + {\mathcal N_Z} \label{eq:rans_zbar} \\
\erho\fav{D}_t \fav{j}_z = & -\nabla_r f_{jz}  + {\mathcal N_{jz}} \label{eq:rans_jz} 
\end{align}
\end{table}


\subsection{Mean turbulent mass flux and mean density-specific volume covariance equation (second-order moments)}

\begin{table}[!h]
\label{tab:rans}
\begin{align}
\eht{\rho}\fht{D}_t \eht{u''_r} =&  -(\eht{\rho'u'_ru'_r}/\eht{\rho})\partial_r\eht{\rho} + (\fht{R}_{rr}/\eht{\rho})/\partial_r \eht{\rho} - \eht{\rho} \nabla_r (\eht{u''_r} \ \eht{u''_r}) + \nabla_r \overline{\rho' u'_r u'_r} - \eht{\rho}\eht{u''_r} \nabla_r \eht{u}_r + \eht{\rho} \eht{u'_r d''} - b\partial_r \eht{P} + \eht{\rho' v \partial_r P'} +{\mathcal G_a} + {\mathcal N_a} \label{eq:rans_a}\\
\eht{D}_t b = &  +\eht{v} \nabla_r \eht{\rho} \eht{u''_r} -\eht{\rho}\nabla_r (\eht{u'_r v'}) + 2\eht{\rho}\eht{v'd'} +  {\mathcal N_b} \label{eq:rans_b}
\end{align}
\end{table}


\newpage

\subsection{Mean Reynolds stress equations (second-order moments)}


\begin{align}
\eht{\rho}\fht{D}_t \left( \fht{R}_{rr} / \eht{\rho} \right) = & -\nabla_r (\fht{F}_{r r r}^R + f_p^r + f_p^r + f_\tau^{rr} + f_\tau^{rr} ) + \eht{u''_r} \partial_r \eht{P} + \eht{u''_r} \partial_r \eht{P} - \fht{R}_{rr}\partial_r \fht{u}_r - \fht{R}_{rr}\partial_r \fht{u}_r + \eht{P'\nabla_r u''_r} + \eht{P'\nabla_r u''_r} - \overline{u''_r G^{M}_r} - \overline{u''_r G^{M}_r} - \eht{G^{R}_{rr}} - \varepsilon_\tau^{rr} - \varepsilon_\tau^{rr} \\
\eht{\rho}\fht{D}_t \left( \fht{R}_{r\theta} / \eht{\rho} \right) = & -\nabla_r (\fht{F}_{r \theta r}^R + f_p^\theta \ \ \ \ \ \ \ + f_\tau^{r \theta} + f_\tau^{\theta r} ) -\eht{u''_\theta} \partial_r \eht{P} \ \ \ \ \ \ \ \ \ \ \ \ \ - \fht{R}_{rr}\partial_r \fht{u}_\theta - \fht{R}_{\theta r}\partial_r \fht{u}_r + \eht{P'\nabla_r u''_\theta} + \eht{P'\nabla_\theta u''_r}  - \eht{u''_r G_\theta^M} - \eht{u''_\theta G_r^M} - \eht{G^{R}_{r\theta}} - \varepsilon_\tau^{r \theta} - \varepsilon_\tau^{\theta r}  \\
\eht{\rho}\fht{D}_t \left( \fht{R}_{r\phi} / \eht{\rho} \right) = & -\nabla_r (\fht{F}_{r \phi r}^R  + f_p^\phi \ \ \ \ \ \ \ + f_\tau^{r \phi} + f_\tau^{\phi r}) -\eht{u''_\phi} \partial_r \eht{P} \ \ \ \ \ \ \ \ \ \ \ \ - \fht{R}_{rr}\partial_r \fht{u}_\phi - \fht{R}_{\phi r}\partial_r \fht{u}_r + \eht{P'\nabla_r u''_\phi} + \eht{P'\nabla_\phi u''_r} - \eht{u''_r G_\phi^M} - \eht{u''_\phi G_r^M}   - \eht{G^{R}_{r\phi}} - \varepsilon_\tau^{r \phi} - \varepsilon_\tau^{\phi r} \\
\eht{\rho}\fht{D}_t \left( \fht{R}_{\theta r} / \eht{\rho} \right) = & -\nabla_r (\fht{F}_{\theta r r}^R + f_p^\theta \ \ \ \ \ \ \ + f_\tau^{\theta r} + f_\tau^{r \theta} ) -\eht{u''_\theta} \partial_r \eht{P} \ \ \ \ \ \ \ \ \ \ \ \ - \fht{R}_{\theta r}\partial_r \fht{u}_r - \fht{R}_{r r}\partial_r \fht{u}_\theta  + \eht{P'\nabla_\theta u''_r} + \eht{P'\nabla_r u''_\theta} - \eht{u''_\theta G_r^M} - \eht{u''_r G_\theta^M}  - \eht{G^{R}_{\theta r}} - \varepsilon_\tau^{\theta r} - \varepsilon_\tau^{r \theta}   \\
\eht{\rho}\fht{D}_t \left( \fht{R}_{\theta \theta} / \eht{\rho} \right) = & -\nabla_r ( \fht{F}_{\theta \theta r}^R \ \ \ \ \ \ \ \ \ \ \ \ \ \ + f_\tau^{\theta \theta} + f_\tau^{\theta \theta}) \ \ \ \ \ \ \ \ \ \ \ \ \ \ \ \ \ \ \ \ \ \ \ \ -\fht{R}_{\theta r}\partial_r \fht{u}_\theta - \fht{R}_{\theta r}\partial_r \fht{u}_\theta  +\eht{P' \nabla_\theta u''_\theta} + \eht{P' \nabla_\theta u''_\theta} - \overline{u''_\theta G^{M}_\theta} - \overline{u''_\theta G^{M}_\theta} - \eht{G^{R}_{\theta \theta}} - \varepsilon_\tau^{\theta \theta}  - \varepsilon_\tau^{\theta \theta} \\
\eht{\rho}\fht{D}_t \left( \fht{R}_{\theta \phi} / \eht{\rho} \right) = & -\nabla_r (\fht{F}_{\theta \phi r}^R \ \ \ \ \ \ \ \ \ \ \ \ \ \ + f_\tau^{\theta \phi} + f_\tau^{\phi \theta}) \ \ \ \ \ \ \ \ \ \ \ \ \ \ \ \ \ \ \ \ \ \ \ \ -\fht{R}_{\theta r}\partial_r \fht{u}_\phi - \fht{R}_{\phi r}\partial_r \fht{u}_\theta  +\eht{P' \nabla_\theta u''_\phi} + \eht{P' \nabla_\phi u''_\theta} - \overline{u''_\theta G^{M}_\phi} - \overline{u''_\phi G^{M}_\theta} - \eht{G^{R}_{\theta \phi}} - \varepsilon_\tau^{\theta \phi}  - \varepsilon_\tau^{\phi \theta}  \\
\eht{\rho}\fht{D}_t \left( \fht{R}_{\phi r} / \eht{\rho} \right) = & -\nabla_r (\fht{F}_{\phi r r}^R + f_p^\phi \ \ \ \ \ \ \ + f_\tau^{\phi r} + f_\tau^{r \phi}) - \eht{u''_\phi} \partial_r \eht{P} \ \ \ \ \ \ \ \ \ \ \ \  -\fht{R}_{\phi r}\partial_r \fht{u}_r - \fht{R}_{r r}\partial_r \fht{u}_\phi +\eht{P' \nabla_\phi u''_r} +\eht{P' \nabla_r u''_\phi} - \overline{u''_\phi G^{M}_r} - \overline{u''_r G^{M}_\phi} - \eht{G^{R}_{\phi r}} - \varepsilon_\tau^{\phi r}  - \varepsilon_\tau^{r \phi} \\
\eht{\rho}\fht{D}_t \left( \fht{R}_{\phi \theta} / \eht{\rho} \right) = & -\nabla_r (\fht{F}_{\phi \theta r}^R \ \ \ \ \ \ \ \ \ \ \ \ \ \ + f_\tau^{\phi \theta} + f_\tau^{\theta \phi}) \ \ \ \ \ \ \ \ \ \ \ \ \ \ \ \ \ \ \ \ \ \ \ \  - \fht{R}_{\phi r}\partial_r \fht{u}_\theta -\fht{R}_{\theta r}\partial_r \fht{u}_\phi +\eht{P' \nabla_\phi u''_\theta} +\eht{P' \nabla_\theta u''_\phi} - \overline{u''_\phi G^{M}_\theta} - \overline{u''_\theta G^{M}_\phi} - \eht{G^{R}_{\phi \theta}} - \varepsilon_\tau^{\phi \theta}  - \varepsilon_\tau^{\theta \phi}  \\
 \eht{\rho}\fht{D}_t \left( \fht{R}_{\phi \phi} / \eht{\rho} \right) = & -\nabla_r ( \fht{F}_{\phi \phi r}^R \ \ \ \ \ \ \ \ \ \ \ \ \ \ + f_\tau^{\phi \phi} + f_\tau^{\phi \phi}) \ \ \ \ \ \ \ \ \ \ \ \ \ \ \ \ \ \ \ \ \ \ - \fht{R}_{\phi r}\partial_r \fht{u}_\phi - \fht{R}_{\phi r}\partial_r \fht{u}_\phi + \eht{P' \nabla_\phi u''_\phi} + \eht{P' \nabla_\phi u''_\phi} -\overline{u''_\phi G^{M}_\phi} -\overline{u''_\phi G^{M}_\phi} - \eht{G^{R}_{\phi \phi}} -\varepsilon_\tau^{\phi \phi} -\varepsilon_\tau^{\phi \phi}   
\end{align}

\vspace{-0.5cm}

\begin{table}[!h]
%\caption{1D RANS (Reynolds stress) equations in Lagrangian form.}
\begin{align}
\eht{\rho}\fht{D}_t \left( \fht{R}_{rr} / \eht{\rho} \right) = & -\nabla_r ( 2 f_k^r + 2 f_P ) + 2 W_b - 2\fht{R}_{rr}\partial_r \fht{u}_r + 2\eht{P'\nabla_r u''_r} + 2 {\mathcal G_k^r} + {\mathcal N_{Rrr}} \\
\eht{\rho}\fht{D}_t \left( \fht{R}_{\theta \theta} / \eht{\rho} \right) = & -\nabla_r ( 2 f_k^\theta )- 2\fht{R}_{\theta r}\partial_r \fht{u}_\theta +2\eht{P' \nabla_\theta u''_\theta}  + 2 {\mathcal G_k^\theta} + {\mathcal N_{R \theta \theta}}  \\
\eht{\rho}\fht{D}_t \left( \fht{R}_{\phi \phi} / \eht{\rho} \right) = & - \nabla_r ( 2 f_k^\phi ) - 2\fht{R}_{\phi r}\partial_r \fht{u}_\phi + 2\eht{P' \nabla_\phi u''_\phi}   + 2 {\mathcal G_k^\phi} + {\mathcal N_{R \phi \phi}}
%\fht{k} = (1/2)\fht{R}_{ii}/\eht{\rho} & \ \ \  \fht{k}^r = (1/2)\fht{R}_{rr}/\eht{\rho} \ \ \  \fht{k}^h = (1/2) (\fht{R}_{\theta \theta}/\eht{\rho} + \fht{R}_{\phi \phi}/\eht{\rho})
\end{align}
\end{table}

\subsection{Mean turbulent kinetic energy equations (second-order moments)}
%\subsubsection{Turbulent kinetic energy}

\begin{table}[!h]
%\caption{1D RANS equations (turbulent kinetic energy and relevant quantities) in Lagrangian form.}
\label{tab:rans}
\begin{align}
% kinetic energy
\av{\rho} \fav{D}_t \fav{k}^{ } = &  -\nabla_r ( f_k +  f_P ) - \fht{R}_{ir}\partial_r \fht{u}_i + W_b + W_P + {\mathcal N_k}  \label{eq:rans_tke} \\
\av{\rho} \fav{D}_t \fav{k}^r =  & -\nabla_r  ( f_k^r + f_P )  - \fht{R}_{rr}\partial_r \fht{u}_r + W_b  + \eht{P'\nabla_r u''_r} + {\mathcal G_k^r} +  {\mathcal N_{kr}} \label{eq:rans_ekin_r} \\
\av{\rho} \fav{D}_t \fav{k}^h =  &  -\nabla_r f_k^h - (\fht{R}_{\theta r}\partial_r \fht{u}_\theta + \fht{R}_{\phi r}\partial_r \fht{u}_\phi) + (\eht{P' \nabla_\theta u''_\theta} + \eht{P' \nabla_\phi u''_\phi}) + {\mathcal G_k^h} + {\mathcal N_{kh}} \label{eq:rans_ekin_h} \\
\end{align}
\end{table}

\newpage

\subsection{Mean flux equations (second-order moments)}

\begin{table}[!h]
%\caption{1D RANS (flux) equations in Lagrangian form.}
\begin{align}
% Internal energy flux equation
\erho \fav{D}_t (f_I / \eht{\rho}) = &  -\nabla_r f_I^r  - f_I \partial_r \fht{u}_r  - \fht{R}_{rr} \partial_r \fht{\epsilon_I} - \eht{\epsilon''_I} \partial_r \eht{P} - \eht{\epsilon''_I \partial_r P'}  - \eht{u''_r \left( P d \right)}  + \overline{u''_r ({\mathcal S} + \nabla \cdot F_T)} + {\mathcal G_I} +  {\mathcal N_{fI}} \label{eq:rans_fi} \\
% Enthalpy flux
\erho \fav{D}_t (f_h / \eht{\rho}) = &  -\nabla_r f_h^r - f_h \partial_r \fht{u}_r - \fht{R}_{rr} \partial_r \fht{h} -\eht{h''}\partial_r \eht{P} - \eht{h''\partial_r P'} - \Gamma_1\eht{u''_r \left( P d \right) } + \Gamma_3 \overline{u''_r ({\mathcal S} + \nabla \cdot F_T)} + {\mathcal G_h} + {\mathcal N_{h \ }} \label{eq:rans_fh} \\
% Entropy flux equation
\erho \fav{D}_t (f_s / \eht{\rho}) = & -\nabla_r f_s^r - f_s \partial_r \fht{u}_r - \fht{R}_{rr} \partial_r \fht{s} -\eht{s''}\partial_r \eht{P} - \eht{s''\partial_r P'} + \eht{u''_r ( {\mathcal S} + \nabla \cdot F_T)  / T} + {\mathcal G_s} + {\mathcal N_{fs}}  \label{eq:rans_fs} \\
% Composition flux equations
\erho \fav{D}_t (f_{\alpha r} / \eht{\rho}) = &  -\nabla_r f_\alpha^r  - f_{\alpha r} \partial_r \fht{u}_r - \fht{R}_{rr} \partial_r \fht{X}_\alpha -\eht{X''_\alpha} \partial_r \eht{P} - \eht{X''_\alpha \partial_r P'} + \overline{u''_r \rho \dot{X}_\alpha^{\rm nuc}} + {\mathcal G_\alpha} + {\mathcal N_{f\alpha}}  \label{eq:rans_falphar} \\
\erho \fav{D}_t (f_{\alpha \theta} / \eht{\rho}) = &  -\nabla_r f_\alpha^\theta  - f_{\alpha r} \partial_r \fht{u}_\theta - \fht{R}_{\theta r} \partial_r \fht{X}_\alpha + \overline{u''_\theta \rho \dot{X}_\alpha^{\rm nuc}} + {\mathcal G_{\alpha \theta}} + {\mathcal N_{f\alpha \theta}}  \label{eq:rans_falphat} \\
\erho \fav{D}_t (f_{\alpha \phi} / \eht{\rho}) = &  -\nabla_r f_\alpha^\phi  - f_{\alpha r} \partial_r \fht{u}_\phi - \fht{R}_{\phi r} \partial_r \fht{X}_\alpha + \overline{u''_\phi \rho \dot{X}_\alpha^{\rm nuc}} + {\mathcal G_{\alpha \phi}} + {\mathcal N_{f\alpha \phi}}  \label{eq:rans_falphap} \\
\erho \fav{D}_t (f_A / \eht{\rho}) = &  -\nabla_r f_A^r - f_A \partial_r \fht{u}_r - \fht{R}_{rr} \partial_r \fht{A} -\eht{A''} \partial_r \eht{P} - \eht{A'' \partial_r P'} - \overline{u''_r \rho A^2\Sigma_\alpha \dot{X}_\alpha^{\rm nuc} / A_\alpha} + {\mathcal G_A} + {\mathcal N_{fA}}                \label{eq:rans_fabar} \\
\erho \fav{D}_t (f_Z / \eht{\rho}) = &  -\nabla_r f_Z^r  - f_Z \partial_r \fht{u}_r - \fht{R}_{rr} \partial_r \fht{Z} -\eht{Z''} \partial_r \eht{P} - \eht{Z'' \partial_r P'} - \overline{u''_r \rho Z A \Sigma_\alpha (\dot{X}_\alpha^{\rm nuc}/ A_\alpha)} - \overline{u''_r \rho A \Sigma_\alpha (Z_\alpha \dot{X}_\alpha^{\rm nuc} / A_\alpha)}  + {\mathcal G_Z} + {\mathcal N_{fZ}}   \label{eq:rans_fzbar}  \\
% Angular momentum flux equation (z comp.)
\erho \fav{D}_t (f_{jz} / \rho) = & -\nabla_r f_{jz}^r  - f_{jz} \partial_r \fht{u}_r - \fht{R}_{rr} \partial_r \fht{j_z} -\eht{j''_z} \partial_r \eht{P} - \eht{j''_z \partial_r P'} + {\mathcal G_{jz}} + {\mathcal N_{jz}} \label{eq:rans_fjz} \\
% Heat flux equation 
\fht{D}_t f_T = & -\nabla_r f_T^r - f_T \partial_r \eht{u}_r - \eht{u'_r u''_r} \partial_r \eht{T} - \eht{T'\partial_r P / \rho} - (\Gamma_3 -1)(\eht{T} \ \eht{u'_r d''}+\fht{d} \ \eht{u'_r T'} + \eht{u'_r T'd''}) +\eht{T'u'_rd''} + \eht{u'_r \epsilon_{\rm nuc}/c_v} +  \eht{u'_r \nabla \cdot F_T / \rho c_v} + {\mathcal G_{T}} +  {\mathcal N_{fT}} \\
% TKE flux equation
\erho \fav{D}_t (f_k / \eht{\rho}) = &  -\nabla_r ( 2 f_k^r + f_P^r ) - (1/2)(\nabla_r \eht{u''_iu''_iP'} + \nabla_r 2 k \partial_r \fht{u}_r - \fht{u''_iu''_r}\nabla_r \fht{R}_{rr} - 2 k \partial_r \eht{P} + \eht{u''_iu''_i}\partial_r \eht{P} - \eht{P'\nabla_r u''_iu''_i}) - \\
& - (k_i^r\partial_r \fht{u}_i - \fht{u''_iu''_r} \nabla_r \fht{u''_i u''_r} - 2 k^r \partial_r \eht{P} + \eht{u''_r u''_r} \partial_r \eht{P} - \eht{P'\nabla_iu''_iu''_r}) + {\mathcal G_{fk}} + {\mathcal N_{fk}} \\
% Acoustic flux equations
\fht{D}_t f_{pr} = & -\nabla_r f_p^r - f_{pr} \partial_r \eht{u}_r + \eht{u'_r u''_r}\partial_r \eht{P} + \Gamma_1 \eht{u'_r P d} + (\Gamma_3 -1)\eht{u'_r \rho \varepsilon_{nuc}} + \eht{P'u''_r d''} - \eht{P' G_r^M / \rho} - \eht{P'\partial_r P/ \rho}  + {\mathcal N_{fpr}} \\
\fht{D}_t f_{p\theta} = & -\nabla_r f_p^\theta - f_{pr} \partial_r \eht{u}_\theta + \eht{u'_\theta u''_r}\partial_r \eht{P} + \Gamma_1 \eht{u'_\theta P d} + (\Gamma_3 -1)\eht{u'_\theta \rho \varepsilon_{nuc}} + \eht{P'u''_\theta d''} - \eht{P' G_\theta^M / \rho} - \eht{P'\partial_\theta P/ \rho r}  + {\mathcal N_{fp\theta}} \\
\fht{D}_t f_{p\phi} = & -\nabla_r f_p^\phi - f_{pr} \partial_r \eht{u}_\phi + \eht{u'_\phi u''_r}\partial_r \eht{P} + \Gamma_1 \eht{u'_\phi P d} + (\Gamma_3 -1)\eht{u'_\phi \rho \varepsilon_{nuc}} + \eht{P'u''_\phi d''} - \eht{P' G_\phi^M / \rho} - \eht{P'\partial_\phi P/ \rho r \sin{\theta}}  + {\mathcal N_{fp\phi}}
\end{align} 
\end{table}

\subsection{Mean variance equations (second-order moments)}

\begin{table}[!h]
%\caption{1D RANS (Reynolds variance) equations in Lagrangian form.}
\label{tab:rans-variances1}
\begin{align}
% sigma density
\fht{D}_t \sigma_\rho =  &  - \nabla_r \eht{(\rho' \rho ' u''_r)}  - 2\eht{\rho} \ \eht{\rho'd''} - 2 \eht{\rho'u''_r} \partial_r \eht{\rho} - 2 \fht{d} \ \sigma_\rho - \eht{\rho'\rho'd''} + {\mathcal N_{\sigma_\rho}} \\
% sigma P
\fht{D}_t \sigma_P = & - \nabla_r \eht{(P' P' u''_r)} - 2\Gamma_1 \eht{P} \ W_P - 2 f_P \partial_r \eht{P} - 2\Gamma_1 \widetilde{d} \ \sigma_P - (2 \Gamma_1 -1) \eht{P'P'd''} + 2(\Gamma_3 - 1)\eht{P' {\mathcal S}} + {\mathcal N_{\sigma_P}} \\
\fht{D}_t \sigma_T = & -\nabla_r \eht{(T' T' u''_r)} - 2(\Gamma_3 -1)\eht{T} \ \eht{T'd''} - 2\eht{T'u''_r} \partial_r \eht{T} - 2(\Gamma_3-1)\fht{d}\sigma_T + (3-2\Gamma_3)\eht{T'T'd''} + \eht{2 T' \nabla \cdot F_T/ \rho c_v} + \eht{2 T' \epsilon_{\rm nuc} / c_v} + {\mathcal N_{\sigma T}}  \\
% sigma ur
\eht{\rho} \fht{D}_t \sigma_{ur} = & -\nabla_r ( \eht{\rho u''_r u''_r u''_r} ) + 2 \nabla_r f_P + 2 W_b - 2\fht{R}_{rr}\partial_r \fht{u}_r + 2 \overline{P'\nabla_r u''_r} + {\mathcal G_{\sigma_{ur}}} + {\mathcal N_{\sigma_{ur}}} 
\end{align}
\end{table}

\begin{table}[!h]
%\caption{1D RANS (Reynolds variance) equations in Lagrangian form.}
\label{tab:rans-variances2}
\begin{align}
% sigma T
\fht{D}_t \sigma_T = & -\nabla_r \eht{(T' T' u''_r)} - 2(\Gamma_3 -1)\eht{T} \ \eht{T'd''} - 2\eht{T'u''_r} \partial_r \eht{T} - 2(\Gamma_3-1)\fht{d}\sigma_T + (3-2\Gamma_3)\eht{T'T'd''} + \eht{2 T' \nabla \cdot F_T/ \rho c_v} + \eht{2 T' \epsilon_{\rm nuc} / c_v} + {\mathcal N_{\sigma T}}  \\
% sigma ur
\eht{\rho} \fht{D}_t \sigma_{ur} = & -\nabla_r ( \eht{\rho u''_r u''_r u''_r} ) + 2 \nabla_r f_P + 2 W_b - 2\fht{R}_{rr}\partial_r \fht{u}_r + 2 \overline{P'\nabla_r u''_r} + {\mathcal G_{\sigma_{ur}}} + {\mathcal N_{\sigma_{ur}}} \\
% sigma internal energy
\eht{\rho} \fht{D}_t \sigma_{\epsilon I} = &  -\nabla_r (\eht{\rho \epsilon''_I \epsilon''_I u''_r} ) - 2 f_I \partial_r \fht{\epsilon_I} - 2\overline{\epsilon''_I}\ \eht{P} \ \fht{d} - 2\eht{P} \ \eht{\epsilon''_I d''} - 2\fht{d} \ \eht{\epsilon''_I P'} - 2\overline{\epsilon''_I P' d''} + 2\eht{\epsilon''_I {\mathcal S}} + {\mathcal N_{\sigma_{\epsilon I}}} \\
% sigma enthalpy
\eht{\rho} \fht{D}_t \sigma_{h} = & -\nabla_r (\eht{\rho h'' h'' u''_r} )  - 2 f_h \partial \widetilde{h} - 2\Gamma_1 \overline{h''Pd} + 2\Gamma_3 \overline{h'' \rho \mathcal{S}} + {\mathcal N_{\sigma_{h}}} \\
% sigma entropy
\eht{\rho} \fht{D}_t \sigma_s = & -\nabla_r ( \eht{\rho s'' s'' u''_r} ) - 2 f_s \partial_r \fht{s} - 2\eht{s'' \nabla \cdot F_T / T} + 2 \eht{s'' {\mathcal S} / T} + {\mathcal N_{\sigma_{s}}}\\
% sigma composition
\eht{\rho} \fht{D}_t \sigma_\alpha = & -\nabla_r (\eht{\rho X''_\alpha X''_\alpha u''_r} ) - 2 f_\alpha \partial_r \fht{X}_\alpha + 2 \eht{X''_\alpha \rho \dot{X}_\alpha^{\rm nuc}} + {\mathcal N_{\sigma_\alpha}}
\end{align}
\end{table}

%\newpage

%\subsection{Mean Reynolds stress flux equations (third-order moments)}

%\begin{table}[!h]
%\caption{1D RANS (Reynolds stress flux) equations in Lagrangian form.}
%\begin{align}
%\eht{\rho}\fht{D}_t \left( 2 f_k^r / \eht{\rho} \right) = & -\nabla_r (\fht{F}_{rrrr}^R + 3\eht{u''_r u''_r P'} - 3\eht{u''_r u''_r \tau'_{rr}} ) + 3 \eht{u''_r u''_r \rho g_r} - 3 \fht{F}_{rrr}^R \partial_r \fht{u}_r + 3\fht{u''_r u''_r}\nabla_r \fht{R}_{rr} + 3 \fht{u''_r u''_r}\partial_r \eht{P} - \nonumber \\
%& - 3 \eht{u''_r u''_r}\partial_r \eht{P} + 3\eht{P'\nabla_r u''_r u''_r} - 3 \fht{R}_{rr}\fht{g_r} + 3 \fht{u''_ru''_r} \eht{G_r^M} - 3\eht{u''_ru''_r G_r^M} - \eht{G_{rrr}^R} - 3\eht{\rho}\varepsilon_{rrr}^R \\
%\eht{\rho}\fht{D}_t \left( 2 f_k^\theta / \eht{\rho} \right) = & -\nabla_r (\fht{F}_{\theta \theta rr}^R + \eht{u''_\theta u''_\theta P'} - \eht{u''_\theta u''_\theta \tau'_{rr}} - 2\eht{u''_\theta u''_r \tau'_{\theta r}}) + \eht{u''_\theta u''_\theta \rho g_r} - (\fht{F}_{\theta \theta r}^R \partial_r \fht{u}_r + 2\fht{F}_{\theta rr}\partial_r \fht{u}_\theta ) + \nonumber \\ 
%& +(\fht{u''_\theta u''_\theta}\nabla_r \fht{R}_{rr} + 2\fht{u''_\theta u''_r}\nabla_r \fht{R}_{\theta r} ) - \fht{R}_{\theta \theta} \fht{g_r} + (\fht{u''_\theta u''_\theta}\partial_r \eht{P} - \eht{u''_\theta u''_\theta}\partial_r \eht{P}) + (\eht{P'\nabla_r u''_\theta u''_\theta} + 2\eht{P'\nabla_\theta u''_\theta u''_r }) + \nonumber \\ 
%& + (\fht{u''_\theta u''_\theta} \eht{G_r^M} - \eht{u''_\theta u''_\theta G_r^M} + 2\fht{u''_\theta u''_r}\eht{G_\theta^M}- 2\eht{u''_\theta u''_r G_\theta^R}) - \eht{G}_{\theta \theta r}^R - \eht{\rho}\varepsilon_{r\theta \theta}^R - 2\eht{\rho}\varepsilon_{\theta \theta r}^R \\
%\eht{\rho}\fht{D}_t \left( 2 f_k^\phi / \eht{\rho} \right) = &  -\nabla_r (\fht{F}_{\phi \phi rr}^R + \eht{u''_\phi u''_\phi P'} - \eht{u''_\phi u''_\phi \tau'_{rr}} - 2\eht{u''_\phi u''_r \tau'_{\phi r}}) + \eht{u''_\phi u''_\phi \rho g_r} - (\fht{F}_{\phi \phi r}^R \partial_r \fht{u}_r + 2\fht{F}_{\phi rr}\partial_r \fht{u}_\phi ) + \nonumber \\ 
%& +(\fht{u''_\phi u''_\phi}\nabla_r \fht{R}_{rr} + 2\fht{u''_\phi u''_r}\nabla_r \fht{R}_{\phi r} ) - \fht{R}_{\phi \phi} \fht{g_r} + (\fht{u''_\phi u''_\phi}\partial_r \eht{P} - \eht{u''_\phi u''_\phi}\partial_r \eht{P}) + (\eht{P'\nabla_r u''_\phi u''_\phi} + 2\eht{P'\nabla_\phi u''_\phi u''_r }) + \nonumber \\ 
%& + ( \fht{u''_\phi u''_\phi} \eht{G_r^M} - \eht{u''_\phi u''_\phi G_r^M} + 2\fht{u''_\phi u''_r}\eht{G_\phi^M}- 2\eht{u''_\phi u''_r G_\phi^R}) - \eht{G}_{\phi \phi r}^R - \eht{\rho}\varepsilon_{r\phi \phi}^R - 2\eht{\rho}\varepsilon_{\phi \phi r}^R 
%\end{align}
%\end{table}

\newpage

\begin{table*}
\label{tab:rans-cont}
\caption{Definitions:}
\begin{align}                                                      
& \rho \ \ \mbox{density}                                           & & g_r  \ \ \mbox{radial gravitational acceleration} \nonumber \\
& T \ \ \mbox{temperature}                                          & & {\mathcal S} = \rho \epsilon_\mathrm{nuc} (q) \ \ \mbox{nuclear energy production (cooling function)} \nonumber \\
& P \ \ \mbox{pressure}                                             & & \tau_{ij} = 2\mu S_{ij} \ \ \mbox{viscous stress tensor}  \ \ (\mu \ \ \mbox{kinematic viscosity}) \nonumber \\ 
& u_r, u_\theta, u_\phi \ \ \mbox{velocity components}                 & & S_{ij} = (1/2)(\partial_i u_j + \partial_j u_i) \ \ \mbox{strain rate} \nonumber \\
& {\bf u} = u (u_r, u_\theta, u_\phi) \ \ \mbox{velocity}               & & \fht{R}_{ij} = \eht{\rho}\fht{u''_i u''_j} \ \ \mbox{Reynolds stress tensor} \nonumber \\              
& j_z = r \sin{\theta} \ u_\phi \ \ \mbox{specific angular momentum} & & F_T = \chi \partial_r T \ \ \mbox{heat flux}   \nonumber \\
& d = \nabla \cdot {\bf u} \ \ \mbox{dilatation}                     & & \Gamma_1 = (d \ ln \ P/ d \ ln \ \rho)|_s   \nonumber \\ 
& \epsilon_I \ \ \mbox{specific internal energy}                     & & \Gamma_2 / (\Gamma_2 -1) =  (d \ ln \ P/ d \ ln \ T)|_s \nonumber \\
& h \ \ \mbox{specific enthalpy}                                    & &  \Gamma_3 -1 =  (d \ ln \ T/ d \ ln \ \rho)|_s \nonumber    \\  
& k = (1/2) \fht{u''_iu''_i} \ \ \mbox{turbulent kinetic energy}    & & \fht{k}^r = (1/2) \fht{u''_ru''_r} = (1/2) \fht{R}_{rr}/\eht{\rho} \ \ \mbox{radial turbulent kinetic energy}  \nonumber \\  
& \epsilon_k \ \ \mbox{specific kinetic energy}                      & & \fht{k}^\theta = (1/2)\fht{u''_\theta u''_\theta} = (1/2)\fht{R}_{\theta \theta}/\eht{\rho} \ \ \mbox{angular turbulent kinetic energy}  \nonumber \\
& \epsilon_t \ \ \mbox{specific total energy}                        & & \fht{k}^\phi = (1/2)\fht{u''_\phi u''_\phi} = (1/2) \fht{R}_{\phi \phi}/\eht{\rho} \ \ \mbox{angular turbulent kinetic energy} \nonumber \\ 
& s \ \ \mbox{specific entropy}                                      & & \fht{k}^h = \fht{k}^\theta + \fht{k}^\phi \ \ \mbox{horizontal turbulent kinetic energy}                                    \nonumber \\
& v = 1/\rho \ \ \mbox{specific volume}                               & & f_k = (1/2)\eht{\rho} \fht{u''_i u''_i u''_r} \ \ \mbox{turbulent kinetic energy flux}                                    \nonumber \\    
& X_\alpha \ \ \mbox{mass fraction of isotope $\alpha$}               & & f_k^r = (1/2)\eht{\rho} \fht{u''_r u''_r u''_r} \ \ \mbox{radial turbulent kinetic energy flux}                          \nonumber \\
& \dot{X}_\alpha^{\mathrm nuc} \ \ \mbox{rate of change of $X_\alpha$}     & & f_k^\theta = (1/2)\eht{\rho} \fht{u''_\theta u''_\theta u''_r} \ \ \mbox{angular turbulent kinetic energy flux}              \nonumber \\    
& A_\alpha \ \ \mbox{number of nucleons in isotope $\alpha$}           & & f_k^\phi = (1/2)\eht{\rho} \fht{u''_\phi u''_\phi u''_r} \ \ \mbox{angular turbulent kinetic energy flux}                   \nonumber \\ 
& Z_\alpha \ \ \mbox{charge of isotope $\alpha$}                     & & f_k^h = f_k^\theta + f_k^\phi \ \ \mbox{horizontal turbulent kinetic energy flux}                                           \nonumber \\   
& A \ \ \mbox{mean number of nucleons per isotope}          & & W_p = \eht{P'd''} \ \ \mbox{turbulent pressure dilatation}      \nonumber \\                                            
& Z \ \ \mbox{mean charge per isotope}                     & &   W_b = \eht{\rho} \eht{u''_r} \fht{g}_r \ \ \mbox{buoyancy}          \nonumber  \\
& f_P = \eht{P' u'_r} \ \ \mbox{acoustic flux}                                    & &  f_T = -\eht{T'u'_r} \ \ \mbox{heat flux ($\chi$ thermal conductivity})             \nonumber 
\end{align} 
\end{table*}

\newpage

\begin{table*}
\label{tab:rans-cont}
\caption{Definitions (continued):}
\begin{align}
& f_I = \eht{\rho} \fht{\epsilon''_I u''_r} \ \ \mbox{internal energy flux}        & & f_\alpha = \eht{\rho} \fht{X''_\alpha u''_r} \ \ \mbox{$X_\alpha$ flux}                   \nonumber \\        
& f_s = \eht{\rho} \fht{s'' u''_r} \ \ \mbox{entropy flux}                & & f_{jz} = \eht{\rho}\fht{j''_z u''_r} \ \ \mbox{angular momentum flux}    \nonumber \\ 
& f_T = \eht{u'_r T'} \ \ \mbox{turbulent heat flux}                      & & f_A = \eht{\rho}\fht{A''u''_r} \ \ \mbox{A (mean number of nucleons per isotope) flux}  \nonumber \\
& f_h = \eht{\rho}\fht{h''u''_r} \ \ \mbox{enthalpy flux}        & &  f_Z = \eht{\rho}\fht{Z''u''_r} \ \ \mbox{Z (mean charge per isotope) flux}  \nonumber \\         
& b = \overline{v'\rho'} \ \ \mbox{density-specific volume covariance}                & & \mathcal N_\rho, \mathcal N_{ur}, \mathcal N_{u\theta}, \mathcal N_{u\phi}, \mathcal N_{jz}, \mathcal N_{\alpha}, \mathcal N_{A}, \mathcal N_{Z} \ \ \mbox{numerical effect} \nonumber \\              
& f_\tau = f_\tau^r + f_\tau^\theta + f_\tau^\phi \ \ \mbox{viscous flux}                    & &  \mathcal N_{\epsilon I} = -\nabla_r f_\tau +\varepsilon_k \ \ \mbox{numerical effect} \nonumber \\ 
& f_\tau^r = -\eht{\tau'_{rr}u'_r}  \ \ \mbox{viscous flux}                               & &  \mathcal N_{\epsilon k} = -\varepsilon_k \ \ \mbox{numerical effect} \nonumber \\              
& f_\tau^\theta = -\eht{\tau'_{\theta r} u'_\theta }  \ \ \mbox{viscous flux}                 & &  \mathcal N_{\epsilon t} = -\nabla_r f_\tau \ \ \mbox{numerical effect} \nonumber \\  
& f_\tau^\phi = -\eht{\tau'_{\phi r} u'_\phi}  \ \ \mbox{viscous flux}                       & &   \mathcal N_{s} = \eht{-\varepsilon_k/T} \ \ \mbox{numerical effect} \nonumber \\          
& f_\tau^h = f_\tau^\theta + f_\tau^\phi   \ \ \mbox{viscous flux}                           & & \mathcal N_{h} = -\nabla_r f_\tau + (\Gamma_3 -1)\varepsilon_k \ \ \mbox{numerical effect} \nonumber \\              
& f_I^r = \eht{\rho}\fht{\epsilon''_I u''_r u''_r} \ \ \mbox{radial flux of $f_I$}      & &  \mathcal N_{P} = +(\Gamma_3 -1)\varepsilon_k \ \ \mbox{numerical effect} \nonumber \\
& f_s^r = \eht{\rho}\fht{s'' u''_r u''_r} \ \ \mbox{radial flux of $f_s$}               & &  \mathcal N_{T} = +\eht{(\tau_{ij} \partial_j u_i)/(c_v \rho)}  \ \ \mbox{numerical effect} \nonumber \\ 
& f_h^r = \eht{\rho}\fht{h'' u''_r u''_r} \ \ \mbox{radial flux of $f_h$}               & & \mathcal N_{Rrr} = -2\nabla_r f_\tau^r - 2\varepsilon_k^r \ \ \mbox{numerical effect} \nonumber \\
& f_T^r = \eht{T' u'_r u'_r} \ \ \mbox{radial flux of $f_T$}                            & &  \mathcal N_{R\theta \theta} = -2\nabla_r f_\tau^\theta - 2\varepsilon_k^\theta  \ \ \mbox{numerical effect} \nonumber \\
& f_{jz}^r = \eht{\rho}\fht{j''_z u''_r u''_r} \ \ \mbox{radial flux of $f_{jz}$}                           & &  \mathcal N_{R\phi \phi} = -2\nabla_r f_\tau^\phi - 2\varepsilon_k^\phi \ \ \mbox{numerical effect} \nonumber \\
& f_\alpha^r =  \eht{\rho}\fht{X''_\alpha u''_r u''_r} \ \ \mbox{radial flux of $f_\alpha$} & &  \mathcal N_{k \ } = -\nabla_r f_\tau - \varepsilon_k  \ \ \mbox{numerical effect} \nonumber \\                          
& f_A^r = \eht{\rho}\fht{A'' u''_r u''_r} \ \ \mbox{radial flux of $f_A$}               & &  \mathcal N_{kr} = -\nabla_r f_\tau^r - \varepsilon_k^r \ \ \mbox{numerical effect} \nonumber \\                                 
& f_Z^r = \eht{\rho}\fht{Z'' u''_r u''_r} \ \ \mbox{radial flux of $f_Z$}               & &  \mathcal N_{kh} = -\nabla_r f_\tau^h - \varepsilon_k^h \ \ \mbox{numerical effect} \nonumber \\                                        
& \mathcal G_k^r = -(1/2)\eht{G_{rr}^R} - \eht{u''_rG_r^M}                 & &  \mathcal N_a = -\varepsilon_a \ \ \mbox{numerical effect} \nonumber                          
\end{align}
\end{table*} 



\newpage

\begin{table*}
\label{tab:rans-cont}
\caption{Definitions (continued):}
\begin{align} 
& \mathcal G_k^\theta = -(1/2)\eht{G_{\theta \theta}^R} - \eht{u''_\theta G_\theta^M}  & &  \mathcal N_b \ \ \mbox{numerical effect} \nonumber \\               
& \mathcal G_k^\phi  = -(1/2)\eht{G_{\phi \phi}^R} - \eht{u''_\phi G_\phi^M}   & &  \mathcal N_{fI} = -\nabla_r (\eht{\epsilon''_I \tau'_{rr}}) + \eht{u''_r \tau_{ij} \partial_i u_j} -\varepsilon_I  \ \ \mbox{numerical effect} \nonumber \\               
& \mathcal G_k^h = +\mathcal G_k^\theta + \mathcal G_k^\phi                  & &  \mathcal N_{fh} = -\nabla_r (\eht{h'' \tau'_{rr}}) + \eht{u''_r (\Gamma_3 - 1) \tau_{ij} \partial_i u_j} - \eht{ u''_r \nabla_i u_i \tau_{ji}} - \varepsilon_h   \ \ \mbox{numerical effect} \nonumber \\                                   
& \mathcal G_a = +\eht{\rho' v G_r^M}                                     & &  \mathcal N_{fs} = -\nabla_r (\eht{s''\tau'_{rr}}) + \eht{u''_r \tau_{ij} \partial_i u_j/T} -\varepsilon_s  \ \ \mbox{numerical effect} \nonumber \\     
& \mathcal G_I = -\eht{G_{r}^I} - \eht{\epsilon''_I G_r^M}  & &  \mathcal N_{fA} = -\nabla_r (\eht{A''\tau'_{rr}}) - \varepsilon_A \ \ \mbox{numerical effect} \nonumber \\                                       
& \mathcal G_\alpha =  -\eht{G_{r}^\alpha} - \eht{X''_\alpha G_r^M} & &  \mathcal N_{fZ} = -\nabla_r (\eht{Z''\tau'_{rr}}) - \varepsilon_Z \ \ \mbox{numerical effect} \nonumber \\                             
& \mathcal G_A =  -\eht{G_{r}^A} - \eht{A'' G_r^M}  & & \mathcal N_{f\alpha} = -\nabla_r (\eht{\alpha''\tau'_{rr}}) - \varepsilon_\alpha  \ \ \mbox{numerical effect} \nonumber \\
& \mathcal G_Z =  -\eht{G_{r}^Z} - \eht{Z'' G_r^M}  & & \mathcal N_{fjz} =  -\nabla_r (\eht{j''_z \tau'_{rr}}) - \varepsilon_{jz}  \ \ \mbox{numerical effect} \nonumber \\ 
& \mathcal G_h =  -\eht{G_{r}^h} - \eht{h'' G_r^M}  & & \mathcal N_{fT} =  + \eht{T'\partial_i \tau_{ri} / \rho} + \eht{u'_r \tau_{ij} \partial_i u_j / \rho c_v} \ \ \mbox{numerical effect} \nonumber \\ 
& \mathcal G_T =  -\eht{G_{r}^T} - \eht{T' G_r^M}  & & \nonumber  \\
& \mathcal G_{s} = -\eht{G_{r}^s} - \eht{s'' G_r^M} & & \nonumber \\
& \mathcal G_{jz} = -\eht{G_{r}^{jz}} - \eht{j''_z G_r^M}  & &  \nonumber \\
& \sigma_\rho = \eht{\rho'\rho'} & & \mathcal N_{\sigma_\rho} \ \ \mbox{numerical effect} \nonumber \\
& \sigma_P = \eht{P'P'} & & \mathcal N_{\sigma_P} = +2 (\Gamma_3 -1)\eht{P'\tau_{ij}\partial_i u_j} \ \ \mbox{numerical effect} \nonumber \\
& \sigma_T = \eht{T'T'} & & \mathcal N_{\sigma_T} = +\eht{2 T' \tau_{ij} \partial_i u_j / \rho c_v} \ \ \mbox{numerical effect} \nonumber \\
& \sigma_{ur} = \fht{u''_r u''_r} & & \mathcal N_{\sigma_{ur}} = +2\nabla_r f_{\tau}^r - 2\varepsilon_{k}^r \ \ \mbox{numerical effect} \nonumber \\
& \sigma_{s} =  \fht{s''s''} & & \mathcal N_{\sigma_s} = +  2 \eht{s'' \tau_{ij} \partial_j u_i / T} \ \ \mbox{numerical effect} \nonumber \\
& \sigma_\alpha = \fht{X''_\alpha X''_\alpha} & & \mathcal N_{\sigma_\alpha} \ \ \mbox{numerical effect} \ \ \mbox{numerical effect} \nonumber \\
& \sigma_{\epsilon I} = \fht{\epsilon''_I \epsilon''_I} & &  N_{\sigma \epsilon_I} =  + 2\eht{\epsilon''_I \tau_{ij} \partial_j u_i} \ \ \mbox{numerical effect} \nonumber
\end{align}
\end{table*}

\newpage

\begin{table*}
\label{tab:rans-cont}
\caption{Definitions (continued):}
\begin{align} 
& \varepsilon_k^r = \eht{\tau'_{rr}\partial_r u''_r} + \eht{\tau'_{r\theta}(1/r)\partial_\theta u''_r} + \eht{\tau'_{r\phi}(1/r\sin{\theta})\partial_\phi u''_r} & & \eht{G^{M}_r}     = -\eht{\rho u_\theta u_\theta/r} - \eht{\rho u_\phi u_\phi/r} \nonumber \\
& \varepsilon_k^\theta = \eht{\tau'_{\theta r}\partial_r u''_\theta} + \eht{\tau'_{\theta \theta}(1/r)\partial_\theta u''_\theta} + \eht{\tau'_{\theta \phi}(1/r\sin{\theta})\partial_\phi u''_\theta} & & \eht{G_\theta^M} = +\eht{\rho u_\theta u_r/r} - \eht{\rho u_\phi u_\phi/(r\tan{\theta})} \nonumber \\                                       
&  \varepsilon_k^\phi = \eht{\tau'_{\phi r}\partial_r u''_\phi} + \eht{\tau'_{\phi \theta}(1/r)\partial_\theta u''_\phi} + \eht{\tau'_{\phi \phi}(1/r\sin{\theta})\partial_\phi u''_\phi} & & \eht{G_\phi^M} = +\eht{\rho u_\phi u_r/r} + \eht{\rho u_{\phi} u_{\theta}/(r \tan{\theta})}  \nonumber \\
& \varepsilon_k = \varepsilon_k^r + \varepsilon_k^\theta + \varepsilon_k^\phi  & & \eht{G^{R}_{rr}}  = -\eht{\rho u''_\theta u''_\theta u''_r/r} - \eht{\rho u''_\theta u''_r u''_\theta/r} - \eht{\rho u''_\phi u''_\phi u''_r/r} - \eht{\rho u''_\phi u''_r u''_\phi /r}  \nonumber \\
& \varepsilon_k^h = \varepsilon_k^\theta + \varepsilon_k^\phi & & \eht{G^{R}_{\theta \theta}} = +\eht{\rho u''_\theta u''_r u''_\theta/r} + \eht{\rho u''_\theta u''_\theta u''_r/r} - \eht{\rho u''_\phi u''_\phi u''_\theta/(r\tan{\theta})} - \eht{u''_\phi u''_\theta u''_\phi/(r\tan{\theta})}  \nonumber \\
& \varepsilon_a = \eht{\rho' v \nabla_r  \tau'_{rr}} & & \eht{G^{R}_{\phi \phi}} = +\eht{\rho u''_\phi u''_r r_\phi /r} + \eht{\rho u''_\phi u''_\theta u''_\phi /(r\tan{\theta})} + \eht{\rho u''_\phi u''_\phi u''_r/r} + \eht{\rho u''_\phi u''_\phi u''_\theta / (r\tan{\theta})} \nonumber \\
& \varepsilon_I = \eht{\tau'_{rr}\partial_r \epsilon''_I} + \eht{\tau'_{r\theta}(1/r)\partial_\theta \epsilon''_I} + \eht{\tau'_{r\phi}(1/r\sin{\theta})\partial_\phi \epsilon''_I} & & \eht{G^{I}_r} = -\eht{\rho \epsilon''_I u''_\theta u''_\theta/r} - \eht{\rho \epsilon''_I u''_\phi u''_\phi/r} \nonumber \\
& \varepsilon_s = \eht{\tau'_{rr}\partial_r s''} + \eht{\tau'_{r\theta}(1/r)\partial_\theta s''} + \eht{\tau'_{r\phi}(1/r\sin{\theta})\partial_\phi s''} & & \eht{G^{s}_r} = -\eht{\rho s'' u''_\theta u''_\theta/r} - \eht{\rho s'' u''_\phi u''_\phi/r} \nonumber \\
& \varepsilon_\alpha = \eht{\tau'_{rr}\partial_r X''_\alpha} + \eht{\tau'_{r\theta}(1/r)\partial_\theta X''_\alpha} + \eht{\tau'_{r\phi}(1/r\sin{\theta})\partial_\phi X''_\alpha} & & \eht{G^{\alpha}_r} =  -\eht{\rho X''_\alpha u''_\theta u''_\theta/r} - \eht{\rho X''_\alpha u''_\phi u''_\phi/r} \nonumber \\  
& \varepsilon_A = \eht{\tau'_{rr}\partial_r A''} + \eht{\tau'_{r\theta}(1/r)\partial_\theta A''} + \eht{\tau'_{r\phi}(1/r\sin{\theta})\partial_\phi A''} & & \eht{G^A_r} =  -\eht{\rho A'' u''_\theta u''_\theta/r} - \eht{\rho A'' u''_\phi u''_\phi/r} \nonumber \\
& \varepsilon_Z = \eht{\tau'_{rr}\partial_r Z''} + \eht{\tau'_{r\theta}(1/r)\partial_\theta Z''} + \eht{\tau'_{r\phi}(1/r\sin{\theta})\partial_\phi Z''}& & \eht{G^Z_r} =  -\eht{\rho Z'' u''_\theta u''_\theta/r} - \eht{\rho Z'' u''_\phi u''_\phi/r} \nonumber \\
& \varepsilon_{h} = \eht{\tau'_{rr}\partial_r h''} + \eht{\tau'_{r\theta}(1/r)\partial_\theta h''} + \eht{\tau'_{r\phi}(1/r\sin{\theta})\partial_\phi h''} & & \eht{G^h_r} = -\eht{\rho h'' u''_\theta u''_\theta/r} - \eht{\rho h'' u''_\phi u''_\phi/r} \nonumber \\
& \varepsilon_{jz} = \eht{\tau'_{rr}\partial_r j''_z} + \eht{\tau'_{r\theta}(1/r)\partial_\theta j''_z} + \eht{\tau'_{r\phi}(1/r\sin{\theta})\partial_\phi j''_z} & & \eht{G^T_r} = -\eht{\rho T' u'_\theta u'_\theta/r} - \eht{\rho T' u'_\phi u'_\phi/r} \nonumber \\
& & & \eht{G^{jz}_r} = -\eht{\rho j''_z u''_\theta u''_\theta/r} - \eht{\rho j''_z u''_\phi u''_\phi/r} \nonumber 
\end{align}
\end{table*}

\newpage

\begin{table*}
\label{tab:rans-cont}
\caption{Definitions (continued):}
\begin{align}                                                      
  & \fht{F}_{i j k}^R = \eht{\rho} \fht{u''_i u''_j u''_k}  \ \ \mbox{Reynolds stress flux} & &  \nonumber \\
  & f_\tau^{i j} = \eht{u''_i \tau_{j r}} \ \ \mbox{viscous flux}  & & \nonumber \\
  & \varepsilon_\tau^{ij} = \eht{\tau'_{ir} \partial_r u''_i} + \eht{\tau'_{i \theta} (1/r)\partial_\theta u''_j} + \eht{\tau'_{i \phi} (1/ r \sin{\theta}) \partial_\phi u''_j} \ \ \mbox{viscous dissipation} \nonumber \\
  & G_r^M = -(\rho u_{\theta}^{2} - \tau_{\theta\theta})/r - (\rho u_{\phi}^{2} - \tau_{\phi\phi})/r  & &  \nonumber \\
  & G_\theta^M = +(\rho u_{\theta} u_{r} - \tau_{\theta r})/r - (\rho u_{\phi}^{2} - \tau_{\phi\phi}) \cos{\theta}/(r \sin{\theta})  & & \nonumber \\
  & G_\phi^M = +(\rho u_{\phi} u_{r} - \tau_{\phi r})/r + (\rho u_{\phi} u_{\theta} - \tau_{\phi \theta})\cos{\theta}/(r \sin{\theta})   & & \nonumber \\
  & & & \nonumber \\
  & \mbox{Below are non-vanishing terms of space-time average over $\nabla \cdot F_{ijk}$ ($F_{ijk}$ is $3^{rd}$ order tensor)} & & \nonumber \\
  & & & \nonumber \\  
  & \eht{G_{rr}^R} =  -\eht{F^R_{\theta\theta r}/r} - \eht{F^R_{\theta r \theta}/r} - \eht{F^R_{\phi\phi r}/r} - \eht{F^R_{\phi r \phi}/r} \ & & \nonumber \\ 
  & \eht{G_{r\theta}^R} = - \eht{F^R_{\theta \theta \theta}/r} + \eht{F^R_{\theta rr}/r} - \eht{F^R_{\phi \phi \theta}/r} - \eht{F^R_{\phi r \phi} \cos{\theta}/(r\sin{\theta})}   & & \nonumber \\
  & \eht{G_{r\phi}^R} = + \eht{F^R_{\theta \theta \phi}/r} - \eht{F^R_{\phi \phi \phi}} + \eht{F^R_{\phi r \phi} \cos{\theta}/(r\sin{\theta})}  & & \nonumber \\  
  & \eht{G_{\theta r}^R} = + \eht{F^R_{\theta rr}/r} - \eht{F^R_{\theta \theta \theta}/r} -\eht{F^R_{\phi \phi r} \cos{\theta}/(r \sin{\theta})} - \eht{F^R_{\phi \theta \phi}/r}  & & \nonumber \\
  & \eht{G_{\theta \theta}^R} =  +\eht{F^R_{\theta r \theta}/r} + \eht{F^R_{\theta \theta r}/r} - \eht{F^R_{\phi \phi \theta} \cos{\theta}/(r\sin{\theta})} - \eht{F^R_{\phi \theta \phi} \cos{\theta}/(r\sin{\theta})}  & & \nonumber \\
  & \eht{G_{\theta \phi}^R} = + \eht{F^R_{\theta r \phi}/r} + \eht{F^R_{\phi \theta r}/r} + \eht{F^R_{\phi \theta \theta} \cos{\theta}/(r\sin{\theta})}  & & \nonumber \\  
  & \eht{G_{\phi r}^R} = - \eht{F^R_{\theta \phi \theta}/r} + \eht{F^R_{\phi rr}/r} + \eht{F^R_{\phi \theta r} \cos{\theta}/(r\sin{\theta})} - \eht{F^R_{\phi \phi \phi}/r}  & & \nonumber \\
  & \eht{G_{\phi \theta}^R} = + \eht{F^R_{\theta \phi r}/r} + \eht{F^R_{\phi r \theta}/r} + \eht{F^R_{\phi \theta \theta} \cos{\theta}/(r\sin{\theta})} - \eht{F^R_{\phi \theta \phi} \cos{\theta}/(r\sin{\theta})}   & & \nonumber \\
  & \eht{G_{\phi \phi}^R} = +\eht{F^R_{\phi r \phi}/r} + \eht{F^R_{\phi \theta \phi} \cos{\theta}/(r\sin{\theta})} + \eht{F^R_{\phi \phi r}/r} + \eht{F^R_{\phi \phi \theta} \cos{\theta}/(r\sin{\theta})}  & & \nonumber \\
\end{align} 

%\centerline{Differential operators}
\begin{align}
\nabla (.) = \nabla_r (.) + \nabla_\theta (.) + \nabla_\phi (.) = \frac{1}{r^2} \partial_r (r^2 . ) + \frac{1}{r\sin{\theta}} \partial_\theta (\sin{\theta} . ) + \frac{1}{r\sin{\theta}} \partial_\phi (.) \nonumber
\end{align}
%\tablecomments{Notes: $f$ is flux, $\varepsilon$ is viscous dissipation, $\mathcal G$ are geometric term, $\mathcal N$ is numerical effect}
\end{table*}


\clearpage

\section{Properties of our oxygen shell burning and red giant data}

\begin{figure}[!h]
\centerline{
\includegraphics[width=7.cm]{obmrez_tavg230_initial_model_rho_t.eps}
\includegraphics[width=7.cm]{obmrez_tavg230_mfields_rms_tpr_insf.eps}
\includegraphics[width=7.cm]{obmrez_tavg230_mean_velocities_insf.eps}}

\centerline{
\includegraphics[width=7.3cm]{rgmrez_tavg800_initial_model_rho_t.eps}
\includegraphics[width=7.3cm]{rgmrez_tavg800_mfields_rms_tpr_insf.eps}
\includegraphics[width=7.3cm]{rgmrez_tavg800_mean_velocities_insf.eps}}
\caption{Properties of our data. Model {\sf ob.3D.mr} (upper panels) and model {\sf rg.3D.mr} (lower panels). \label{fig:data}}
\end{figure}


\newpage

\subsection{Snapshots of turbulent kinetic energy in a meridional plane}

\vspace{1.cm}

\begin{figure}[!h]
\centerline{
\includegraphics[width=11.8cm]{ob3d_mrez_tke.jpg}
\includegraphics[width=11.8cm]{rg3d_mrez_tke.jpg}}
\caption{Snapshots of turbulent kinetic energy (in erg g$^{-1}$) in a meridional plane of 3D oxygen burning shell model ob.3D.mr (left) and red giant envelope convection model rg.3D.mr (right). Convectively unstable (CVZ) and stable layers (STABLE) are separated by dashed lines.}
\label{fig:ob-rg-tke-cuts}
\end{figure}

\newpage

\subsection{Summary of the Oxygen Burning Simulations and their properties}

\vspace{1.cm}

% OB Model Table
\begin{table}[!h]
\centerline{
\begin{tabular}{|l c c c c c c c c|}
\hline
%\tablecaption{Summary of the Oxygen Burning Simulations.}
{\bf Parameter} & {\sf ob.3D.lr} & {\sf ob.3D.mr} & {\sf ob.3D.hr} & {\sf ob.3D.1hp} & {\sf ob.3D.2hp} & {\sf ob.3D.4hp} & {\sf ob.3d.1hp.vc} & {\sf ob.3d.1hp.vh} \\
& & & & & & & & \\
Grid zoning &   192$\times 128^2$ & 384$\times 256^2$ & 786$\times 512^2$ & 200$\times 50^2$ & 400$\times 100^2$ & 320$\times 50^2$ & 200$\times 50^2$ & 400$\times 100^2$ \\
$r_\mathrm{in}$, $r_\mathrm{out}$ ($10^{9}$ cm)  & 0.3, 1.0 & 0.3, 1.0 & 0.3, 1.0 & 0.3, 0.9 & 0.3, 1.0 & 0.3, 1.6 & 0.3, 0.9 & 0.3, 0.9 \\
$r_\mathrm{b}^c$, $r_\mathrm{t}^c$ ($10^{9}$ cm)  &  0.43, 0.85 &  0.43, 0.85 &  0.43, 0.84 & 0.43, 0.68 & 0.43, 0.84 & 0.42, 1.4 & 0.43, 0.65 & 0.43, 0.68 \\
$\Delta \theta$, $\Delta \phi$  &  45$^\circ$ & 45$^\circ$ &45$^\circ$ & 27.5$^\circ$ & 27.5$^\circ$ & 27.5$^\circ$ & 27.5$^\circ$ & 27.5$^\circ$ \\
CZ stratification ($H_P$) & 1.9 & 1.9 & 1.9 & 1.2 & 1.9 & 4.1 & 1.1 & 1.2  \\
$\Delta t_\mathrm{av}$ (s) & $230$ & $230$ & $165$ & $900$ & $230$ & $500$ & 300 & 300    \\
$v_\mathrm{rms}$  ($10^6$ cm/s)       & 10.7 & 10.9 & 10.9 & 5.28  & 9.15 & 4.88 & 4.66 & 4.97 \\
$\tau_\mathrm{conv}$  (s)             & 78.2 & 77.1 & 75.6 & 94.7  & 89.2 & 403. & 94.6 & 95.6  \\
$P_{turb}/P_{gas} (10^{-4})$            & 3.88 &  4.05 &  4.03 &  0.96  &  3.01 &  1.73 & 0.79 & 0.96   \\
$L$ ($10^{46}$ erg/s)                & 2.74 & 2.63 & 2.58 & 0.44 & 2.86 & 0.26 & 0.40 & -0.42  \\
$L_\mathrm{d}$ ($10^{46}$  erg/s)      & 0.29 & 0.28 & 0.26 & 0.04 & 0.31  & 0.08 & 0.03 & 0.03 \\
$l_\mathrm{d}$  ($10^{8}$ cm)          & 7.39 & 7.92 & 8.73 & 3.87 & 4.15 & 5.1 & 2.85 & 4.1  \\
$\tau_\mathrm{d}$  (s)                & 34.48 & 36.47 & 39.98 & 36.72 & 22.64 & 52.04 & 30.56 & 41.1 \\
$\tau_\mathrm{dr}$ (s)                & 38.06 & 39.27 & - & 75.65 & 46.59 & 130.73 & 90.20 & 90.8  \\
$\tau_\mathrm{dh}$ (s)                & 30.77 & 32.14 & - & 22.91 & 14.21 & 28.42 & 18.4 & 25.24  \\
\hline
\end{tabular}}
\caption{boundaries of computational domain $r_\mathrm{in}$, $r_\mathrm{out}$; boundaries of convection zone at bottom and top $r_\mathrm{b}^c$, $r_\mathrm{t}^c$; angular size of computational domain $\Delta \theta$, $\Delta \phi$ ; depth of convection zone ``CZ stratification'' in pressure scale height $H_P$; averaging timescale of mean fields analysis $\Delta t_\mathrm{av}$; global rms velocity $v_\mathrm{rms}$; convective turnover timescale $\tau_\mathrm{conv}$; average ratio of turbulent ram pressure and gas pressure $p_{turb}/p_{gas}$; total luminosity of the hydrodynamic model $L$; total rate of kinetic energy dissipation $L_d$; dissipation length-scale  $l_d$; turbulent kinetic energy dissipation time-scale $\tau_d$; radial turbulent kinetic energy dissipation time-scale $\tau_{dr}$; horizontal turbulent kinetic energy dissipation time-scale $\tau_{dh}$. The numerical values may vary in time up to 20$\%$ due to limited amount of data for averaging out the time dependence.}
\label{tab:ob-models} 
\end{table}

\newpage

\subsection{Summary of Red Giant Simulations and their properties}

\vspace{1.cm}

% RG
\begin{table}[!h]
\centerline{
\begin{tabular}{|l c c c|}
\hline
%\tablecaption{Summary of the Red Giant Simulations.}
{\bf Parameter}  & {\sf rg.3D.lr} & {\sf rg.3D.mr} & {\sf rg.3D.4hp} \\
& & & \\
Grid zoning & 216$\times 128^2$ & 432$\times 256^2$ & 176$\times 128^2$ \\
$r_\mathrm{in}$, $r_\mathrm{out}$ ($10^{12}$ cm) & 0.82, 4.09 & 0.82, 4.09 & 0.82, 0.34\\
$r^c_\mathrm{in}$, $r^c_\mathrm{out}$ ($10^{12}$ cm) & 2.05, 3.86 & 2.07, 3.88 & 2.16, 3.33\\
$\Delta \theta$, $\Delta \phi$ & 45$^\circ$ & 45$^\circ$ & 45$^\circ$\\
CZ stratification ($H_p$)         & 7.0 & 7.2 & 3.5\\
$\Delta t_\mathrm{av}$ (days)       & $800$ & $800$ & $800$ \\
$v_\mathrm{rms}$ ($10^5$ cm/s)      & 2.59 & 2.66  & 2.01 \\
$\tau_\mathrm{conv}$ (days)         & 161. & 158.  & 134. \\
$P_{turb}/P_{gas} (10^{-3})$          & 4.68 &  4.98   & 0.98   \\ 
$L_\mathrm{cool}$ ($10^{36}$ erg/s)  & -8.57 & -7.13  & -9.2\\
$L_\mathrm{d}$ ($10^{36}$ erg/s)     & 7.26 & 7.24  & 2.27 \\
$l_\mathrm{d}$ ($10^{11}$ cm)        & 9.95 & 10.4  & 11.6\\
$\tau_\mathrm{d}$ (days)            & 22.2 & 22.7  & 33.3\\
$\tau_\mathrm{dr}$ (days)           & 36.7 & 44.7  & 53.0 \\
$\tau_\mathrm{dh}$ (days)           & 18.3 & 17.9  & 28.2 \\
\hline
\end{tabular}}
\caption{boundaries of computational domain $r_\mathrm{in}$, $r_\mathrm{out}$; boundaries of convection zone at bottom and top $r_\mathrm{b}^c$, $r_\mathrm{t}^c$; angular size of computational domain $\Delta \theta$, $\Delta \phi$ ; depth of convection zone ``CZ stratification'' in pressure scale height $H_P$; averaging timescale of mean fields analysis $\Delta t_\mathrm{av}$; global rms velocity $v_\mathrm{rms}$; convective turnover timescale $\tau_\mathrm{conv}$; average ratio of turbulent ram pressure and gas pressure $p_{turb}/p_{gas}$; total luminosity of the hydrodynamic model $L$; total rate of kinetic energy dissipation $L_d$; dissipation length-scale  $l_d$; turbulent kinetic energy dissipation time-scale $\tau_d$; radial turbulent kinetic energy dissipation time-scale $\tau_{dr}$; horizontal turbulent kinetic energy dissipation time-scale $\tau_{dh}$. The numerical values may vary in time up to 20$\%$ due to limited amount of data for averaging out the time dependence.}
\label{tab:rg-models} 
\end{table}


\clearpage


\section{Profiles and intergral budgets of mean fields equations}

\subsection{Mean continuity equation}

\begin{align}
\fav{D}_t \av{\rho} =& -\av{\rho} \fav{d} + {\mathcal N_\rho}  \label{eq:rans_density}
\end{align}

\begin{figure}[!h]
\centerline{
\includegraphics[width=7.3cm]{obmrez_tavg230_mean_rho_ransdat.eps}
\includegraphics[width=7.3cm]{obmrez_tavg230_continuity_equation_ransdat.eps}
\includegraphics[width=7.3cm]{obmrez_tavg230_continuity_equation_ransdat_bar.eps}}

\centerline{
\includegraphics[width=7.3cm]{rgmrez_tavg800_mean_rho_insf.eps}
\includegraphics[width=7.3cm]{rgmrez_tavg800_continuity_equation_insf.eps}
\includegraphics[width=7.3cm]{rgmrez_tavg800_continuity_equation_insf_bar.eps}}
\caption{Mean continuity equation. Model {\sf ob.3D.mr} (upper panels) and model {\sf rg.3D.mr} (lower panels). \label{fig:continuity-equation}}
\end{figure}

\newpage

\subsection{Mean radial momentum equation}

\begin{align}
\av{\rho}\fav{D}_t\fav{u}_r = & -\nabla_r \fav{R}_{rr} -\av{G^{M}_r} - \partial_r \av{P} + \av{\rho}\fav{g_r} + {\mathcal N_{ur}}
\end{align}

\begin{figure}[!h]
\centerline{
\includegraphics[width=7.3cm]{obmrez_tavg230_mean_ur_insf.eps}
\includegraphics[width=7.3cm]{obmrez_tavg230_rmomentum_equation_ransdat.eps}
\includegraphics[width=7.3cm]{obmrez_tavg230_rmomentum_equation_ransdat_bar.eps}}

\centerline{
\includegraphics[width=7.3cm]{rgmrez_tavg800_mean_ur_insf.eps}
\includegraphics[width=7.3cm]{rgmrez_tavg800_rmomentum_equation_insf.eps}
\includegraphics[width=7.3cm]{rgmrez_tavg800_rmomentum_equation_insf_bar.eps}}
\caption{Mean radial momentum equation. Model {\sf ob.3D.mr} (upper panels) and model {\sf rg.3D.mr} (lower panels). \label{fig:rm-equation}}
\end{figure}

\newpage

\subsection{Mean azimutal momentum equation}

\begin{align}
\av{\rho}\fav{D}_t\fav{u}_\theta = &  -\nabla_r \fav{R}_{\theta r} -\av{G^{M}_\theta} - (1/r)\av{\partial_\theta P} + {\mathcal N_{u \theta}}
\end{align}

\begin{figure}[!h]
\centerline{
\includegraphics[width=7.3cm]{obmrez_tavg230_mean_ut_insf.eps}
\includegraphics[width=7.3cm]{obmrez_tavg230_tmomentum_equation_insf.eps}
\includegraphics[width=7.3cm]{obmrez_tavg230_tmomentum_equation_insf_bar.eps}}

\centerline{
\includegraphics[width=7.3cm]{rgmrez_tavg800_mean_ut_insf.eps}
\includegraphics[width=7.3cm]{rgmrez_tavg800_tmomentum_equation_insf.eps}
\includegraphics[width=7.3cm]{rgmrez_tavg800_tmomentum_equation_insf_bar.eps}}
\caption{Mean azimuthal momentum equation. Model {\sf ob.3D.mr} (upper panels) and model {\sf rg.3D.mr} (lower panels). \label{fig:tm-equation}}
\end{figure}

\newpage

\subsection{Mean polar momentum equation}

\begin{align}
\av{\rho}\fav{D}_t\fav{u}_\phi = & -\nabla_r \fav{R}_{\phi r} -\av{G^{M}_\phi} + {\mathcal N_{u \phi}}
\end{align}

\begin{figure}[!h]
\centerline{
\includegraphics[width=7.3cm]{obmrez_tavg230_mean_up_insf.eps}
\includegraphics[width=7.3cm]{obmrez_tavg230_pmomentum_equation_ransdat.eps}
\includegraphics[width=7.3cm]{obmrez_tavg230_pmomentum_equation_ransdat_bar.eps}}

\centerline{
\includegraphics[width=7.3cm]{rgmrez_tavg800_mean_up_insf.eps}
\includegraphics[width=7.3cm]{rgmrez_tavg800_pmomentum_equation_insf.eps}
\includegraphics[width=7.3cm]{rgmrez_tavg800_pmomentum_equation_insf_bar.eps}}
\caption{Mean polar omentum equation. Model {\sf ob.3D.mr} (upper panels) and model {\sf rg.3D.mr} (lower panels). \label{fig:pm-equation}}
\end{figure}

\newpage

\subsection{Mean internal energy equation}

\begin{align}
\av{\rho} \fav{D}_t \fav{\epsilon}_I = & - \nabla_r  ( f_I + f_T ) - \av{P} \ \av{d} - W_P  + {\mathcal S} + {\mathcal N_{\epsilon I}}
\end{align}

\begin{figure}[!h]
\centerline{
\includegraphics[width=7.3cm]{obmrez_tavg230_mean_ei_ransdat.eps}
\includegraphics[width=7.3cm]{obmrez_tavg230_internal_energy_equation_ransdat.eps}
\includegraphics[width=7.3cm]{obmrez_tavg230_internal_energy_equation_ransdat_bar.eps}}

\centerline{
\includegraphics[width=7.3cm]{rgmrez_tavg800_mean_ei_insf.eps}                      
\includegraphics[width=7.3cm]{rgmrez_tavg800_internal_energy_equation_insf.eps}     
\includegraphics[width=7.3cm]{rgmrez_tavg800_internal_energy_equation_insf_bar.eps}}
\caption{Mean internal energy equation. Model {\sf ob.3D.mr} (upper panels) and model {\sf rg.3D.mr} (lower panels). \label{fig:ei-equation}}
\end{figure}

\newpage

\subsection{Mean kinetic energy equation}

\begin{align}
\av{\rho} \fav{D}_t \fav{\epsilon}_k = &  -\nabla_r  ( f_k +  f_P ) - \fht{R}_{ir}\partial_r \fht{u}_i + W_b + W_P +\av{\rho}\fav{D}_t (\fav{u}_i \fav{u}_i / 2) + {\mathcal N_{\epsilon k}} \label{eq:rans_mke} 
\end{align}

\begin{figure}[!h]
\centerline{
\includegraphics[width=7.3cm]{obmrez_tavg230_mean_ek_ransdat.eps}
\includegraphics[width=7.3cm]{obmrez_tavg230_kinetic_energy_equation_ransdat.eps}
\includegraphics[width=7.3cm]{obmrez_tavg230_kinetic_energy_equation_ransdat_bar.eps}}

\centerline{
\includegraphics[width=7.3cm]{rgmrez_tavg800_mean_ek_insf.eps}                      
\includegraphics[width=7.3cm]{rgmrez_tavg800_kinetic_energy_equation_insf.eps}     
\includegraphics[width=7.3cm]{rgmrez_tavg800_kinetic_energy_equation_insf_bar.eps}}
\caption{Mean kinetic energy equation. Model {\sf ob.3D.mr} (upper panels) and model {\sf rg.3D.mr} (lower panels). \label{fig:ek-equation}}
\end{figure}

\newpage

\subsection{Mean total energy equation}

\begin{align}
\av{\rho} \fav{D}_t \fav{\epsilon}_t = &  - \nabla_r ( f_I + f_T + f_k + f_P ) - \fht{R}_{ir}\partial_r \fht{u}_i - \av{P} \ \av{d} + W_b + {\mathcal S} + \av{\rho}\fav{D}_t (\fav{u}_i \fav{u}_i / 2) + {\mathcal N_{\epsilon t}} \label{eq:rans_etot}
\end{align}

\begin{figure}[!h]
\centerline{
\includegraphics[width=7.3cm]{obmrez_tavg230_mean_et_ransdat.eps}
\includegraphics[width=7.3cm]{obmrez_tavg230_total_energy_equation_ransdat.eps}
\includegraphics[width=7.3cm]{obmrez_tavg230_total_energy_equation_ransdat_bar.eps}}

\centerline{
\includegraphics[width=7.3cm]{rgmrez_tavg800_mean_et_insf.eps}                      
\includegraphics[width=7.3cm]{rgmrez_tavg800_total_energy_equation_insf.eps}     
\includegraphics[width=7.3cm]{rgmrez_tavg800_total_energy_equation_insf_bar.eps}}
\caption{Mean total energy equation. Model {\sf ob.3D.mr} (upper panels) and model {\sf rg.3D.mr} (lower panels). \label{fig:et-equation}}
\end{figure}

\newpage

\subsection{Mean entropy equation}

\begin{align}
\av{\rho} \fav{D}_t \fav{s} = & - \nabla_r  f_s    - \av{(\nabla \cdot F_T)/T}+ \av{{\mathcal S}/T} + {\mathcal N_s}  \label{eq:rans_entropy} 
\end{align}


\begin{figure}[!h]
\centerline{
\includegraphics[width=7.3cm]{obmrez_tavg230_mean_entropy_ransdat.eps}
\includegraphics[width=7.3cm]{obmrez_tavg230_entropy_equation_ransdat.eps}
\includegraphics[width=7.3cm]{obmrez_tavg230_entropy_equation_ransdat_bar.eps}}

\centerline{
\includegraphics[width=7.3cm]{rgmrez_tavg800_mean_entropy_insf.eps}                      
\includegraphics[width=7.3cm]{rgmrez_tavg800_entropy_equation_insf.eps}     
\includegraphics[width=7.3cm]{rgmrez_tavg800_entropy_equation_insf_bar.eps}}
\caption{Mean entropy equation. Model {\sf ob.3D.2hp} (upper panels) and model {\sf rg.3D.mr} (lower panels). \label{fig:ss-equation}}
\end{figure}

\newpage

\subsection{Mean pressure equation}

\begin{align}
\av{D}_t \av{P} = & -\nabla_r f_P - \Gamma_1 \eht{P} \ \eht{d} + (1 -\Gamma_1) W_P + (\Gamma_3 -1){\mathcal S} + (\Gamma_3 - 1)\nabla_r f_T + {\mathcal N_P}
\end{align}

\begin{figure}[!h]
\centerline{
\includegraphics[width=7.3cm]{obmrez_tavg230_mean_pressure_insf.eps}
\includegraphics[width=7.3cm]{obmrez_tavg230_pressure_equation_insf.eps}
\includegraphics[width=7.3cm]{obmrez_tavg230_pressure_equation_insf_bar.eps}}

\centerline{
\includegraphics[width=7.3cm]{rgmrez_tavg800_mean_pressure_insf.eps}                      
\includegraphics[width=7.3cm]{rgmrez_tavg800_pressure_equation_insf.eps}     
\includegraphics[width=7.3cm]{rgmrez_tavg800_pressure_equation_insf_bar.eps}}
\caption{Mean pressure equation. Model {\sf ob.3D.mr} (upper panels) and model {\sf rg.3D.mr} (lower panels). \label{fig:pp-equation}}
\end{figure}

\newpage

\subsection{Mean enthalpy equation}

\begin{align}
\erho\fav{D}_t \fav{h} = & -\nabla_r f_h - \Gamma_1\eht{P} \ \eht{d} - \Gamma_1 W_P + \Gamma_3 {\mathcal S} + \Gamma_3 \nabla_r f_T +  {\mathcal N_h} \label{eq:rans_h}
\end{align}

\begin{figure}[!h]
\centerline{
\includegraphics[width=7.3cm]{obmrez_tavg230_mean_enthalpy_insf.eps}
\includegraphics[width=7.3cm]{obmrez_tavg230_enthalpy_equation_insf.eps}
\includegraphics[width=7.3cm]{obmrez_tavg230_enthalpy_equation_insf_bar.eps}}

\centerline{
\includegraphics[width=7.3cm]{rgmrez_tavg800_mean_enthalpy_insf.eps}                      
\includegraphics[width=7.3cm]{rgmrez_tavg800_enthalpy_equation_insf.eps}     
\includegraphics[width=7.3cm]{rgmrez_tavg800_enthalpy_equation_insf_bar.eps}}
\caption{Mean enthalpy equation. Model {\sf ob.3D.mr} (upper panels) and model {\sf rg.3D.mr} (lower panels). \label{fig:hh-equation}}
\end{figure}

\newpage

\subsection{Mean angular momentum equation (z-component)}

\begin{align}
\erho\fav{D}_t \fav{j}_z = & -\nabla_r f_{jz} + {\mathcal N_{jz}}
\end{align}

\begin{figure}[!h]
\centerline{
\includegraphics[width=7.3cm]{obmrez_tavg230_mean_jz_insf.eps}
\includegraphics[width=7.3cm]{obmrez_tavg230_jz_equation_insf.eps}
\includegraphics[width=7.3cm]{obmrez_tavg230_jz_equation_insf_bar.eps}}

\centerline{
\includegraphics[width=7.3cm]{rgmrez_tavg800_mean_jz_insf.eps}                      
\includegraphics[width=7.3cm]{rgmrez_tavg800_jz_equation_insf.eps}     
\includegraphics[width=7.3cm]{rgmrez_tavg800_jz_equation_insf_bar.eps}}
\caption{Mean angular momentum equation. Model {\sf ob.3D.mr} (upper panels) and model {\sf rg.3D.mr} (lower panels). \label{fig:jz-equation}}
\end{figure}

\newpage

\subsection{Mean composition equation}

\begin{align}
\erho\fav{D}_t \fav{X}_\alpha = & -\nabla_r f_\alpha + \av{\rho}\fav{\dot{X}}_\alpha^{\rm nuc} + {\mathcal N_\alpha} & \erho\fav{D}_t \fav{A} = & -\nabla_r f_A - \av{\rho A^2\Sigma_\alpha (\dot{X}_\alpha^{\rm nuc} / A_\alpha)} + {\mathcal N_A}  \label{eq:rans_abar}
\end{align}

\begin{figure}[!h]
\centerline{
\includegraphics[width=7.3cm]{ob3dB_tavg300_mean_xo16_insf.eps}
\includegraphics[width=7.3cm]{ob3dB_tavg300_xo16_equation_insf.eps}
\includegraphics[width=7.3cm]{ob3dB_tavg300_xo16_equation_insf_bar.eps}}

\centerline{
\includegraphics[width=7.3cm]{ob3dB_tavg300_mean_abar_insf.eps}
\includegraphics[width=7.3cm]{ob3dB_tavg300_abar_equation_insf.eps}
\includegraphics[width=7.3cm]{ob3dB_tavg300_abar_equation_insf_bar.eps}}
\caption{Mean composition equation for O$^{16}$ and equation for mean number of nucleons per isotope. Model {\sf ob.3D.2hp}. \label{fig:alpha-abar-equation}}
\end{figure}

%\subsection{Properties of ob.3d.B}

%tavg = 300 s, timec = 1038 s

%\begin{figure}[!h]
%\centerline{
%\includegraphics[width=7.3cm]{ob3dB_tavg300_mean_rho_insf.eps}
%\includegraphics[width=7.3cm]{ob3dB_tavg300_mean_temp_insf.eps}
%\includegraphics[width=7.3cm]{ob3dB_tavg300_mfields_rms_tpr_insf.eps}}

%\centerline{
%\includegraphics[width=7.3cm]{ob3dB_tavg300_mean_urms_insf.eps}
%\includegraphics[width=7.3cm]{ob3dB_tavg300_mean_utms_insf.eps}
%\includegraphics[width=7.3cm]{ob3dB_tavg300_mean_upms_insf.eps}}
%\caption{Properties of ob.3d.B. Model {\sf ob.3D.mr} (upper panels) and model {\sf rg.3D.mr} (lower panels). \label{fig:abar-equation}}
%\end{figure}

%\newpage

\newpage

\subsection{Mean turbulent kinetic energy equation}

\begin{align}
%\fht{k} = & \frac{1}{2}\fht{R}_{ii} / \eht{\rho}  \\
\av{\rho} \fav{D}_t \fav{k}^{ } = & -\nabla_r ( f_k +  f_P ) - \fht{R}_{ir}\partial_r \fht{u}_i + W_b + W_P + {\mathcal N_k}  \label{eq:rans_tke}
\end{align}

\begin{figure}[!h]
\centerline{
\includegraphics[width=7.3cm]{obmrez_tavg230_mean_k_insf.eps}
\includegraphics[width=7.3cm]{obmrez_tavg230_mfields_k_equation_insf.eps}
\includegraphics[width=7.3cm]{obmrez_tavg230_mfields_k_equation_insf_bar.eps}}

\centerline{
\includegraphics[width=7.3cm]{rgmrez_tavg800_mean_k_insf.eps}
\includegraphics[width=7.3cm]{rgmrez_tavg800_mfields_k_equation_insf.eps}
\includegraphics[width=7.3cm]{rgmrez_tavg800_mfields_k_equation_insf_bar.eps}}
\caption{Turbulent kinetic energy equation. Model {\sf ob.3D.mr} (upper panels) and model {\sf rg.3D.mr} (lower panels). \label{fig:k-equation}}
\end{figure}

\newpage

\subsection{Mean turbulent kinetic energy equation (radial part)}

\begin{align}
%\fht{k}^r = & \frac{1}{2}\fht{R}_{rr} / \eht{\rho}  \\
\av{\rho} \fav{D}_t \fav{k}^r =  &  -\nabla_r  ( f_k^r + f_P )  - \fht{R}_{rr}\partial_r \fht{u}_r + W_b  + \eht{P'\nabla_r u''_r} + {\mathcal G_k^r} + {\mathcal N_{kr}} \label{eq:rans_ekin_r}
\end{align}

\begin{figure}[!h]
\centerline{
\includegraphics[width=7.3cm]{obmrez_tavg230_mean_kr_insf.eps}
\includegraphics[width=7.3cm]{obmrez_tavg230_mfields_k_equation_rad_insf.eps}
\includegraphics[width=7.3cm]{obmrez_tavg230_mfields_k_equation_rad_insf_bar.eps}}

\centerline{
\includegraphics[width=7.3cm]{rgmrez_tavg800_mean_kr_insf.eps}
\includegraphics[width=7.3cm]{rgmrez_tavg800_mfields_k_equation_rad_insf.eps}
\includegraphics[width=7.3cm]{rgmrez_tavg800_mfields_k_equation_rad_insf_bar.eps}}
\caption{Turbulent radial kinetic energy equation. Model {\sf ob.3D.mr} (upper panels) and model {\sf rg.3D.mr} (lower panels). \label{fig:kr-equation}}
\end{figure}

\newpage


\subsection{Mean turbulent kinetic energy equation (horizontal part)}

\begin{align}
%\fht{k}^h = & \fht{k}_\theta + \fht{k}_\phi = \frac{1}{2} \big( \fht{R}_{\theta \theta} + \fht{R}_{\phi \phi} \big) / \eht{\rho} \\
\av{\rho} \fav{D}_t \fav{k}^h =  &  -\nabla_r f_k^h - (\fht{R}_{\theta r}\partial_r \fht{u}_\theta + \fht{R}_{\phi r}\partial_r \fht{u}_\phi) + (\eht{P' \nabla_\theta u''_\theta} + \eht{P' \nabla_\phi u''_\phi}) + {\mathcal G_k^h} + {\mathcal N_{kh}} \label{eq:rans_ekin_h} 
\end{align}

\begin{figure}[!h]
\centerline{
\includegraphics[width=6.3cm]{obmrez_tavg230_mean_kh_insf.eps}
\includegraphics[width=6.3cm]{obmrez_tavg230_mfields_k_equation_hor_insf.eps}
\includegraphics[width=6.3cm]{obmrez_tavg230_mfields_k_equation_hor_insf_bar.eps}}

\centerline{
\includegraphics[width=6.3cm]{rgmrez_tavg800_mean_kh_insf.eps}
\includegraphics[width=6.3cm]{rgmrez_tavg800_mfields_k_equation_hor_insf.eps}
\includegraphics[width=6.3cm]{rgmrez_tavg800_mfields_k_equation_hor_insf_bar.eps}}
\caption{Turbulent horizontal kinetic energy equation. Model {\sf ob.3D.mr} (upper panels) and model {\sf rg.3D.mr} (lower panels). \label{fig:kh-equation}}
\end{figure}


\newpage

\subsection{Mean turbulent mass flux equation}

\begin{align}
\eht{\rho}\fht{D}_t \eht{u''_r} = & -(\eht{\rho'u'_ru'_r}/\eht{\rho})\partial_r\eht{\rho} + (\fht{R}_{rr}/\eht{\rho})/\partial_r \eht{\rho} - \eht{\rho} \nabla_r (\eht{u''_r} \ \eht{u''_r}) + \nabla_r \overline{\rho' u'_r u'_r} - \eht{\rho}\eht{u''_r} \nabla_r \eht{u}_r + \eht{\rho} \eht{u'_r d''} - b\partial_r \eht{P} + \eht{\rho' v \partial_r P'} +{\mathcal G_a} + {\mathcal N_a}
\end{align}

%\begin{align}
%\eht{\rho}\fht{D}_t \eht{u''_i} = -(\eht{\rho'u'_ru'_r}/\eht{\rho})\partial_r\eht{\rho} + (\fht{R}_{rr}/\eht{\rho})/\partial_r \eht{\rho} - \eht{\rho} \nabla_r (\eht{u''_r} \ \eht{u''_r}) + \nabla_r \overline{\rho' u'_r u'_r} - \eht{\rho}\eht{u''_r} \nabla_r \eht{u}_r + \eht{\rho} \eht{u'_r d''} - b\partial_r \eht{P} + \eht{\rho' v \partial_r P'} - \eht{\rho' v \nabla_r  \tau'_{rr} }  + \eht{\rho' v G_r^M}
%\end{align}

%\newpage

%\begin{align}
%\eht{\rho}\fht{D}_t \eht{u''_r} =&  \ {\mathcal N_a} -(\eht{\rho'u'_ru'_r}/\eht{\rho})\partial_r\eht{\rho} + (\fht{R}_{rr}/\eht{\rho})/\partial_r \eht{\rho} - \eht{\rho} \nabla_r (\eht{u''_r} \ \eht{u''_r}) + \nabla_r \overline{\rho' u'_r u'_r} - \eht{\rho}\eht{u''_r} \nabla_r \eht{u}_r + \eht{\rho} \eht{u'_r d''} - b\partial_r \eht{P} + \eht{\rho' v \partial_r P'} +{\mathcal G_a} \label{eq:rans_a}
%\end{align}

\begin{figure}[!h]
\centerline{
\includegraphics[width=7.3cm]{obmrez_tavg230_mean_a_insf.eps}
\includegraphics[width=7.3cm]{obmrez_tavg230_mfields_a_equation_insf.eps}
\includegraphics[width=7.3cm]{obmrez_tavg230_mfields_a_equation_insf_bar.eps}}

\centerline{
\includegraphics[width=7.3cm]{rgmrez_tavg800_mean_a_insf.eps}
\includegraphics[width=7.3cm]{rgmrez_tavg800_mfields_a_equation_insf.eps}
\includegraphics[width=7.3cm]{rgmrez_tavg800_mfields_a_equation_insf_bar.eps}}
\caption{Turbulent mass flux equation. Model {\sf ob.3D.mr} (upper panels) and model {\sf rg.3D.mr} (lower panels). \label{fig:a-equation}}
\end{figure}

\newpage

\subsection{Mean density-specific volume covariance equation}

\begin{align}
\eht{D}_t b = &  +\eht{v} \nabla_r \eht{\rho} \eht{u''_r} -\eht{\rho}\nabla_r (\eht{u'_r v'}) + 2\eht{\rho}\eht{v'd'}+ {\mathcal N_b} \label{eq:rans_b}
\end{align}

\begin{figure}[!h]
\centerline{
\includegraphics[width=6.3cm]{obmrez_tavg230_mean_b_insf.eps}
\includegraphics[width=6.3cm]{obmrez_tavg230_mfields_b_equation_insf.eps}
\includegraphics[width=6.3cm]{obmrez_tavg230_mfields_b_equation_insf_bar.eps}}

\centerline{
\includegraphics[width=6.3cm]{rgmrez_tavg800_mean_b_insf.eps}
\includegraphics[width=6.3cm]{rgmrez_tavg800_mfields_b_equation_insf.eps}
\includegraphics[width=6.3cm]{rgmrez_tavg800_mfields_b_equation_insf_bar.eps}}
\caption{Density-specific volume covariance equation. Model {\sf ob.3D.mr} (upper panels) and model {\sf rg.3D.mr} (lower panels). \label{fig:b-equation}}
\end{figure}

\newpage

\subsection{Mean internal energy flux equation}

\begin{align}
%\overline{\rho}\widetilde{D}_t \widetilde{c''d''} = & \overline{c'' \rho D_t d} - \overline{\rho} \widetilde{c''u''_n}\partial_n \widetilde{d} + \overline{d'' \rho D_t c} - \overline{\rho} \widetilde{d''u''_n}\partial_n \widetilde{c} - \overline{\partial_n \rho c''d''u''_n} 
\erho \fav{D}_t (f_I / \eht{\rho}) = &  {\mathcal N_{fI}} -\nabla_r f_I^r  - f_I \partial_r \fht{u}_r  - \fht{R}_{rr} \partial_r \fht{\epsilon_I} - \eht{\epsilon''_I} \partial_r \eht{P} - \eht{\epsilon''_I \partial_r P'}  - \eht{u''_r \left( P d \right)}  + \overline{u''_r ({\mathcal S} + \nabla \cdot f_T)} + {\mathcal G_I} + {\mathcal N_{fI}}\label{eq:rans_fi}
\end{align}

\begin{figure}[!h]
\centerline{
\includegraphics[width=7.3cm]{obmrez_tavg230_mean_fi_insf.eps}
\includegraphics[width=7.3cm]{obmrez_tavg230_mfields_i_equation_insf.eps}
\includegraphics[width=7.3cm]{obmrez_tavg230_mfields_i_equation_insf_bar.eps}}

\centerline{
\includegraphics[width=7.3cm]{rgmrez_tavg800_mean_fi_insf.eps}
\includegraphics[width=7.3cm]{rgmrez_tavg800_mfields_i_equation_insf.eps}
\includegraphics[width=7.3cm]{rgmrez_tavg800_mfields_i_equation_insf_bar.eps}}
\caption{Mean internal energy flux equation. Model {\sf ob.3D.mr} (upper panels) and model {\sf rg.3D.mr} (lower panels). \label{fig:fi-equation}}
\end{figure}

\newpage

\subsection{Mean entropy flux equation}

\begin{align}
%\overline{\rho}\widetilde{D}_t \widetilde{c''d''} = & \overline{c'' \rho D_t d} - \overline{\rho} \widetilde{c''u''_n}\partial_n \widetilde{d} + \overline{d'' \rho D_t c} - \overline{\rho} \widetilde{d''u''_n}\partial_n \widetilde{c} - \overline{\partial_n \rho c''d''u''_n} \\
\erho \fav{D}_t (f_s / \eht{\rho}) = &  -\nabla_r f_s^r - f_s \partial_r \fht{u}_r - \fht{R}_{rr} \partial_r \fht{s} -\eht{s''}\partial_r \eht{P} - \eht{s''\partial_r P'} + \eht{u''_r ( {\mathcal S} + \nabla \cdot F_T)  / T} + {\mathcal G_s} + {\mathcal N_{fs}}  \label{eq:rans_fs}
\end{align}

\begin{figure}[!h]
\centerline{
\includegraphics[width=7.3cm]{ob3dB_tavg300_mean_fs_insf.eps}
\includegraphics[width=7.3cm]{ob3dB_tavg300_mfields_s_equation_insf.eps}
\includegraphics[width=7.3cm]{ob3dB_tavg300_mfields_s_equation_insf_bar.eps}}

\centerline{
\includegraphics[width=7.3cm]{rgmrez_tavg800_mean_fs_insf.eps}
\includegraphics[width=7.3cm]{rgmrez_tavg800_mfields_s_equation_insf.eps}
\includegraphics[width=7.3cm]{rgmrez_tavg800_mfields_s_equation_insf_bar.eps}}
\caption{Mean entropy flux equation. Model {\sf ob.3D.2hp} (upper panels) and model {\sf rg.3D.mr} (lower panels). \label{fig:fs-equation}}
\end{figure}

\newpage

\subsection{Mean composition flux equation and mean A and Z flux equations}

\begin{align}
%\overline{\rho}\widetilde{D}_t \widetilde{c''d''} = & \overline{c'' \rho D_t d} - \overline{\rho} \widetilde{c''u''_n}\partial_n \widetilde{d} + \overline{d'' \rho D_t c} - \overline{\rho} \widetilde{d''u''_n}\partial_n \widetilde{c} - \overline{\partial_n \rho c''d''u''_n} \\
\erho \fav{D}_t (f_\alpha / \eht{\rho}) = &  -\nabla_r f_\alpha^r  - f_\alpha \partial_r \fht{u}_r - \fht{R}_{rr} \partial_r \fht{X}_\alpha -\eht{X''_\alpha} \partial_r \eht{P} - \eht{X''_\alpha \partial_r P'} + \overline{u''_r \rho \dot{X}_\alpha^{\rm nuc}} + {\mathcal G_\alpha} + {\mathcal N_{f\alpha}} \label{eq:rans_falpha} \\
\erho \fav{D}_t (f_A / \eht{\rho}) = &  \ {\mathcal N_{fA}} -\nabla_r f_A^r - f_A \partial_r \fht{u}_r - \fht{R}_{rr} \partial_r \fht{A} -\eht{A''} \partial_r \eht{P} - \eht{A'' \partial_r P'} - \overline{u''_r \rho A^2\Sigma_\alpha \dot{X}_\alpha^{\rm nuc} / A_\alpha} + {\mathcal G_A}                 \label{eq:rans_fabar} \\
\erho \fav{D}_t (f_Z / \eht{\rho}) = &  \ {\mathcal N_{fZ}} -\nabla_r f_Z^r  - f_Z \partial_r \fht{u}_r - \fht{R}_{rr} \partial_r \fht{Z} -\eht{Z''} \partial_r \eht{P} - \eht{Z'' \partial_r P'} - \overline{u''_r \rho Z A \Sigma_\alpha (\dot{X}_\alpha^{\rm nuc}/ A_\alpha)} - \nonumber \\ 
& - \overline{u''_r \rho A \Sigma_\alpha (Z_\alpha \dot{X}_\alpha^{\rm nuc} / A_\alpha)}  + {\mathcal G_Z}   \label{eq:rans_fzbar} 
\end{align}

\vspace{1.cm}

\begin{figure}[!h]
\centerline{
\includegraphics[width=7.3cm]{ob3dB_tavg300_mean_o16_insf.eps}
\includegraphics[width=7.3cm]{ob3dB_tavg300_mfields_fo16_equation_insf.eps}
\includegraphics[width=7.3cm]{ob3dB_tavg300_mfields_fo16_equation_insf_bar.eps}}
\caption{Mean composition flux equation. Model {\sf ob.3D.2hp}.\label{fig:fcomp-equation}}
\end{figure}

\newpage

\subsection{Mean A and Z flux equations}

\vspace{1.cm}

\begin{figure}[!h]
\centerline{
\includegraphics[width=7.3cm]{ob3dB_tavg300_mean_fabar_insf.eps}
\includegraphics[width=7.3cm]{ob3dB_tavg300_fabar_equation_insf.eps}
\includegraphics[width=7.3cm]{ob3dB_tavg300_fabar_equation_insf_bar.eps}}

\centerline{
\includegraphics[width=7.3cm]{ob3dB_tavg300_mean_fzbar_insf.eps}
\includegraphics[width=7.3cm]{ob3dB_tavg300_fzbar_equation_insf.eps}
\includegraphics[width=7.3cm]{ob3dB_tavg300_fzbar_equation_insf_bar.eps}}
\caption{Mean A and Z flux equation. Model {\sf ob.3D.2hp}. \label{fig:fcomp-equation}}
\end{figure}

\clearpage

\section{Mean composition equations (for all elements in oxygen burning model ob.3D.2hp)}

%\subsection{Mean H$^1$ and neutron equation}

%\begin{figure}[!h]
%\centerline{
%\includegraphics[width=7.3cm]{ob3dB_tavg300_mean_rhoxprot_insf.eps}
%\includegraphics[width=7.3cm]{ob3dB_tavg300_xprot_equation_insf.eps}
%\includegraphics[width=7.3cm]{ob3dB_tavg300_xprot_equation_insf_bar.eps}}

%\centerline{
%\includegraphics[width=7.3cm]{ob3dB_tavg300_mean_rhoxneut_insf.eps}
%\includegraphics[width=7.3cm]{ob3dB_tavg300_xneut_equation_insf.eps}
%\includegraphics[width=7.3cm]{ob3dB_tavg300_xneut_equation_insf_bar.eps}}
%\caption{Mean composition equation. Model {\sf ob.3D.2hp}. \label{fig:x1-n-equations}}
%\end{figure}

%\newpage

\subsection{Mean C$^{12}$ and O$^{16}$ equation}

\begin{figure}[!h]
%\centerline{
%\includegraphics[width=7.3cm]{ob3dB_tavg300_mean_rhoxhe4_insf.eps}
%\includegraphics[width=7.3cm]{ob3dB_tavg300_xhe4_equation_insf.eps}
%\includegraphics[width=7.3cm]{ob3dB_tavg300_xhe4_equation_insf_bar.eps}}
\centerline{
\includegraphics[width=6.8cm]{ob3dB_tavg300_mean_xc12_insf.eps}
\includegraphics[width=6.8cm]{ob3dB_tavg300_xc12_equation_insf.eps}
\includegraphics[width=6.8cm]{ob3dB_tavg300_xc12_equation_insf_bar.eps}}

\centerline{
\includegraphics[width=6.8cm]{ob3dB_tavg300_mean_xo16_insf.eps}
\includegraphics[width=6.8cm]{ob3dB_tavg300_xo16_equation_insf.eps}
\includegraphics[width=6.8cm]{ob3dB_tavg300_xo16_equation_insf_bar.eps}}
\caption{Mean composition equations. Model {\sf ob.3D.2hp}. \label{fig:xhe4-xc12-equations}}
\end{figure}

\newpage

\subsection{Mean Ne$^{20}$ and Na$^{23}$ equation}

\begin{figure}[!h]
\centerline{
\includegraphics[width=6.8cm]{ob3dB_tavg300_mean_xne20_insf.eps}
\includegraphics[width=6.8cm]{ob3dB_tavg300_xne20_equation_insf.eps}
\includegraphics[width=6.8cm]{ob3dB_tavg300_xne20_equation_insf_bar.eps}}

\centerline{
\includegraphics[width=6.8cm]{ob3dB_tavg300_mean_xna23_insf.eps}
\includegraphics[width=6.8cm]{ob3dB_tavg300_xna23_equation_insf.eps}
\includegraphics[width=6.8cm]{ob3dB_tavg300_xna23_equation_insf_bar.eps}}
\caption{Mean composition equations. Model {\sf ob.3D.2hp}. \label{fig:xo16-xne20-equations}}
\end{figure}


\newpage

\subsection{Mean Mg$^{24}$ and Si$^{28}$ equation}

\begin{figure}[!h]
\centerline{
\includegraphics[width=6.8cm]{ob3dB_tavg300_mean_xmg24_insf.eps}
\includegraphics[width=6.8cm]{ob3dB_tavg300_xmg24_equation_insf.eps}
\includegraphics[width=6.8cm]{ob3dB_tavg300_xmg24_equation_insf_bar.eps}}

\centerline{
\includegraphics[width=6.8cm]{ob3dB_tavg300_mean_xsi28_insf.eps}
\includegraphics[width=6.8cm]{ob3dB_tavg300_xsi28_equation_insf.eps}
\includegraphics[width=6.8cm]{ob3dB_tavg300_xsi28_equation_insf_bar.eps}}
\caption{Mean composition equations. Model {\sf ob.3D.2hp}. \label{fig:xna23-xmg24-equations}}
\end{figure}

\newpage

\subsection{Mean P$^{31}$ and S$^{32}$ equation}

\begin{figure}[!h]
\centerline{
\includegraphics[width=6.8cm]{ob3dB_tavg300_mean_xp31_insf.eps}
\includegraphics[width=6.8cm]{ob3dB_tavg300_xp31_equation_insf.eps}
\includegraphics[width=6.8cm]{ob3dB_tavg300_xp31_equation_insf_bar.eps}}

\centerline{
\includegraphics[width=6.8cm]{ob3dB_tavg300_mean_xs32_insf.eps}
\includegraphics[width=6.8cm]{ob3dB_tavg300_xs32_equation_insf.eps}
\includegraphics[width=6.8cm]{ob3dB_tavg300_xs32_equation_insf_bar.eps}}
\caption{Mean composition equations. Model {\sf ob.3D.2hp}. \label{fig:xsi28-xp31-equations}}
\end{figure}

\newpage

\subsection{Mean S$^{34}$ and Cl$^{35}$ equation}

\begin{figure}[!h]
\centerline{
\includegraphics[width=6.8cm]{ob3dB_tavg300_mean_xs34_insf.eps}
\includegraphics[width=6.8cm]{ob3dB_tavg300_xs34_equation_insf.eps}
\includegraphics[width=6.8cm]{ob3dB_tavg300_xs34_equation_insf_bar.eps}}

\centerline{
\includegraphics[width=6.8cm]{ob3dB_tavg300_mean_xcl35_insf.eps}
\includegraphics[width=6.8cm]{ob3dB_tavg300_xcl35_equation_insf.eps}
\includegraphics[width=6.8cm]{ob3dB_tavg300_xcl35_equation_insf_bar.eps}}
\caption{Mean composition equations. Model {\sf ob.3D.2hp}. \label{fig:xs32-xs34-equations}}
\end{figure}

\newpage

\subsection{Mean Ar$^{36}$ and Ar$^{38}$ equation}

\begin{figure}[!h]
\centerline{
\includegraphics[width=6.8cm]{ob3dB_tavg300_mean_xar36_insf.eps}
\includegraphics[width=6.8cm]{ob3dB_tavg300_xar36_equation_insf.eps}
\includegraphics[width=6.8cm]{ob3dB_tavg300_xar36_equation_insf_bar.eps}}

\centerline{
\includegraphics[width=6.8cm]{ob3dB_tavg300_mean_xar38_insf.eps}
\includegraphics[width=6.8cm]{ob3dB_tavg300_xar38_equation_insf.eps}
\includegraphics[width=6.8cm]{ob3dB_tavg300_xar38_equation_insf_bar.eps}}
\caption{Mean composition equations. Model {\sf ob.3D.2hp}. \label{fig:xcl35-xar36-equations}}
\end{figure}

\newpage

\subsection{Mean K$^{39}$ and Ca$^{40}$ equation}

\begin{figure}[!h]
\centerline{
\includegraphics[width=6.8cm]{ob3dB_tavg300_mean_xk39_insf.eps}
\includegraphics[width=6.8cm]{ob3dB_tavg300_xk39_equation_insf.eps}
\includegraphics[width=6.8cm]{ob3dB_tavg300_xk39_equation_insf_bar.eps}}

\centerline{
\includegraphics[width=6.8cm]{ob3dB_tavg300_mean_xca40_insf.eps}
\includegraphics[width=6.8cm]{ob3dB_tavg300_xca40_equation_insf.eps}
\includegraphics[width=6.8cm]{ob3dB_tavg300_xca40_equation_insf_bar.eps}}
\caption{Mean composition equations. Model {\sf ob.3D.2hp}. \label{fig:xar38-xk39-equations}}
\end{figure}

\newpage

\subsection{Mean Ca$^{42}$ and Ti$^{44}$ equation}

\begin{figure}[!h]
\centerline{
\includegraphics[width=6.8cm]{ob3dB_tavg300_mean_xca42_insf.eps}
\includegraphics[width=6.8cm]{ob3dB_tavg300_xca42_equation_insf.eps}
\includegraphics[width=6.8cm]{ob3dB_tavg300_xca42_equation_insf_bar.eps}}

\centerline{
\includegraphics[width=6.8cm]{ob3dB_tavg300_mean_xti44_insf.eps}
\includegraphics[width=6.8cm]{ob3dB_tavg300_xti44_equation_insf.eps}
\includegraphics[width=6.8cm]{ob3dB_tavg300_xti44_equation_insf_bar.eps}}
\caption{Mean composition equations. Model {\sf ob.3D.2hp}. \label{fig:xca40-ca42-equations}}
\end{figure}

\newpage

\subsection{Mean Ti$^{46}$ and Cr$^{48}$ equation}

\begin{figure}[!h]
\centerline{
\includegraphics[width=6.8cm]{ob3dB_tavg300_mean_xti46_insf.eps}
\includegraphics[width=6.8cm]{ob3dB_tavg300_xti46_equation_insf.eps}
\includegraphics[width=6.8cm]{ob3dB_tavg300_xti46_equation_insf_bar.eps}}

\centerline{
\includegraphics[width=6.8cm]{ob3dB_tavg300_mean_xcr48_insf.eps}
\includegraphics[width=6.8cm]{ob3dB_tavg300_xcr48_equation_insf.eps}
\includegraphics[width=6.8cm]{ob3dB_tavg300_xcr48_equation_insf_bar.eps}}
\caption{Mean composition equations. Model {\sf ob.3D.2hp}. \label{fig:xti44-xti46-equations}}
\end{figure}

\newpage

\subsection{Mean Cr$^{50}$ and Fe$^{52}$ equation}

\begin{figure}[!h]
\centerline{
\includegraphics[width=6.8cm]{ob3dB_tavg300_mean_xcr50_insf.eps}
\includegraphics[width=6.8cm]{ob3dB_tavg300_xcr50_equation_insf.eps}
\includegraphics[width=6.8cm]{ob3dB_tavg300_xcr50_equation_insf_bar.eps}}

\centerline{
\includegraphics[width=6.8cm]{ob3dB_tavg300_mean_xfe52_insf.eps}
\includegraphics[width=6.8cm]{ob3dB_tavg300_xfe52_equation_insf.eps}
\includegraphics[width=6.8cm]{ob3dB_tavg300_xfe52_equation_insf_bar.eps}}
\caption{Mean composition equations. Model {\sf ob.3D.2hp}. \label{fig:xcr48-xcr50-equations}}
\end{figure}

\newpage

\subsection{Mean Fe$^{54}$ and Ni$^{56}$ equation}

\begin{figure}[!h]
\centerline{
\includegraphics[width=6.8cm]{ob3dB_tavg300_mean_xfe54_insf.eps}
\includegraphics[width=6.8cm]{ob3dB_tavg300_xfe54_equation_insf.eps}
\includegraphics[width=6.8cm]{ob3dB_tavg300_xfe54_equation_insf_bar.eps}}

\centerline{
\includegraphics[width=6.8cm]{ob3dB_tavg300_mean_xni56_insf.eps}
\includegraphics[width=6.8cm]{ob3dB_tavg300_xni56_equation_insf.eps}
\includegraphics[width=6.8cm]{ob3dB_tavg300_xni56_equation_insf_bar.eps}}
\caption{Mean composition equations. Model {\sf ob.3D.2hp}. \label{fig:xfe52-xfe54-equations}}
\end{figure}

\newpage

\section{Resolution effects}

\subsection{Oxygen burning shell models}

\subsubsection{Mean continuity equation and mean radial momentum equation}

\begin{figure}[!h]
\centerline{
\includegraphics[width=7.3cm]{oblrez_tavg230_continuity_equation_ransdat.eps}
\includegraphics[width=7.3cm]{obmrez_tavg230_continuity_equation_ransdat.eps}
\includegraphics[width=7.3cm]{obhrez_tavg230_continuity_equation_ransdat.eps}}

\centerline{
\includegraphics[width=7.3cm]{oblrez_tavg230_rmomentum_equation_ransdat.eps}
\includegraphics[width=7.3cm]{obmrez_tavg230_rmomentum_equation_ransdat.eps}
\includegraphics[width=7.3cm]{obhrez_tavg230_rmomentum_equation_ransdat.eps}}
\caption{Mean continuity equation (upper panels) and radial momentum equation (lower panels). Model {\sf ob.3D.lr} (left), {\sf ob.3D.mr} (middle), {\sf ob.3D.hr} (right) \label{fig:ob-res-cont-rmomentum-equation}}
\end{figure}


\newpage

\subsubsection{Mean azimuthal and polar momentum equations}

\begin{figure}[!h]
\centerline{
\includegraphics[width=7.3cm]{oblrez_tavg230_tmomentum_equation_ransdat.eps}
\includegraphics[width=7.3cm]{obmrez_tavg230_tmomentum_equation_ransdat.eps}
\includegraphics[width=7.3cm]{obhrez_tavg230_tmomentum_equation_ransdat.eps}}

\centerline{
\includegraphics[width=7.3cm]{oblrez_tavg230_pmomentum_equation_ransdat.eps}
\includegraphics[width=7.3cm]{obmrez_tavg230_pmomentum_equation_ransdat.eps}
\includegraphics[width=7.3cm]{obhrez_tavg230_pmomentum_equation_ransdat.eps}}
\caption{Mean azimuthal momentum equation (upper panels) and polar momentum equation (lower panels). Model {\sf ob.3D.lr} (left), {\sf ob.3D.mr} (middle), {\sf ob.3D.hr} (right) \label{fig:ob-res-tmomentum-pmomentum-equation}}
\end{figure}

\newpage

\subsubsection{Mean internal and kinetic energy equation}

\begin{figure}[!h]
\centerline{
\includegraphics[width=7.3cm]{oblrez_tavg230_internal_energy_equation_ransdat.eps}
\includegraphics[width=7.3cm]{obmrez_tavg230_internal_energy_equation_ransdat.eps}
\includegraphics[width=7.3cm]{obhrez_tavg230_internal_energy_equation_ransdat.eps}}

\centerline{
\includegraphics[width=7.3cm]{oblrez_tavg230_kinetic_energy_equation_ransdat.eps}
\includegraphics[width=7.3cm]{obmrez_tavg230_kinetic_energy_equation_ransdat.eps}
\includegraphics[width=7.3cm]{obhrez_tavg230_kinetic_energy_equation_ransdat.eps}}
\caption{Mean internal energy equation (upper panels) and kinetic energy equation (lower panels). Model {\sf ob.3D.lr} (left), {\sf ob.3D.mr} (middle), {\sf ob.3D.hr} (right) \label{fig:ob-res-ei-ek-equation}}
\end{figure}

\newpage

\subsubsection{Mean total energy and entropy equation}

\begin{figure}[!h]
\centerline{
\includegraphics[width=7.3cm]{oblrez_tavg230_total_energy_equation_ransdat.eps}
\includegraphics[width=7.3cm]{obmrez_tavg230_total_energy_equation_ransdat.eps}
\includegraphics[width=7.3cm]{obhrez_tavg230_total_energy_equation_ransdat.eps}}

\centerline{
\includegraphics[width=7.3cm]{oblrez_tavg230_entropy_equation_ransdat.eps}
\includegraphics[width=7.3cm]{obmrez_tavg230_entropy_equation_ransdat.eps}
\includegraphics[width=7.3cm]{obhrez_tavg230_entropy_equation_ransdat.eps}}
\caption{Mean total energy equation (upper panels) and mean entropy equation (lower panels). Model {\sf ob.3D.lr} (left), {\sf ob.3D.mr} (middle), {\sf ob.3D.hr} (right) \label{fig:ob-res-et-ss-equation}}
\end{figure}

\newpage

\subsubsection{Mean density-specific volume covariance equation and mean number of nucleons per isotope equation}

\begin{figure}[!h]
\centerline{
\includegraphics[width=7.3cm]{oblrez_tavg230_mfields_b_equation_ransdat.eps}
\includegraphics[width=7.3cm]{obmrez_tavg230_mfields_b_equation_ransdat.eps}
\includegraphics[width=7.3cm]{obhrez_tavg230_mfields_b_equation_ransdat.eps}}

\centerline{
\includegraphics[width=7.3cm]{oblrez_tavg230_abar_equation_ransdat.eps}
\includegraphics[width=7.3cm]{obmrez_tavg230_abar_equation_ransdat.eps}
\includegraphics[width=7.3cm]{obhrez_tavg230_abar_equation_ransdat.eps}}
\caption{Mean density-specific volume covariance equation (upper panels) and mean number of nucleons per isotope equation (lower panels). Model {\sf ob.3D.lr} (left), {\sf ob.3D.mr} (middle), {\sf ob.3D.hr} (right) \label{fig:ob-res-b-A-equation}}
\end{figure}

\newpage

\subsubsection{Mean turbulent kinetic energy equation and mean velocities}

\begin{figure}[!h]
\centerline{
\includegraphics[width=7.3cm]{oblrez_tavg230_mfields_k_equation_ransdat.eps}
\includegraphics[width=7.3cm]{obmrez_tavg230_mfields_k_equation_ransdat.eps}
\includegraphics[width=7.3cm]{obhrez_tavg230_mfields_k_equation_ransdat.eps}}

\centerline{
\includegraphics[width=7.3cm]{oblrez_tavg230_mean_velocities_ransdat.eps}
\includegraphics[width=7.3cm]{obmrez_tavg230_mean_velocities_ransdat.eps}
\includegraphics[width=7.3cm]{obhrez_tavg230_mean_velocities_ransdat.eps}}
\caption{Mean turbulent kinetic energy equation (upper panels) and mean velocities (lower panels). Model {\sf ob.3D.lr} (left), {\sf ob.3D.mr} (middle), {\sf ob.3D.hr} (right) \label{fig:ob-res-k-vel-equation}}
\end{figure}

\newpage

\subsection{Red giant envelope convection}

\subsubsection{Mean continuity equation and mean radial momentum equation}

\begin{figure}[!h]
\centerline{
\includegraphics[width=7.3cm]{rglrez_tavg800_continuity_equation_insf.eps}
\includegraphics[width=7.3cm]{rgmrez_tavg800_continuity_equation_insf.eps}}

\centerline{
\includegraphics[width=7.3cm]{rglrez_tavg800_rmomentum_equation_insf.eps}
\includegraphics[width=7.3cm]{rgmrez_tavg800_rmomentum_equation_insf.eps}}
\caption{Mean continuity equation (upper panels) and radial momentum equation (lower panels). Model {\sf rg.3D.lr} (left) and {\sf rg.3D.mr} (right) \label{fig:rg-res-cont-rmomentum-equation}}
\end{figure}


\newpage

\subsubsection{Mean azimuthal and polar momentum equation}

\begin{figure}[!h]
\centerline{
\includegraphics[width=7.3cm]{rglrez_tavg800_tmomentum_equation_insf.eps}
\includegraphics[width=7.3cm]{rgmrez_tavg800_tmomentum_equation_insf.eps}}

\centerline{
\includegraphics[width=7.3cm]{rglrez_tavg800_pmomentum_equation_insf.eps}
\includegraphics[width=7.3cm]{rgmrez_tavg800_pmomentum_equation_insf.eps}}
\caption{Mean azimuthal momentum (upper panels) and polar momentum equation (lower panels). Model {\sf rg.3D.lr} (left) and {\sf rg.3D.mr} (right) \label{fig:rg-res-rmomentum-tmomentum-equation}}
\end{figure}

\newpage

\subsubsection{Mean internal and kinetic energy equation}

\begin{figure}[!h]
\centerline{
\includegraphics[width=7.3cm]{rglrez_tavg800_internal_energy_equation_insf.eps}
\includegraphics[width=7.3cm]{rgmrez_tavg800_internal_energy_equation_insf.eps}}

\centerline{
\includegraphics[width=7.3cm]{rglrez_tavg800_kinetic_energy_equation_insf.eps}
\includegraphics[width=7.3cm]{rgmrez_tavg800_kinetic_energy_equation_insf.eps}}
\caption{Mean internal energy equation (upper panels) and kinetic energy equation (lower panels). Model {\sf rg.3D.lr} (left) and {\sf rg.3D.mr} (right) \label{fig:rg-res-ei-ek-equation}}
\end{figure}

\newpage

\subsubsection{Mean total energy equation and mean entropy equation}

\begin{figure}[!h]
\centerline{
\includegraphics[width=7.3cm]{rglrez_tavg800_total_energy_equation_insf.eps}
\includegraphics[width=7.3cm]{rgmrez_tavg800_total_energy_equation_insf.eps}}

\centerline{
\includegraphics[width=7.3cm]{rglrez_tavg800_entropy_equation_insf.eps}
\includegraphics[width=7.3cm]{rgmrez_tavg800_entropy_equation_insf.eps}}
\caption{Mean total energy equation (upper panels) and mean entropy equation (lower panels). Model {\sf rg.3D.lr} (left) and {\sf rg.3D.mr} (right) \label{fig:rg-res-et-ss-equation}}
\end{figure}

\newpage

\subsubsection{Mean density-specific volume covariance and entropy flux equation}

\begin{figure}[!h]
\centerline{
\includegraphics[width=7.3cm]{rglrez_tavg800_mfields_b_equation_insf.eps}
\includegraphics[width=7.3cm]{rgmrez_tavg800_mfields_b_equation_insf.eps}}

\centerline{
\includegraphics[width=7.3cm]{rglrez_tavg800_mfields_s_equation_insf.eps}
\includegraphics[width=7.3cm]{rgmrez_tavg800_mfields_s_equation_insf.eps}}
\caption{Mean density-specific volume covariance equation (upper panels) and entropy flux equation (lower panels). Model {\sf rg.3D.lr} (left) and {\sf rg.3D.mr} (right) \label{fig:rg-res-ss-fssx-equation}}
\end{figure}

\newpage

\subsubsection{Mean turbulent kinetic energy equation and mean turbulent mass flux equation}

\begin{figure}[!h]
\centerline{
\includegraphics[width=7.3cm]{rglrez_tavg800_mfields_k_equation_insf.eps}
\includegraphics[width=7.3cm]{rgmrez_tavg800_mfields_k_equation_insf.eps}}

\centerline{
\includegraphics[width=7.3cm]{rglrez_tavg800_mfields_a_equation_insf.eps}
\includegraphics[width=7.3cm]{rgmrez_tavg800_mfields_a_equation_insf.eps}}

\caption{Mean turbulent kinetic energy equation (upper panels) and mean turbulent mass flux equation (lower panels). Model {\sf rg.3D.lr} (left) and {\sf rg.3D.mr} (right) \label{fig:rg-res-k-vel-equation}}
\end{figure}

\newpage

\section{Wedge-size effects}

\subsection{Oxygen burning shell models}

\subsubsection{Mean continuity equation and mean radial momentum equation}

\begin{figure}[!h]
\centerline{
\includegraphics[width=7.3cm]{obmrez_tavg230_continuity_equation_ransdat.eps}
\includegraphics[width=7.3cm]{ob3dB_tavg230_continuity_equation_insf.eps}}

\centerline{
\includegraphics[width=7.3cm]{obmrez_tavg230_rmomentum_equation_insf.eps}
\includegraphics[width=7.3cm]{ob3dB_tavg230_rmomentum_equation_insf.eps}}
\caption{Continuity equation (upper panels) and radial momentum equation (lower panels). Model {\sf ob.3D.mr} (45\dgr wedge - left) and {\sf ob.3D.2hp} (27.5\dgr wedge - right). \label{fig:ob-wedge-effects-cont-rmomentum-eq}}
\end{figure}

\newpage

\subsubsection{Mean azimuthal momentum and polar momentum equation}

\begin{figure}[!h]
\centerline{
\includegraphics[width=7.3cm]{obmrez_tavg230_tmomentum_equation_insf.eps}
\includegraphics[width=7.3cm]{ob3dB_tavg230_tmomentum_equation_insf.eps}}

\centerline{
\includegraphics[width=7.3cm]{obmrez_tavg230_pmomentum_equation_insf.eps}
\includegraphics[width=7.3cm]{ob3dB_tavg230_pmomentum_equation_insf.eps}}
\caption{Mean azimuthal momentum (upper panels) and polar momentum equation (lower panels). Model {\sf ob.3D.mr} (45\dgr wedge - left) and {\sf ob.3D.2hp} (27.5\dgr wedge - right). \label{fig:ob-wedge-pmomentum-tmomentum-equation-eq}}
\end{figure}

\newpage

\subsubsection{Mean internal energy equation and total energy equation}

\begin{figure}[!h]
\centerline{
\includegraphics[width=7.3cm]{obmrez_tavg230_internal_energy_equation_insf.eps}
\includegraphics[width=7.3cm]{obmrez_tavg230_internal_energy_equation_insf.eps}}

\centerline{
\includegraphics[width=7.3cm]{obmrez_tavg230_total_energy_equation_insf.eps}
\includegraphics[width=7.3cm]{ob3dB_tavg230_total_energy_equation_insf.eps}}
\caption{Mean internal energy (upper panels) equations and mean total energy equation (lower panels). Model {\sf ob.3D.mr} (45\dgr wedge - left) and {\sf ob.3D.2hp} (27.5\dgr wedge - right). \label{fig:obwedge-ei-et-eq}}
\end{figure}

\newpage

\subsubsection{Mean turbulent kinetic energy equation and turbulent mass flux equation}

\begin{figure}[!h]
\centerline{
\includegraphics[width=7.3cm]{obmrez_tavg230_mfields_k_equation_insf.eps}
\includegraphics[width=7.3cm]{ob3dB_tavg230_mfields_k_equation_insf.eps}}

\centerline{
\includegraphics[width=7.3cm]{obmrez_tavg230_mfields_a_equation_insf.eps}
\includegraphics[width=7.3cm]{ob3dB_tavg230_mfields_a_equation_insf.eps}}
\caption{Mean turbulent kinetic energy equation and turbulent mass flux equation. Model {\sf ob.3D.mr} (45\dgr wedge - left) and {\sf ob.3D.2hp} (27.5\dgr wedge - right). \label{fig:ob-wedge-k-a-eq}}
\end{figure}

\newpage

\subsubsection{Mean density-specific volume covariance and internal energy flux equation}

\begin{figure}[!h]
\centerline{
\includegraphics[width=7.3cm]{obmrez_tavg230_mfields_b_equation_insf.eps}
\includegraphics[width=7.3cm]{ob3dB_tavg230_mfields_b_equation_insf.eps}}

\centerline{
\includegraphics[width=7.3cm]{obmrez_tavg230_mfields_i_equation_insf.eps}
\includegraphics[width=7.3cm]{ob3dB_tavg230_mfields_i_equation_insf.eps}}
\caption{Mean density-specific volume covariance equation (upper panels) and mean internal energy flux equation (lower panels). Model {\sf ob.3D.mr} (45\dgr wedge - left) and {\sf ob.3D.2hp} (27.5\dgr wedge - right). \label{fig:ob-wedge-b-i-eq}}
\end{figure}

\newpage

\subsubsection{Mean kinetic energy equation and mean velocities}

\begin{figure}[!h]
\centerline{
\includegraphics[width=7.3cm]{obmrez_tavg230_mean_ek_insf.eps}
\includegraphics[width=7.3cm]{ob3dB_tavg230_mean_ek_insf.eps}}

\centerline{
\includegraphics[width=7.3cm]{obmrez_tavg230_mean_velocities2_insf.eps}
\includegraphics[width=7.3cm]{ob3dB_tavg230_mean_velocities_insf.eps}}
\caption{Mean kinetic energy equation (upper panels) mean velocities (lower panels). Model {\sf ob.3D.mr} (45\dgr wedge - left) and {\sf ob.3D.2hp} (27.5\dgr wedge - right).  \label{fig:ob-wedge-ek-vel}}
\end{figure}



%\newpage

%\section{RANSDAT versus data calculated by tseries insf.pro}

%\subsection{Oxygen burning shell models}
 
%\subsubsection{Favrian mean velocities}
%
%\begin{figure}[!ht]
%\centerline{
%\includegraphics[width=7.3cm]{obmrez_tavg230_mean_ur_ransdat.eps}
%\includegraphics[width=7.3cm]{obmrez_tavg230_mean_ur_insf.eps}}

%\centerline{
%\includegraphics[width=7.3cm]{obmrez_tavg230_mean_ut_ransdat.eps}
%\includegraphics[width=7.3cm]{obmrez_tavg230_mean_ut_insf.eps}}
%\caption{Mean velocities \label{fig:ransdat-vs-insf-mean-vel}}
%\end{figure}

%\newpage

%\subsubsection{Favrian mean velocities}

%\begin{figure}[!ht]
%\centerline{
%\includegraphics[width=7.3cm]{obmrez_tavg230_mean_up_ransdat.eps}
%\includegraphics[width=7.3cm]{obmrez_tavg230_mean_up_insf.eps}}

%\centerline{
%\includegraphics[width=7.3cm]{obmrez_tavg230_continuity_equation_ransdat.eps}
%\includegraphics[width=7.3cm]{obmrez_tavg230_continuity_equation_insf.eps}}
%\caption{Mean velocities \label{fig:ransdat-vs-insf-mean-vel}}
%\end{figure}

%\newpage

%\subsubsection{Mean velocities and mean kinetic energy equation}

%\begin{figure}[!h]
%\centerline{
%\includegraphics[width=7.3cm]{obmrez_tavg230_mean_velocities_ransdat.eps}
%\includegraphics[width=7.3cm]{obmrez_tavg230_mean_velocities_insf.eps}}

%\centerline{
%\includegraphics[width=7.3cm]{obmrez_tavg230_mean_ek_ransdat.eps}
%\includegraphics[width=7.3cm]{obmrez_tavg230_mean_ek_insf.eps}}
%\caption{Continuity equation \label{continuity}}
%\end{figure}

%\newpage

%\subsubsection{Mean momentum equations}

%\begin{figure}[!h]
%\centerline{
%\includegraphics[width=7.3cm]{obmrez_tavg230_rmomentum_equation_ransdat.eps}
%\includegraphics[width=7.3cm]{obmrez_tavg230_rmomentum_equation_insf.eps}}

%\centerline{
%\includegraphics[width=7.3cm]{obmrez_tavg230_tmomentum_equation_ransdat.eps}
%\includegraphics[width=7.3cm]{obmrez_tavg230_tmomentum_equation_insf.eps}}
%\caption{Momentum equations, multiplied by 4$\pi r^2$ \label{fig:ransdat-vs-insf-momentum-eq}}
%\end{figure}

%\newpage

%\subsubsection{Mean p momentum equation and mean internal energy equation}

%\begin{figure}[!h]
%\centerline{
%\includegraphics[width=7.3cm]{obmrez_tavg230_pmomentum_equation_ransdat.eps}
%\includegraphics[width=7.3cm]{obmrez_tavg230_pmomentum_equation_insf.eps}}

%\centerline{
%\includegraphics[width=7.3cm]{obmrez_tavg230_internal_energy_equation_ransdat.eps}
%\includegraphics[width=7.3cm]{obmrez_tavg230_internal_energy_equation_insf.eps}}
%\caption{Mean energy equations, multiplied by 4$\pi r^2$ \label{fig:ransdat-vs-insf-energy-eq}}
%\end{figure}

\newpage

%\subsubsection{Mean kinetic energy equation and total energy equation}

%\begin{figure}[!h]
%\centerline{
%\includegraphics[width=7.3cm]{obmrez_tavg230_kinetic_energy_equation_ransdat.eps}
%\includegraphics[width=7.3cm]{obmrez_tavg230_kinetic_energy_equation_insf.eps}}

%\centerline{
%\includegraphics[width=7.3cm]{obmrez_tavg230_total_energy_equation_ransdat.eps}
%\includegraphics[width=7.3cm]{obmrez_tavg230_total_energy_equation_insf.eps}}
%\caption{Mean energy equations, multiplied by 4$\pi r^2$ \label{fig:ransdat-vs-insf-energy-eq}}
%\end{figure}

%\newpage

%\subsubsection{Density-specific volume covariance, turbulent kinetic energy equation}

%\begin{figure}[!h]
%\centerline{
%\includegraphics[width=7.3cm]{obmrez_tavg230_mfields_b_equation_ransdat.eps}
%\includegraphics[width=7.3cm]{obmrez_tavg230_mfields_b_equation_insf.eps}}

%\centerline{
%\includegraphics[width=7.3cm]{obmrez_tavg230_mfields_k_equation_ransdat.eps}
%\includegraphics[width=7.3cm]{obmrez_tavg230_mfields_k_equation_insf.eps}}
%\caption{Mean energy equations, multiplied by 4$\pi r^2$ \label{fig:ransdat-vs-insf-energy-eq}}
%\end{figure}

\newpage

\section{Sensitivity to averaging window}

\subsection{Oxygen burning shell model}

\subsubsection{Mean continuity equation and mean radial momentum equation}

\begin{figure}[!h]
\centerline{
\includegraphics[width=7.3cm]{obmrez275_tavg150_continuity_equation_ransdat.eps}
\includegraphics[width=7.3cm]{obmrez275_tavg230_continuity_equation_ransdat.eps}
\includegraphics[width=7.3cm]{obmrez275_tavg460_continuity_equation_ransdat.eps}}

\centerline{
\includegraphics[width=7.3cm]{obmrez275_tavg150_rmomentum_equation_ransdat.eps}
\includegraphics[width=7.3cm]{obmrez275_tavg230_rmomentum_equation_ransdat.eps}
\includegraphics[width=7.3cm]{obmrez275_tavg460_rmomentum_equation_ransdat.eps}}
\caption{Mean continuity equation (upper panels) and radial momentum equation (lower panels) from model {\sf ob.3D.2hp}. Averaging window over roughly 2 convective turnover timescales 150 s (left), 3 convective turnover timescales 230 s (middle) and 4 convective turnover timescales 460 s (right).}
\end{figure}

\newpage

\subsubsection{Mean azimuthal and polar momentum equations}

\begin{figure}[!h]
\centerline{
\includegraphics[width=7.3cm]{obmrez275_tavg150_tmomentum_equation_ransdat.eps}
\includegraphics[width=7.3cm]{obmrez275_tavg230_tmomentum_equation_ransdat.eps}
\includegraphics[width=7.3cm]{obmrez275_tavg460_tmomentum_equation_ransdat.eps}}

\centerline{
\includegraphics[width=7.3cm]{obmrez275_tavg230_pmomentum_equation_ransdat.eps}
\includegraphics[width=7.3cm]{obmrez275_tavg230_pmomentum_equation_ransdat.eps}
\includegraphics[width=7.3cm]{obmrez275_tavg460_pmomentum_equation_ransdat.eps}}
\caption{Mean azimuthal equation (upper panels) and mean polar momentum equation (lower panels) from model {\sf ob.3D.2hp}. Averaging window over roughly 2 convective turnover timescales 150 s (left), 3 convective turnover timescales 230 s (middle) and 4 convective turnover timescales 460 s (right). }
\end{figure}

\newpage

\subsubsection{Mean total energy equation and mean turbulent kinetic energy equation}

\begin{figure}[!h]
\centerline{
\includegraphics[width=7.3cm]{obmrez275_tavg150_total_energy_equation_ransdat.eps}
\includegraphics[width=7.3cm]{obmrez275_tavg230_total_energy_equation_ransdat.eps}
\includegraphics[width=7.3cm]{obmrez275_tavg460_total_energy_equation_ransdat.eps}}

\centerline{
\includegraphics[width=7.3cm]{obmrez275_tavg150_mfields_k_equation_ransdat.eps}
\includegraphics[width=7.3cm]{obmrez275_tavg230_mfields_k_equation_ransdat.eps}
\includegraphics[width=7.3cm]{obmrez275_tavg460_mfields_k_equation_ransdat.eps}}
\caption{Mean total energy equation (upper panels) and mean turbulent kinetic energy equation (lower panels) from model {ob.3D.2hp}. Averaging window over roughly 2 convective turnover timescales 150 s (left), 3 convective turnover timescales 230 s (middle) and 4 convective turnover timescales 460 s (right).}
\end{figure}

\newpage

\subsubsection{Mean entropy equation and mean number of nucleons per isotope equation}

\begin{figure}[!h]
\centerline{
\includegraphics[width=7.3cm]{obmrez275_tavg150_entropy_equation_ransdat.eps}
\includegraphics[width=7.3cm]{obmrez275_tavg230_entropy_equation_ransdat.eps}
\includegraphics[width=7.3cm]{obmrez275_tavg460_entropy_equation_ransdat.eps}}

\centerline{
\includegraphics[width=7.3cm]{obmrez275_tavg150_abar_equation_ransdat.eps}
\includegraphics[width=7.3cm]{obmrez275_tavg230_abar_equation_ransdat.eps}
\includegraphics[width=7.3cm]{obmrez275_tavg460_abar_equation_ransdat.eps}}
\caption{Mean entropy equtaion (upper panels) and mean number of nucleons per isotope (upper panels) from model {\sf ob.3D.2hp}. Averaging window over roughly 2 convective turnover timescales 150 s (left), 3 convective turnover timescales 230 s (middle) and 4 convective turnover timescales 460 s (right).}
\end{figure}

\newpage

\subsubsection{Mean turbulent kinetic energy and mean velocities}

\begin{figure}[!h]
\centerline{
\includegraphics[width=7.3cm]{obmrez275_tavg150_mean_k_ransdat.eps}
\includegraphics[width=7.3cm]{obmrez275_tavg230_mean_k_ransdat.eps}
\includegraphics[width=7.3cm]{obmrez275_tavg460_mean_k_ransdat.eps}}

\centerline{
\includegraphics[width=7.3cm]{obmrez275_tavg150_mean_velocities_ransdat.eps}
\includegraphics[width=7.3cm]{obmrez275_tavg230_mean_velocities_ransdat.eps}
\includegraphics[width=7.3cm]{obmrez275_tavg460_mean_velocities_ransdat.eps}}
\caption{Mean turbulent kinetic energy equation (upper panels) and mean velocities from model {\sf ob.3D.2hp}. Averaging window over roughly 2 convective turnover timescales 150 s (left), 3 convective turnover timescales 230 s (middle) and 4 convective turnover timescales 460 s (right).}
\end{figure}


\newpage

\section{Depth dependence}

\subsection{Oxygen burning shell model}

\subsubsection{Mean continuity equation and mean radial momentum equation}

\begin{figure}[!h]
\centerline{
\includegraphics[width=7.3cm]{ob1hp_tavg900_continuity_equation_insf.eps}
\includegraphics[width=7.3cm]{ob3dB_tavg230_continuity_equation_insf.eps}
\includegraphics[width=7.3cm]{ob4hp_tavg500_continuity_equation_insf.eps}}

\centerline{
\includegraphics[width=7.3cm]{ob1hp_tavg900_rmomentum_equation_insf.eps}
\includegraphics[width=7.3cm]{ob3dB_tavg230_rmomentum_equation_insf.eps}
\includegraphics[width=7.3cm]{ob4hp_tavg500_rmomentum_equation_insf.eps}}
\caption{Mean continuity equation (upper panels) and radial momentum equation (lower panels). 1 Hp model {\sf ob.3D.1hp} (left), 2 Hp model {\sf ob.3D.2hp} (middle) and 4 Hp model {\sf ob.3D.4hp} (right)}
\end{figure}

\newpage

\subsubsection{Mean azimuthal and polar momentum equation}

\begin{figure}[!h]
\centerline{
\includegraphics[width=7.0cm]{ob1hp_tavg900_tmomentum_equation_insf.eps}
\includegraphics[width=7.0cm]{ob3dB_tavg230_tmomentum_equation_insf.eps}
\includegraphics[width=7.0cm]{ob4hp_tavg500_tmomentum_equation_insf.eps}}

\centerline{
\includegraphics[width=7.0cm]{ob1hp_tavg900_pmomentum_equation_insf.eps}
\includegraphics[width=7.0cm]{ob3dB_tavg230_pmomentum_equation_insf.eps}
\includegraphics[width=7.0cm]{ob4hp_tavg500_pmomentum_equation_insf.eps}}
\caption{Mean azimuthal equation (upper panels) and mean polar momentum equation (lower panels). 1 Hp model {\sf ob.3D.1hp} (left), 2 Hp model {\sf ob.3D.2hp} (middle) and 4 Hp model {\sf ob.3D.4hp} (right).}
\end{figure}


%\newpage


%\subsubsection{Pressure and enthalpy equations}

%\begin{figure}[!h]
%\centerline{
%\includegraphics[width=7.3cm]{ob1hp_tavg900_pressure_equation_insf.eps}
%\includegraphics[width=7.3cm]{ob3dB_tavg230_pressure_equation_insf.eps}
%\includegraphics[width=7.3cm]{ob4hp_tavg500_pressure_equation_insf.eps}}

%\centerline{
%\includegraphics[width=7.3cm]{ob1hp_tavg900_enthalpy_equation_insf.eps}
%\includegraphics[width=7.3cm]{ob3dB_tavg230_enthalpy_equation_insf.eps}
%\includegraphics[width=7.3cm]{ob4hp_tavg500_enthalpy_equation_insf.eps}}
%\caption{1 Hp model {\sf ob.3D.1hp} (left), 2 Hp model {\sf ob.3D.2hp} (left) and 4 Hp model {\sf ob.3D.4hp} (right).}
%\end{figure}

\newpage

\subsubsection{Mean total energy equation and turbulent kinetic energy equation}

\begin{figure}[!h]
\centerline{
\includegraphics[width=7.3cm]{ob1hp_tavg900_total_energy_equation_insf.eps}
\includegraphics[width=7.3cm]{ob3dB_tavg230_total_energy_equation_insf.eps}
\includegraphics[width=7.3cm]{ob4hp_tavg500_total_energy_equation_insf.eps}}

\centerline{
\includegraphics[width=7.3cm]{ob1hp_tavg900_mfields_k_equation_insf.eps}
\includegraphics[width=7.3cm]{ob3dB_tavg230_mfields_k_equation_insf.eps}
\includegraphics[width=7.3cm]{ob4hp_tavg500_mfields_k_equation_insf.eps}}
%\centerline{
%\includegraphics[width=7.3cm]{ob1hp_tavg900_mfields_k_equation_rad_insf_bar.eps}
%\includegraphics[width=7.3cm]{ob3dB_tavg230_mfields_k_equation_rad_insf_bar.eps}
%\includegraphics[width=7.3cm]{ob4hp_tavg500_mfields_k_equation_rad_insf_bar.eps}}
\caption{Mean total energy equation (upper panels) and mean turbulent kinetic energy equation (lower panels). 1 Hp model {\sf ob.3D.1hp} (left), 2 Hp model {\sf ob.3D.2hp} (middle) and 4 Hp model {\sf ob.3D.4hp} (right).}
\end{figure}

\newpage

\subsubsection{Mean turbulent kinetic energy equation (radial + horizontal part)}

\begin{figure}[!h]
\centerline{
\includegraphics[width=7.3cm]{ob1hp_tavg900_mfields_k_equation_rad_insf.eps}
\includegraphics[width=7.3cm]{ob3dB_tavg230_mfields_k_equation_rad_insf.eps}
\includegraphics[width=7.3cm]{ob4hp_tavg500_mfields_k_equation_rad_insf.eps}}

\centerline{
\includegraphics[width=7.3cm]{ob1hp_tavg900_mfields_k_equation_hor_insf.eps}
\includegraphics[width=7.3cm]{ob3dB_tavg230_mfields_k_equation_hor_insf.eps}
\includegraphics[width=7.3cm]{ob4hp_tavg500_mfields_k_equation_hor_insf.eps}}
\caption{Radial (upper panels) and horizontal (lower panels) part of the mean turbulent kinetic energy equation. 1 Hp model {\sf ob.3D.1hp} (left), 2 Hp model {\sf ob.3D.2hp} (middle) and 4 Hp model {\sf ob.3D.4hp} (right).}
\end{figure}

\newpage

\subsubsection{Mean turbulent mass flux equation and mean density-specific volume covariance equation}

\begin{figure}[!h]
\centerline{
\includegraphics[width=7.3cm]{ob1hp_tavg900_mfields_a_equation_insf.eps}
\includegraphics[width=7.3cm]{ob3dB_tavg230_mfields_a_equation_insf.eps}
\includegraphics[width=7.3cm]{ob4hp_tavg500_mfields_a_equation_insf.eps}}

\centerline{
\includegraphics[width=7.3cm]{ob1hp_tavg900_mfields_b_equation_insf.eps}
\includegraphics[width=7.3cm]{ob3dB_tavg230_mfields_b_equation_insf.eps}
\includegraphics[width=7.3cm]{ob4hp_tavg500_mfields_b_equation_insf.eps}}
\caption{Mean turbulent mass flux equation (upper panels) and density-specific volume covariance equation (lower panels). 1 Hp model {\sf ob.3D.1hp} (left), 2 Hp model {\sf ob.3D.2hp} (middle) and 4 Hp model {\sf ob.3D.4hp} (right).}
\end{figure}

\newpage

\subsubsection{Mean specific angular momentum equation and internal energy flux equation}

\begin{figure}[!h]
\centerline{
\includegraphics[width=7.3cm]{ob1hp_tavg900_jz_equation_insf.eps}
\includegraphics[width=7.3cm]{ob3dB_tavg230_jz_equation_insf.eps}
\includegraphics[width=7.3cm]{ob4hp_tavg500_jz_equation_insf.eps}}

\centerline{
\includegraphics[width=7.3cm]{ob1hp_tavg900_mfields_i_equation_insf.eps}
\includegraphics[width=7.3cm]{ob3dB_tavg230_mfields_i_equation_insf.eps}
\includegraphics[width=7.3cm]{ob4hp_tavg500_mfields_i_equation_insf.eps}}
\caption{Mean specific angular momentum equation (upper panels) and mean turbulent internal energy flux equation (lower panels). 1 Hp model {\sf ob.3D.1hp} (left), 2 Hp model {\sf ob.3D.2hp} (middle) and 4 Hp model {\sf ob.3D.4hp} (right).}
\end{figure}

\newpage

\subsubsection{Mean entropy equation and mean entropy flux equation}

\begin{figure}[!h]
\centerline{
\includegraphics[width=7.3cm]{ob1hp_tavg900_entropy_equation_insf.eps}
\includegraphics[width=7.3cm]{ob3dB_tavg230_entropy_equation_insf.eps}
\includegraphics[width=7.3cm]{ob4hp_tavg500_entropy_equation_insf.eps}}

\centerline{
\includegraphics[width=7.3cm]{ob1hp_tavg900_mfields_s_equation_insf.eps}
\includegraphics[width=7.3cm]{ob3dB_tavg230_mfields_s_equation_insf.eps}
\includegraphics[width=7.3cm]{ob4hp_tavg500_mfields_s_equation_insf.eps}}
\caption{Mean entropy equation (upper panels) and mean entropy flux equation (lower panels). 1 Hp model {\sf ob.3D.1hp} (left), 2 Hp model {\sf ob.3D.2hp} (middle) and 4 Hp model {\sf ob.3D.4hp} (right).}
\end{figure}

\newpage

\subsubsection{Mean turbulent kinetic energy and mean velocities}

\begin{figure}[!h]
\centerline{
\includegraphics[width=7.3cm]{ob1hp_tavg900_mean_k_insf.eps}
\includegraphics[width=7.3cm]{ob3dB_tavg230_mean_k_insf.eps}
\includegraphics[width=7.3cm]{ob4hp_tavg500_mean_k_insf.eps}}

\centerline{
\includegraphics[width=7.3cm]{ob1hp_tavg900_mean_velocities_insf.eps}
\includegraphics[width=7.3cm]{ob3dB_tavg230_mean_velocities_insf.eps}
\includegraphics[width=7.3cm]{ob4hp_tavg500_mean_velocities_insf.eps}}
\caption{Mean turbulent kinetic energy (upper panels) and mean velocities (lower panels). 1 Hp model {\sf ob.3D.1hp} (left), 2 Hp model {\sf ob.3D.2hp} (middle) and 4 Hp model {\sf ob.3D.4hp} (right).}
\end{figure}

\newpage

\subsection{Red giant convection envelope model}

\subsubsection{Mean continuity equation and mean radial momentum equation}

\begin{figure}[!h]
\centerline{
\includegraphics[width=7.3cm]{rg4hp_tavg800_continuity_equation_insf.eps}
\includegraphics[width=7.3cm]{rgmrez_tavg800_continuity_equation_insf.eps}}

\centerline{
\includegraphics[width=7.3cm]{rg4hp_tavg800_rmomentum_equation_insf.eps}
\includegraphics[width=7.3cm]{rgmrez_tavg800_rmomentum_equation_insf.eps}}
\caption{Mean continuity equation (upper panels) and radial momentum equation (lower panels). 4 Hp model {\sf rg.3D.4hp} (left) and 7 Hp model {\sf rg.3D.mrez} (right).}
\end{figure}
 
\newpage

\subsubsection{Mean azimuthal and polar momentum equation}

\begin{figure}[!h]
\centerline{
\includegraphics[width=7.3cm]{rg4hp_tavg800_tmomentum_equation_insf.eps}
\includegraphics[width=7.3cm]{rgmrez_tavg800_tmomentum_equation_insf.eps}}

\centerline{
\includegraphics[width=7.3cm]{rg4hp_tavg800_pmomentum_equation_insf.eps}
\includegraphics[width=7.3cm]{rgmrez_tavg800_pmomentum_equation_insf.eps}}
\caption{Mean azimuthal equation (upper panels) and mean polar momentum equation (lower panels). 4 Hp model {\sf rg.3D.4hp} (left) and 7 Hp model {\sf rg.3D.mrez} (right).}
\end{figure}

\newpage

\subsubsection{Mean total energy equation and mean turbulent kinetic energy equation}

\begin{figure}[!h]
\centerline{
\includegraphics[width=7.3cm]{rg4hp_tavg800_total_energy_equation_insf.eps}
\includegraphics[width=7.3cm]{rgmrez_tavg800_total_energy_equation_insf.eps}}

\centerline{
\includegraphics[width=7.3cm]{rg4hp_tavg800_mfields_k_equation_insf.eps}
\includegraphics[width=7.3cm]{rgmrez_tavg800_mfields_k_equation_insf.eps}}
\caption{Mean total energy equation (upper panels) and mean turbulent kinetic energy equation (lower panels). 4 Hp model {\sf rg.3D.4hp} (left) and 7 Hp model {\sf rg.3D.mrez} (right).}
\end{figure}

\newpage

\subsubsection{Mean turbulent kinetic energy equation (radial + horizontal part)}

\begin{figure}[!h]
\centerline{
\includegraphics[width=7.3cm]{rg4hp_tavg800_mfields_k_equation_rad_insf.eps}
\includegraphics[width=7.3cm]{rgmrez_tavg800_mfields_k_equation_rad_insf.eps}}

\centerline{
\includegraphics[width=7.3cm]{rg4hp_tavg800_mfields_k_equation_hor_insf.eps}
\includegraphics[width=7.3cm]{rgmrez_tavg800_mfields_k_equation_hor_insf.eps}}
\caption{Radial (upper panels) and horizontal (lower panels) part of the mean turbulent kinetic energy equation. 4 Hp model {\sf rg.3D.4hp} (left) and 7 Hp model {\sf rg.3D.mrez} (right).}
\end{figure}

\newpage

\subsubsection{Mean turbulent mass flux and mean density-specific volume covariance equation}

\begin{figure}[!h]
\centerline{
\includegraphics[width=7.3cm]{rg4hp_tavg800_mfields_a_equation_insf.eps}
\includegraphics[width=7.3cm]{rgmrez_tavg800_mfields_a_equation_insf.eps}}

\centerline{
\includegraphics[width=7.3cm]{rg4hp_tavg800_mfields_b_equation_insf.eps}
\includegraphics[width=7.3cm]{rgmrez_tavg800_mfields_b_equation_insf.eps}}
\caption{Mean turbulent mass flux equation (upper panels) and density-specific volume covariance equation (lower panels). 4 Hp model {\sf rg.3D.4hp} (left) and 7 Hp model {\sf rg.3D.mrez} (right).}
\end{figure}

\newpage

\subsubsection{Mean specific angular momentum equation and internal energy flux equation}

\begin{figure}[!h]
\centerline{
\includegraphics[width=7.3cm]{rg4hp_tavg800_jz_equation_insf.eps}
\includegraphics[width=7.3cm]{rgmrez_tavg800_jz_equation_insf.eps}}

\centerline{
\includegraphics[width=7.3cm]{rg4hp_tavg800_mfields_i_equation_insf.eps}
\includegraphics[width=7.3cm]{rgmrez_tavg800_mfields_i_equation_insf.eps}}
\caption{Mean specific angular momentum equation (upper panels) and mean turbulent internal energy flux equation (lower panels). 4 Hp model {\sf rg.3D.4hp} (left) and 7 Hp model {\sf rg.3D.mrez} (right).}
\end{figure}

\newpage

\subsubsection{Mean entropy equation and mean entropy flux equation}

\begin{figure}[!h]
\centerline{
\includegraphics[width=7.3cm]{rg4hp_tavg800_entropy_equation_insf.eps}
\includegraphics[width=7.3cm]{rgmrez_tavg800_entropy_equation_insf.eps}}

\centerline{
\includegraphics[width=7.3cm]{rg4hp_tavg800_mfields_s_equation_insf.eps}
\includegraphics[width=7.3cm]{rgmrez_tavg800_mfields_s_equation_insf.eps}}
\caption{Mean entropy equation (upper panels) and mean entropy flux equation (lower panels). 4 Hp model {\sf rg.3D.4hp} (left) and 7 Hp model {\sf rg.3D.mrez} (right).}
\end{figure}

\newpage

\subsubsection{Mean turbulent kinetic energy and mean velocities}

\begin{figure}[!h]
\centerline{
\includegraphics[width=7.3cm]{rg4hp_tavg800_mean_k_insf.eps}
\includegraphics[width=7.3cm]{rgmrez_tavg800_mean_k_insf.eps}}

\centerline{
\includegraphics[width=7.3cm]{rg4hp_tavg800_mean_velocities_insf.eps}
\includegraphics[width=7.3cm]{rgmrez_tavg800_mean_velocities2_insf.eps}}
\caption{Mean turbulent kinetic energy (upper panels) and mean velocities (lower panels). 4 Hp model {\sf rg.3D.4hp} (left) and 7 Hp model {\sf rg.3D.mrez} (right).}
\end{figure}

\newpage

\section{Position of convection driving source}

\subsection{Oxygen burning shell models}

\subsubsection{Mean continuity equation and mean radial momentum equation}

\begin{figure}[!h]
\centerline{
\includegraphics[width=7.3cm]{ob1hpvh_tavg300_continuity_equation_insf.eps}
\includegraphics[width=7.3cm]{ob1hpvc_tavg300_continuity_equation_insf.eps}}

\centerline{
\includegraphics[width=7.3cm]{ob1hpvh_tavg300_rmomentum_equation_insf.eps}
\includegraphics[width=7.3cm]{ob1hpvc_tavg300_rmomentum_equation_insf.eps}}
\caption{Mean continuity equation (upper panels) and radial momentum equation (lower panels). Model with volumetric heating at the bottom of convection zone {\sf ob.3D.1hp.vh} (left) and model with volumetric cooling at the top of convection zone {\sf ob.3D.1hp.vc} (right).}
\end{figure}

\newpage

\subsubsection{Mean azimuthal and polar momentum equation}

\begin{figure}[!h]
\centerline{
\includegraphics[width=7.3cm]{ob1hpvh_tavg300_tmomentum_equation_insf.eps}
\includegraphics[width=7.3cm]{ob1hpvc_tavg300_tmomentum_equation_insf.eps}}

\centerline{
\includegraphics[width=7.3cm]{ob1hpvh_tavg300_pmomentum_equation_insf.eps}
\includegraphics[width=7.3cm]{ob1hpvc_tavg300_pmomentum_equation_insf.eps}}
\caption{Mean azimuthal equation (upper panels) and mean polar momentum equation (lower panels). Model with volumetric heating at the bottom of convection zone {\sf ob.3D.1hp.vh} (left) and model with volumetric cooling at the top of convection zone {\sf ob.3D.1hp.vc} (right).}
\end{figure}


\newpage

\subsubsection{Mean total energy equation and mean turbulent kinetic energy equation}

\begin{figure}[!h]
\centerline{
\includegraphics[width=7.3cm]{ob1hpvh_tavg300_total_energy_equation_insf.eps}
\includegraphics[width=7.3cm]{ob1hpvc_tavg300_total_energy_equation_insf.eps}}

\centerline{
\includegraphics[width=7.3cm]{ob1hpvh_tavg300_mfields_k_equation_insf.eps}
\includegraphics[width=7.3cm]{ob1hpvc_tavg300_mfields_k_equation_insf.eps}}
\caption{Mean total energy equation (upper panels) and mean turbulent kinetic energy equation (lower panels). Model with volumetric heating at the bottom of convection zone {\sf ob.3D.1hp.vh} (left) and model with volumetric cooling at the top of convection zone {\sf ob.3D.1hp.vc} (right).}
\end{figure}

\newpage

%\newpage

%\subsubsection{Mean pressure and enthalpy equation}

%\begin{figure}[!h]
%\centerline{
%\includegraphics[width=7.3cm]{ob1hpvh_tavg300_pressure_equation_insf.eps}
%\includegraphics[width=7.3cm]{ob1hpvc_tavg300_pressure_equation_insf.eps}}

%\centerline{
%\includegraphics[width=7.3cm]{ob1hpvh_tavg300_enthalpy_equation_insf.eps}
%\includegraphics[width=7.3cm]{ob1hpvc_tavg300_enthalpy_equation_insf.eps}}
%\caption{Model with volumetric heating at the bottom of convection zone {\sf ob.3D.1hp.vh} (left) and model with volumetric cooling at the top of convection zone {\sf ob.3D.1hp.vc} (right).}
%\end{figure}


\newpage

\subsubsection{Mean turbulent kinetic energy equations (radial + horizontal part)}

\begin{figure}[!h]
\centerline{
\includegraphics[width=7.3cm]{ob1hpvh_tavg300_mfields_k_equation_rad_insf.eps}
\includegraphics[width=7.3cm]{ob1hpvc_tavg300_mfields_k_equation_rad_insf.eps}}

\centerline{
\includegraphics[width=7.3cm]{ob1hpvh_tavg300_mfields_k_equation_hor_insf.eps}
\includegraphics[width=7.3cm]{ob1hpvc_tavg300_mfields_k_equation_hor_insf.eps}}
\caption{Radial (upper panels) and horizontal (lower panels) part of the mean turbulent kinetic energy equation. Model with volumetric heating at the bottom of convection zone {\sf ob.3D.1hp.vh} (left) and model with volumetric cooling at the top of convection zone {\sf ob.3D.1hp.vc} (right).}
\end{figure}

\newpage

\subsubsection{Mean turbulent mass flux and mean density-specific volume covariance equations}

\begin{figure}[!h]
\centerline{
\includegraphics[width=7.3cm]{ob1hpvh_tavg300_mfields_a_equation_insf.eps}
\includegraphics[width=7.3cm]{ob1hpvc_tavg300_mfields_a_equation_insf.eps}}

\centerline{
\includegraphics[width=7.3cm]{ob1hpvh_tavg300_mfields_b_equation_insf.eps}
\includegraphics[width=7.3cm]{ob1hpvc_tavg300_mfields_b_equation_insf.eps}}
\caption{Mean turbulent mass flux equation (upper panels) and density-specific volume covariance equation (lower panels). Model with volumetric heating at the bottom of convection zone {\sf ob.3D.1hp.vh} (left) and model with volumetric cooling at the top of convection zone {\sf ob.3D.1hp.vc} (right).}
\end{figure}

\newpage

\subsubsection{Mean specific angular momentum equation and mean internal energy flux equation}

\begin{figure}[!h]
\centerline{
\includegraphics[width=7.3cm]{ob1hpvh_tavg300_jz_equation_insf.eps}
\includegraphics[width=7.3cm]{ob1hpvc_tavg300_jz_equation_insf.eps}}

\centerline{
\includegraphics[width=7.3cm]{ob1hpvh_tavg300_mfields_i_equation_insf.eps}
\includegraphics[width=7.3cm]{ob1hpvc_tavg300_mfields_i_equation_insf.eps}}
\caption{Mean specific angular momentum equation (upper panels) and mean turbulent internal energy flux equation (lower panels). Model with volumetric heating at the bottom of convection zone {\sf ob.3D.1hp.vh} (left) and model with volumetric cooling at the top of convection zone {\sf ob.3D.1hp.vc} (right).}
\end{figure}

\newpage

\subsubsection{Mean turbulent kinetic energy and mean velocities}

\begin{figure}[!h]
\centerline{
\includegraphics[width=7.3cm]{ob1hpvh_tavg300_mean_k_insf.eps}
\includegraphics[width=7.3cm]{ob1hpvc_tavg300_mean_k_insf.eps}}

\centerline{
\includegraphics[width=7.3cm]{ob1hpvh_tavg300_mean_velocities_insf.eps}
\includegraphics[width=7.3cm]{ob1hpvc_tavg300_mean_velocities_insf.eps}}
\caption{Mean turbulent kinetic energy (upper panels) and mean background velocities (lower panels). Model with volumetric heating at the bottom of convection zone {\sf ob.3D.1hp.vh} (left) and model with volumetric cooling at the top of convection zone {\sf ob.3D.1hp.vc} (right).}
\end{figure}


\newpage

\section{Analysis of some turbulence one-point closure models}

\subsection{Downgradient approximation theory background}

\begin{align}
{\color{red}\fht{F}_i^q \sim -\Gamma_t \ \frac{\partial \fht{q}}{\partial x_i}} \ \ \ \mbox{($\Gamma_t$ is turbulence diffusivity and $\fht{F}^q_i = \eht{\rho q''u''_i}$ is a flux of $q$)} \nonumber
\end{align}

\begin{itemize}
\item can be derived from a transport equation of a diffusive passive scalar ({\color{blue} Harlow $\&$ Hirt, 1969; Daly $\&$ Harlow, 1970}):

\begin{align}
\partial_t \fht{F}_i^q  - \eht{u''_i q''\partial_t \rho} {\color{red}- \fht{R}_{in}\partial_n \fht{q}} + \fht{u}_n \eht{\rho \partial_n u''_i q''} + \fht{F}_n^q \partial_n \fht{u}_i + \partial_n \eht{\rho u''_n u''_i q''} - \eht{u''_iq''\partial_n \rho u''_n} = -\eht{q''}\partial_i \eht{P} - \eht{q''\partial_i P'} + \partial_n (\eht{\lambda \rho u''_i \partial_n q'')} {\color{red}+ f \fht{F}_i^q} \nonumber 
\end{align}

where q is the passive scalar governed by a diffusion equation $D_t q = \lambda \nabla^2 q$. \\

{\bf It implies, that the downgradient approximation holds only for:}

\begin{itemize} 
\item a transport of a diffusive passive scalar
\item a flow in steady state ($\partial_t \fht{F}_i^q = 0$)
\item an incompressible flow ($\partial_t \rho = 0$)
\item a flow with no background velocities ($\fht{u}_i = 0$)
\item a flow with no pressure-scalar correlations ($\eht{q''}\partial_i \eht{P} = \eht{q''\partial_i P'} = 0$)
\item a homogeneous flow ($\partial_n \eht{\rho u''_n u''_i q''} = 0$) 
\item an isotropic flow (decay-rate assumption: $\overline{\partial_n q'' \partial_n \rho u''_i} \sim {\color{red} f \fht{F}_i^q}$) 
\end{itemize}

%\begin{figure}
%\includegraphics[width=10.cm]{ob_fht_fekx_downgrad_comb1.eps}
%\caption{Downgradient approximations to the turbulent kinetic energy flux $\fht{F}_r^k = \eht{\rho u''_r k''}$ derived from 3D oxygen burning shell model.}
%\end{figure}

%\begin{textblock}{9}(11.,3.2)
%\begin{figure}
%\includegraphics[width=8.cm]{ob_fht_feix_downgrad_comb.eps}
%\caption{Red giant envelope}
%\end{figure}
%\end{textblock}

{\bf But, stellar turbulent convection is:}
\begin{itemize}
\item stratified (not homogeneous)
\item anisotropic
\item compressible on expanding/contracting background
\end{itemize}

\begin{itemize}
\item {\bf {\LARGE downgradient approximation is not suitable for modelling stellar processes}}
\end{itemize}

\end{itemize}


\newpage

\subsection{Downgradient approximations}

\begin{align}
f_k = (C \ \rho \ \sqrt{\fht{k}} \ l_d) \ \partial_r \fht{k} \ \ \ \ \ \ \ \ \ \ \ f_I = (C \ \rho \ \sqrt{\fht{k}} \ l_d) \ \partial_r \fht{\epsilon}_I \ \ \ \ \ \ \ \ \ \ \ f_I = (C \ \rho \ \fht{k}^2 / \varepsilon_k) \ \partial_r \fht{\epsilon}_I
\end{align}

\begin{figure}[!h]
\centerline{
\includegraphics[width=5.2cm]{obmrez_tavg230_mean_k_insf.eps}
\includegraphics[width=5.2cm]{obmrez_tavg230_mean_ei_insf.eps}
\includegraphics[width=5.2cm]{rgmrez_tavg800_mean_k_insf.eps}
\includegraphics[width=5.2cm]{rgmrez_tavg800_mean_ei_insf.eps}}

\centerline{
\includegraphics[width=5.2cm]{obmrez_tavg230_fht_fekx_downgrad.eps}
\includegraphics[width=5.2cm]{obmrez_tavg230_fht_feix_downgrad_comb.eps}
\includegraphics[width=5.2cm]{rgmrez_tavg800_fht_fekx_downgrad.eps}
\includegraphics[width=5.2cm]{rgmrez_tavg800_fht_feix_downgrad_comb.eps}}
\caption{Profiles of mean turbulent kinetic energy and mean internal energy (upper panels) and various downgradient approximations to turbulent kinetic energy flux and internal energy flux (lower panels) derived from oxygen burning shell model {\sf ob.3D.mr} (ob) and red giant envelope convection {\sf rg.3D.mr} (rg). $l_d$ is dissipation length scale and $C$ is model constant. \label{fig:downgrad_approx}}
\end{figure}

\newpage

\subsection{Various approximations taken from Besnard-Harlow-Rauenzahn (BHR) model}

\begin{align}
\eht{\rho'u'_ru'_r} = C_{1a} \frac{l_d}{\sqrt{\fht{k}}} \fht{R}_{rr} \partial_r \overline{u''_r} \ \ \ \ \ \ \ \ \ \ \eht{v' \partial_r P'} = -C_{2a} \frac{\sqrt{k}}{l_d} \eht{u''_r}  \ \ \ \ \ \ \ \ \ \ \eht{v'u'_r} = -C_{1b} \frac{l_d}{\sqrt{k}} \frac{\fht{R}_{rr}}{\eht{\rho}}\frac{\partial}{\partial r} \left( \frac{1+b}{\eht{\rho}} \right) \ \ \ \ \ \ \ \ \ \ \eht{v'd'} =  -C_{2b} \frac{\sqrt{k}}{l_d} \frac{b}{\eht{\rho}} \nonumber
\end{align}

\begin{figure}[!h]
\centerline{
\includegraphics[width=5.3cm]{obmrez_tavg230_mfields_bhr1.eps}
\includegraphics[width=5.3cm]{obmrez_tavg230_mfields_bhr2.eps}
\includegraphics[width=5.3cm]{obmrez_tavg230_mfields_bhr3.eps}
\includegraphics[width=5.3cm]{obmrez_tavg230_mfields_bhr4.eps}}

\centerline{
\includegraphics[width=5.3cm]{rgmrez_tavg800_mfields_bhr1.eps}
\includegraphics[width=5.3cm]{rgmrez_tavg800_mfields_bhr2.eps}
\includegraphics[width=5.3cm]{rgmrez_tavg800_mfields_bhr3.eps}
\includegraphics[width=5.3cm]{rgmrez_tavg800_mfields_bhr4.eps}}
\caption{Various approximations taken from Besnard-Harlow-Rauenzahn (BHR) model derived from  oxygen burning shell model {\sf ob.3D.mr} (ob) and red giant envelope convection {\sf rg.3D.mr} (rg). $l_d$ is dissipation length scale and $C$ is model constant. \label{fig:bhr-models}}
\end{figure}

\newpage

\subsection{Quasi-normal approximation and decay-rate assumption model}

\begin{align}
\eht{a'b'c'd'} = & \ \overline{a'b'} \ \overline{c'd'} + \overline{a'c'} \ \overline{b'd'} + \overline{a'd'} \ \overline{b'c'} \hspace{2.cm} & \overline{a'b'} = & \ (l_d^2 / \Delta) \overline{\partial_k a' \partial_k b'} & \ \ \ \mbox{(original formulations)} \\
\fht{a''b''c''d''} = & \ \fht{a''b''} \ \fht{c''d''} + \fht{a''c''} \ \fht{b''d''} + \fht{a''d''} \ \fht{b''c''} \hspace{2.cm} & \fht{a''b''} = & \ (l_d^2 / \Delta) \overline{\partial_k a'' \partial_k b''} & \ \ \ \mbox{(assumed Favre eqvivalents)}
\end{align}

\begin{figure}[!h]
\centerline{
\includegraphics[width=5.9cm]{obmrez_tavg230_fht_fuiuiuxux_qnapprox.eps}
\includegraphics[width=5.9cm]{obmrez_tavg230_eht_feix_decayrate_ass.eps}
\includegraphics[width=5.9cm]{obmrez_tavg230_eht_rxx_decayrate_ass.eps}}

\centerline{
\includegraphics[width=5.9cm]{rgmrez_tavg800_fht_fuiuiuxux_qnapprox.eps}
\includegraphics[width=5.9cm]{rgmrez_tavg800_eht_feix_decayrate_ass.eps}
\includegraphics[width=5.9cm]{rgmrez_tavg800_eht_rxx_decayrate_ass.eps}}
\caption{Quasi-normal approximations (left panels) and decay-rate assumption models (middle, right panels) derived from  oxygen burning shell model {\sf ob.3D.mr} (ob) and red giant envelope convection {\sf rg.3D.mr} (rg). \label{fig:qn-dc-models}}
\end{figure}

\newpage

\subsection{Integral models}

\begin{align}
f_I = \int (\epsilon_{\rm nuc} + \varepsilon_k/ \eht{\rho} - \eht{P} \ \eht{d} / \eht{\rho} - W_P / \eht{\rho} - T \dot{\fht{s}}) \ dr \hspace{2.cm} f_k = \int ( \ -(f_I (\Gamma_3 -1) - f_P))/((\Gamma_3-\Gamma_1-1)H_P) - \varepsilon_k \ - \nabla_r f_P + W_P \ ) \ dr \nonumber
\end{align}

\begin{figure}[!h]
\centerline{
\includegraphics[width=6.0cm]{obmrez_tavg230_fht_feix_integral.eps}
\includegraphics[width=6.0cm]{obmrez_tavg230_fht_fekx_integral.eps}}

\centerline{
\includegraphics[width=6.0cm]{rgmrez_tavg800_fht_feix_integral.eps}
\includegraphics[width=6.0cm]{rgmrez_tavg800_fht_fekx_integral.eps}}
\caption{Integral models for internal energy flux $f_I$ (left) and turbulent kinetic energy flux (right) derived from  oxygen burning shell model {\sf ob.3D.mr} (ob) and red giant envelope convection {\sf rg.3D.mr} (rg). \label{fig:integral-models}}
\end{figure}

\newpage

\section{Fourier scale analysis}

We perform scale analysis on fluctuation correlations containing two constituents by using Fourier analysis and taking advantage of Parceval's theorem. If $a''(r,\theta,\phi)\delta(r')$ and $b''(r,\theta,\phi)\delta(r')$ are real-valued functions (in our case fluctuations in a two dimensional plane at radius $r'$) and their Fourier transforms are $\widehat{a''}$ and $\widehat{b''}$ then ``energy'' contribution from large and small scales to horizontally averaged correlation $a''b''$ can be calculated as:

\begin{align}
E(k) =  \frac{1}{\Delta \Omega}\int_\Omega a'' b'' d\Omega = \int_k \widehat{a''}(k)\widehat{b''}^{\star}(k) dk = \nonumber \\
= \underbrace{\int_{k \le k_T} \widehat{a''}(k)\widehat{b''}^{\star}(k) dk}_\text{large scales} + \underbrace{\int_{k > k_T} \widehat{a''}(k)\widehat{b''}^{\star}(k) dk}_\text{small scales}
\label{eq:fourier_e}
\end{align}

where this integral is evaluated by summing the integrand over spherical shells of constant radius $k = \sqrt{(k_\theta^2 + k_\phi^2)}$ and unit wavenumber. $k_T$ is a threshold wave number which separates large and small scales. $\Omega$ is our two-dimensional cut domain.

The angular size of a structure in a horizontal plane can be at a given radius and wave number $k$ expressed as $l \sim N / k$, where $N$ is the angular size of simulated wedge. It means that in a 45\dgr wedge simulation will structures at wave number $k = 10$ have approximately size 4.5\dgr.

For a separation of contribution from large and small scales to a given correlation field, we choose a length scale equivalent to 1 $H_P$. This number is close to average vertical correlation length scales for $u'_r,\rho',P'$ in middle of convection zones calculated by \citet{VialletMeakin2013}. It is about $1.5 \times 10^8$ cm in the 2 $H_P$ oxygen burning models and $3 \times 10^{11}$ cm in 7 $H_P$ red giant models. Corresponding separation wave numbers $k_T$ are $4$ for the oxygen case and $11$ for the red giant case. These separation scales should be considered as a lower limit for a size of actual large scales, as we know that they are elongated in radial direction \citep{MeakinArnett2007}. Here, we Fourier transform our data in two dimensional horizontal planes and essentially looking only at horizontal scales. 

% 1 Hp in ob (2 Hp) 2.0e8 cm   (l_cvz/2) 
% 1 Hp in rg (7 Hp) 2.9e11 cm  (l_cvz/7)

% (ob) k_\theta, k_\phi \sim 3 (at r=6.e8  cm) but k_T = sqrt(k_theta^2 + k_\phi^2) \sim 4
% (rg) k_\theta, k_\phi \sim 8 (at r=3.e12 cm) but k_T = sqrt(k_theta^2 + k_\phi^2) \sim 11

% NOTE: if you go away from center of convection zone, all these numbers change. They are not very robust, but also not very sensitive to exact values.

We present our scale analysis in a form of the cumulative Fourier spectra, which are calculated according the formula:

\begin{align}
\eht{E}_c(k) = \sum_{i \le k} \eht{E(i)/E_t}
\end{align}

where $i$ is a wave number and $E(i)$ defined by Eq.~\ref{eq:fourier_e}. $E_t$ is total energy of a correlation field or in other words a horizontally averaged value of the correlation field. The over bar represent time averaging of instantaneous values. Such a spectrum shows how big energy contribution to a field comes from first $k$ Fourier modes.

\begin{figure}[!h]
\centerline{
\includegraphics[width=7.3cm]{oblrez_ft_sumcum_spectra.eps}
\includegraphics[width=7.3cm]{obmrez_ft_sumcum_spectra.eps}
\includegraphics[width=7.3cm]{obhrez_ft_sumcum_spectra.eps}} 

\centerline{
\includegraphics[width=8.3cm]{rglrez_ft_sumcum_spectra.eps}}
\caption{Cumulative Fourier spectra of relevant mean fields derived from oxygen burning shell models (upper panels) {\sf ob.3D.lr} (left), {\sf ob.3D.mr} (middle) and {\sf ob.3D.hr} (right) derived at radius where corresponding mean field has a maximum value. The same is done for red giant envelope convection model (lower panel) {\sf rg.3D.lr}. The shaded vertical lines separates cumulative contributions from large and small scales.}
\end{figure}

%\section{Variance equations (preliminary)}

%\subsection{Oxygen burning shell}

%\begin{align}
%\fht{D}_t \sigma_\rho =  &  - \nabla_r \eht{(\rho' \rho ' u''_r)}  - 2\eht{\rho} \ \eht{\rho'd''} - 2 \eht{\rho'u''_r} \partial_r \eht{\rho} - 2 \fht{d} \ \sigma_\rho - \eht{\rho'\rho'd''} + {\mathcal N_{\sigma_\rho}} \\
%\eht{\rho} \fht{D}_t \sigma_{\epsilon I} = &  -\nabla_r (\eht{\rho \epsilon''_I \epsilon''_I u''_r} ) - 2 f_I \partial_r \fht{\epsilon_I} - 2\overline{\epsilon''_I}\ \eht{P} \ \fht{d} - 2\eht{P} \ \eht{\epsilon''_I d''} - 2\fht{d} \ \eht{\epsilon''_I P'} - 2\overline{\epsilon''_I P' d''} + 2\eht{\epsilon''_I {\mathcal S}} + 2\eht{\epsilon''_I \tau_{ij} \partial_j u_i} + {\mathcal N_{\sigma_{\epsilon I}}}
%\end{align}

%\begin{figure}[!h]
%\centerline{
%\includegraphics[width=7.3cm]{ob3d_mrez_tavg230_sigma_dd_dist.eps}
%\includegraphics[width=7.3cm]{ob3d_mrez_tavg230_sigma_dd.eps}}

%\centerline{
%\includegraphics[width=7.3cm]{ob3d_mrez_tavg230_sigma_ei_dist.eps}
%\includegraphics[width=7.3cm]{ob3d_mrez_tavg230_sigma_ei.eps}}
%\caption{Variances ob.data.mrez \label{fig:ob-variances}}
%\end{figure}

%\newpage

%\begin{align}
%\eht{\rho} \fht{D}_t \sigma_{ur} = & -\nabla_r ( \eht{\rho u''_r u''_r u''_r} ) + 2 \nabla_r f_P + 2 W_b - 2\fht{R}_{rr}\partial_r \fht{u}_r + 2 \overline{P'\nabla_r u''_r} + {\mathcal G_{\sigma_{ur}}} + {\mathcal N_{\sigma_{ur}}} \\
%\fht{D}_t \sigma_P = & - \nabla_r \eht{(P' P' u''_r)} - 2\Gamma_1 \eht{P} \ W_P - 2 f_P \partial_r \eht{P} - 2\Gamma_1 \widetilde{d} \ \sigma_P - (2 \Gamma_1 -1) \eht{P'P'd''} + 2(\Gamma_3 - 1)(\eht{P' {\mathcal S}} + \eht{P'\tau_{ij}\partial_j u_i})
%\end{align}

%\begin{figure}[!h]
%\centerline{
%\includegraphics[width=7.3cm]{ob3d_mrez_tavg230_sigma_ur_dist.eps}
%\includegraphics[width=7.3cm]{ob3d_mrez_tavg230_sigma_ur.eps}}

%\centerline{
%\includegraphics[width=7.3cm]{ob3d_mrez_tavg230_sigma_pp_dist.eps}
%\includegraphics[width=7.3cm]{ob3d_mrez_tavg230_sigma_pp.eps}}
%\caption{Variances ob.data.mrez \label{fig:ob-variances}}
%\end{figure}

%\newpage

%\begin{align}
%\eht{\rho} \fht{D}_t \sigma_s = & -\nabla_r ( \eht{\rho s'' s'' u''_r} ) - 2 f_s \partial_r \fht{s} + 2 \eht{s'' {\mathcal S} / T} +  2 \eht{s'' \tau_{ij} \partial_j u_i / T} + {\mathcal N_{\sigma_{s}}}\\
%\eht{\rho} \fht{D}_t \sigma_\alpha = & -\nabla_r (\eht{\rho \alpha'' \alpha'' u''_r} ) - 2 f_\alpha \partial_r \fht{\alpha} + 2 \eht{\alpha'' \rho \dot{X}_\alpha^{\rm nuc}} + {\mathcal N_{\sigma_\alpha}}
%\end{align}

%\begin{figure}[!h]
%\centerline{
%\includegraphics[width=7.3cm]{ob3dB_tavg300_sigma_ss_dist.eps}
%\includegraphics[width=7.3cm]{ob3db_tavg300_sigma_ss.eps}}

%\centerline{
%\includegraphics[width=7.3cm]{ob3dB_tavg300_sigma_aa_dist.eps}
%\includegraphics[width=7.3cm]{ob3db_tavg300_sigma_aa.eps}}
%\caption{Variances ob.3d.B \label{fig:ob-variances}}
%\end{figure}

%\newpage

%\subsection{Red giant convection envelope}

%\begin{align}
%\fht{D}_t \sigma_\rho =  &  - \nabla_r \eht{(\rho' \rho ' u''_r)}  - 2\eht{\rho} \ \eht{\rho'd''} - 2 \eht{\rho'u''_r} \partial_r \eht{\rho} - 2 \fht{d} \ \sigma_\rho - \eht{\rho'\rho'd''} + {\mathcal N_{\sigma_\rho}} \\
%\eht{\rho} \fht{D}_t \sigma_{\epsilon I} = &  -\nabla_r (\eht{\rho \epsilon''_I \epsilon''_I u''_r} ) - 2 f_I \partial_r \fht{\epsilon_I} - 2\overline{\epsilon''_I}\ \eht{P} \ \fht{d} - 2\eht{P} \ \eht{\epsilon''_I d''} - 2\fht{d} \ \eht{\epsilon''_I P'} - 2\overline{\epsilon''_I P' d''} + 2\eht{\epsilon''_I {\mathcal S}} + 2\eht{\epsilon''_I \tau_{ij} \partial_j u_i} + {\mathcal N_{\sigma_{\epsilon I}}}
%\end{align}

%\begin{figure}[!h]
%\centerline{
%\includegraphics[width=7.3cm]{rglrez_tavg800_sigma_dd_dist.eps}
%\includegraphics[width=7.3cm]{rglrez_tavg800_sigma_dd.eps}}

%\centerline{
%\includegraphics[width=7.3cm]{rglrez_tavg800_sigma_ei_dist.eps}
%\includegraphics[width=7.3cm]{rglrez_tavg800_sigma_ei.eps}}
%\caption{Variances ob.data.mrez \label{fig:ob-variances}}
%\end{figure}

%\newpage

%\begin{align}
%\eht{\rho} \fht{D}_t \sigma_{ur} = & -\nabla_r ( \eht{\rho u''_r u''_r u''_r} ) + 2 \nabla_r f_P + 2 W_b - 2\fht{R}_{rr}\partial_r \fht{u}_r + 2 \overline{P'\nabla_r u''_r} + {\mathcal G_{\sigma_{ur}}} + {\mathcal N_{\sigma_{ur}}} \\
%\fht{D}_t \sigma_P = & - \nabla_r \eht{(P' P' u''_r)} - 2\Gamma_1 \eht{P} \ W_P - 2 f_P \partial_r \eht{P} - 2\Gamma_1 \widetilde{d} \ \sigma_P - (2 \Gamma_1 -1) \eht{P'P'd''} + 2(\Gamma_3 - 1)(\eht{P' {\mathcal S}} + \eht{P'\tau_{ij}\partial_j u_i})
%\end{align}

%\begin{figure}[!h]
%\centerline{
%\includegraphics[width=7.3cm]{rglrez_tavg800_sigma_ur_dist.eps}
%\includegraphics[width=7.3cm]{rglrez_tavg800_sigma_ur.eps}}

%\centerline{
%\includegraphics[width=7.3cm]{rglrez_tavg800_sigma_pp_dist.eps}
%\includegraphics[width=7.3cm]{rglrez_tavg800_sigma_pp.eps}}
%\caption{Variances ob.data.mrez \label{fig:ob-variances}}
%\end{figure}

%\newpage

%\begin{align}
%\eht{\rho} \fht{D}_t \sigma_s = & -\nabla_r ( \eht{\rho s'' s'' u''_r} ) - 2 f_s \partial_r \fht{s} + 2 \eht{s'' {\mathcal S} / T} +  2 \eht{s'' \tau_{ij} \partial_j u_i / T} + {\mathcal N_{\sigma_{s}}}\\
%\eht{\rho} \fht{D}_t \sigma_\alpha = & -\nabla_r (\eht{\rho \alpha'' \alpha'' u''_r} ) - 2 f_\alpha \partial_r \fht{\alpha} + 2 \eht{\alpha'' \rho \dot{X}_\alpha^{\rm nuc}} + {\mathcal N_{\sigma_\alpha}}
%\end{align}

%\begin{figure}[!h]
%\centerline{
%\includegraphics[width=7.3cm]{rglrez_tavg800_sigma_ss_dist.eps}
%\includegraphics[width=7.3cm]{rglrez_tavg800_sigma_ss.eps}}
%\caption{Variances ob.3d.B \label{fig:ob-variances}}
%\end{figure}

%\newpage

%\section{Third-order moments (preliminary)}

%\begin{figure}[!h]
%\centerline{
%\includegraphics[width=7.3cm]{obmrez_tavg230_mean_fk_insf.eps}
%\includegraphics[width=17.3cm]{obmrez_tavg230_mfields_fk_equation_insf.eps}}

%\centerline{
%\includegraphics[width=7.3cm]{rgmrez_tavg800_mean_fk_insf.eps}
%\includegraphics[width=17.3cm]{rgmrez_tavg800_mfields_fk_equation_insf.eps}}
%\caption{Turbulent kinetic energy flux equation \label{fig:ob-variances}}
%\end{figure}

\newpage


\section{Properties of our hydrogen injection flash and core helium flash data}

\begin{figure}[!h]
\centerline{
\includegraphics[width=7.3cm]{hif3d_tavg4000_initial_model_rho_t.eps}
\includegraphics[width=7.3cm]{hif3d_tavg4000_mfields_rms_tpr_insf.eps}
\includegraphics[width=7.3cm]{hif3d_tavg4000_mean_velocities_insf.eps}}

\centerline{
\includegraphics[width=7.3cm]{chf3d_tavg18000_initial_model_rho_t.eps}
\includegraphics[width=7.3cm]{chf3d_tavg18000_mfields_rms_tpr_insf.eps}
\includegraphics[width=7.3cm]{chf3d_tavg18000_mean_velocities_insf.eps}}
\caption{Properties of our data. Model {\sf hif.3D} (upper panels) and model {\sf chf.3D} (lower panels). \label{fig:data}}
\end{figure}


\newpage

\subsection{Snapshots of turbulent kinetic energy in a meridional plane}

\vspace{1.cm}

\begin{figure}[!h]
\centerline{
\includegraphics[width=11.8cm]{hif3d_tke.jpg}
\includegraphics[width=11.8cm]{chf3d_tke.jpg}}
\caption{Snapshots of turbulent kinetic energy (in erg g$^{-1}$) in a meridional plane.}
\label{fig:ob-rg-tke-cuts}
\end{figure}

\newpage

\subsection{Summary of the hydrogen injection flash and core helium flash simulations and their properties}

\vspace{1.cm}

% HIF
\begin{table}[!h]
\centerline{
\begin{tabular}{|l c c c|}
\hline
{\bf Parameter}  & {\sf hif.3D} & {\sf hif.3D} & {\sf chf.3D} \\
Grid zoning & 375$\times 45^2$ & 375$\times 45^2$ & 270$\times 30^2$ \\
$r_\mathrm{in}$, $r_\mathrm{out}$ ($10^{8}$ cm) & 2., 16. & 2., 16. & 2., 12. \\
$\Delta \theta$, $\Delta \phi$ & 45$^\circ$ & & 30$^\circ$ \\
& & & \\
{\bf Convection zone} & He-burning (CVZ-1) & CNO-burning (CVZ-2) & He-burning (CVZ)\\
$r^c_\mathrm{in}$, $r^c_\mathrm{out}$ ($10^{8}$ cm) & 4.7, 9.55 & 9.55, 11.3 & 4.7, 9.4\\
CZ stratification ($H_p$)         & 2.4   & 1.3 & 2.3 \\
$\Delta t_\mathrm{av}$ (s)          & $4000$ & $4000$ & $18000$ \\
$v_\mathrm{rms}$ ($10^5$ cm/s)      & 8.2 & 11.8 & 8.5 \\
$\tau_\mathrm{conv}$ (s)     & 1180  & 290 & 1100 \\
$P_{turb}/P_{gas} (10^{-5})$          & 2.6 & 3.7  & 3.3 \\ 
$L_\mathrm{heat}$ ($10^{43}$ erg/s)  & 1.2  & 1.8 & 1.2 \\
$L_\mathrm{d}$ ($10^{41}$ erg/s)     & 5.3  & 6.1 & 6. \\
$l_\mathrm{d}$ ($10^{8}$ cm)        & 5.6  & 2.9  & 4.8 \\
$\tau_\mathrm{d}$ (s)            & 337.4   & 123.6 & 281.9\\
$\tau_\mathrm{dr}$ (s)           & 719.7   & 175.8 & 659.7 \\
$\tau_\mathrm{dh}$ (s)           & 211.2   & 116.1 & 142.5 \\
\hline
\end{tabular}}
\caption{boundaries of computational domain $r_\mathrm{in}$, $r_\mathrm{out}$; boundaries of convection zone at bottom and top $r_\mathrm{b}^c$, $r_\mathrm{t}^c$; angular size of computational domain $\Delta \theta$, $\Delta \phi$ ; depth of convection zone ``CZ stratification'' in pressure scale height $H_P$; averaging timescale of mean fields analysis $\Delta t_\mathrm{av}$; global rms velocity $v_\mathrm{rms}$; convective turnover timescale $\tau_\mathrm{conv}$; average ratio of turbulent ram pressure and gas pressure $p_{turb}/p_{gas}$; total luminosity of the hydrodynamic model $L$; total rate of kinetic energy dissipation $L_d$; dissipation length-scale  $l_d$; turbulent kinetic energy dissipation time-scale $\tau_d$; radial turbulent kinetic energy dissipation time-scale $\tau_{dr}$; horizontal turbulent kinetic energy dissipation time-scale $\tau_{dh}$. The numerical values may vary in time up to 20$\%$ due to limited amount of data for averaging out the time dependence.}
\label{tab:rg-models} 
\end{table}

\newpage

\section{Profiles and integral budgets of mean fields equations}

\subsection{Mean continuity equation}

\begin{align}
\fav{D}_t \av{\rho} =& -\av{\rho} \fav{d} + {\mathcal N_\rho}  
\end{align}

\begin{figure}[!h]
\centerline{
\includegraphics[width=7.3cm]{hif3d_tavg4000_initial_model_rho_t.eps}
\includegraphics[width=7.3cm]{hif3d_tavg4000_continuity_equation_insf.eps}
\includegraphics[width=7.3cm]{hif3d_tavg4000_continuity_equation_insf_bar.eps}}

\centerline{
\includegraphics[width=7.3cm]{chf3d_tavg18000_mean_rho_insf.eps}
\includegraphics[width=7.3cm]{chf3d_tavg18000_continuity_equation_insf.eps}
\includegraphics[width=7.3cm]{chf3d_tavg18000_continuity_equation_insf_bar.eps}}
\caption{Mean continuity equation. Model {\sf hif.3D} (upper panels) and model {\sf chf.3D} (lower panels)}
\end{figure}

\newpage

\subsection{Mean radial momentum equation}

\begin{align}
\av{\rho}\fav{D}_t\fav{u}_r = & -\nabla_r \fav{R}_{rr} -\av{G^{M}_r} - \partial_r \av{P} + \av{\rho}\fav{g_r} + {\mathcal N_{ur}}
\end{align}

\begin{figure}[!h]
\centerline{
\includegraphics[width=7.3cm]{hif3d_tavg4000_mean_ur_insf.eps}
\includegraphics[width=7.3cm]{hif3d_tavg4000_rmomentum_equation_insf.eps}
\includegraphics[width=7.3cm]{hif3d_tavg4000_rmomentum_equation_insf_bar.eps}}

\centerline{
\includegraphics[width=7.3cm]{chf3d_tavg18000_mean_ur_insf.eps}
\includegraphics[width=7.3cm]{chf3d_tavg18000_rmomentum_equation_insf.eps}
\includegraphics[width=7.3cm]{chf3d_tavg18000_rmomentum_equation_insf_bar.eps}}
\caption{Mean radial momentum equation. Model {\sf hif.3D} (upper panels) and model {\sf chf.3D} (lower panels)}
\end{figure}

\newpage

\subsection{Mean azimuthal momentum equation}

\begin{align}
\av{\rho}\fav{D}_t\fav{u}_\theta = & -\nabla_r \fav{R}_{\theta r} -\av{G^{M}_\theta} - (1/r)\overline{\partial_\theta P} + {\mathcal N_{u \theta}} 
\end{align}

\begin{figure}[!h]
\centerline{
\includegraphics[width=7.3cm]{hif3d_tavg4000_mean_ut_insf.eps}
\includegraphics[width=7.3cm]{hif3d_tavg4000_tmomentum_equation_insf.eps}
\includegraphics[width=7.3cm]{hif3d_tavg4000_tmomentum_equation_insf_bar.eps}}

\centerline{
\includegraphics[width=7.3cm]{chf3d_tavg18000_mean_ut_insf.eps}
\includegraphics[width=7.3cm]{chf3d_tavg18000_tmomentum_equation_insf.eps}
\includegraphics[width=7.3cm]{chf3d_tavg18000_tmomentum_equation_insf_bar.eps}}

\caption{Mean azimuthal momentum equation. Model {\sf hif.3D} (upper panels) and model {\sf chf.3D} (lower panels)}
\end{figure}

\newpage

\subsection{Mean polar momentum equation}

\begin{align}
\av{\rho}\fav{D}_t\fav{u}_\phi = & -\nabla_r \fav{R}_{\phi r} -\av{G^{M}_\phi} + {\mathcal N_{u \phi}}
\end{align}

\begin{figure}[!h]
\centerline{
\includegraphics[width=7.3cm]{hif3d_tavg4000_mean_up_insf.eps}
\includegraphics[width=7.3cm]{hif3d_tavg4000_pmomentum_equation_insf.eps}
\includegraphics[width=7.3cm]{hif3d_tavg4000_pmomentum_equation_insf_bar.eps}}

\centerline{
\includegraphics[width=7.3cm]{chf3d_tavg18000_mean_up_insf.eps}
\includegraphics[width=7.3cm]{chf3d_tavg18000_pmomentum_equation_insf.eps}
\includegraphics[width=7.3cm]{chf3d_tavg18000_pmomentum_equation_insf_bar.eps}}

\caption{Mean polar momentum equation. Model {\sf hif.3D} (upper panels) and model {\sf chf.3D} (lower panels)}
\end{figure}

\newpage
 
\subsection{Mean internal energy equation}

\begin{align}
\av{\rho} \fav{D}_t \fav{\epsilon}_I = & - \nabla_r  ( f_I + f_T ) - \av{P} \ \av{d} - W_P  + {\mathcal S} + {\mathcal N_{\epsilon I}} 
\end{align}

\begin{figure}[!h]
\centerline{
\includegraphics[width=7.3cm]{hif3d_tavg4000_mean_ei_insf.eps}
\includegraphics[width=7.3cm]{hif3d_tavg4000_internal_energy_equation_insf.eps}
\includegraphics[width=7.3cm]{hif3d_tavg4000_internal_energy_equation_insf_bar.eps}}

\centerline{
\includegraphics[width=7.3cm]{chf3d_tavg18000_mean_ei_insf.eps}
\includegraphics[width=7.3cm]{chf3d_tavg18000_internal_energy_equation_insf.eps}
\includegraphics[width=7.3cm]{chf3d_tavg18000_internal_energy_equation_insf_bar.eps}}

\caption{Mean internal energy equation. Model {\sf hif.3D} (upper panels) and model {\sf chf.3D} (lower panels)}
\end{figure}

\newpage
 
\subsection{Mean kinetic energy equation}

\begin{align}
\av{\rho} \fav{D}_t \fav{\epsilon}_k = &  -\nabla_r  ( f_k +  f_P ) - \fht{R}_{ir}\partial_r \fht{u}_i + W_b + W_P +\av{\rho}\fav{D}_t (\fav{u}_i \fav{u}_i / 2) + {\mathcal N_{\epsilon k}} \label{eq:rans_mke} 
\end{align}

\begin{figure}[!h]
\centerline{
\includegraphics[width=7.3cm]{hif3d_tavg4000_mean_ek_insf.eps}
\includegraphics[width=7.3cm]{hif3d_tavg4000_kinetic_energy_equation_insf.eps}
\includegraphics[width=7.3cm]{hif3d_tavg4000_kinetic_energy_equation_insf_bar.eps}}

\centerline{
\includegraphics[width=7.3cm]{chf3d_tavg18000_mean_ek_insf.eps}
\includegraphics[width=7.3cm]{chf3d_tavg18000_kinetic_energy_equation_insf.eps}
\includegraphics[width=7.3cm]{chf3d_tavg18000_kinetic_energy_equation_insf_bar.eps}}

\caption{Mean kinetic energy equation. Model {\sf hif.3D} (upper panels) and model {\sf chf.3D} (lower panels)}
\end{figure}

\newpage

\subsection{Mean total energy equation}

\begin{align}
\av{\rho} \fav{D}_t \fav{\epsilon}_t = &  - \nabla_r ( f_I + f_T + f_k + f_P ) - \fht{R}_{ir}\partial_r \fht{u}_i - \av{P} \ \av{d} + W_b + {\mathcal S} + \av{\rho}\fav{D}_t (\fav{u}_i \fav{u}_i / 2) + {\mathcal N_{\epsilon t}} 
\end{align}

\begin{figure}[!h]
\centerline{
\includegraphics[width=7.3cm]{hif3d_tavg4000_mean_et_insf.eps}
\includegraphics[width=7.3cm]{hif3d_tavg4000_total_energy_equation_insf.eps}
\includegraphics[width=7.3cm]{hif3d_tavg4000_total_energy_equation_insf_bar.eps}}

\centerline{
\includegraphics[width=7.3cm]{chf3d_tavg18000_mean_et_insf.eps}
\includegraphics[width=7.3cm]{chf3d_tavg18000_total_energy_equation_insf.eps}
\includegraphics[width=7.3cm]{chf3d_tavg18000_total_energy_equation_insf_bar.eps}}

\caption{Mean total energy equation. Model {\sf hif.3D} (upper panels) and model {\sf chf.3D} (lower panels)}
\end{figure}


\newpage

\subsection{Mean entropy equation}

\begin{align}
\av{\rho} \fav{D}_t \fav{s} = & - \nabla_r  f_s    - \av{(\nabla \cdot F_T)/T}+ \av{{\mathcal S}/T} + {\mathcal N_s}   
\end{align}

\begin{figure}[!h]
\centerline{
\includegraphics[width=7.3cm]{hif3d_tavg4000_mean_entropy_insf.eps}
\includegraphics[width=7.3cm]{hif3d_tavg4000_entropy_equation_insf.eps}
\includegraphics[width=7.3cm]{hif3d_tavg4000_entropy_equation_insf_bar.eps}}

\centerline{
\includegraphics[width=7.3cm]{chf3d_tavg18000_mean_entropy_insf.eps}
\includegraphics[width=7.3cm]{chf3d_tavg18000_entropy_equation_insf.eps}
\includegraphics[width=7.3cm]{chf3d_tavg18000_entropy_equation_insf_bar.eps}}

\caption{Mean entropy equation. Model {\sf hif.3D} (upper panels) and model {\sf chf.3D} (lower panels)}
\end{figure}

\newpage

\subsection{Mean pressure equation }

\begin{align}
\av{D}_t \av{P} = & -\nabla_r f_P - \Gamma_1 \eht{P} \ \eht{d} + (1 -\Gamma_1) W_P + (\Gamma_3 -1){\mathcal S} + (\Gamma_3 - 1)\nabla_r f_T + {\mathcal N_P} 
\end{align}

\begin{figure}[!h]
\centerline{
\includegraphics[width=7.3cm]{hif3d_tavg4000_mean_pressure_insf.eps}
\includegraphics[width=7.3cm]{hif3d_tavg4000_pressure_equation_insf.eps}
\includegraphics[width=7.3cm]{hif3d_tavg4000_pressure_equation_insf_bar.eps}}

\centerline{
\includegraphics[width=7.3cm]{chf3d_tavg18000_mean_pressure_insf.eps}
\includegraphics[width=7.3cm]{chf3d_tavg18000_pressure_equation_insf.eps}
\includegraphics[width=7.3cm]{chf3d_tavg18000_pressure_equation_insf_bar.eps}}
\caption{Mean pressure equation. Model {\sf hif.3D} (upper panels) and model {\sf chf.3D} (lower panels)}
\end{figure}

\newpage

\subsection{Mean enthalpy equation}

\begin{align}
\erho\fav{D}_t \fav{h} = & -\nabla_r f_h - \Gamma_1\eht{P} \ \eht{d} - \Gamma_1 W_P + \Gamma_3 {\mathcal S} + \Gamma_3 \nabla_r f_T +  {\mathcal N_h} \label{eq:rans_h}
\end{align}

\begin{figure}[!h]
\centerline{
\includegraphics[width=7.3cm]{hif3d_tavg4000_mean_enthalpy_insf.eps}
\includegraphics[width=7.3cm]{hif3d_tavg4000_enthalpy_equation_insf.eps}
\includegraphics[width=7.3cm]{hif3d_tavg4000_enthalpy_equation_insf_bar.eps}}

\centerline{
\includegraphics[width=7.3cm]{chf3d_tavg18000_mean_enthalpy_insf.eps}
\includegraphics[width=7.3cm]{chf3d_tavg18000_enthalpy_equation_insf.eps}
\includegraphics[width=7.3cm]{chf3d_tavg18000_enthalpy_equation_insf_bar.eps}}

\caption{Mean enthalpy equation. Model {\sf hif.3D} (upper panels) and model {\sf chf.3D} (lower panels)}
\end{figure}


\newpage

\subsection{Mean angular momentum equation (z-component)}

\begin{align}
\erho\fav{D}_t \fav{j}_z = & -\nabla_r f_{jz} + {\mathcal N_{jz}} 
\end{align}

\begin{figure}[!h]
\centerline{
\includegraphics[width=7.3cm]{hif3d_tavg4000_mean_jz_insf.eps}
\includegraphics[width=7.3cm]{hif3d_tavg4000_jz_equation_insf.eps}
\includegraphics[width=7.3cm]{hif3d_tavg4000_jz_equation_insf_bar.eps}}

\centerline{
\includegraphics[width=7.3cm]{chf3d_tavg18000_mean_jz_insf.eps}
\includegraphics[width=7.3cm]{chf3d_tavg18000_jz_equation_insf.eps}
\includegraphics[width=7.3cm]{chf3d_tavg18000_jz_equation_insf_bar.eps}}

\caption{Mean angular momentum equation (z-component). Model {\sf hif.3D} (upper panels) and model {\sf chf.3D} (lower panels)}
\end{figure}


\newpage

\subsection{Mean composition equations}

\begin{align}
\erho\fav{D}_t \fav{X}_\alpha = & -\nabla_r f_\alpha + \av{\rho}\fav{\dot{X}}_\alpha^{\rm nuc} + {\mathcal N_\alpha} 
\end{align}

\begin{figure}[!h]
\centerline{
\includegraphics[width=7.3cm]{chf3d_tavg12000_mean_xhe4_insf.eps}
\includegraphics[width=7.3cm]{chf3d_tavg12000_xhe4_equation_insf.eps}
\includegraphics[width=7.3cm]{chf3d_tavg12000_xhe4_equation_insf_bar.eps}}

\centerline{
\includegraphics[width=7.3cm]{hif3d_tavg4000_mean_xprot_insf.eps}
\includegraphics[width=7.3cm]{hif3d_tavg4000_xprot_equation_insf.eps}
\includegraphics[width=7.3cm]{hif3d_tavg4000_xprot_equation_insf_bar.eps}}
\caption{Mean composition equations. Model {\sf hif.3D} (upper panels) and model {\sf chf.3D} (lower panels)}
\end{figure}

\newpage

\subsection{Mean turbulent kinetic energy equation}

\begin{align}
\av{\rho} \fav{D}_t \fav{k}^{ } = & -\nabla_r ( f_k +  f_P ) - \fht{R}_{ir}\partial_r \fht{u}_i + W_b + W_P + {\mathcal N_k}   
\end{align}

\begin{figure}[!h]
\centerline{
\includegraphics[width=7.3cm]{hif3d_tavg4000_mean_k_insf.eps}
\includegraphics[width=7.3cm]{hif3d_tavg4000_mfields_k_equation_insf.eps}
\includegraphics[width=7.3cm]{hif3d_tavg4000_mfields_k_equation_insf_bar.eps}}

\centerline{
\includegraphics[width=7.3cm]{chf3d_tavg18000_mean_k_insf.eps}
\includegraphics[width=7.3cm]{chf3d_tavg18000_mfields_k_equation_insf.eps}
\includegraphics[width=7.3cm]{chf3d_tavg18000_mfields_k_equation_insf_bar.eps}}

\caption{Mean turbulent kinetic energy equation. Model {\sf hif.3D} (upper panels) and model {\sf chf.3D} (lower panels)}
\end{figure}

\newpage

\subsection{Radial part of mean turbulent kinetic energy equation}

\begin{align}
\av{\rho} \fav{D}_t \fav{k}^r =  &  -\nabla_r  ( f_k^r + f_P )  - \fht{R}_{rr}\partial_r \fht{u}_r + W_b  + \eht{P'\nabla_r u''_r} + {\mathcal G_k^r} + {\mathcal N_{kr}} 
\end{align}

\begin{figure}[!h]

\centerline{
\includegraphics[width=7.3cm]{hif3d_tavg4000_mean_kr_insf.eps}
\includegraphics[width=7.3cm]{hif3d_tavg4000_mfields_k_equation_rad_insf.eps}
\includegraphics[width=7.3cm]{hif3d_tavg4000_mfields_k_equation_rad_insf_bar.eps}}

\centerline{
\includegraphics[width=7.3cm]{chf3d_tavg18000_mean_kr_insf.eps}
\includegraphics[width=7.3cm]{chf3d_tavg18000_mfields_k_equation_rad_insf.eps}
\includegraphics[width=7.3cm]{chf3d_tavg18000_mfields_k_equation_rad_insf_bar.eps}}

\caption{Radial turbulent kinetic energy equation. Model {\sf hif.3D} (upper panels) and model {\sf chf.3D} (lower panels)}
\end{figure}


\newpage

\subsection{Horizontal part of mean turbulent kinetic energy equation}

\begin{align}
\av{\rho} \fav{D}_t \fav{k}^h =  &  -\nabla_r f_k^h - (\fht{R}_{\theta r}\partial_r \fht{u}_\theta + \fht{R}_{\phi r}\partial_r \fht{u}_\phi) + (\eht{P' \nabla_\theta u''_\theta} + \eht{P' \nabla_\phi u''_\phi}) + {\mathcal G_k^h} + {\mathcal N_{kh}} 
\end{align}

\begin{figure}[!h]
\centerline{
\includegraphics[width=7.3cm]{hif3d_tavg4000_mean_kh_insf.eps}
\includegraphics[width=7.3cm]{hif3d_tavg4000_mfields_k_equation_hor_insf.eps}
\includegraphics[width=7.3cm]{hif3d_tavg4000_mfields_k_equation_hor_insf_bar.eps}}

\centerline{
\includegraphics[width=7.3cm]{chf3d_tavg18000_mean_kh_insf.eps}
\includegraphics[width=7.3cm]{chf3d_tavg18000_mfields_k_equation_hor_insf.eps}
\includegraphics[width=7.3cm]{chf3d_tavg18000_mfields_k_equation_hor_insf_bar.eps}}

\caption{Horizontal turbulent kinetic energy equation. Model {\sf hif.3D} (upper panels) and model {\sf chf.3D} (lower panels)}
\end{figure}

\newpage

\section{Compositon equations derived from hydrogen injection flash data}

\subsection{Mean H$^1$ and He$^3$ equation}

\begin{figure}[!h]
\centerline{
\includegraphics[width=6.7cm]{hif3d_tavg7000_mean_xprot_insf.eps}
\includegraphics[width=6.7cm]{hif3d_tavg7000_xprot_equation_insf.eps}
\includegraphics[width=6.7cm]{hif3d_tavg7000_xprot_equation_insf_bar.eps}}

\centerline{
\includegraphics[width=6.7cm]{hif3d_tavg7000_mean_xhe3_insf.eps}
\includegraphics[width=6.7cm]{hif3d_tavg7000_xhe3_equation_insf.eps}
\includegraphics[width=6.7cm]{hif3d_tavg7000_xhe3_equation_insf_bar.eps}}
\caption{Mean composition equations derived from {\sf hif.3D}. \label{fig:x-equations}}
\end{figure}

\newpage

\subsection{Mean He$^4$ and C$^{12}$ equation}

\begin{figure}[!h]
\centerline{
\includegraphics[width=6.7cm]{hif3d_tavg7000_mean_xhe4_insf.eps}
\includegraphics[width=6.7cm]{hif3d_tavg7000_xhe4_equation_insf.eps}
\includegraphics[width=6.7cm]{hif3d_tavg7000_xhe4_equation_insf_bar.eps}}

\centerline{
\includegraphics[width=6.7cm]{hif3d_tavg7000_mean_xc12_insf.eps}
\includegraphics[width=6.7cm]{hif3d_tavg7000_xc12_equation_insf.eps}
\includegraphics[width=6.7cm]{hif3d_tavg7000_xc12_equation_insf_bar.eps}}
\caption{Mean composition equations derived from {\sf hif.3D}. \label{fig:x-equations}}
\end{figure}

\newpage


\subsection{Mean C$^{13}$ and N$^{13}$ equation}

\begin{figure}[!h]
\centerline{
\includegraphics[width=6.7cm]{hif3d_tavg7000_mean_xc13_insf.eps}
\includegraphics[width=6.7cm]{hif3d_tavg7000_xc13_equation_insf.eps}
\includegraphics[width=6.7cm]{hif3d_tavg7000_xc13_equation_insf_bar.eps}}

\centerline{
\includegraphics[width=6.7cm]{hif3d_tavg7000_mean_xn13_insf.eps}
\includegraphics[width=6.7cm]{hif3d_tavg7000_xn13_equation_insf.eps}
\includegraphics[width=6.7cm]{hif3d_tavg7000_xn13_equation_insf_bar.eps}}
\caption{Mean composition equations derived from {\sf hif.3D}. \label{fig:x-equations}}
\end{figure}

\newpage

\subsection{Mean N$^{14}$ and N$^{15}$ equation}

\begin{figure}[!h]
\centerline{
\includegraphics[width=6.7cm]{hif3d_tavg7000_mean_xn14_insf.eps}
\includegraphics[width=6.7cm]{hif3d_tavg7000_xn14_equation_insf.eps}
\includegraphics[width=6.7cm]{hif3d_tavg7000_xn14_equation_insf_bar.eps}}

\centerline{
\includegraphics[width=6.7cm]{hif3d_tavg7000_mean_xn15_insf.eps}
\includegraphics[width=6.7cm]{hif3d_tavg7000_xn15_equation_insf.eps}
\includegraphics[width=6.7cm]{hif3d_tavg7000_xn15_equation_insf_bar.eps}}
\caption{Mean composition equations derived from {\sf hif.3D}. \label{fig:x-equations}}
\end{figure}

\newpage

\subsection{Mean O$^{15}$ and O$^{16}$ equation}

\begin{figure}[!h]
\centerline{
\includegraphics[width=6.7cm]{hif3d_tavg7000_mean_xo15_insf.eps}
\includegraphics[width=6.7cm]{hif3d_tavg7000_xo15_equation_insf.eps}
\includegraphics[width=6.7cm]{hif3d_tavg7000_xo15_equation_insf_bar.eps}}

\centerline{
\includegraphics[width=6.7cm]{hif3d_tavg7000_mean_xo16_insf.eps}
\includegraphics[width=6.7cm]{hif3d_tavg7000_xo16_equation_insf.eps}
\includegraphics[width=6.7cm]{hif3d_tavg7000_xo16_equation_insf_bar.eps}}
\caption{Mean composition equations derived from {\sf hif.3D}. \label{fig:x-equations}}
\end{figure}

\newpage

\subsection{Mean O$^{17}$ and Ne$^{20}$ equation}

\begin{figure}[!h]
\centerline{
\includegraphics[width=6.7cm]{hif3d_tavg7000_mean_xo17_insf.eps}
\includegraphics[width=6.7cm]{hif3d_tavg7000_xo17_equation_insf.eps}
\includegraphics[width=6.7cm]{hif3d_tavg7000_xo17_equation_insf_bar.eps}}

\centerline{
\includegraphics[width=6.7cm]{hif3d_tavg7000_mean_xne20_insf.eps}
\includegraphics[width=6.7cm]{hif3d_tavg7000_xne20_equation_insf.eps}
\includegraphics[width=6.7cm]{hif3d_tavg7000_xne20_equation_insf_bar.eps}}
\caption{Mean composition equations derived from {\sf hif.3D}. \label{fig:x-equations}}
\end{figure}

\newpage

\subsection{Mean Mg$^{24}$ and Si$^{28}$ equation}

\begin{figure}[!h]
\centerline{
\includegraphics[width=6.7cm]{hif3d_tavg7000_mean_xmg24_insf.eps}
\includegraphics[width=6.7cm]{hif3d_tavg7000_xmg24_equation_insf.eps}
\includegraphics[width=6.7cm]{hif3d_tavg7000_xmg24_equation_insf_bar.eps}}

\centerline{
\includegraphics[width=6.7cm]{hif3d_tavg7000_mean_xsi28_insf.eps}
\includegraphics[width=6.7cm]{hif3d_tavg7000_xsi28_equation_insf.eps}
\includegraphics[width=6.7cm]{hif3d_tavg7000_xsi28_equation_insf_bar.eps}}
\caption{Mean composition equations derived from {\sf hif.3D} \label{fig:x-equations}}
\end{figure}

\newpage

\section{Compositon equations derived from core helium flash data}

\subsection{Mean He$^{4}$ and C$^{12}$ equation}

\begin{figure}[!h]
\centerline{
\includegraphics[width=6.7cm]{chf3d_tavg12000_mean_xhe4_insf.eps}
\includegraphics[width=6.7cm]{chf3d_tavg12000_xhe4_equation_insf.eps}
\includegraphics[width=6.7cm]{chf3d_tavg12000_xhe4_equation_insf_bar.eps}}

\centerline{
\includegraphics[width=6.7cm]{chf3d_tavg12000_mean_xc12_insf.eps}
\includegraphics[width=6.7cm]{chf3d_tavg12000_xc12_equation_insf.eps}
\includegraphics[width=6.7cm]{chf3d_tavg12000_xc12_equation_insf_bar.eps}}
\caption{Mean composition equations derived from {\sf chf.3D}. \label{fig:x-equations}}
\end{figure}

\newpage

\subsection{Mean O$^{16}$ and Ne$^{20}$ equation}

\begin{figure}[!h]
\centerline{
\includegraphics[width=6.7cm]{chf3d_tavg12000_mean_xo16_insf.eps}
\includegraphics[width=6.7cm]{chf3d_tavg12000_xo16_equation_insf.eps}
\includegraphics[width=6.7cm]{chf3d_tavg12000_xo16_equation_insf_bar.eps}}

\centerline{
\includegraphics[width=6.7cm]{chf3d_tavg12000_mean_xne20_insf.eps}
\includegraphics[width=6.7cm]{chf3d_tavg12000_xne20_equation_insf.eps}
\includegraphics[width=6.7cm]{chf3d_tavg12000_xne20_equation_insf_bar.eps}}
\caption{Mean composition equations derived from {\sf chf.3D}.\label{fig:x-equations}}
\end{figure}



\clearpage


\section{Instantaneous hydrodynamic equations in spherical coordinates (Eulerian form)}
\label{sect:mean-field-derivation}

The hydrodynamic equations of a viscous multi-component reactive gas subject to gravity and thermal transport in spherical coordinates ($r, \theta, \phi$):
%

%\begin{align}
%\partial_t \rho \ \ \ = &  -\partial_i \rho U_i \\
%\partial_t \rho U_i = & - \partial_j(\rho U_i U_j + \delta_{ij} P - \tau_{ij}) - \rho \partial_i \Phi \\
%\partial_t \rho E \ = & - \partial_i(U_i(\rho E + P) - U_j \tau_{ij} - K\partial_i T) - \rho U_j \partial_j \Phi + \rho \dot{S} \\
%\partial_t \rho X_k = & - \partial_i \rho U_i X_k + \rho \dot{X}_k
%\end{align}

%\vspace{-0.5cm}

%first parameter is font size and the other line spacing
\fontsize{9pt}{20pt}

\begin{align}
\partial_{t} \big(\rho\big) \ \ \ = & - \left( \frac{1}{r^{2}}\partial_{r}\big(r^{2}[\rho u_{r}~]\big) + \frac{1}{r\sin{\theta}}\partial_{\theta} \big(\sin{\theta} [\rho u_{\theta}]\big) + \frac{1}{r\sin{\theta}}\partial_{\phi} [\rho u_{\phi}] \right) \\ 
\partial_{t} \big(\rho u_{r}\big) = & -\left( \frac{1}{r^{2}} \partial_{r} \big( r^{2} [\rho u_{r}^{2} - \tau_{rr}]\big) + \frac{1}{r\sin{\theta}}\partial_{\theta}(\sin{\theta}[\rho u_{r} u_{\theta} - \tau_{r\theta}]\big) + \frac{1}{r\sin{\theta}}\partial_{\phi}\big([\rho u_{r} u_{\phi} - \tau_{r\phi}]\big) + G_r^M + \partial_{r} P \right) - \rho \partial_{r} \Phi \\
\partial_{t} \big(\rho u_{\theta}\big) = & -\left( \frac{1}{r^{2}} \partial_{r} \big(r^{2} [\rho u_{\theta} u_{r} - \tau_{\theta r}]\big) + \frac{1}{r\sin{\theta}}\partial_{\theta}\big(\sin{\theta}[\rho u_{\theta}^{2} - \tau_{\theta \theta}]\big) + \frac{1}{r\sin{\theta}}\partial_{\phi}[\rho u_{\theta} u_{\phi} - \tau_{\theta \phi}]\big) + G_\theta^M + \frac{1}{r} \partial_{\theta} P \right) - \rho \frac{1}{r} \partial_{\theta} \Phi \\
\partial_{t} \big(\rho u_{\phi}\big) = & -\left( \frac{1}{r^{2}} \partial_{r} \big(r^{2} [\rho u_{\phi} u_r - \tau_{\phi r}]\big) + \frac{1}{r\sin{\theta}}\partial_{\theta}\big(\sin{\theta}[\rho u_{\theta}u_\phi - \tau_{\theta \phi}]\big) + \frac{1}{r\sin{\theta}}\partial_{\phi}\big([\rho u_{\phi}^2 - \tau_{\phi \phi}]\big) + G_\phi^M + \frac{1}{r\sin{\theta}} \partial_{\phi} P \right) - \rho \frac{1}{r \sin{\theta}} \partial_{\phi} \Phi \\
\partial_{t} \big(\rho \epsilon_T\big)  = & -\left( \frac{1}{r^{2}} \partial_{r} (r^{2} [u_{r} \big(\rho \epsilon_T + P\big) - K \partial_{r} T ]\big) + \frac{1}{r\sin{\theta}} \partial_{\theta} \big(\sin{\theta} [u_{\theta}\big(\rho \epsilon_T + P\big) - K\frac{1}{r} \partial_{\theta} T ]\big) + \frac{1}{r \sin{\theta}} \partial_{\phi} [u_{\phi}\big(\rho \epsilon_T + P\big) - K \frac{1}{r \sin{\theta}} \partial_{\phi} T] \right) + \nonumber \\ 
& + \left( \frac{1}{r^2}\partial_r \big( r^2 [u_j \tau_{jr}] \big) + \frac{1}{r\sin{\theta}}\partial_\theta \big( \sin{\theta} [u_j\tau_{j\theta}] \big) + \frac{1}{r\sin{\theta}}\partial_\phi [u_j\tau_{j\phi}] \right) -\rho \big(u_{r}\partial_{r} \Phi + u_{\theta}\frac{1}{r} \partial_{\theta} \Phi + u_{\phi}\frac{1}{r \sin{\theta}}\partial_{\phi} \Phi) + \rho \epsilon_{nuc} \\
\partial_{t} \big(\rho \epsilon_I\big)  = &  - \left( \frac{1}{r^{2}}\partial_{r}\big(r^{2}[\rho u_{r} \epsilon_I]\big) + \frac{1}{r\sin{\theta}}\partial_{\theta} \big(\sin{\theta} [\rho u_{\theta} \epsilon_I]\big) + \frac{1}{r\sin{\theta}}\partial_{\phi} [\rho u_{\phi} \epsilon_I] \right) - P \left( \frac{1}{r^{2}}\partial_{r}\big(r^{2}[u_{r}~]\big) + \frac{1}{r\sin{\theta}}\partial_{\theta} \big(\sin{\theta} [u_{\theta}]\big) + \frac{1}{r\sin{\theta}}\partial_{\phi} [u_{\phi}]  \right) + \nonumber \\
& + \left( \frac{1}{r^{2}}\partial_{r}\big(r^{2}[K\partial_r T]\big) + \frac{1}{r\sin{\theta}}\partial_{\theta} \big(\sin{\theta} [K\frac{1}{r}\partial_\theta T]\big) + \frac{1}{r\sin{\theta}}\partial_{\phi} [K\frac{1}{r\sin{\theta}}\partial_\phi T] \right) + \left( \tau_{jr} \partial_r u_j + \tau_{j\theta} \frac{1}{r}\partial_\theta u_j + \tau_{j\phi} \frac{1}{r\sin{\theta}}\partial_\phi u_j  \right) + \rho \epsilon_{nuc} \\
\partial_{t} \big(\rho \epsilon_K\big)  = & -\left( \frac{1}{r^2}\partial_r [r^2 \big(\rho u_r \epsilon_K - u_j\tau_{jr}\big)] + \frac{1}{r\sin{\theta}} \partial_\theta [\sin{\theta} \big(\rho u_\theta \epsilon_K - u_j\tau_{j\theta} \big)] + \frac{1}{r\sin{\theta}} \partial_\phi [\rho u_\phi \epsilon_K - u_j\tau_{j\phi}] \right) - \nonumber \\
& -\left( \frac{1}{r^2} \partial_r \big( r^2 [P u_r] \big) + \frac{1}{r \sin{\theta}} \partial_\theta \big(\sin{\theta}[P u_\theta] \big) + \frac{1}{r\sin{\theta}} \partial_\phi [ P u_\phi ] \right) + P \left( \frac{1}{r^{2}}\partial_{r}\big(r^{2}[u_{r}]\big) + \frac{1}{r\sin{\theta}}\partial_{\theta} \big(\sin{\theta} [u_{\theta}]\big) + \frac{1}{r\sin{\theta}}\partial_{\phi} [u_{\phi}]  \right) - \nonumber \\
& -  \left( \tau_{jr} \partial_r u_j + \tau_{j\theta} \frac{1}{r}\partial_\theta u_j + \tau_{j\phi} \frac{1}{r\sin{\theta}}\partial_\phi u_j  \right)  - \rho \big(u_{r}\partial_{r} \Phi + u_{\theta}\frac{1}{r} \partial_{\theta} \Phi + u_{\phi}\frac{1}{r \sin{\theta}}\partial_{\phi} \Phi) \\   
\partial_{t} \big(\rho X_{k}\big) = & -\left( \frac{1}{r^{2}} \partial_{r}(r^{2} [\rho u_{r} X_{k} ] \big) + \frac{1}{r\sin{\theta}}\partial_{\theta}\big(\sin{\theta} [\rho u_{\theta} X_{k} ]\big) + \frac{1}{r \sin{\theta}} \partial_{\phi} [\rho u_{\phi} X_{k}] \right) + \rho \dot{X}_{k}^{n} \ \ \ \ \ \ \ \ \ \ \ \  k = 1 ... N_{nuc} 
\end{align}

\begin{align}
G_r^M = -\frac{(\rho u_{\theta}^{2} - \tau_{\theta\theta})}{r} - \frac{(\rho u_{\phi}^{2} - \tau_{\phi\phi})}{r} \ \ \ G_\theta^M = + \frac{(\rho u_{\theta} u_{r} - \tau_{\theta r})}{r} -  \frac{(\rho u_{\phi}^{2} - \tau_{\phi\phi}) \cos{\theta}}{r \sin{\theta}} \ \ \ G_\phi^M = + \frac{(\rho u_{\phi} u_{r} - \tau_{\phi r})}{r} +  \frac{(\rho u_{\phi} u_{\theta} - \tau_{\phi \theta})\cos{\theta}}{r \sin{\theta}}
\end{align}

%\fontsize{12pt}{20pt}

%\noindent
%where $\rho$, $u_r$, $u_{\theta}$, $u_{\phi}$, $P$, $\epsilon_T$, $\epsilon_I$, $\epsilon_K$, $T$, $\epsilon_{nuc}$, $X_k$, and $\dot{X}_k^n$ are the density, the radial velocity, the $\theta$-velocity, the rotation velocity, the pressure, the total specific energy, the specific internal energy, the specific kinetic energy, the temperature, the energy generation rate per mass due to reactions, the mass fraction of species $k$, and the change of this mass fraction due to reactions, respectively.  $N_{nuc}$ is the number of species the gas is composed. $\tau_{ij} = 2 \mu S_{ij} {\color{red}- 2/3 \mu \nabla \cdot {\bf u} \delta_{ij}}$ {\color{red} (check Mihalas and Mihalas book)} is the viscous stress, where $S_{ij} = \frac{1}{2} (\partial_i u_j + \partial_j u_i)$ is strain-rate, $\mu = \rho \nu$ is dynamic viscosity and $\nu$ is the kinematic viscosity. K is thermal conductivity. $g_i$ is gravitational acceleration in $r, \theta, \phi$ and $\Phi$ is gravitational potential. $G$ are geometric terms. 

%\begin{align}
%G_r^M = -\frac{(u_{\theta}^{2} - \tau_{\theta\theta}/\rho)}{r} - \frac{(u_{\phi}^{2} - \tau_{\phi\phi}/\rho)}{r} \ \ \ G_\theta^M = + \frac{(u_{\theta} u_{r} - \tau_{\theta r}/\rho)}{r} -  \frac{(u_{\phi}^{2} - \tau_{\phi\phi}/\rho) \cos{\theta}}{r \sin{\theta}} \ \ \ G_\phi^M = + \frac{(u_{\phi} u_{r} - \tau_{\phi r}/\rho)}{r} +  \frac{(u_{\phi} u_{\theta} - \tau_{\phi \theta}/\rho)\cos{\theta}}{r \sin{\theta}}
%\end{align}

%\noindent
%where $\rho$, $u_r$, $u_{\theta}$, $u_{\phi}$, $P$, $\epsilon_T$, $T$, $\dot{S}$, $X_k$, and $\dot{X}_k$ are the density, the radial velocity, the $\theta$-velocity, the rotation velocity, the pressure, the total specific energy, the temperature, the energy generation rate per mass due to reactions, the mass fraction of species $k$, and the change of this mass fraction due to reactions, respectively.  $N_{nuc}$ is the number of species the gas is composed. $\tau_{ij} = 2 \mu S_{ij}$ is the viscous stress (Newtonian dynamic stress), where $S_{ij} = \frac{1}{2} (\partial_i u_j + \partial_j u_i)$ is strain-rate, $\mu = \rho \nu$ is dynamic viscosity and $\nu$ is the kinematic viscosity. $G^M$ are geometric terms. 

\newpage 

\section{Instantaneous hydrodynamic equations in spherical coordinates (Lagrangian form)}
\label{sect:hydro-inst-lag}

The hydrodynamic equations of a viscous multi-component reactive gas subject to gravity and thermal transport in spherical coordinates ($r, \theta, \phi$) are ($D_t (.) = \partial_t (.) + u_n \partial_n (.)$ is advective derivative):

%\begin{align}
%\partial_t \rho \ \ \ = &  -\partial_i \rho U_i \\
%\partial_t \rho U_i = & - \partial_j(\rho U_i U_j + \delta_{ij} P - \tau_{ij}) - \rho \partial_i \Phi \\
%\partial_t \rho E \ = & - \partial_i(U_i(\rho E + P) - U_j \tau_{ij} - K\partial_i T) - \rho U_j \partial_j \Phi + \rho \dot{S} \\
%\partial_t \rho X_k = & - \partial_i \rho U_i X_k + \rho \dot{X}_k
%\end{align}

%\vspace{-0.5cm}

%first parameter is font size and the other line spacing
\fontsize{9pt}{20pt}


\begin{align}
D_{t} \big(\rho\big) \ \ \ = & - \rho \left( \frac{1}{r^{2}}\partial_{r}\big(r^{2}[u_{r}~]\big) + \frac{1}{r\sin{\theta}}\partial_{\theta} \big(\sin{\theta} [u_{\theta}]\big) + \frac{1}{r\sin{\theta}}\partial_{\phi} [u_{\phi}] \right) \\ 
\rho D_{t} \big(u_{r}\big) = & +\left( \frac{1}{r^{2}} \partial_{r} \big( r^{2} [\tau_{rr}]\big) + \frac{1}{r\sin{\theta}}\partial_{\theta}(\sin{\theta}[\tau_{r\theta}]\big) + \frac{1}{r\sin{\theta}}\partial_{\phi}\big([\tau_{r\phi}]\big) - G_r^M - \partial_{r} P \right) + \rho g_r \\
\rho D_{t} \big(u_{\theta}\big) = & +\left( \frac{1}{r^{2}} \partial_{r} \big(r^{2} [\tau_{\theta r}]\big) + \frac{1}{r\sin{\theta}}\partial_{\theta}\big(\sin{\theta}[\tau_{\theta \theta}]\big) + \frac{1}{r\sin{\theta}}\partial_{\phi}[\tau_{\theta \phi}]\big) - G_\theta^M - \frac{1}{r} \partial_{\theta} P \right) + \rho g_\theta \\
\rho D_{t} \big(u_{\phi}\big) = & +\left( \frac{1}{r^{2}} \partial_{r} \big(r^{2} [\tau_{\phi r}]\big) + \frac{1}{r\sin{\theta}}\partial_{\theta}\big(\sin{\theta}[\tau_{\phi \theta}]\big) + \frac{1}{r\sin{\theta}}\partial_{\phi}\big([\tau_{\phi \phi}]\big) - G_\phi^M - \frac{1}{r\sin{\theta}} \partial_{\phi} P \right) + \rho g_\phi \\
\rho D_{t} \big(\epsilon_T\big) = & -\left( \frac{1}{r^{2}} \partial_{r} \big( r^{2} [u_{r} P - K \partial_{r} T ]\big) + \frac{1}{r\sin{\theta}} \partial_{\theta} \big(\sin{\theta} [u_{\theta} P - K\frac{1}{r} \partial_{\theta} T ]\big) + \frac{1}{r \sin{\theta}} \partial_{\phi} [u_{\phi}P - K \frac{1}{r \sin{\theta}} \partial_{\phi} T] \right) - \nonumber \\
& +\left( \frac{1}{r^2}\partial_r [r^2 \big(u_j\tau_{jr} \big)] + \frac{1}{r\sin{\theta}} \partial_\theta [\sin{\theta} \big(u_j\tau_{j\theta} \big)] + \frac{1}{r\sin{\theta}} \partial_\phi [u_j\tau_{j\phi}] \right) +\rho \big(u_{r} g_r + u_{\theta} g_\theta + u_{\phi} g_\phi) + \rho \epsilon_{nuc} \\
\rho D_{t} \big(\epsilon_I\big) = & -P \left( \frac{1}{r^{2}}\partial_{r}\big(r^{2}[u_{r}]\big) + \frac{1}{r\sin{\theta}}\partial_{\theta} \big(\sin{\theta} [u_{\theta}]\big) + \frac{1}{r\sin{\theta}}\partial_{\phi} [u_{\phi}]  \right) + \left( \frac{1}{r^{2}}\partial_{r}\big(r^{2}[K\partial_r T]\big) + \frac{1}{r\sin{\theta}}\partial_{\theta} \big(\sin{\theta} [K\frac{1}{r}\partial_\theta T]\big) + \frac{1}{r\sin{\theta}}\partial_{\phi} [K\frac{1}{r\sin{\theta}}\partial_\phi T] \right) + \nonumber \\
& + \left( \tau_{jr} \partial_r u_j + \tau_{j\theta} \frac{1}{r}\partial_\theta u_j + \tau_{j\phi} \frac{1}{r\sin{\theta}}\partial_\phi u_j  \right) + \rho \epsilon_{nuc} \\
\rho D_{t} \big(\epsilon_K\big) = & +\left( \frac{1}{r^2}\partial_r [r^2 \big(u_j\tau_{jr} \big)] + \frac{1}{r\sin{\theta}} \partial_\theta [\sin{\theta} \big(u_j\tau_{j\theta} \big)] + \frac{1}{r\sin{\theta}} \partial_\phi [u_j\tau_{j\phi}] \right) -\left( \frac{1}{r^2} \partial_r \big( r^2 [P u_r] \big) + \frac{1}{r \sin{\theta}} \partial_\theta \big(\sin{\theta}[P u_\theta] \big) + \frac{1}{r\sin{\theta}} \partial_\phi [ P u_\phi ] \right) + \nonumber \\
& + P \left( \frac{1}{r^{2}}\partial_{r}\big(r^{2}[u_{r}~]\big) + \frac{1}{r\sin{\theta}}\partial_{\theta} \big(\sin{\theta} [u_{\theta}]\big) + \frac{1}{r\sin{\theta}}\partial_{\phi} [u_{\phi}]  \right) - \left( \tau_{jr} \partial_r u_j + \tau_{j\theta} \frac{1}{r}\partial_\theta u_j + \tau_{j\phi} \frac{1}{r\sin{\theta}}\partial_\phi u_j  \right) +\rho \big(u_{r} g_r + u_{\theta} g_\theta + u_{\phi} g_\phi) \\
\rho D_{t} \big(X_{k}\big) = & + \rho \dot{X}_{k}^{n} \ \ \ \ \ \ \ \ \ \ \ \  k = 1 ... N_{nuc}
\end{align}

\begin{align}
G_r^M = -\frac{(\rho u_{\theta}^{2} - \tau_{\theta\theta})}{r} - \frac{(\rho u_{\phi}^{2} - \tau_{\phi\phi})}{r} \ \ \ G_\theta^M = + \frac{(\rho u_{\theta} u_{r} - \tau_{\theta r})}{r} -  \frac{(\rho u_{\phi}^{2} - \tau_{\phi\phi}) \cos{\theta}}{r \sin{\theta}} \ \ \ G_\phi^M = + \frac{(\rho u_{\phi} u_{r} - \tau_{\phi r})}{r} +  \frac{(\rho u_{\phi} u_{\theta} - \tau_{\phi \theta})\cos{\theta}}{r \sin{\theta}}
\end{align}

%first parameter is font size and the other line spacing
\fontsize{12pt}{20pt}

\noindent
where $\rho$, $u_r$, $u_{\theta}$, $u_{\phi}$, $P$, $\epsilon_T$, $\epsilon_I$, $\epsilon_K$, $T$, $\epsilon_{nuc}$, $X_k$, and $\dot{X}_k^n$ are the density, the radial velocity, the $\theta$-velocity, the rotation velocity, the pressure, the total specific energy, the specific internal energy, the specific kinetic energy, the temperature, the energy generation rate per mass due to reactions, the mass fraction of species $k$, and the change of this mass fraction due to reactions, respectively.  $N_{nuc}$ is the number of species the gas is composed. $\tau_{ij} = 2 \mu S_{ij} - 2/3 \mu \nabla \cdot {\bf u} \delta_{ij}$ is the viscous stress, where $S_{ij} = \frac{1}{2} (\partial_i u_j + \partial_j u_i)$ is strain-rate, $\mu = \rho \nu$ is dynamic viscosity and $\nu$ is the kinematic viscosity. K is thermal conductivity. $g_i$ is gravitational acceleration in $r, \theta, \phi$ and $\Phi$ is gravitational potential. $G$ are geometric terms. 

\section{Reynolds decomposition}

\noindent 
Reynolds decomposition:

\begin{equation}
A(r,\theta,\phi) = \eht{A}(r) + A'(r,\theta,\phi)
\end{equation}

\noindent
Definition of the averaging space-time operator:

\begin{equation}
\eht{A(r)} = \frac{1}{\Delta T \Delta \Omega} \int_{\Delta T}\int_{\Delta \Omega} A(r,
\theta,\phi) \ dt \ d\Omega
\end{equation}

\noindent
Some properties of the operator:

\begin{equation}
\eht{A'} = 0 
\end{equation}

\begin{equation}
\eht{\eht{A}}=\eht{A}
\end{equation}

\begin{equation}
(A')'=A'
\end{equation}

\begin{equation}
\eht{\eht{A}B} = \eht{A} \ \eht{B}
\end{equation}

\begin{equation}
\eht{AB} = \eht{A} \ \eht{B} + \eht{A'B'}
\end{equation}

\begin{equation}
\eht{A'B'} = \eht{A'B}
\end{equation}

\section{Favre decomposition}

\noindent 
Favre decomposition:

\begin{equation}
F = \fht{F}(r) + F''(r,\theta,\phi)
\end{equation}

\noindent
Definition of the averaging operator:

\begin{equation}
\fht{F} = \frac{\eht{\rho F}}{\eht{\rho}}
\end{equation}

\noindent
Some properties of the operator: 

\begin{equation}
\eht{\rho F''} = \fht{F''} = 0
\end{equation}

\begin{equation}
\fht{F} = \eht{F} + \frac{\eht{\rho F'}}{\eht{\rho}} 
\end{equation}

\begin{equation}
F'' = F' - \frac{\eht{\rho F'}}{\eht{\rho}} \rightarrow \eht{F''} = -\frac{\eht{\rho F'}}{\eht{\rho}}
\end{equation}

\begin{equation}
\eht{\rho F''G''} = \eht{\rho}\fht{F''G''}
\end{equation}

\newpage

\section{Derivation of first order moments}

\subsection{Mean continuity equation}

We begin by instantaneous 3D continuity equation and apply "ensemble'' (space-time) averaging (Sect.\ref{sect:intro}):

%first parameter is font size and the other line spacing
\fontsize{9pt}{20pt}

\begin{align}
\partial_{t} \rho = & - \left( \frac{1}{r^{2}}\partial_{r}\big(r^{2}[\rho u_{r}~]\big) + \frac{1}{r\sin{\theta}}\partial_{\theta} \big(\sin{\theta} [\rho u_{\theta}]\big) + \frac{1}{r\sin{\theta}}\partial_{\phi} [\rho u_{\phi}] \right) \\
\partial_{t} \eht{\rho} = & -\left( \eht{\frac{1}{r^{2}}\partial_{r}\big(r^{2}[\rho u_{r}~]\big)} + \cancelto{0}{\eht{\frac{1}{r\sin{\theta}}\partial_{\theta} \big(\sin{\theta} [\rho u_{\theta}]\big)}} + \cancelto{0}{\eht{\frac{1}{r\sin{\theta}}\partial_{\phi} [\rho u_{\phi}]}} \right) \ \ \ \mbox{``ensemble'' (space-time) averaging} \\
\partial_t \eht{\rho} = & -\frac{1}{r^{2}}\partial_{r}\big(r^{2}[\eht{\rho} \fht{u}_{r}~]\big) \\
\partial_t \eht{\rho} = & -\fht{u}_r \partial_r \eht{\rho} -\eht{\rho} \frac{1}{r^{2}}\partial_{r} (r^2 \fht{u}_r) \\
\partial_t \eht{\rho} + \fht{u}_r \partial_r \eht{\rho} = & -\eht{\rho}\frac{1}{r^{2}}\partial_{r} (r^2 \fht{u}_r) \\
{\color{red} \fht{D}_t \eht{\rho} =} & {\color{red} -\eht{\rho} \fht{d}}
\end{align}

%first parameter is font size and the other line spacing
\fontsize{12pt}{20pt}

\subsection{Mean radial momentum equation}

We begin by instantaneous 3D radial momentum equation and apply "ensemble'' (space-time) averaging (Sect.\ref{sect:intro}):

%first parameter is font size and the other line spacing
\fontsize{9pt}{20pt}

\begin{align}
\partial_{t} \rho u_{r} = & -\left( \frac{1}{r^{2}} \partial_{r} \big( r^{2} [\rho u_{r}^{2} - \tau_{rr}]\big) + \frac{1}{r\sin{\theta}}\partial_{\theta}(\sin{\theta}[\rho u_{r} u_{\theta} - \tau_{r\theta}]\big) + \frac{1}{r\sin{\theta}}\partial_{\phi}\big([\rho u_{r} u_{\phi} - \tau_{r\phi}]\big) + G_r^M + \partial_{r} P \right) + \rho g_r \\
\partial_{t} \eht{\rho u_{r}}= & -\left( \eht{\frac{1}{r^{2}} \partial_{r} \big( r^{2} [\rho u_{r}^{2} - \tau_{rr}]\big)} + \cancelto{0}{\eht{\frac{1}{r\sin{\theta}}\partial_{\theta}(\sin{\theta}[\rho u_{r} u_{\theta} - \tau_{r\theta}]\big)}} + \cancelto{0}{\eht{\frac{1}{r\sin{\theta}}\partial_{\phi}\big([\rho u_{r} u_{\phi} - \tau_{r\phi}]\big)}} + \eht{G_r^M} + \partial_{r} \eht{P} \right) + \eht{\rho g_r} \\
\partial_{t} \eht{\rho {u}_{r}}= & -\frac{1}{r^{2}} \partial_{r} \big( r^{2} [\eht{\rho u_{r} u_{r}}] \big) - \cancelto{0}{\frac{1}{r^{2}} \partial_{r} \big(r^2 \eht{\tau}_{rr}\big)} - \eht{G_r^M} - \partial_{r} \eht{P} + \eht{\rho g_r} \ \ \ \mbox{we neglect mean viscosity $\eht{\tau}$} 
\end{align}

\newpage


%first parameter is font size and the other line spacing
\fontsize{9pt}{20pt}

\begin{align}
\partial_{t} \eht{\rho} \fht{u_{r}} = & -\frac{1}{r^{2}} \partial_{r} \big( r^{2} [\eht{\rho} \fht{u_{r} u_{r}}] \big) - \eht{G_r^M} - \partial_{r} \eht{P} + \eht{\rho} \fht{g}_r \\
\partial_{t} \eht{\rho} \fht{u_{r}} = & -\frac{1}{r^{2}} \partial_{r} \big( r^{2} [\eht{\rho} \fht{u_{r}} \fht{u_{r}} + \eht{\rho}\fht{u''_r u''_r}] \big) - \eht{G_r^M} - \partial_{r} \eht{P} + \eht{\rho} \fht{g}_r \\
\partial_{t} \eht{\rho} \fht{u_{r}} + \frac{1}{r^{2}} \partial_{r} \big( r^{2} [\eht{\rho} \fht{u_{r}} \fht{u_{r}}] \big) = & -\frac{1}{r^{2}} \partial_{r} \big( \eht{\rho}\fht{u''_r u''_r} \big) - \eht{G_r^M} - \partial_{r} \eht{P} + \eht{\rho} \fht{g}_r \\
{\color{red} \eht{\rho} \fht{D}_t \fht{u}_r = } & {\color{red}-\nabla_r \fht{R}_{rr}  - \eht{G_r^M} - \partial_{r} \eht{P} + \eht{\rho} \fht{g}_r}
\end{align}

%first parameter is font size and the other line spacing
\fontsize{12pt}{20pt}

\subsection{Mean polar momentum equation}

We begin by instantaneous 3D polar momentum equation and apply "ensemble'' (space-time) averaging (Sect.\ref{sect:intro}):

%first parameter is font size and the other line spacing
\fontsize{9pt}{20pt}

\begin{align}
\partial_{t} \rho u_{\theta} = & -\left( \frac{1}{r^{2}} \partial_{r} \big(r^{2} [\rho u_{\theta} u_{r} - \tau_{\theta r}]\big) + \frac{1}{r\sin{\theta}}\partial_{\theta}\big(\sin{\theta}[\rho u_{\theta}^{2} - \tau_{\theta \theta}]\big) + \frac{1}{r\sin{\theta}}\partial_{\phi}[\rho u_{\theta} u_{\phi} - \tau_{\theta \phi}]\big) + G_\theta^M + \frac{1}{r} \partial_{\theta} P \right) - \rho \frac{1}{r} \partial_{\theta} \Phi \\
\partial_{t} \eht{\rho u_{\theta}} = & -\left( \eht{\frac{1}{r^{2}} \partial_{r} \big(r^{2} [\rho u_{\theta} u_{r} - \tau_{\theta r}]\big)} + \cancelto{0}{\eht{\frac{1}{r\sin{\theta}}\partial_{\theta}\big(\sin{\theta}[\rho u_{\theta}^{2} - \tau_{\theta \theta}]\big)}} + \cancelto{0}{\eht{\frac{1}{r\sin{\theta}}\partial_{\phi}[\rho u_{\theta} u_{\phi} - \tau_{\theta \phi}]\big)}} + \eht{G_\theta^M} + \frac{1}{r} \eht{\partial_{\theta} P} + \eht{\rho g_\theta}\right) \\
\partial_{t} \eht{\rho u_{\theta}} = & -\frac{1}{r^{2}} \partial_{r} \big( r^{2} [\eht{\rho u_{\theta} u_{r}}] \big) + \cancelto{0}{\frac{1}{r^{2}} \partial_{r} \big(r^{2} \eht{\tau_{\theta r}} \big)} - \eht{G_\theta^M} - \frac{1}{r} \eht{\partial_{\theta} P} + \cancelto{0}{\eht{\rho g_\theta}} \ \ \ \mbox{we neglect mean viscosity $\eht{\tau}$ and gravity in $\theta$} \\
\partial_{t} \eht{\rho} \fht{u}_{\theta} = & -\frac{1}{r^{2}} \partial_{r} \big( r^{2} [\eht{\rho} \fht{u_{\theta} u_{r}}] \big) - \eht{G_\theta^M} - \frac{1}{r} \eht{\partial_{\theta} P}\\
\partial_{t} \eht{\rho} \fht{u}_{\theta} = & -\frac{1}{r^{2}} \partial_{r} \big( r^{2} [\eht{\rho} \fht{u}_{\theta} \fht{u}_{r} + \eht{\rho}\fht{u''_\theta u''_r} \big) - \eht{G_\theta^M} - \frac{1}{r} \eht{\partial_{\theta} P}\\ 
\partial_{t} \eht{\rho} \fht{u}_{\theta} + \frac{1}{r^{2}} \partial_{r} \big( r^{2} [\eht{\rho} \fht{u}_{\theta} \fht{u}_{r}] \big) = & -\frac{1}{r^{2}} \partial_{r} \big( r^{2} \eht{\rho}\fht{u''_\theta u''_r} \big) - \eht{G_\theta^M} - \frac{1}{r} \eht{\partial_{\theta} P}\\
{\color{red} \eht{\rho}\fht{D}_t \fht{u}_\theta =} & {\color{red} -\nabla_r \fht{R}_{\theta r} - \eht{G_\theta^M} - (1/r) \eht{\partial_{\theta} P}}
\end{align}

%first parameter is font size and the other line spacing
\fontsize{12pt}{20pt}

\subsection{Mean azimutal momentum equation}

We begin by instantaneous 3D azimutal momentum equation and apply "ensemble'' (space-time) averaging (Sect.\ref{sect:intro}):

%first parameter is font size and the other line spacing
\fontsize{9pt}{20pt}

\begin{align}
\partial_{t} \big(\rho u_{\phi}\big) = & -\left( \frac{1}{r^{2}} \partial_{r} \big(r^{2} [\rho u_{\phi} u_r - \tau_{\phi r}]\big) + \frac{1}{r\sin{\theta}}\partial_{\theta}\big(\sin{\theta}[\rho u_{\theta}u_\phi - \tau_{\theta \phi}]\big) + \frac{1}{r\sin{\theta}}\partial_{\phi}\big([\rho u_{\phi}^2 - \tau_{\phi \phi}]\big) + G_\phi^M + \frac{1}{r\sin{\theta}} \partial_{\phi} P \right) - \\
& - \rho \frac{1}{r \sin{\theta}} \partial_{\phi} \Phi \nonumber \\
\partial_{t} \eht{\rho u_{\phi}} = & -\left( \eht{\frac{1}{r^{2}} \partial_{r} \big(r^{2} [\rho u_{\phi} u_r - \tau_{\phi r}]\big)} + \cancelto{0}{\eht{\frac{1}{r\sin{\theta}}\partial_{\theta}\big(\sin{\theta}[\rho u_{\theta}u_\phi - \tau_{\theta \phi}]\big)}} + \cancelto{0}{\eht{\frac{1}{r\sin{\theta}}\partial_{\phi}\big([\rho u_{\phi}^2 - \tau_{\phi \phi}]\big)}} + \eht{G_\phi^M} + \cancelto{0}{\eht{\frac{1}{r\sin{\theta}} \partial_{\phi} P}} \right) + \nonumber \\ 
+ \eht{\rho g_\phi} \\
\partial_{t} \eht{\rho u_{\phi}} = & -\frac{1}{r^{2}} \partial_{r} \big(r^{2} [\eht{\rho u_{\phi} u_r}] \big) - \cancelto{0}{\frac{1}{r^{2}} \partial_{r} \big(r^{2} \eht{\tau_{\phi r}} \big)} + \eht{G_\phi^M} - \cancelto{0}{\eht{\rho g_\phi}} \ \ \ \mbox{we neglect mean viscosity $\tau$ and gravity in $\phi$} \\
\partial_{t} \eht{\rho} \fht{u}_{\phi} = & -\frac{1}{r^{2}} \partial_{r} \big(r^{2} [\eht{\rho} \fht{u_{\phi} u_r}] \big) + \eht{G_\phi^M} \\
\partial_{t} \eht{\rho} \fht{u}_{\phi} = & -\frac{1}{r^{2}} \partial_{r} \big(r^{2} [\eht{\rho} \fht{u}_{\phi} \fht{u}_r - \eht{\rho} \fht{u''_\phi u''_r}] \big) - \eht{G_\phi^M} \\
\partial_{t} \eht{\rho} \fht{u}_{\phi} + \frac{1}{r^{2}} \partial_{r} \big(r^{2} [\eht{\rho} \fht{u}_{\phi} \fht{u}_r]\big) = & -\frac{1}{r^{2}} \partial_{r} \big(r^{2} \eht{\rho} \fht{u''_\phi u''_r} \big) - \eht{G_\phi^M} \\
{\color{red} \eht{\rho}\fht{D}_t \fht{u}_\phi =} & {\color{red} -\nabla_r \fht{R}_{\phi r} - \eht{G_\phi^M}}
\end{align}

%first parameter is font size and the other line spacing
\fontsize{12pt}{20pt}

\subsection{Mean internal energy equation}
\label{sect:mean-internal-energy-eq}

We begin by instantaneous 3D internal energy equation and apply "ensemble'' (space-time) averaging (Sect.\ref{sect:intro}):

%first parameter is font size and the other line spacing
\fontsize{9pt}{20pt}

\begin{align}
\partial_{t} \big(\rho \epsilon_I\big)  = &  - \left( \frac{1}{r^{2}}\partial_{r}\big(r^{2}[\rho u_{r} \epsilon_I]\big) + \frac{1}{r\sin{\theta}}\partial_{\theta} \big(\sin{\theta} [\rho u_{\theta} \epsilon_I]\big) + \frac{1}{r\sin{\theta}}\partial_{\phi} [\rho u_{\phi} \epsilon_I] \right) - P \left( \frac{1}{r^{2}}\partial_{r}\big(r^{2}[u_{r}~]\big) + \frac{1}{r\sin{\theta}}\partial_{\theta} \big(\sin{\theta} [u_{\theta}]\big) + \frac{1}{r\sin{\theta}}\partial_{\phi} [u_{\phi}]  \right) + \nonumber \\
& + \left( \frac{1}{r^{2}}\partial_{r}\big(r^{2}[K\partial_r T]\big) + \frac{1}{r\sin{\theta}}\partial_{\theta} \big(\sin{\theta} [K\frac{1}{r}\partial_\theta T]\big) + \frac{1}{r\sin{\theta}}\partial_{\phi} [K\frac{1}{r\sin{\theta}}\partial_\phi T] \right) + \left( \tau_{jr} \partial_r u_j + \tau_{j\theta} \frac{1}{r}\partial_\theta u_j + \tau_{j\phi} \frac{1}{r\sin{\theta}}\partial_\phi u_j  \right) + \rho \epsilon_{\rm nuc} 
\end{align}

\begin{align}
\partial_{t} \eht{\rho \epsilon_I}  = &  - \left( \eht{\frac{1}{r^{2}}\partial_{r}\big(r^{2}[\rho u_{r} \epsilon_I]\big)} + \cancelto{0}{\eht{\frac{1}{r\sin{\theta}}\partial_{\theta} \big(\sin{\theta} [\rho u_{\theta} \epsilon_I]\big)}} + \cancelto{0}{\eht{\frac{1}{r\sin{\theta}}\partial_{\phi} [\rho u_{\phi} \epsilon_I]}} \right) - \eht{P \left( \frac{1}{r^{2}}\partial_{r}\big(r^{2}[u_{r}~]\big) + \frac{1}{r\sin{\theta}}\partial_{\theta} \big(\sin{\theta} [u_{\theta}]\big) + \frac{1}{r\sin{\theta}}\partial_{\phi} [u_{\phi}]  \right)} + \nonumber \\
& + \left( \eht{\frac{1}{r^{2}}\partial_{r}\big(r^{2}[K\partial_r T]\big)} + \cancelto{0}{\eht{\frac{1}{r\sin{\theta}}\partial_{\theta} \big(\sin{\theta} [K\frac{1}{r}\partial_\theta T]\big)}} + \cancelto{0}{\eht{\frac{1}{r\sin{\theta}}\partial_{\phi} [K\frac{1}{r\sin{\theta}}\partial_\phi T]}} \right) + \left( \eht{\tau_{jr} \partial_r u_j + \tau_{j\theta} \frac{1}{r}\partial_\theta u_j + \tau_{j\phi} \frac{1}{r\sin{\theta}}\partial_\phi u_j}  \right) + \eht{\rho \epsilon_{\rm nuc}} \\
\partial_{t} \eht{\rho \epsilon_I}  = &  - \frac{1}{r^{2}}\partial_{r}\big(r^{2}[\eht{\rho u_{r} \epsilon_I}]\big) - \eht{P d} + \frac{1}{r^{2}}\partial_{r}\big(r^{2}[\eht{\chi \partial_r T}]\big) + \eht{\tau_{ji} \partial_i u_j} + \eht{\rho \epsilon_{\rm nuc}} \\
\partial_{t} \eht{\rho} \fht{\epsilon_I}  = &  - \frac{1}{r^{2}}\partial_{r}\big(r^{2}[\eht{\rho} \fht{u_{r} \epsilon_I}]\big) - \eht{P d} + \frac{1}{r^{2}}\partial_{r}\big(r^{2}[\eht{\chi \partial_r T}]\big) + \eht{\tau_{ji} \partial_i u_j} + \eht{\rho \epsilon_{\rm nuc}} \\
\partial_{t} \eht{\rho} \fht{\epsilon_I}  = &  - \frac{1}{r^{2}}\partial_{r}\big(r^{2}[\eht{\rho} \fht{u}_{r} \fht{\epsilon}_I]\big) -\frac{1}{r^{2}}\partial_{r}\big(r^{2}[\eht{\rho} \fht{u''_{r} \epsilon''_I}]\big)  - \eht{P d} + \frac{1}{r^{2}}\partial_{r}\big(r^{2}[\eht{\chi \partial_r T}]\big) + \cancelto{0}{\eht{\tau_{ji}} \partial_i \eht{u_j}} + \eht{\tau'_{ji}\partial_i u'_j} + \eht{\rho} \fht{\epsilon}_{\rm nuc} \ \ \ \mbox{we neglect $\eht{\tau}$} \\
\eht{\rho}\fht{D}_t \fht{\epsilon}_I = & -\nabla_r f_I - \eht{P d} + \nabla_r f_T + \eht{\tau'_{ji}\partial_i u'_j} + \eht{\rho} \fht{\epsilon}_{\rm nuc} \\
\eht{\rho}\fht{D}_t \fht{\epsilon}_I = & -\nabla_r (f_I + f_T) - \eht{P} \ \eht{d} - \eht{P'd'} + \varepsilon_k + \eht{\rho} \fht{\epsilon}_{\rm nuc} \\
{\color{red} \eht{\rho}\fht{D}_t \fht{\epsilon}_I =} & {\color{red}-\nabla_r (f_I + f_T) - \eht{P} \ \eht{d} - W_P + \varepsilon_k + \eht{\rho} \fht{\epsilon}_{\rm nuc}}
\end{align}

%first parameter is font size and the other line spacing
\fontsize{12pt}{20pt}

\subsection{Mean kinetic energy equation}

We begin by instantaneous 3D kinetic energy equation and apply "ensemble'' (space-time) averaging (Sect.\ref{sect:intro}):

%first parameter is font size and the other line spacing
\fontsize{9pt}{20pt}

\begin{align}
\partial_{t} \big(\rho \epsilon_K\big)  = & -\left( \frac{1}{r^2}\partial_r [r^2 \big(\rho u_r \epsilon_K - u_j\tau_{jr}\big)] + \frac{1}{r\sin{\theta}} \partial_\theta [\sin{\theta} \big(\rho u_\theta \epsilon_K - u_j\tau_{j\theta} \big)] + \frac{1}{r\sin{\theta}} \partial_\phi [\rho u_\phi \epsilon_K - u_j\tau_{j\phi}] \right) - \nonumber \\
& -\left( \frac{1}{r^2} \partial_r \big( r^2 [P u_r] \big) + \frac{1}{r \sin{\theta}} \partial_\theta \big(\sin{\theta}[P u_\theta] \big) + \frac{1}{r\sin{\theta}} \partial_\phi [ P u_\phi ] \right) + P \left( \frac{1}{r^{2}}\partial_{r}\big(r^{2}[u_{r}]\big) + \frac{1}{r\sin{\theta}}\partial_{\theta} \big(\sin{\theta} [u_{\theta}]\big) + \frac{1}{r\sin{\theta}}\partial_{\phi} [u_{\phi}]  \right) - \nonumber \\
& -  \left( \tau_{jr} \partial_r u_j + \tau_{j\theta} \frac{1}{r}\partial_\theta u_j + \tau_{j\phi} \frac{1}{r\sin{\theta}}\partial_\phi u_j  \right)  - \rho \big(u_{r}\partial_{r} \Phi + u_{\theta}\frac{1}{r} \partial_{\theta} \Phi + u_{\phi}\frac{1}{r \sin{\theta}}\partial_{\phi} \Phi) 
\end{align}

\begin{align}
\partial_{t} \eht{\rho \epsilon_K} = & -\left( \eht{\frac{1}{r^2}\partial_r [r^2 \big(\rho u_r \epsilon_K - u_j\tau_{jr}\big)]} + \cancelto{0}{\eht{\frac{1}{r\sin{\theta}} \partial_\theta [\sin{\theta} \big(\rho u_\theta \epsilon_K - u_j\tau_{j\theta} \big)]}} + \cancelto{0}{\eht{\frac{1}{r\sin{\theta}} \partial_\phi [\rho u_\phi \epsilon_K - u_j\tau_{j\phi}]}} \right) - \nonumber \\
& -\left( \eht{\frac{1}{r^2} \partial_r \big( r^2 [P u_r] \big)} + \cancelto{0}{\eht{\frac{1}{r \sin{\theta}} \partial_\theta \big(\sin{\theta}[P u_\theta] \big)}} + \cancelto{0}{\eht{\frac{1}{r\sin{\theta}} \partial_\phi [ P u_\phi ]}} \right) + \eht{P \left( \frac{1}{r^{2}}\partial_{r}\big(r^{2}[u_{r}]\big) + \frac{1}{r\sin{\theta}}\partial_{\theta} \big(\sin{\theta} [u_{\theta}]\big) + \frac{1}{r\sin{\theta}}\partial_{\phi} [u_{\phi}]  \right)} - \nonumber \\
& -  \left( \eht{\tau_{jr} \partial_r u_j + \tau_{j\theta} \frac{1}{r}\partial_\theta u_j + \tau_{j\phi} \frac{1}{r\sin{\theta}}\partial_\phi u_j}  \right)  + \eht{\rho \big( u_{r} g_r + \cancelto{0}{u_{\theta} g_\theta} + \cancelto{0}{u_{\phi} g_\phi \big)}} \\
\partial_{t} \eht{\rho \epsilon_K} = & -\frac{1}{r^2}\partial_r [r^2 \big(\eht{\rho u_r \epsilon_K}\big)] -\frac{1}{r^2}\partial_r [r^2 \eht{(u_j\tau_{jr})}] -\frac{1}{r^2} \partial_r \big( r^2 [\eht{P u_r}] \big) + \eht{P d} -\eht{\tau_{ji} \partial_i u_j}  + \eht{\rho u_{r} g_r} \\
\eht{\rho}\fht{D}_t \fht{\epsilon}_K = & -\frac{1}{r^2}\partial_r [r^2 \big(\eht{\rho u''_r \epsilon''_K}\big)] -\cancelto{0}{\frac{1}{r^2}\partial_r [r^2 (\eht{u_j} \ \eht{\tau_{jr}})]} - \frac{1}{r^2}\partial_r [r^2 \eht{(u'_j \tau'_{jr})}] - \frac{1}{r^2} \partial_r \big( r^2 [\eht{P} \eht{u_r}] \big) - \frac{1}{r^2} \partial_r \big( r^2 [\eht{P' u'_r}] \big) + \eht{P} \ \eht{d} + \eht{P'd'} -\varepsilon_k  - \eht{\rho} \eht{u''_{r}} \fht{g}_r + \eht{\rho} \ \eht{u}_r \fht{g}_r \\ 
\eht{\rho}\fht{D}_t \fht{\epsilon}_K = & -\frac{1}{r^2}\partial_r [r^2 \big(\eht{\rho u''_r \epsilon''_K}\big)] - \frac{1}{r^2}\partial_r [r^2 \eht{(u'_j \tau'_{jr})}] - \frac{1}{r^2} \partial_r \big( r^2 [\eht{P} \eht{u_r}] \big) - \frac{1}{r^2} \partial_r \big( r^2 [\eht{P' u'_r}] \big) + \eht{P} \ \eht{d} + W_P -\varepsilon_k  + W_b + \eht{\rho} \ \eht{u}_r \fht{g}_r \\
\eht{\rho}\fht{D}_t \fht{\epsilon}_K = & -\nabla_r \eht{\rho u''_r \epsilon''_K} - \nabla_r f_\tau - (\eht{P}\nabla_r \eht{u}_r + \eht{u}_r \partial_r \eht{P}) - \nabla_r f_P + \eht{P} \ \eht{d} + W_P -\varepsilon_k  + W_b + \eht{\rho} \ \eht{u}_r \fht{g}_r \\
\eht{\rho}\fht{D}_t \fht{\epsilon}_K = & -\nabla_r \eht{\rho u''_r \epsilon''_K} - \nabla_r f_\tau - \eht{P}\ \eht{d} - \eht{\rho} \ \eht{u}_r \fht{g}_r - \nabla_r f_P + \eht{P} \ \eht{d} + W_P -\varepsilon_k  + W_b + \eht{\rho} \ \eht{u}_r \fht{g}_r \\
{\color{red} \eht{\rho}\fht{D}_t \fht{\epsilon}_K = } & {\color{red} -\nabla_r \eht{\rho u''_r \epsilon''_K} - \nabla_r (f_\tau + f_P ) + W_P -\varepsilon_k  + W_b }
\end{align}

Second way:

\begin{align}
\eht{\rho}\fht{D}_t \fht{\epsilon}_K = & +\eht{\rho}\fht{D}_t \fht{u}_i \fht{u}_i + \eht{\rho}\fht{D}_t \fht{u''_i u''_i} \\
\eht{\rho}\fht{D}_t \fht{\epsilon}_K = & +\eht{\rho}\fht{D}_t \fht{u}_i \fht{u}_i + \eht{\rho}\fht{D}_t \fht{k} \\
{\color{red} \eht{\rho}\fht{D}_t \fht{\epsilon}_K = }& {\color{red} -\nabla_r  ( f_k +  f_P + f_\tau) - \fht{R}_{ir}\partial_r \fht{u}_i + W_b + W_P -\varepsilon_k +\av{\rho}\fav{D}_t (\fav{u}_i \fav{u}_i / 2) }
\end{align}

where equation for the $\fht{k}$ is derived later.

%first parameter is font size and the other line spacing
\fontsize{12pt}{20pt}

\subsection{Mean total energy equation}

%first parameter is font size and the other line spacing
\fontsize{9pt}{20pt}

\begin{align}
\eht{\rho}\fht{D}_t \fht{\epsilon}_t = & + \eht{\rho}\fht{D}_t \fht{\epsilon}_I + \eht{\rho}\fht{D}_t \fht{\epsilon}_K \\
{\color{red} \av{\rho} \fav{D}_t \fav{\epsilon}_t =} & {\color{red} -\nabla_r ( f_I + f_T + f_k + f_P + f_\tau)  - \av{P} \ \av{d} - \fht{R}_{ir}\partial_r \fht{u}_i + W_b + {\mathcal S} + \av{\rho}\fav{D}_t (\fav{u}_i \fav{u}_i / 2)} 
\end{align}

%first parameter is font size and the other line spacing
\fontsize{12pt}{20pt}

\subsection{Mean pressure equation}
\label{sect:mean-pressure-eq}

We begin by deriving 3D instantaneous pressure equation and then apply "ensemble'' (space-time) averaging (Sect.\ref{sect:intro}):

%first parameter is font size and the other line spacing
\fontsize{9pt}{20pt}

\begin{align}
dP = & \frac{\partial P}{\partial \rho}\big|_{\epsilon_I} d\rho + \frac{\partial P}{\partial \epsilon_I}\big|_\rho d\epsilon_I = \frac{P}{\rho}(1-\Gamma_3+\Gamma_1)d\rho + \rho(\Gamma_3 -1)d\epsilon_I \\
D_t P = & + \frac{P}{\rho}(1-\Gamma_3+\Gamma_1) D_t \rho + (\Gamma_3 -1)\rho D_t\epsilon_I \\
D_t P = & -(1-\Gamma_3+\Gamma_1)Pd + (\Gamma_3 -1)(-Pd + {\mathcal S} + \nabla \cdot F_T + \tau_{ij}\partial_i u_j) \\
\partial_t P =  & - u_n \partial_n P -(1-\Gamma_3+\Gamma_1)Pd + (\Gamma_3 -1)(-Pd + {\mathcal S} + \nabla \cdot F_T + \tau_{ij}\partial_i u_j) \\
\partial_t P = & -\left( \frac{1}{r^{2}}\partial_{r}\big(r^{2}[P u_{r}~]\big) + \frac{1}{r\sin{\theta}}\partial_{\theta} \big(\sin{\theta} [P u_{\theta}]\big) + \frac{1}{r\sin{\theta}}\partial_{\phi} [P u_{\phi}] \right) + (1-\Gamma_1)Pd + (\Gamma_3-1)({\mathcal S} + \nabla \cdot F_T + \tau_{ij}\partial_i u_j) \\
\partial_t \eht{P} = & -\left( \eht{\frac{1}{r^{2}}\partial_{r}\big(r^{2}[P u_{r}~]\big)} + \cancelto{0}{\eht{\frac{1}{r\sin{\theta}}\partial_{\theta} \big(\sin{\theta} [P u_{\theta}]\big)}} + \cancelto{0}{\eht{\frac{1}{r\sin{\theta}}\partial_{\phi} [P u_{\phi}]}} \right) + (1-\Gamma_1)\eht{Pd} + (\Gamma_3-1)({\mathcal S} + \eht{\nabla \cdot F_T} + \eht{\tau_{ij}\partial_i u_j}) \\
\partial_t \eht{P} = & -\frac{1}{r^{2}}\partial_{r}\big(r^{2}[\eht{P} \eht{u}_{r}]\big) - \frac{1}{r^{2}}\partial_{r}\big(r^{2}[\eht{P'u'_r}]\big) + (1-\Gamma_1)\eht{P} \ \eht{d} + (1-\Gamma_1)\eht{P'd'} + (\Gamma_3-1)({\mathcal S} + \frac{1}{r^{2}}\partial_{r}\big(r^{2} \eht{\chi \partial_r T} \big) + \cancelto{0}{\eht{\tau_{ij}} \partial_i \eht{{u'_j}}} + \eht{\tau'_{ij}\partial_i u'_j}) \\
\partial_t \eht{P} = & -\nabla_r \eht{P} \eht{u}_{r} - \nabla_r f_P + (1-\Gamma_1)\eht{P} \ \eht{d} + (1-\Gamma_1)W_P + (\Gamma_3-1)({\mathcal S} + \nabla_r f_T + \varepsilon_k) \\
\partial_t \eht{P} = & -\eht{u}_r \partial_r \eht{P}- \eht{P} \ \eht{d} - \nabla_r f_P + (1-\Gamma_1)\eht{P} \ \eht{d} + (1-\Gamma_1)W_P + (\Gamma_3-1)(\eht{{\mathcal S}} + \nabla_r f_T + \varepsilon_k) \\
\partial_t \eht{P} + \eht{u}_r \partial_r \eht{P} = & -\nabla_r f_P  -\Gamma_1 \eht{P} \ \eht{d} + (1-\Gamma_1)W_P + (\Gamma_3-1)({\mathcal S} + \nabla_r f_T + \varepsilon_k) \\
{\color{red} \eht{D}_t \eht{P} = }& {\color{red} -\nabla_r f_P  -\Gamma_1 \eht{P} \ \eht{d} + (1-\Gamma_1)W_P + (\Gamma_3-1)({\mathcal S} + \nabla_r f_T + \varepsilon_k)}
\end{align}

%first parameter is font size and the other line spacing
\fontsize{12pt}{20pt}

\subsection{Mean enthalpy equation}

We start from the total energy equation, where we can subsitute $\rho \epsilon_t = \rho h + \rho \epsilon_k - P$ (for clarity reasons we use here more compact vector notation):

%first parameter is font size and the other line spacing
\fontsize{9pt}{20pt}

\begin{align}
\partial_t \epsilon_t + \vec{\nabla} \cdot \big( (\rho \epsilon_t + P) \vec{u} \big) = \rho \vec{u} \cdot \vec{g} + \vec{\nabla} \cdot F_T + {\mathcal S}  &&  \\
\partial_t (\rho h + \rho \epsilon_k - P) + \vec{\nabla} \cdot (\rho h \vec{u} + \rho \epsilon_k \vec{u} ) = \rho \vec{u} \cdot \vec{g} + \vec{\nabla} \cdot F_T + {\mathcal S} & & \\ 
\partial_t \rho h + {\color{brown}\partial_t \rho \epsilon_k} - {\color{blue}\partial_t P} = - \vec{\nabla} \cdot (\rho h \vec{u} + \rho \epsilon_k \vec{u}) + \rho \vec{u} \cdot \vec{g} + \vec{\nabla} \cdot F_T + {\mathcal S} & & \\
\partial_t \rho h + {\color{brown}\left[-\vec{\nabla} \cdot (\rho \epsilon_k \vec{u}) - \vec{\nabla} \cdot (P \vec{u}) + P (\vec{\nabla} \cdot \vec{u}) + \rho \vec{u} \cdot \vec{g} + \nabla_i u_j \tau_{ji} \right]} - {\color{blue}\left[-\vec{\nabla} \cdot (P \vec{u}) + (1-\Gamma_1)P \vec{\nabla} \cdot \vec{u} + (\Gamma_3 - 1)({\mathcal S} + \vec{\nabla} \cdot F_T + \tau_{ij}\partial_j u_i) \right]} = && \\
= - \vec{\nabla} \cdot (\rho h \vec{u} + \rho \epsilon_k \vec{u}) + \rho \vec{u} \cdot \vec{g} + \vec{\nabla} \cdot F_T + {\mathcal S} &&   \\
\partial_t \rho h + {\color{brown}\left[-\cancel{\vec{\nabla} \cdot (\rho \epsilon_k \vec{u})} - \cancel{\vec{\nabla} \cdot (P \vec{u})} + \cancel{P (\vec{\nabla} \cdot \vec{u})} + \cancel{\rho \vec{u} \cdot \vec{g}} + \nabla_i u_j \tau_{ji} \right]} + {\color{blue}\cancel{\vec{\nabla} \cdot (P \vec{u})} - \cancel{P\vec{\nabla} \cdot \vec{u}} +\Gamma_1 P \vec{\nabla} \cdot \vec{u} - \Gamma_3 ({\mathcal S} + \vec{\nabla} \cdot F_T) + (\cancel{{\mathcal S}} + \cancel{\vec{\nabla} \cdot F_T})} - = && \\
{\color{blue} - (\Gamma_3 -1) \tau_{ij}\partial_j u_i} = - \vec{\nabla} \cdot (\rho h \vec{u} + \cancel{\rho \epsilon_k \vec{u}}) + \cancel{\rho \vec{u} \cdot \vec{g}} + \cancel{\vec{\nabla} \cdot F_T} + \cancel{{\mathcal S}} &&   
\end{align}

So, from the above we have:

\begin{align}
\partial_t \rho h + & \nabla_i u_j \tau_{ji} + \Gamma_1 P \vec{\nabla} \cdot \vec{u} - \Gamma_3 {\mathcal S} - \Gamma_3 \vec{\nabla} \cdot F_T - (\Gamma_3 -1) \tau_{ij}\partial_j u_i = -\vec{\nabla} \cdot (\rho h \vec{u}) & \\
\rho D_t h = & -\Gamma_1 P \vec{\nabla} \cdot \vec{u} + \Gamma_3 {\mathcal S} + \Gamma_3 \vec{\nabla} \cdot F_T - \nabla_i u_j \tau_{ji} +  (\Gamma_3 -1) \tau_{ij}\partial_j u_i \\
\eht{\rho}\fht{D}_t \fht{h} = & -\nabla_r f_h - \Gamma_1\eht{P} \ \eht{d} - \Gamma_1 W_P + \Gamma_3 {\mathcal S} + \Gamma_3 \nabla_r f_T - \nabla_r \eht{u_j \tau_{jr}} +  (\Gamma_3 -1) \eht{\tau_{ij}\partial_j u_i} \\
%\eht{\rho}\fht{D}_t \fht{h} = & -\nabla_r f_h - \Gamma_1\eht{P} \ \eht{d} - \Gamma_1 W_P + \Gamma_3 {\mathcal S} + \Gamma_3 \nabla_r f_T - \cancelto{0}{\nabla_r \eht{u_j} \ \eht{\tau_{jr}}} - \nabla_r \eht{u'_j \tau'_{jr}} +  \cancelto{0}{(\Gamma_3 -1) \eht{\tau_{ij}} \ \eht{\partial_j u_i}} + (\Gamma_3 -1) \eht{\tau'_{ij} \partial_j u'_i} \\
{\color{red} \eht{\rho}\fht{D}_t \fht{h} =} & {\color{red}-\nabla_r (f_h + f_\tau) - \Gamma_1\eht{P} \ \eht{d} - \Gamma_1 W_P + \Gamma_3 {\mathcal S} + \Gamma_3 \nabla_r f_T +  (\Gamma_3 -1)\varepsilon_k}  
%{\color{red} \eht{\rho}\fht{D}_t \fht{h} = }& {\color{red}-\nabla_r f_h - \Gamma_1\eht{P} \ \eht{d} - \Gamma_1 W_P + \Gamma_3 {\mathcal S} + \Gamma_3 \nabla_r f_T + {\mathcal N_h}}
\end{align}

%first parameter is font size and the other line spacing
\fontsize{12pt}{20pt}

%Corresponding mean enthlapy equation is:

\subsection{Mean temperature equation}

We begin by deriving 3D instantaneous temperature equation and then apply "ensemble'' (space-time) averaging (Sect.\ref{sect:intro}):

%first parameter is font size and the other line spacing
\fontsize{9pt}{20pt}

\begin{align}
dT = & \frac{\partial T}{\partial \rho}\big|_{\epsilon_I} d\rho + \frac{\partial T}{\partial \epsilon_I}\big|_\rho d\epsilon_I = \left( \frac{T}{\rho}(\Gamma_3 -1) - \frac{P}{\rho^2}\frac{1}{c_v} \right) d\rho + \frac{1}{c_v} d\epsilon_I \\
D_t T = & +\frac{T}{\rho}(\Gamma_3 -1) D_t\rho - \frac{P}{\rho^2}\frac{1}{c_v} D_t \rho + \frac{1}{c_v} D_t \epsilon_I \\
D_t T = & -\frac{T}{\rho}(\Gamma_3 -1)(\rho d) + \frac{P}{\rho^2}\frac{1}{c_v}(\rho d) + \frac{1}{c_v}\left(-\frac{Pd}{\rho} + \frac{\nabla \cdot F_T}{\rho} + \frac{\tau_{ij}\partial_j u_i}{\rho} + \frac{{\mathcal S}}{\rho} \right) \\
D_t T = & -(\Gamma_3 -1)Td + \frac{\nabla \cdot F_T}{c_v \rho} + \frac{\tau_{ij}\partial_j u_i}{c_v \rho} + \frac{{\mathcal S}}{c_v \rho} \\
\partial_t T = & -u_n \partial_n T  -(\Gamma_3 -1)Td + \frac{\nabla \cdot F_T}{c_v \rho} + \frac{\tau_{ij}\partial_j u_i}{c_v \rho} + \frac{{\mathcal S}}{c_v \rho} \\
\partial_t T = & -\left( \frac{1}{r^{2}}\partial_{r}\big(r^{2}[T u_{r}~]\big) + \frac{1}{r\sin{\theta}}\partial_{\theta} \big(\sin{\theta} [T u_{\theta}]\big) + \frac{1}{r\sin{\theta}}\partial_{\phi} [T u_{\phi}] \right) + Td  -(\Gamma_3 -1)Td + \frac{\nabla \cdot F_T}{c_v \rho} + \frac{\tau_{ij}\partial_j u_i}{c_v \rho} + \frac{{\mathcal S}}{c_v \rho} \\
\partial_t \eht{T} = & -\left( \eht{\frac{1}{r^{2}}\partial_{r}\big(r^{2}[T u_{r}~]\big)} + \cancelto{0}{\eht{\frac{1}{r\sin{\theta}}\partial_{\theta} \big(\sin{\theta} [T u_{\theta}]\big)}} + \cancelto{0}{\eht{\frac{1}{r\sin{\theta}}\partial_{\phi} [T u_{\phi}]}} \right) + \eht{Td}  -(\Gamma_3 -1)\eht{Td} + \eht{\frac{\nabla \cdot F_T}{c_v \rho}} + \eht{\frac{\tau_{ij}\partial_j u_i}{c_v \rho}} + \eht{\frac{{\mathcal S}}{c_v \rho}} \\
\partial_t \eht{T} = & -\nabla_r \eht{T}\eht{u}_r -\nabla_r f_T + \eht{Td}  -\Gamma_3 \eht{Td} + \eht{Td} + \eht{(\nabla \cdot F_T)/(c_v \rho)} + \eht{(\tau_{ij}\partial_j u_i)/(c_v \rho)} + \eht{{\mathcal S}/(c_v \rho)} \\
\partial_t \eht{T} + \eht{u}_r \partial_r \eht{T} = & -\eht{T}\ \eht{d} -\nabla_r f_T + \eht{Td}  -\Gamma_3 \eht{Td} + \eht{Td} + \eht{(\nabla \cdot F_T)/(c_v \rho)} + \eht{(\tau_{ij}\partial_j u_i)/(c_v \rho)} + \eht{{\mathcal S}/(c_v \rho)} \\
\eht{D}_t \eht{T} = & -\eht{T}\ \eht{d} -\nabla_r f_T + \eht{Td}  -\Gamma_3 \eht{Td} + \eht{Td} + \eht{(\nabla \cdot F_T)/(c_v \rho)} + \eht{(\tau_{ij}\partial_j u_i)/(c_v \rho)} + \eht{{\mathcal S}/(c_v \rho)} \\
\eht{D}_t \eht{T} = & -\eht{T}\ \eht{d} -\nabla_r f_T + \eht{Td}  -\Gamma_3 \eht{Td} + \eht{T} \ \eht{d} + \eht{T'd'} + \eht{(\nabla \cdot F_T)/(c_v \rho)} + \eht{(\tau_{ij}\partial_j u_i)/(c_v \rho)} + \eht{{\mathcal S}/(c_v \rho)} \\
\eht{D}_t \eht{T} = & -\nabla_r f_T + \eht{Td}  -\Gamma_3 \eht{Td} + \eht{T'd'} + \eht{(\nabla \cdot F_T)/(c_v \rho)} + \eht{(\tau_{ij}\partial_j u_i)/(c_v \rho)} + \eht{{\mathcal S}/(c_v \rho)} \\
\eht{D}_t \eht{T} = & -\nabla_r f_T + \eht{T} \ \eht{d} + \eht{T'd'} -\Gamma_3 (\eht{T} \ \eht{d} + \eht{T'd'}) + \eht{T'd'} + \eht{(\nabla \cdot F_T)/(c_v \rho)} + \eht{(\tau_{ij}\partial_j u_i)/(c_v \rho)} + \eht{{\mathcal S}/(c_v \rho)}\\
{\color{red} \eht{D}_t \eht{T} =} & {\color{red}-\nabla_r f_T + (1-\Gamma_3)\eht{T} \ \eht{d} + (2-\Gamma_3)\eht{T'd'} + \eht{(\nabla \cdot F_T)/(c_v \rho)} + \eht{(\tau_{ij}\partial_j u_i)/(c_v \rho)} + \eht{{\mathcal S}/(c_v \rho)}}
\end{align}

%first parameter is font size and the other line spacing
\fontsize{12pt}{20pt}

\subsection{Mean angular momentum equation (z-component)}

Z component of the specific angular momentum is defined as  $j_z = r \sin{\theta} u_\phi$. We begin by multiplying the instantaneous 3D azimutal momentum equation by $r \sin{\theta}$, neglect viscosity and $\phi$ component of gravity and obtain (Sect.\ref{sect:intro}):

%first parameter is font size and the other line spacing
\fontsize{9pt}{20pt}

\begin{align}
r\sin{\theta} \partial_{t} \big(\rho u_{\phi}\big) = & -r\sin{\theta}\left( \frac{1}{r^{2}} \partial_{r} \big(r^{2} [\rho u_{\phi} u_r - \cancelto{0}{\tau_{\phi r}}]\big) + \frac{1}{r\sin{\theta}}\partial_{\theta}\big(\sin{\theta}[\rho u_{\theta}u_\phi - \cancelto{0}{\tau_{\theta \phi}}]\big) + \frac{1}{r\sin{\theta}}\partial_{\phi}\big([\rho u_{\phi}^2 - \cancelto{0}{\tau_{\phi \phi}}]\big) + G_\phi^M + \frac{1}{r\sin{\theta}} \partial_{\phi} P \right) - \nonumber \\
- \cancelto{0}{r\sin{\theta} \rho \frac{1}{r \sin{\theta}} \partial_{\phi} \Phi} \\
r\sin{\theta} \partial_{t} \big(\rho u_{\phi}\big) = & -r\sin{\theta}\left( \frac{1}{r^{2}} \partial_{r} \big(r^{2} [\rho u_{\phi} u_r]\big) + \frac{1}{r\sin{\theta}}\partial_{\theta}\big(\sin{\theta}[\rho u_{\theta}u_\phi]\big) + \frac{1}{r\sin{\theta}}\partial_{\phi}\big([\rho u_{\phi}^2]\big) + G_\phi^M + \frac{1}{r\sin{\theta}} \partial_{\phi} P \right) \\
\partial_t \rho j_z = & -\frac{1}{r^2}\partial_r (r^2 \rho j_z u_r) - \frac{1}{r\sin{\theta}}\partial_\theta (\sin{\theta}\rho j_z u_\theta) - \frac{1}{r\sin{\theta}}\partial_\phi (\rho j_z u_\phi) - \partial_\phi P \\
\partial_t \eht{\rho j_z} = & -\eht{\frac{1}{r^2}\partial_r (r^2 \rho j_z u_r)} - \cancelto{0}{\eht{\frac{1}{r\sin{\theta}}\partial_\theta (\sin{\theta}\rho j_z u_\theta)}} - \cancelto{0}{\eht{\frac{1}{r\sin{\theta}}\partial_\phi (\rho j_z u_\phi)}} - \cancelto{0}{\eht{\partial_\phi P}}\\
\partial_t \eht{\rho j_z} = & -\frac{1}{r^2}\partial_r (r^2 \eht{\rho j_z u_r})  \\
\partial_t \eht{\rho} \fht{j_z} = & -\frac{1}{r^2}\partial_r (r^2 \eht{\rho} \fht{j_z} \fht{u_r}) -\frac{1}{r^2}\partial_r (r^2 \eht{\rho} \fht{j''_z u''_r})  \\
\partial_t \eht{\rho} \fht{j_z} + \frac{1}{r^2}\partial_r (r^2 \eht{\rho} \fht{j_z} \fht{u_r}) = & -\frac{1}{r^2}\partial_r (r^2 \eht{\rho} \fht{j''_z u''_r})  \\
{\color{red} \eht{\rho}\fht{D}_t \fht{j_z} =} & {\color{red} -\nabla_r f_{jz}} 
\end{align}

%first parameter is font size and the other line spacing
\fontsize{12pt}{20pt}

%\subsection{Mean composition equations}

\subsection{Mean $\alpha$ equation}

We begin by 3D instantaneous composition equation and then apply "ensemble'' (space-time) averaging (Sect.\ref{sect:intro}):

%first parameter is font size and the other line spacing
\fontsize{9pt}{20pt}

\begin{align}
\partial_{t} \big(\rho X_{\alpha}\big) = & -\left( \frac{1}{r^{2}} \partial_{r}(r^{2} [\rho u_{r} X_{\alpha} ] \big) + \frac{1}{r\sin{\theta}}\partial_{\theta}\big(\sin{\theta} [\rho u_{\theta} X_{\alpha} ]\big) + \frac{1}{r \sin{\theta}} \partial_{\phi} [\rho u_{\phi} X_{\alpha}] \right) + \rho \dot{X}_{\alpha}^{\rm nuc} \ \ \ \ \ \ \ \ \ \ \ \  \alpha = 1 ... N_{nuc} \\
\partial_{t} \eht{\rho X_{\alpha}} = & -\left( \eht{\frac{1}{r^{2}} \partial_{r}(r^{2} [\rho u_{r} X_{\alpha} ] \big)} + \cancelto{0}{\eht{\frac{1}{r\sin{\theta}}\partial_{\theta}\big(\sin{\theta} [\rho u_{\theta} X_{\alpha} ]\big)}} + \cancelto{0}{\eht{\frac{1}{r \sin{\theta}} \partial_{\phi} [\rho u_{\phi} X_{\alpha}]}} \right) + \eht{\rho \dot{X}_{\alpha}^{\rm nuc}} \\
%\partial_{t} \eht{\rho X_{\alpha}} = & -\frac{1}{r^{2}} \partial_{r}(r^{2} [\eht{\rho u_{r} X_{\alpha}} ] \big) + \eht{\rho \dot{X}_{\alpha}^{\rm nuc}} \\
%\partial_{t} \eht{\rho} \fht{X}_{\alpha} = & -\frac{1}{r^{2}} \partial_{r}(r^{2} [\eht{\rho} \fht{u_{r} X_{\alpha}} ] \big) + \eht{\rho \dot{X}_{\alpha}^{\rm nuc}} \\
\partial_{t} \eht{\rho} \fht{X}_{\alpha} = & -\frac{1}{r^{2}} \partial_{r} ( r^{2} [\eht{\rho} \fht{u}_{r} \fht{X}_{\alpha} + \eht{\rho}\fht{u''_r X''_\alpha} ] ) + \eht{\rho} \fht{\dot{X}}_{\alpha}^{\rm nuc} \\
\partial_{t} \eht{\rho} \fht{X}_{\alpha} +\frac{1}{r^{2}} \partial_{r} ( r^{2} [\eht{\rho} \fht{u}_{r} \fht{X}_{\alpha}] ) = & -\frac{1}{r^{2}} \partial_{r} (\eht{\rho}\fht{u''_r X''_\alpha}) + \eht{\rho} \fht{\dot{X}}_{\alpha}^{\rm nuc} \\
{\color{red} \eht{\rho}\fht{D}_t \fht{X}_\alpha = }& {\color{red} -\nabla_r f_\alpha + \eht{\rho} \fht{\dot{X}}_{\alpha}^{\rm nuc}}
\end{align}

%first parameter is font size and the other line spacing
\fontsize{12pt}{20pt}

\newpage

\subsection{Mean number of nucleons per isotope ($A$) equation} 

We begin by deriving 3D instantaneous A equation and then apply "ensemble'' (space-time) averaging (Sect.\ref{sect:intro}):

%first parameter is font size and the other line spacing
\fontsize{9pt}{20pt}

\begin{align}
A = & +\left(\sum_\alpha \frac{X_\alpha}{A_\alpha}  \right)^{-1} \\
D_t A = & +D_t \left(\sum_\alpha \frac{X_\alpha}{A_\alpha}  \right)^{-1} = +D_t \frac{1}{\sum_\alpha (X_\alpha / A_\alpha)} = -\frac{D_t \sum_\alpha (X_\alpha/A_\alpha)}{[\sum_\alpha (X_\alpha/A_\alpha)]^2} = -A^2 D_t \sum_\alpha \frac{X_\alpha}{A_\alpha} \\
D_t A = & -A^2 D_t \sum_\alpha \frac{X_\alpha}{A_\alpha} = -A^2 \sum_\alpha \frac{A_\alpha D_t X_\alpha - X_\alpha D_t A_\alpha}{A^2_\alpha} = -A^2 \sum_\alpha \frac{A_\alpha D_t X_\alpha}{A^2_\alpha} \\
D_t A = & -A^2 \sum_\alpha \frac{A_\alpha D_t X_\alpha}{A^2_\alpha} = -A^2 \sum_\alpha \frac{A_\alpha \dot{X}_\alpha^{\rm nuc}}{A^2_\alpha} = -A^2 \sum_\alpha \frac{\dot{X}_\alpha^{\rm nuc}}{A_\alpha} \\
\rho D_t A = & -\rho A^2 \sum_\alpha \frac{\dot{X}_\alpha^{\rm nuc}}{A_\alpha} \\
\partial_t \rho A = & -\left( \frac{1}{r^{2}} \partial_{r}(r^{2} [\rho u_{r} A ] \big) + \frac{1}{r\sin{\theta}}\partial_{\theta}\big(\sin{\theta} [\rho u_{\theta} A ]\big) + \frac{1}{r \sin{\theta}} \partial_{\phi} [\rho u_{\phi} A] \right) -\rho A^2 \sum_\alpha \frac{\dot{X}_\alpha^{\rm nuc}}{A_\alpha} \\
\partial_t \overline{\rho A} = & -\left( \eht{\frac{1}{r^{2}} \partial_{r}(r^{2} [\rho u_{r} A ] \big)} + \cancelto{0}{\eht{\frac{1}{r\sin{\theta}}\partial_{\theta}\big(\sin{\theta} [\rho u_{\theta} A ]\big)}} + \cancelto{0}{\eht{\frac{1}{r \sin{\theta}} \partial_{\phi} [\rho u_{\phi} A]}} \right) - \eht{\rho A^2 \sum_\alpha \frac{\dot{X}_\alpha^{\rm nuc}}{A_\alpha}} \\
\partial_t \overline{\rho A} = & -\frac{1}{r^{2}} \partial_{r}(r^{2} [\eht{\rho u_{r} A }] \big) - \eht{\rho A^2 \sum_\alpha \frac{\dot{X}_\alpha^{\rm nuc}}{A_\alpha}} \\
\partial_t \overline{\rho} \fht{A} = & -\frac{1}{r^{2}} \partial_{r}(r^{2} [\eht{\rho} \fht{u_{r} A}] \big) - \eht{\rho A^2 \sum_\alpha \frac{\dot{X}_\alpha^{\rm nuc}}{A_\alpha}} \\
\partial_t \overline{\rho} \fht{A} = & -\frac{1}{r^{2}} \partial_{r}(r^{2} [\eht{\rho} \fht{u}_{r} \fht{A}])  -\frac{1}{r^{2}} \partial_{r}(r^{2} [\eht{\rho} \fht{u''_r A''}]) - \eht{\rho A^2 \sum_\alpha \frac{\dot{X}_\alpha^{\rm nuc}}{A_\alpha}} \\
\partial_t \overline{\rho} \fht{A} + \frac{1}{r^{2}} \partial_{r}(r^{2} [\eht{\rho} \fht{u}_{r} \fht{A}]) = &   -\frac{1}{r^{2}} \partial_{r}(r^{2} [\eht{\rho} \fht{u''_r A''}]) - \eht{\rho A^2 \sum_\alpha \frac{\dot{X}_\alpha^{\rm nuc}}{A_\alpha}} \\
{\color{red} \rho \fht{D}_t\fht{A} =} & {\color{red} -\nabla_r f_A - \eht{\rho A^2 \sum_\alpha \frac{\dot{X}_\alpha^{\rm nuc}}{A_\alpha}}}
\end{align}

%first parameter is font size and the other line spacing
\fontsize{12pt}{20pt}

\newpage

\subsection{Mean charge per isotope ($Z$) equation}

We begin by deriving 3D instantaneous Z equation and then apply "ensemble'' (space-time) averaging (Sect.\ref{sect:intro}):

%first parameter is font size and the other line spacing
\fontsize{9pt}{20pt}

\begin{align}
Z = & + A \sum_\alpha \frac{Z_\alpha A_\alpha}{A_\alpha} \\
D_t Z = & +D_t \left(A \sum_i \frac{Z_\alpha A_\alpha}{A_\alpha}\right) = \sum_\alpha \frac{Z_\alpha X_\alpha}{A_\alpha} D_t A + A\sum_\alpha D_t \frac{Z_\alpha X_\alpha}{A_\alpha} \\
D_t Z = & -\sum_\alpha \frac{Z_\alpha X_\alpha}{A_\alpha}  A^2 \sum_\alpha \frac{\dot{X}_\alpha^{\rm nuc}}{A_\alpha} + A\sum_\alpha \frac{A_\alpha D_t Z_\alpha X_\alpha - Z_\alpha X_\alpha D_t A_\alpha}{A_\alpha^2} \\
D_t Z = & -Z A \sum_\alpha \frac{\dot{X}_\alpha^{\rm nuc}}{A_\alpha} + A \sum_\alpha \frac{A_\alpha D_t Z_\alpha X_\alpha}{A_\alpha^2} \\
D_t Z = & -Z A \sum_\alpha \frac{\dot{X}_\alpha^{\rm nuc}}{A_\alpha} + A\sum_\alpha \frac{Z_\alpha D_t X_\alpha + X_\alpha D_t Z_\alpha}{A^2_\alpha} \\
D_t Z = & -Z A \sum_\alpha \frac{\dot{X}_\alpha^{\rm nuc}}{A_\alpha} + A\sum_\alpha \frac{Z_\alpha \dot{X}_\alpha^{\rm nuc}}{A_\alpha} \\
\rho D_t Z = & -\rho Z A \sum_\alpha \frac{\dot{X}_\alpha^{\rm nuc}}{A_\alpha} + \rho A \sum_\alpha \frac{Z_\alpha \dot{X}_\alpha^{\rm nuc}}{A_\alpha} \\
\partial_t \overline{\rho Z} = & -\left( \eht{\frac{1}{r^{2}} \partial_{r}(r^{2} [\rho u_{r} Z ] \big)} + \cancelto{0}{\eht{\frac{1}{r\sin{\theta}}\partial_{\theta}\big(\sin{\theta} [\rho u_{\theta} Z ]\big)}} + \cancelto{0}{\eht{\frac{1}{r \sin{\theta}} \partial_{\phi} [\rho u_{\phi} Z]}} \right) -\eht{\rho Z A \sum_\alpha \frac{\dot{X}_\alpha^{\rm nuc}}{A_\alpha}} + \eht{\rho A \sum_\alpha \frac{Z_\alpha \dot{X}_\alpha^{\rm nuc}}{A_\alpha}}\\
\partial_t \overline{\rho Z} = & -\frac{1}{r^{2}} \partial_{r}(r^{2} [\eht{\rho u_{r} Z} ] \big) -\eht{\rho Z A \sum_\alpha \frac{\dot{X}_\alpha^{\rm nuc}}{A_\alpha}} + \eht{\rho A \sum_\alpha \frac{Z_\alpha \dot{X}_\alpha^{\rm nuc}}{A_\alpha}} \\
\partial_t \overline{\rho} \fht{Z} = & -\frac{1}{r^{2}} \partial_{r}(r^{2} [\eht{\rho} \fht{u_{r} Z}]) -\eht{\rho Z A \sum_\alpha \frac{\dot{X}_\alpha^{\rm nuc}}{A_\alpha}} + \eht{\rho A \sum_\alpha \frac{Z_\alpha \dot{X}_\alpha^{\rm nuc}}{A_\alpha}} \\
\partial_t \overline{\rho} \fht{Z} = & -\frac{1}{r^{2}} \partial_{r}(r^{2} [\eht{\rho} \fht{u_{r}} \fht{Z}]) -\frac{1}{r^{2}} \partial_{r}(r^{2} [\eht{\rho} \fht{u''_{r} Z''}]) -\eht{\rho Z A \sum_\alpha \frac{\dot{X}_\alpha^{\rm nuc}}{A_\alpha}} + \eht{\rho A \sum_\alpha \frac{Z_\alpha \dot{X}_\alpha^{\rm nuc}}{A_\alpha}} \\
\partial_t \overline{\rho} \fht{Z}  +\frac{1}{r^{2}} \partial_{r}(r^{2} [\eht{\rho} \fht{u_{r}} \fht{Z}]) = & -\frac{1}{r^{2}} \partial_{r}(r^{2} [\eht{\rho} \fht{u''_{r} Z''}]) -\eht{\rho Z A \sum_\alpha \frac{\dot{X}_\alpha^{\rm nuc}}{A_\alpha}} + \eht{\rho A \sum_\alpha \frac{Z_\alpha \dot{X}_\alpha^{\rm nuc}}{A_\alpha}} \\
{\color{red} \eht{\rho}\fht{D}_t \fht{Z} =} & {\color{red} -\nabla_r f_Z -\eht{\rho Z A \sum_\alpha \frac{\dot{X}_\alpha^{\rm nuc}}{A_\alpha}} + \eht{\rho A \sum_\alpha \frac{Z_\alpha \dot{X}_\alpha^{\rm nuc}}{A_\alpha}}}
\end{align}

%first parameter is font size and the other line spacing
\fontsize{12pt}{20pt}

\subsection{Mean entropy equation}

We can derive the mean entropy equation in the following way (Sect.\ref{sect:intro}):

%first parameter is font size and the other line spacing
\fontsize{9pt}{20pt}

\begin{align}
\rho D_t s = & +(\nabla \cdot F_T + {\mathcal S} + \varepsilon_k)/T \\
\partial_t \rho s + \nabla_r (\rho s u_r) =  & +(\nabla \cdot F_T + {\mathcal S} + \varepsilon_k)/T \\
\partial_t \eht{\rho s} + \nabla_r \eht{\rho u_r s} = & +\eht{(\nabla \cdot F_T + {\mathcal S} + \varepsilon_k)/T} \\
\partial_t \eht{\rho}\fht{s} + \nabla_r (\eht{\rho} \fht{u}_r \fht{s}) = & -\nabla_r (\eht{\rho} \fht{s''u''_r}) + \eht{\nabla \cdot F_T / T} + \eht{{\mathcal S}/T} + \eht{{\varepsilon_k}/T} \\
{\color{red} \eht{\rho} \fht{D}_t \fht{s} =} & {\color{red}-\nabla_r f_s + \eht{\nabla \cdot F_T / T} + \eht{{\mathcal S}/T} + \eht{{\varepsilon_k}/T} }
\end{align}

%first parameter is font size and the other line spacing
\fontsize{12pt}{20pt}

\newpage

\section{General formula for second and third order moments and variances}

\subsection{Second-order moments}

In order to calculate evolution equations for correlations of two arbitrary fluctuations, we can derive the following general formula.

%first parameter is font size and the other line spacing
\fontsize{11pt}{20pt}

\begin{align}
\eht{\rho}\fht{D}_t \fht{c''d''} - \eht{\rho D_t c''d''} & = \eht{\rho} \big(\partial_t \fht{c''d''} + \fht{u}_n \partial_n \fht{c''d''} \big) - \eht{\rho \big( \partial_t c''d'' + u_n \partial_n c''d'' \big)} = \eht{\rho} \partial_t \fht{c''d''} + \eht{\rho}\fht{u}_n \partial_n \fht{c''d''} - \eht{\rho \partial_t c''d''} - \eht{\rho u_n \partial_n c''d''} = \nonumber \\
& = \eht{\rho} \partial_t \fht{c''d''} + \eht{\rho} \fht{u}_n \partial_n \fht{c''d''} - \big(\eht{\partial_t \rho c''d''} - \eht{c''d''\partial_t \rho} \big) - \eht{\rho u_n \partial_n c''d''} = \\
& = \eht{\rho} \partial_t \fht{c''d''} + \eht{\rho} \fht{u}_n \partial_n \fht{c''d''} - \partial_t \eht{\rho} \fht{c''d''} - \eht{c''d''\partial_n \rho u_n} - \eht{\rho u_n \partial_n c''d''} = \\
& = \eht{\rho} \partial_t \fht{c''d''} + \eht{\rho} \fht{u}_n \partial_n \fht{c''d''} - \big(\eht{\rho} \partial_t \fht{c''d''} + \fht{c''d''} \partial_t \eht{\rho} \big) - \eht{c''d''\partial_n \rho u_n} - \eht{\rho u_n \partial_n c''d''} = \\
& = \eht{\rho} \partial_t \fht{c''d''} + \eht{\rho} \fht{u}_n \partial_n \fht{c''d''} - \big(\eht{\rho} \partial_t \fht{c''d''} - \fht{c''d''} \partial_n \eht{\rho}\fht{u}_n \big) - \eht{c''d''\partial_n \rho u_n} - \eht{\rho u_n \partial_n c''d''} = \\
& =  \eht{\rho} \partial_t \fht{c''d''} -  \eht{\rho} \partial_t \fht{c''d''} + \eht{\rho} \fht{u}_n \partial_n \fht{c''d''} +  \fht{c''d''} \partial_n \eht{\rho}\fht{u}_n - \eht{\partial_n \rho u_n c''d''} = \\
& = \partial_n \eht{\rho} \fht{u}_n \fht{c''d''} - \big( \eht{\partial_n \rho \fht{u}_n c''d''} + \eht{\partial_n \rho u''_n c''d'' }  \big) = -\eht{\partial_n \rho u''_n c''d''}
\end{align}

\begin{align}
\eht{\rho}\fht{D}_t \fht{c''d''} = \eht{\rho D_t c''d''} -\eht{\partial_n \rho u''_n c''d''} = \eht{c'' \rho D_t d''} + \eht{d'' \rho D_t c''} -\eht{\partial_n \rho u''_n c''d''}
\end{align}

\begin{align}
\rho D_t c'' = \rho D_t c - \rho D_t \fht{c} = \rho D_t c - \rho \fht{D}_t \fht{c} - \rho u''_n \partial_n \fht{c} =  \rho D_t c - \frac{\rho}{\eht{\rho}} \left[ \eht{\rho} \fht{D}_t \fht{c} \right] - \rho u''_n \partial_n \fht{c} \\
\rho D_t d'' = \rho D_t d - \rho D_t \fht{d} = \rho D_t d - \rho \fht{D}_t \fht{d} - \rho u''_n \partial_n \fht{d} =  \rho D_t d - \frac{\rho}{\eht{\rho}} \left[ \eht{\rho} \fht{D}_t \fht{d} \right] - \rho u''_n \partial_n \fht{d}  
\end{align}

\begin{align}
\eht{\rho}\fht{D}_t \fht{c''d''} =  & \ \eht{c''\left(\rho D_t d - \frac{\rho}{\eht{\rho}}\left[\eht{\rho}\fht{D}_t \fht{d} \right] - \rho u''_n\partial_n\fht{d} \right)} + \ \eht{d''\left(\rho D_t c - \frac{\rho}{\eht{\rho}}\left[\eht{\rho}\fht{D}_t \fht{c} \right] - \rho u''_n\partial_n\fht{c} \right)} - \eht{\partial_n {\rho c''d'' u''_n}}
\end{align}

\begin{align}
{\color{red} \eht{\rho}\fht{D}_t \fht{c''d''} = } & {\color{red} +\eht{c''\rho D_t d} - \eht{\rho}\fht{c''u''_n}\partial_n\fht{d} + \eht{d''\rho D_t c} - \eht{\rho}\fht{d''u''_n}\partial_n\fht{c} - \eht{\partial_n{\rho c''d'' u''_n}} }
\label{eq:second-order-moments}
\end{align}

%first parameter is font size and the other line spacing
\fontsize{12pt}{20pt}

\newpage

\subsection{Third-order moments}

In order to calculate evolution equations for correlations of three arbitrary fluctuations, we can derive the following general formula.

%first parameter is font size and the other line spacing
\fontsize{11pt}{20pt}

\begin{align}
\eht{\rho}\fht{D}_t \fht{c''d''e''} - \eht{\rho D_t c''d''e''} & = \ \eht{\rho}\big(\partial_t \fht{c''d''e''} + \fht{u}_n\partial_n \fht{c''d''e''} \big) - \eht{\rho ( \partial_t c''d''e'' + u_n \partial_n c''d''e'' )} = \\
& = \ \eht{\rho}\partial_t \fht{c''d''e''} + \eht{\rho}\fht{u}_n\partial_n \fht{c''d''e''} - \eht{\rho \partial_t c''d''e''} - \eht{\rho u_n \partial_n c''d''e''} = \\
& = \ \eht{\rho}\partial_t \fht{c''d''e''} + \eht{\rho}\fht{u}_n\partial_n \fht{c''d''e''} - (\eht{\partial_t \rho c''d''e''} - \eht{c''d''e''\partial_t \rho}) - \eht{\rho u_n \partial_n c''d''e''} = \\
& = \ \eht{\rho}\partial_t \fht{c''d''e''} + \eht{\rho}\fht{u}_n\partial_n \fht{c''d''e''} - \partial_t \eht{\rho} \fht{c''d''e''} - (\eht{c''d''e''\partial_n \rho u_n} + \eht{\rho u_n \partial_n c''d''e''}) = \\
& = \ \eht{\rho}\partial_t \fht{c''d''e''} + \eht{\rho}\fht{u}_n\partial_n \fht{c''d''e''} - (\eht{\rho}\partial_t \fht{c''d''e''} + \fht{c''d''e''}\partial_t \eht{\rho} ) - \eht{\partial_n c''d''e'' \rho u_n} =  \\
& = \ \eht{\rho}\fht{u}_n\partial_n \fht{c''d''e''} + \fht{c''d''e''}\partial_n \eht{\rho}\fht{u}_n  - \eht{\partial_n c''d''e'' \rho \fht{u}_n} - \eht{\partial_n c''d''e'' \rho u''_n} = \\
& = \ - \eht{\partial_n c''d''e'' \rho u''_n} \\
\end{align}

\begin{align}
\eht{\rho D_t c''d''e''} = & \ \eht{\rho c''d'' D_t e''} + \eht{\rho c''e'' D_t d''} + \eht{\rho d''e'' D_t c''}
\end{align}

\begin{align}
\eht{c''d''\rho D_t e''} & = \ \eht{c''d''\rho D_t e} - \eht{c''d''\rho D_t \fht{e}} = \eht{c''d''\rho D_t e} - \eht{c''d''\rho (\partial_t \fht{e} + u_n \partial_n \fht{e}) } = \\
& = \ \eht{c''d''\rho D_t e} - \eht{c''d''\rho \partial_t \fht{e}} - \eht{c''d''\rho u_n \partial_n \fht{e}} = \eht{c''d''\rho D_t e} - \eht{\rho} \fht{c''d''}\partial_t \fht{e} - (\eht{c''d''\rho \fht{u}_n \partial_n \fht{e}} + \eht{c''d''\rho u''_n \partial_n \fht{e}} ) =  \\
& = \ \eht{c''d''\rho D_t e} - (\eht{\rho}\fht{c''d''}\partial_t \fht{e} + \eht{\rho}\fht{c''d''}\fht{u}_n \partial_n \fht{e}) - \eht{\rho} \fht{c''d''u''_n}\partial_n\fht{e} = \\
& \ = \eht{c''d''\rho D_t e} - \eht{\rho}\fht{c''d''}\fht{D}_t \fht{e} - \eht{\rho}\fht{c''d''u''_n} \partial_n \fht{e} 
\end{align}

\begin{align}
{\color{red} \eht{\rho}\fht{D}_t \fht{c''d''e''}} & = {\color{red} \ \eht{c''d''\rho D_t e} - \eht{\rho}\fht{c''d''}\fht{D}_t \fht{e} - \eht{\rho}\fht{c''d''u''_n} \partial_n \fht{e} +} \\
& {\color{red}\ + \eht{c''e''\rho D_t d} - \eht{\rho}\fht{c''e''}\fht{D}_t \fht{d} - \eht{\rho}\fht{c''e''u''_n} \partial_n \fht{d} +} \\
& {\color{red} \ + \eht{d''e''\rho D_t c} - \eht{\rho}\fht{d''e''}\fht{D}_t \fht{c} - \eht{\rho}\fht{d''e''u''_n} \partial_n \fht{c} -} \\
& {\color{red} \ - \eht{\partial_n c''d''e'' \rho u''_n}}    
\end{align}

%first parameter is font size and the other line spacing
\fontsize{12pt}{20pt}

\subsection{Reynolds and Favrian variance}
\label{sect:reynolds-favrian-variance}

The Reynolds variance can be derived in following way:

%first parameter is font size and the other line spacing
\fontsize{11pt}{20pt}

\begin{align}
\fht{D}_t \eht{c'd'} - \eht{D_t c'd'} & = +\partial_t \eht{c'd'} + \fht{u}_n \partial_n \eht{c'd'} - (\partial_t \eht{c'd'} + \eht{u_n \partial_n c'd'}) = \\
& = +\partial_t \eht{c'd'} + \fht{u}_n \partial_n \eht{c'd'} - \partial_t \eht{c'd'} - \fht{u}_n \partial_n \eht{c'd'} - \eht{u''_n \partial_n c'd'} = \\
& = - \eht{u''_n \partial_n c'd'}
\end{align}

Next step:

\begin{align}
\fht{D}_t \eht{c'd'} = \eht{D_t c'd'} - \eht{u''_n \partial_n c'd'} = \eht{c' D_t d'} + \eht{d' D_t c'} - \eht{u''_n \partial_n c'd'}
\label{eq:reynolds-variance-general}
\end{align}

Next step:

\begin{align}
D_t c' = & D_t c - D_t \eht{c} = D_t c - \fht{D}_t \eht{c} - u''_n \partial_n \eht{c} \\
D_t d' = & D_t d - D_t \eht{c} = D_t d - \fht{D}_t \eht{d} - u''_n \partial_n \eht{d}
\end{align}

Now let's put these equations in the Equation~\ref{eq:reynolds-variance-general}:

\begin{align}
\fht{D}_t \eht{c'd'} & = +\eht{c' D_t d'} + \eht{d' D_t c'} - \eht{u''_n \partial_n c'd'} = \\
& = +\eht{c' (D_t d - \fht{D}_t \eht{d} - u''_n \partial_n \eht{d})} + \eht{d' (D_t c - \fht{D}_t \eht{c} - u''_n \partial_n \eht{c})} - \eht{u''_n \partial_n c'd'} = \\
& = +\eht{c'D_t d} - \eht{c'u''_n}\partial_n \eht{d} + \eht{d' D_t c} - \eht{d'u''_n}\partial_n \eht{c} - \eht{u''_n \partial_n c'd'} = \\
& = +\eht{c'D_t d} - \eht{c'u''_n}\partial_n \eht{d} + \eht{d' D_t c} - \eht{d'u''_n}\partial_n \eht{c} - \partial_n \eht{u''_n c'd'} + \eht{c'd'\partial_n u''_n}
\end{align}

From this general formula by substituting $d = c$ we get:

\begin{align}
{\color{red} \fht{D}_t \eht{c'c'}} & = + 2 \eht{c' D_t c} - 2 \eht{c'u''_n} \partial_n \eht{c} - \eht{u''_n \partial_n c'c'} = \\
& = {\color{red}+2 \eht{c' D_t c} - 2 \eht{c'u''_n} \partial_n \eht{c} - \partial_n \eht{u''_n c' c'} + \eht{c' c' \partial_n u''_n}}
\end{align}

\noindent
The Favrian variance can be easily derived from general equation for second-order moments Equation~\ref{eq:second-order-moments} ie.

\begin{align}
\eht{\rho}\fht{D}_t \fht{c''d''} = & +\eht{c''\rho D_t d} - \eht{\rho}\fht{c''u''_n}\partial_n\fht{d} + \eht{d''\rho D_t c} - \eht{\rho}\fht{d''u''_n}\partial_n\fht{c} - \eht{\partial_n{\rho c''d'' u''_n}}  
\end{align}

Now, substitute $d = c$ and you'll get the equation for Favrian variance:

\begin{align}
{\color{red}\eht{\rho}\fht{D}_t \fht{c''c''} =} & {\color{red}+2 \eht{c''\rho D_t c} - 2\eht{\rho}\fht{c''u''_n}\partial_n\fht{c} - \eht{\partial_n{\rho c''c'' u''_n}}} 
\end{align}

%first parameter is font size and the other line spacing
\fontsize{12pt}{20pt}

\newpage

\section{Derivation of second-order moments equations}

\subsection{Reynolds stress equation}

We can derive the Reynolds stress equation using the general formula for second order moments, where we substitute $c''$ with $u''_i$ and $d''$ with $u''_j$.

\begin{align}
\overline{\rho}\widetilde{D}_t \widetilde{c''d''} = & \ \overline{c'' \rho D_t d} - \overline{\rho} \widetilde{c''u''_n}\partial_n \widetilde{d} + \overline{d'' \rho D_t c} - \overline{\rho} \widetilde{d''u''_n}\partial_n \widetilde{c} - \overline{\partial_n \rho c''d''u''_n} \\
\underbrace{\eht{\rho}\fht{D}_t \fht{u''_i u''_j}}_\text{$\eht{\rho}\fht{D}_t (\fht{R}_{ij} / \eht{\rho})$} = & \ \eht{u''_i \rho D_t u_j} - \underbrace{\eht{\rho} \fht{u''_i u''_n} \partial_n \fht{u}_j}_\text{$\fht{R}_{in}\partial_n \fht{u}_j$} + \eht{u''_j \rho D_t u_i} - \underbrace{\eht{\rho} \fht{u''_j u''_n} \partial_n \fht{u}_i}_\text{$\fht{R}_{jn}\partial_n \fht{u}_i$} - \underbrace{\eht{\partial_n \rho u''_i u''_j u''_n}}_\text{$\eht{\nabla \cdot \rho u''_i u''_j u''_n}$}
\end{align}

\noindent
So, the general formula for Reynolds stress $\fht{R}_{ij}$ is:

\begin{align}
\eht{\rho}\fht{D}_t (\fht{R}_{ij} / \eht{\rho}) = -\left(\fht{R}_{in}\partial_n \fht{u}_j + \fht{R}_{jn}\partial_n \fht{u}_i \right) -\left( \nabla_r \eht{\rho}\fht{u''_i u''_j u''_r} + \eht{G_{rr}^R} \right) + \eht{u''_i \rho D_t u_j} + \eht{u''_j \rho D_t u_i} 
\end{align}

\subsubsection{Mean equation for $\fht{R}_{rr}$}

\begin{align}
\eht{\rho}\fht{D}_t \big( \fht{R}_{rr} / \eht{\rho} \big) = -\left(\fht{R}_{rn}\partial_n \fht{u}_r + \fht{R}_{rn}\partial_n \fht{u}_r \right) -\left( \nabla_r 2 f_k^r + \eht{G_{rr}^R} \right) + 2  \eht{u''_r \rho D_t u_r} 
\end{align}

\noindent
where

\begin{align}
\eht{u''_r \rho D_t u_r} = & \eht{u''_r \left( \frac{1}{r^{2}} \partial_{r} \big( r^{2} [\tau_{rr}]\big) + \frac{1}{r\sin{\theta}}\partial_{\theta}(\sin{\theta}[\tau_{r\theta}]\big) + \frac{1}{r\sin{\theta}}\partial_{\phi}\big([\tau_{r\phi}]\big) - G_r^M - \partial_{r} P + \rho g_r \right) } \\
\eht{u''_r \rho D_t u_r} = & \eht{u''_r \left( \nabla_r \tau_{rr} + \nabla_\theta \tau_{r\theta} + \nabla_\phi \tau_{r\phi}  - G_r^M - \partial_{r} P + \rho g_r \right) }
\end{align}

\noindent
Some terms can be further manipulated in following way:

\begin{align}
+\eht{u''_r  \nabla_r \tau_{rr}} = & \ \eht{u''_r} \nabla_r \eht{\tau_{rr}} + \eht{u''_r \nabla_r \tau'_{rr}} =  \cancelto{0}{\eht{u''_r} \nabla_r \eht{\tau_{rr}}} + \nabla_r (\eht{u''_r \tau'_{rr}}) - \eht{\tau'_{rr}\partial_r u''_r} \\
+\eht{u''_r  \nabla_\theta \tau_{r\theta}} = & \ \cancelto{0}{\eht{u''_r} \nabla_\theta \eht{\tau_{r\theta}}} + \eht{u''_r \nabla_\theta \tau'_{r\theta}} = \cancelto{0}{\eht{\nabla_\theta (u''_r \tau'_{r\theta})}} - \eht{\tau'_{r\theta}\frac{1}{r}\partial_\theta u''_r }  & \\
+\eht{u''_r  \nabla_\phi \tau_{r\phi}} = \ & \cancelto{0}{\eht{u''_r} \nabla_\phi \eht{\tau_{r\phi}}} + \eht{u''_r \nabla_\phi \tau'_{r\phi}} = \cancelto{0}{\eht{\nabla_\phi (u''_r \tau'_{r\phi})}} - \eht{\tau'_{r\phi}\frac{1}{r\sin{\theta}}\partial_\phi u''_r } & \\
-\eht{u''_r G_r^M} = & \ -\eht{u''_r G_r^M} & \\
-\eht{u''_r \partial_r P} = & \ -\eht{u''_r} \partial_r \eht{P} - \eht{u''_r \partial_r P'} = -\eht{u''_r} \partial_r \eht{P} - \nabla_r \eht{(u''_r P')} + \eht{P'\nabla_r u''_r}  & \\ 
+\eht{u''_r \rho g_r} \sim & \ \eht{u''_r \rho} \eht{g}_r = 0 &
\end{align}

\noindent 
Final equation is:

\begin{align}
\eht{\rho}\fht{D}_t \left( \fht{R}_{rr} / \eht{\rho} \right) = & \ -2\fht{R}_{rr}\partial_r \fht{u}_r - \nabla_r \fht{F}_{rrr}^R - \eht{G_{rr}^R}  + 2\nabla_r (\eht{u''_r \tau'_{rr}}) -2\eht{u''_r G_r^M} - 2 \eht{u''_r} \partial_r \eht{P} - 2 \nabla_r \eht{(u''_r P')} + 2 \eht{P'\nabla_r u''_r} - \\
& \ -2 \big( \eht{\tau'_{rr}\partial_r u''_r} + \eht{\tau'_{r\theta}\frac{1}{r}\partial_\theta u''_r } + \eht{\tau'_{r\phi}\frac{1}{r\sin{\theta}}\partial_\phi u''_r }  \big) \nonumber \\
\eht{\rho}\fht{D}_t \left( \fht{R}_{rr} / \eht{\rho} \right) = & \ -\nabla_r ( 2 f_k^r + 2 f_P + 2 f_\tau^r ) + 2 W_b - 2\fht{R}_{rr}\partial_r \fht{u}_r + 2\eht{P'\nabla_r u''_r} - 2\overline{u''_r G^{M}_r} - \eht{G^{R}_{rr}} - 2\varepsilon_k^r
\end{align}

\subsubsection{Mean equation for $\fht{R}_{\theta \theta}$}

\begin{align}
\eht{\rho}\fht{D}_t \big( \fht{R}_{\theta \theta} / \eht{\rho} \big) = -\left(\fht{R}_{\theta n}\partial_n \fht{u}_\theta + \fht{R}_{\theta n}\partial_n \fht{u}_\theta \right) -\left( \nabla_r 2 f_k^\theta - \eht{G_{\theta \theta}^R} \right) + 2  \eht{u''_\theta \rho D_t u_\theta} \\
\end{align}

\noindent
where

\begin{align}
\eht{u''_\theta \rho D_t u_\theta} = & \ \eht{u''_\theta \left( \frac{1}{r^{2}} \partial_{r} \big(r^{2} [\tau_{\theta r}]\big) + \frac{1}{r\sin{\theta}}\partial_{\theta}\big(\sin{\theta}[\tau_{\theta \theta}]\big) + \frac{1}{r\sin{\theta}}\partial_{\phi}[\tau_{\theta \phi}]\big) - G_\theta^M - \frac{1}{r} \partial_{\theta} P + \rho g_\theta \right) } \\
\eht{u''_\theta \rho D_t u_\theta} = & \ \eht{u''_\theta \big( \nabla_r \tau_{\theta r} + \nabla_\theta \tau_{\theta \theta} + \nabla_\phi \tau_{\theta \phi} - G_\theta^M - \frac{1}{r} \partial_{\theta} P + \rho g_\theta \big) }
\end{align}

\noindent
Some terms can be further manipulated in following way:

\begin{align}
+\eht{u''_\theta \nabla_r \tau_{\theta r}} = & \ \cancelto{0}{\eht{u''_\theta}\nabla_r \eht{\tau_{\theta r}}} + \nabla_r (\eht{u''_\theta \tau'_{\theta r}}) - \eht{\tau'_{\theta r}\partial_r u''_\theta}\\
+\eht{u''_\theta \nabla_\theta \tau_{\theta \theta}} = & \ -\eht{\tau'_{\theta \theta} \frac{1}{r}\partial_\theta u''_\theta} \\
+\eht{u''_\theta \nabla_\phi \tau_{\theta \phi}} = & \ -\eht{\tau'_{\theta \phi} \frac{1}{r\sin{\theta}} \partial_\phi u''_\theta} \\
-\eht{u''_\theta G_\theta^M} = & \ -\eht{u''_\theta G_\theta^M} \\
-\eht{u''_\theta \frac{1}{r}\partial_\theta P } = & \ -\eht{u''_\theta}\frac{1}{r}\partial_\theta \eht{P} - \eht{u''_\theta \frac{1}{r}\partial_\theta P'} = -\cancelto{0}{\eht{u''_\theta}\frac{1}{r}\partial_\theta \eht{P}} - \cancelto{0}{\eht{\frac{1}{r\sin{\theta}}\partial_\theta (\sin{\theta} u''_\theta P')}} + \eht{P'\frac{1}{r\sin{\theta}}\partial_\theta (\sin{\theta}u''_\theta}) \\
+\eht{u''_\theta \rho g_\theta} = 0 & \ 
\end{align}

\noindent
Final equation is:

\begin{align}
\eht{\rho}\fht{D}_t \big( \fht{R}_{\theta \theta} / \eht{\rho} \big) = & -2\fht{R}_{\theta r}\partial_r \fht{u}_\theta -\nabla_r 2 f_k^\theta - \eht{G_{\theta \theta}^R} + 2\nabla_r (\eht{u''_\theta \tau'_{\theta r}}) - 2\eht{u''_\theta G_\theta^M} + 2\eht{P'\nabla_\theta u''_\theta} - \nonumber \\ 
& - 2 \left(\eht{\tau'_{\theta r}\partial_r u''_\theta} - \eht{\tau'_{\theta \theta} \frac{1}{r}\partial_\theta u''_\theta} - \eht{\tau'_{\theta \phi} \frac{1}{r\sin{\theta}} \partial_\phi u''_\theta} \right) \\
\eht{\rho}\fht{D}_t \left( \fht{R}_{\theta \theta} / \eht{\rho} \right) = & -\nabla_r ( 2 f_k^\theta + 2 f_\tau^\theta ) - 2\fht{R}_{\theta r}\partial_r \fht{u}_\theta +2\eht{P' \nabla_\theta u''_\theta}  - 2\overline{u''_\theta G^{M}_\theta} - \eht{G^{R}_{\theta \theta}} - 2 \varepsilon_k^\theta 
\end{align}

\subsubsection{Mean equation for $\fht{R}_{\phi \phi}$}

\begin{align}
\eht{\rho}\fht{D}_t \big( \fht{R}_{\phi \phi} / \eht{\rho} \big) = -\left(\fht{R}_{\phi n}\partial_n \fht{u}_\phi + \fht{R}_{\phi n}\partial_n \fht{u}_\phi \right) -\left( \nabla_r 2 f_k^\phi + \eht{G_{\phi \phi}^R} \right) + 2  \eht{u''_\phi \rho D_t u_\phi} 
\end{align}

\noindent
where

\begin{align}
\eht{u''_\phi \rho D_t u_\phi} = & \ \eht{u''_\phi \left( \frac{1}{r^{2}} \partial_{r} \big(r^{2} [\tau_{\phi r}]\big) + \frac{1}{r\sin{\phi}}\partial_{\phi}\big(\sin{\phi}[\tau_{\phi \theta}]\big) + \frac{1}{r\sin{\phi}}\partial_{\phi}[\tau_{\phi \phi}]\big) - G_\phi^M - \frac{1}{r\sin{\theta}} \partial_{\phi} P + \rho g_\phi \right) } \\
\eht{u''_\phi \rho D_t u_\phi} = & \ \eht{u''_\phi \big( \nabla_r \tau_{\phi r} + \nabla_\theta \tau_{\phi \theta} + \nabla_\phi \tau_{\phi \phi} - G_\phi^M - \frac{1}{r\sin{\theta}} \partial_{\phi} P + \rho g_\phi \big) }
\end{align}

\noindent
Some terms can be further manipulated in following way:

\begin{align}
+\eht{u''_\phi \nabla_r \tau_{\phi r}} = & \ \cancelto{0}{\eht{u''_\phi}\nabla_r \eht{\tau_{\phi r}}} + \nabla_r (\eht{u''_\phi \tau'_{\phi r}}) - \eht{\tau'_{\phi r}\partial_r u''_\phi}\\
+\eht{u''_\phi \nabla_\theta \tau_{\phi \theta}} = & \ -\eht{\tau'_{\phi \theta} \frac{1}{r}\partial_\theta u''_\phi} \\
+\eht{u''_\phi \nabla_\phi \tau_{\phi \phi}} = & \ -\eht{\tau'_{\phi \phi} \frac{1}{r\sin{\theta}} \partial_\phi u''_\phi} \\
-\eht{u''_\phi G_\phi^M} = & \ -\eht{u''_\phi G_\phi^M} \\
-\eht{u''_\phi \frac{1}{r\sin{\theta}}\partial_\phi P } = & \ -\eht{u''_\phi}\frac{1}{r\sin{\theta}}\partial_\phi \eht{P} - \eht{u''_\phi \frac{1}{r\sin{\theta}}\partial_\phi P'} = -\cancelto{0}{\eht{u''_\phi}\frac{1}{r\sin{\theta}}\partial_\phi \eht{P}} - \cancelto{0}{\eht{\frac{1}{r\sin{\theta}}\partial_\phi (u''_\phi P')}} + \eht{P'\frac{1}{r\sin{\theta}}\partial_\phi (u''_\phi}) \\
+\eht{u''_\phi \rho g_\phi} = 0 & \ 
\end{align}

\noindent
Final equation is:

\begin{align}
\eht{\rho}\fht{D}_t \big( \fht{R}_{\phi \phi} / \eht{\rho} \big) = & -2\fht{R}_{\phi r}\partial_r \fht{u}_\phi -\nabla_r 2 f_k^\phi - \eht{G_{\phi \phi}^R} + 2\nabla_r (\eht{u''_\phi \tau'_{\phi r}}) - 2\eht{u''_\phi G_\phi^M} + 2\eht{P'\nabla_\phi u''_\phi} - \nonumber \\ 
& - 2 \left(\eht{\tau'_{\phi r}\partial_r u''_\phi} - \eht{\tau'_{\phi \theta} \frac{1}{r}\partial_\theta u''_\phi} - \eht{\tau'_{\phi \phi} \frac{1}{r\sin{\theta}} \partial_\phi u''_\phi} \right) \\
\eht{\rho}\fht{D}_t \left( \fht{R}_{\phi \phi} / \eht{\rho} \right) = & -\nabla_r ( 2 f_k^\phi + 2 f_\tau^\phi) - 2\fht{R}_{\phi r}\partial_r \fht{u}_\phi +2\eht{P' \nabla_\phi u''_\phi}  - 2\overline{u''_\phi G^{M}_\phi} - \eht{G^{R}_{\phi \phi}} - 2\varepsilon_k^\phi 
\end{align}

\newpage

\subsection{Turbulent kinetic energy equations}

\begin{align}
\fht{k} = & +\frac{1}{2}\fht{R}_{ii} / \eht{\rho}  \\
\av{\rho} \fav{D}_t \fav{k}^{ } = & -\nabla_r ( f_k +  f_P ) - \fht{R}_{ir}\partial_r \fht{u}_i + W_b + W_P + {\mathcal N_k}  \label{eq:rans_tke} \\
\fht{k}^r = & +\frac{1}{2}\fht{R}_{rr} / \eht{\rho}  \\
\av{\rho} \fav{D}_t \fav{k}^r =  &  -\nabla_r  ( f_k^r + f_P )  - \fht{R}_{rr}\partial_r \fht{u}_r + W_b  + \eht{P'\nabla_r u''_r} + {\mathcal G_k^r} + {\mathcal N_{kr}} \label{eq:rans_ekin_r} \\
\fht{k}^h = & +\fht{k}_\theta + \fht{k}_\phi = +\frac{1}{2} \big( \fht{R}_{\theta \theta} + \fht{R}_{\phi \phi} \big) / \eht{\rho} \\
\av{\rho} \fav{D}_t \fav{k}^h =  &  -\nabla_r f_k^h - (\fht{R}_{\theta r}\partial_r \fht{u}_\theta + \fht{R}_{\phi r}\partial_r \fht{u}_\phi) + (\eht{P' \nabla_\theta u''_\theta} + \eht{P' \nabla_\phi u''_\phi}) + {\mathcal G_k^h} + {\mathcal N_{kh}} \label{eq:rans_ekin_h} 
\end{align}

\subsection{Turbulent mass flux equation}

The turbulent mass flux equation can be derived in the following way:

%first parameter is font size and the other line spacing
\fontsize{10pt}{20pt}

\begin{align}
\rho D_t \fht{c} - \rho \fht{D}_t \fht{c} = & \ \rho \partial_t \fht{c} + \rho u_n \partial_n \fht{c} - [\rho \partial_t \fht{c} + \rho \fht{u}_n \partial_n \fht{c}_n] = \rho (u_n - \fht{u}_n)\partial_n \fht{c} =  \rho u''_n \partial_n \fht{c} \\
\rho D_t c'' =  & \ \rho D_t c - \rho D_t \fht{c} = \rho D_t c -  \rho \fht{D}_t \fht{c} - \rho u''_n \partial_n \fht{c} = \rho D_t c -  \frac{\rho}{\eht{\rho}} [\eht{\rho}\fht{D}_t \fht{c}] - \rho u''_n \partial_n \fht{c} \\
\rho D_t u_r'' =  & +\left( \frac{1}{r^{2}} \frac{\partial}{\partial r} \big( r^{2} [\tau_{rr}]\big) + \frac{1}{r\sin{\theta}}\frac{\partial}{\partial \theta}(\sin{\theta}[\tau_{r\theta}]\big) + \frac{1}{r\sin{\theta}}\frac{\partial}{\partial \phi}\big([\tau_{r\phi}]\big) - G_r^M - \frac{\partial P}{\partial r} \right) + \rho g_r + \\ 
 & + \frac{\rho}{\eht{\rho}}\left( \frac{1}{r^2}\dr r^2 (\fht{R}_{rr}-\eht{\tau_{rr}}) + \eht{G_r^M} + \dr \eht{P} - \eht{\rho}\fht{g}_r \right) - \rho u''_n \partial_n \fht{u_r}  & \
\end{align}

\begin{align}
\fht{D}_t \eht{u''_i} = & \ \eht{D_t u''_i - u''_n\partial_n u''_i} = \eht{\partial_t u''_i}  + \eht{u_n \partial_n u''_i} - \eht{u''_n \partial_n u''_i} = \eht{\partial_t u''_i}  + \eht{u_n \partial_n u''_i} + \eht{\fht{u_n} \partial_n u''_i} - \eht{u_n \partial_n u''_i} = \eht{\partial_t u''_i} + \eht{\fht{u_n} \partial_n u''_i} = \fht{D}_t \eht{u''_i} \\
\eht{\rho}\fht{D_t}\eht{u''_i} = & \ \eht{\frac{\eht{\rho}}{\rho}[\rho D_t u''_i]} - \eht{\eht{\rho}u''_n\partial_n u''_i} 
\end{align}

\begin{align}
\eht{\rho}\fht{D_t}\eht{u''_r} = & \ \eht{\frac{\eht{\rho}}{\rho}[\rho D_t u''_r]} - \eht{\eht{\rho}u''_n\partial_n u''_r} = \nonumber \\ 
& = \eht{\frac{\eht{\rho}}{\rho} \left[ \frac{1}{r^{2}} \frac{\partial}{\partial r} \big( r^{2} \tau_{rr} \big) - G_r^M - \frac{\partial}{\partial r}P + \rho g_r + \frac{\rho}{\eht{\rho}}\left(\frac{1}{r^2}\dr r^2 (\fht{R}_{rr}-\eht{\tau_{rr}}) + \eht{G_r^M} + \dr \eht{P} - \eht{\rho}\fht{g_r} \right) - \rho u''_n \partial_n \fht{u_r} \right]} - \eht{\eht{\rho}u''_n\partial_n u''_r} = \\
& = \eht{\frac{\eht{\rho}}{\rho} \left[ \frac{1}{r^{2}} \frac{\partial}{\partial r} \big( r^{2} \tau_{rr} \big) \right] -  \left[ \frac{1}{r^{2}} \frac{\partial}{\partial r} \big( r^{2} \tau_{rr} \big) \right]} -\eht{\frac{\eht{\rho}}{\rho}G_r^M + \eht{G_r^M}} - \eht{\frac{\eht{\rho}}{\rho} \left[\frac{\partial}{\partial r}P \right] +  \left[\frac{\partial}{\partial r}P \right]} + \frac{1}{r^2}\dr r^2 (\fht{R}_{rr}) - \eht{\eht{\rho} u''_n \partial_n \fht{u_r}} - \eht{\eht{\rho}u''_n\partial_n u''_r} \\
& = \eht{\left[ \frac{\eht{\rho}}{\rho} - 1 \right] \frac{1}{r^{2}} \frac{\partial}{\partial r} \big( r^{2} \tau_{rr} \big)} - \eht{\left[\frac{\eht{\rho}}{\rho} - 1 \right] G_r^M} - \eht{\left[\frac{\eht{\rho}}{\rho} - 1 \right]\frac{\partial}{\partial r} P} + \frac{1}{r^2}\dr r^2 (\fht{R}_{rr}) - \eht{\rho}\eht{u''_n\partial_n u_r } = \\
& = +\frac{1}{r^2}\dr r^2 (\fht{R}_{rr}) - \eht{\rho}\eht{u''_n\partial_n u_r}  - \eht{\frac{\rho'}{\rho} \frac{1}{r^{2}} \frac{\partial}{\partial r} \big( r^{2} \tau_{rr} \big)} + \eht{\frac{\rho'}{\rho} G_r^M} + \eht{\frac{\rho'}{\rho}\frac{\partial}{\partial r} P} = \\
& = +\frac{1}{r^2}\dr r^2 (\fht{R}_{rr}) - \eht{\rho}\eht{u''_n\partial_n u_r}  - \eht{\rho' v \frac{1}{r^{2}} \frac{\partial}{\partial r} \big( r^{2} (\eht{\tau_{rr}} + \tau'_{rr} \big)} + \eht{\rho' v \frac{\partial}{\partial r} (\eht{P} + P')} + \eht{\rho' v G_r^M} = \\
& = +\frac{1}{r^2}\dr r^2 (\fht{R}_{rr}) - \eht{\rho}\eht{u''_n\partial_n u_r} - \eht{\rho' v} \left(\frac{1}{r^{2}} \frac{\partial}{\partial r} \big( r^{2} \eht{\tau_{rr}} \big) - \frac{\partial}{\partial r} \eht{P} \right) - \eht{\rho' v \left(\frac{1}{r^{2}} \frac{\partial}{\partial r} \big( r^{2} \tau'_{rr} \big) - \frac{\partial}{\partial r} P' \right) }  + \eht{\rho' v G_r^M} = \\
& = +\frac{1}{r^2}\dr r^2 (\fht{R}_{rr}) - \eht{\rho}\eht{u''_n\partial_n u_r} + b\left(\frac{1}{r^{2}} \frac{\partial}{\partial r} \big( r^{2} \eht{\tau_{rr}} \big) - \frac{\partial}{\partial r} \eht{P} \right) - \eht{\rho' v \left(\frac{1}{r^{2}} \frac{\partial}{\partial r} \big( r^{2} \tau'_{rr} \big) - \frac{\partial}{\partial r} P' \right) }  + \eht{\rho' v G_r^M} = \\
& = +\nabla_r ( \fht{R}_{rr} ) - \eht{\rho}\eht{{\bf{u''}} \cdot \nabla u_r} -b \nabla_r \eht{\tau}_{rr} - b\partial_r \eht{P} + \eht{\rho' v \partial_r P'} - \eht{\rho' v \nabla_r  \tau'_{rr} }  + \eht{\rho' v G_r^M} 
\end{align}

\begin{align}
{\color{red} \eht{\rho}\fht{D}_t \eht{u''_r} }& =  +\nabla_r ( \fht{R}_{rr} ) - \eht{\rho}\eht{u''_n \partial_n u_r} -b\nabla_r \eht{\tau}_{rr}- b\partial_r \eht{P} + \eht{\rho' v \partial_r P'} - \eht{\rho' v \nabla_r  \tau'_{rr} }  + \eht{\rho' v G_r^M} = \\
& =  +\nabla_r ( \fht{R}_{rr} ) - \eht{\rho}\eht{u''_n}\partial_n \eht{u}_r - \eht{\rho}\partial_n \eht{u''_n u'_r} + \eht{\rho} \eht{u'_r\partial_n u''_n} - b\partial_r \eht{P} + \eht{\rho' v \partial_r P'} - \eht{\rho' v \nabla_r  \tau'_{rr} }  + \eht{\rho' v G_r^M} = \\
& = (+\nabla_r ( \fht{R}_{rr} ) - \eht{\rho}\nabla_r \eht{u''_r u'_r}) - \eht{\rho}\eht{u''_r} \nabla_r \eht{u}_r + \eht{\rho} \eht{u'_r d''} - b\partial_r \eht{P} + \eht{\rho' v \partial_r P'} - \eht{\rho' v \nabla_r  \tau'_{rr} }  + \eht{\rho' v G_r^M} = \\
& {\color{red} = -(\eht{\rho'u'_ru'_r}/\eht{\rho})\partial_r\eht{\rho} + (\fht{R}_{rr}/\eht{\rho})/\partial_r \eht{\rho} - \eht{\rho} \nabla_r (\eht{u''_r} \ \eht{u''_r}) + \nabla_r \overline{\rho' u'_r u'_r} - \eht{\rho}\eht{u''_r} \nabla_r \eht{u}_r + \eht{\rho} \eht{u'_r d''} - b\partial_r \eht{P} + \eht{\rho' v \partial_r P'} - \eht{\rho' v \nabla_r  \tau'_{rr} }  + \eht{\rho' v G_r^M}} 
\end{align}

\newpage

%first parameter is font size and the other line spacing
\fontsize{12pt}{20pt}

\subsection{Density-specific volume covariance equation}

The density-specific volume ($v = 1/\rho$) covariance ($b = -\eht{\rho' v'}$) equation can be derived from the continuity equation in the following way. 

%first parameter is font size and the other line spacing
\fontsize{9pt}{20pt}

\begin{align}
\partial_t \rho + \partial_n (\rho u_n) = & \ 0 \\
\partial_t \eht{\rho} + \fht{u}_n \partial_n \eht{\rho} = & \ -\eht{\rho} \partial_n \fht{u}_n \\
\partial_t \eht{\rho} + \eht{u}_n \partial_n \eht{\rho} - \eht{u''_n} \partial_n \eht{\rho} = & \ - \eht{\rho}\partial_n \eht{u}_n + \eht{\rho} \partial_n \eht{u''_n} \\
\eht{D}_t \eht{\rho} = & \ +\overline{u''_n} \partial_n \eht{\rho} - \eht{\rho} \partial_n \eht{u}_n + \eht{\rho} \partial_n \eht{u''_n} \\
\eht{D}_t \eht{\rho} = & \ +( \overline{u''_n} \partial_n \eht{\rho} +  \eht{\rho} \partial_n \eht{u''_n} ) - \eht{\rho} \partial_n \eht{u}_n \\
\eht{D}_t \eht{\rho} = & \ - \eht{\rho} \partial_n \eht{u}_n + \partial_n ( \overline{u''_n} \eht{\rho} )
\end{align}

\begin{align}
\partial_t \rho + \partial_n (\rho u_n) = & \ 0 \\
\partial_t (1/v) + \partial_n (u_n / v) = & \ 0 \\
-\partial_t v + v\partial_n u_n - u_n\partial_n v = & \ 0 \\
\partial_t v - v\partial_n u_n - u_n\partial_n v + u_n\partial_n v + u_n\partial_n v = & \ 0 \\
\partial_t v - \partial_n (v u_n) + 2 u_n \partial_n v = & \ 0 \\
\partial_t \eht{v} - \partial_n (\eht{v u_n}) + 2 \eht{u_n \partial_n v} = & \ 0 \\
\eht{D}_t \eht{v} = & +\eht{v}\partial_n \eht{u}_n - \partial_n \eht{u'_n v'} + 2 \eht{v'\partial_n u'_n}  
\end{align}

\begin{align}
b = -\eht{\rho'v'} = \eht{\rho}\eht{v} - 1 \\
\eht{D}_t b = \eht{\rho}\eht{D}_t \eht{v} + \eht{v} \ \eht{D}_t \eht{\rho} 
\end{align}

Using previously derived equations for $\eht{D}_t \eht{v}$ and $\eht{D}_t \eht{\rho}$ we get:

\begin{align}
\eht{D}_t b = -\eht{\rho}\partial_n \eht{\rho' u'_n} + 2 \eht{\rho}\eht{v'\partial_n u'_n} + \eht{v}\partial_n \eht{\rho} \eht{u''_n}
\end{align}

\subsection{Mean internal energy flux equation}

We can derive the internal energy flux equation using the general formula for second order moments, where we substitute $c$ with  $\epsilon_I$ and $d$ with $u_i$.

\begin{align}
\overline{\rho}\widetilde{D}_t \widetilde{c''d''} = & \overline{c'' \rho D_t d} - \overline{\rho} \widetilde{c''u''_n}\partial_n \widetilde{d} + \overline{d'' \rho D_t c} - \overline{\rho} \widetilde{d''u''_n}\partial_n \widetilde{c} - \overline{\partial_n \rho c''d''u''_n} \\
\erho \fav{D}_t (f_I / \eht{\rho}) = &  {\mathcal N_{fI}} -\nabla_r f_I^r  - f_I \partial_r \fht{u}_r  - \fht{R}_{rr} \partial_r \fht{\epsilon_I} - \eht{\epsilon''_I} \partial_r \eht{P} - \eht{\epsilon''_I \partial_r P'}  - \eht{u''_r \left( P d \right)}  + \overline{u''_r ({\mathcal S} + \nabla \cdot f_T)} + {\mathcal G_I} + {\mathcal N_{fI}}\label{eq:rans_fi}
\end{align}

\subsection{Mean enthalpy flux equation}

We can derive the enthalpy flux equation using the general formula for second order moments, where we substitute $c$ with  $h$ and $d$ with $u_i$.

\begin{align}
\overline{\rho}\widetilde{D}_t \widetilde{c''d''} = & \overline{c'' \rho D_t d} - \overline{\rho} \widetilde{c''u''_n}\partial_n \widetilde{d} + \overline{d'' \rho D_t c} - \overline{\rho} \widetilde{d''u''_n}\partial_n \widetilde{c} - \overline{\partial_n \rho c''d''u''_n} \\
\erho \fav{D}_t (f_h / \eht{\rho}) = &  -\nabla_r f_h^r - f_h \partial_r \fht{u}_r - \fht{R}_{rr} \partial_r \fht{h} -\eht{h''}\partial_r \eht{P} - \eht{h''\partial_r P'} - \Gamma_1\eht{u''_r \left( P d \right) } + \Gamma_3 \overline{u''_r ({\mathcal S} + \nabla \cdot F_T)} + {\mathcal G_h} + {\mathcal N_{h \ }} \label{eq:rans_fh} 
\end{align}

\subsection{Mean entropy flux equation}

We can derive the entropy flux equation using the general formula for second order moments, where we substitute $c$ with $s$ and $d$ with $u_i$.

\begin{align}
\overline{\rho}\widetilde{D}_t \widetilde{c''d''} = & \overline{c'' \rho D_t d} - \overline{\rho} \widetilde{c''u''_n}\partial_n \widetilde{d} + \overline{d'' \rho D_t c} - \overline{\rho} \widetilde{d''u''_n}\partial_n \widetilde{c} - \overline{\partial_n \rho c''d''u''_n} \\
\erho \fav{D}_t (f_s / \eht{\rho}) = &  -\nabla_r f_s^r - f_s \partial_r \fht{u}_r - \fht{R}_{rr} \partial_r \fht{s} -\eht{s''}\partial_r \eht{P} - \eht{s''\partial_r P'} + \eht{u''_r ( {\mathcal S} + \nabla \cdot f_T)  / T} + {\mathcal G_s} + {\mathcal N_{fs}}  \label{eq:rans_fs}
\end{align}

\subsection{Mean composition flux equation in $r$ direction.}

We can derive the composition flux equation using the general formula for second order moments, where we substitute $c$ with $X_\alpha$ and $d$ with $u_i$.

\begin{align}
\overline{\rho}\widetilde{D}_t \widetilde{c''d''} = & \overline{c'' \rho D_t d} - \overline{\rho} \widetilde{c''u''_n}\partial_n \widetilde{d} + \overline{d'' \rho D_t c} - \overline{\rho} \widetilde{d''u''_n}\partial_n \widetilde{c} - \overline{\partial_n \rho c''d''u''_n} \\
\erho \fav{D}_t (f_\alpha / \eht{\rho}) = &  -\nabla_r f_\alpha^r  - f_\alpha \partial_r \fht{u}_r - \fht{R}_{rr} \partial_r \fht{X}_\alpha -\eht{X''_\alpha} \partial_r \eht{P} - \eht{X''_\alpha \partial_r P'} + \overline{u''_r \rho \dot{X}_\alpha^{\rm nuc}} + {\mathcal G_\alpha} + {\mathcal N_{f\alpha}} \label{eq:rans_falpha} \\
\end{align}

\noindent Please note that the last term $ {\overline{\partial_n \rho X''_i u''_i u''_n}} \equiv \eht{\nabla \cdot (\rho X''_i {\bf{u''}} {\bf{u''}})} {\mbox{\ is div of 2nd order tensor}} \mbox{ and } \eht{\nabla \cdot (\rho X''_i {\bf{u''}} {\bf{u''}})}({\bf{e_r}}) = \nabla_r \eht{\rho X_i u''_r u''_r}  -\eht{\rho X''_i u''_\theta u''_\theta/r} - \eht{\rho X''_i u''_\phi u''_\phi/r} $\\

\noindent
Further calculation requires the following hydrodynamic equations:

\begin{align}
\rho D_{t} \big(u_{r}\big) = & +\left( \frac{1}{r^{2}} \partial_{r} \big( r^{2} [\tau_{rr}]\big) + \frac{1}{r\sin{\theta}}\partial_{\theta}(\sin{\theta}[\tau_{r\theta}]\big) + \frac{1}{r\sin{\theta}}\partial_{\phi}\big([\tau_{r\phi}]\big) - G_r^M - \partial_{r} P \right) + \rho g_r \\
\rho D_{t} \big(X_{i}\big) = & + \rho \dot{X}_{i}^{n} \ \ \ \ \ \ \ \ \ \ \ \  i = 1 ... N_{nuc}
\end{align}

\noindent
For radial component of the flux, where $u_i = u_r$ and changing order of terms for clarity reasons we get:

\begin{align}
\overline{\rho}\widetilde{D}_t \widetilde{X''_i u''_r} = & \left(- \overline{\partial_n \rho X''_i u''_r u''_n} \right) - \overline{\rho} \widetilde{X''_i u''_n}\partial_n \widetilde{u_r}  - \overline{\rho} \widetilde{u''_r u''_n}\partial_n \widetilde{X_i} + \overline{X''_i \rho D_t u_r} + \overline{u''_r \rho D_t X_i}  \\
\erho \fav{D}_t (f_i / \eht{\rho}) = &  \left( -\nabla_r f_i^r -\eht{\rho X''_i u''_\theta u''_\theta/r} - \eht{\rho X''_i u''_\phi u''_\phi/r} \right) - f_i \partial_r \fht{u}_r - \fht{R}_{rr} \partial_r \fht{X}_i + \eht{X''_{i}(\nabla \cdot \tau_r  - G_r^M - \partial_{r} P + \rho g_r)} + \eht{u''_r \rho  \dot{X}_i^{\rm nuc}}
\end{align}

\noindent
Let's reorganize the terms again and split  $P = \overline{P} + P'$ to get the equation to shape it has in the paper:

%- \eht{\nabla \cdot X''_i \tau'_r} + \eht{\tau'_r \nabla X''_i} 

\begin{align}
\erho \fav{D}_t (f_i / \eht{\rho}) = &  -\nabla_r f_i^r  - f_i \partial_r \fht{u}_r - \fht{R}_{rr} \partial_r \fht{X}_i -\eht{X''_i} \partial_r \eht{P} - \eht{X''_i \partial_r P'} + \cancelto{0}{\eht{X''_i \rho g_r }} + \overline{u''_r \rho \dot{X}_i^{\rm nuc}} \\
& -\eht{\rho X''_i u''_\theta u''_\theta/r} - \eht{\rho X''_i u''_\phi u''_\phi/r}  - \eht{X''_i G_r^M} - \eht{X'' \nabla \cdot \tau_r}  \\
\erho \fav{D}_t (f_i / \eht{\rho}) = &  -\nabla_r f_i^r  - f_i \partial_r \fht{u}_r - \fht{R}_{rr} \partial_r \fht{X}_i -\eht{X''_i} \partial_r \eht{P} - \eht{X''_i \partial_r P'} + \overline{u''_r \rho \dot{X}_i^{\rm nuc}} \\
& -\eht{\rho X''_i u''_\theta u''_\theta/r} - \eht{\rho X''_i u''_\phi u''_\phi/r}  - \eht{X''_i G_r^M} - \eht{\nabla \cdot X''_i \tau_r} + \eht{\tau_r \nabla X''_i}  \\
\end{align}

\noindent
The viscoity $\tau$ is 2nd order tensor and $-\eht{\nabla \cdot X''_i \tau_r}= -\nabla_r \overline{X''_i \tau_{rr}} + \overline{X''\tau_{\theta \theta} \tau_{\theta \theta}/r} + \overline{X''\tau_{\phi \phi} \tau_{\phi \phi}/r}$ 

\begin{align}
\erho \fav{D}_t (f_i / \eht{\rho}) = &  -\nabla_r f_i^r  - f_i \partial_r \fht{u}_r - \fht{R}_{rr} \partial_r \fht{X}_i -\eht{X''_i} \partial_r \eht{P} - \eht{X''_i \partial_r P'} + \overline{u''_r \rho \dot{X}_i^{\rm nuc}} \\
& -\eht{\rho X''_i u''_\theta u''_\theta/r} - \eht{\rho X''_i u''_\phi u''_\phi/r}  - \eht{X''_i G_r^M} -\nabla_r (\eht{X_i''\tau_{rr}}) + \overline{X''\tau_{\theta \theta} \tau_{\theta \theta}/r} + \overline{X''\tau_{\phi \phi} \tau_{\phi \phi}/r}\\
&- \eht{\tau_{rr}\partial_r X''_i} - \eht{\tau_{r\theta}(1/r)\partial_\theta X''_i} - \eht{\tau_{r\phi}(1/r\sin{\theta})\partial_\phi X''_i} \\
\erho \fav{D}_t (f_i / \eht{\rho}) = &  -\nabla_r f_i^r  - f_i \partial_r \fht{u}_r - \fht{R}_{rr} \partial_r \fht{X}_i -\eht{X''_i} \partial_r \eht{P} - \eht{X''_i \partial_r P'} + \overline{u''_r \rho \dot{X}_i^{\rm nuc}} \\
& -\eht{\rho X''_i u''_\theta u''_\theta/r} - \eht{\rho X''_i u''_\phi u''_\phi/r}  - \eht{X''_i G_r^M} {-\nabla_r (\eht{X''_i \tau_{rr}}) - \varepsilon_i}\\
\erho \fav{D}_t (f_i / \eht{\rho}) = &  -\nabla_r f_i^r  - f_i \partial_r \fht{u}_r - \fht{R}_{rr} \partial_r \fht{X}_i -\eht{X''_i} \partial_r \eht{P} - \eht{X''_i \partial_r P'} + \overline{u''_r \rho \dot{X}_i^{\rm nuc}} \\
& -\eht{\rho X''_i u''_\theta u''_\theta/r} - \eht{\rho X''_i u''_\phi u''_\phi/r}  - \eht{X''_i G_r^M} + {\mathcal N_{fi}}
\end{align}

\begin{align}
\erho \fav{D}_t (f_i / \eht{\rho}) = &  -\nabla_r f_i^r  - f_i \partial_r \fht{u}_r - \fht{R}_{rr} \partial_r \fht{X}_i -\eht{X''_i} \partial_r \eht{P} - \eht{X''_i \partial_r P'} + \overline{u''_r \rho \dot{X}_i^{\rm nuc}} \\
& + {\eht{G_{r}^i} - \eht{X''_i G_r^M}} + {\mathcal N_{fi}}\\
\erho \fav{D}_t (f_i / \eht{\rho}) = &  -\nabla_r f_i^r  - f_i \partial_r \fht{u}_r - \fht{R}_{rr} \partial_r \fht{X}_i -\eht{X''_i} \partial_r \eht{P} - \eht{X''_i \partial_r P'} + \overline{u''_r \rho \dot{X}_i^{\rm nuc}} \\
& + {\mathcal G_i} + {\mathcal N_{fi}} \label{eq:rans_falpha} 
\end{align}

\noindent
The exact formulation of the geometry and viscosity terms is:

\begin{align}
{\mathcal G_i} = & {\eht{G_{r}^i} - \eht{X''_i G_r^M}}\\
{\mathcal N_{fi}} = &-\nabla_r (\eht{X''_i {\color{blue}\tau_{rr}}}) - \eht{{\color{blue}\tau_{rr}}\partial_r X''_i} - \eht{{\color{blue}\tau_{r\theta}}(1/r)\partial_\theta X''_i} - \eht{{\color{blue}\tau_{r\phi}}(1/r\sin{\theta})\partial_\phi X''_i}  {\color{red} +\overline{X''\tau_{\theta \theta} \tau_{\theta \theta}/r} + \overline{X''\tau_{\phi \phi} \tau_{\phi \phi}/r}}  
\end{align}

\subsection{Mean composition equation in $\theta$ direction.}

t.b.d


\subsection{Mean A and Z flux equations}

We can derive the composition flux equation using the general formula for second order moments, where we substitute $c$ with $A$ or $Z$ and $d$ with $u_i$.

\begin{align}
\overline{\rho}\widetilde{D}_t \widetilde{c''d''} = & \overline{c'' \rho D_t d} - \overline{\rho} \widetilde{c''u''_n}\partial_n \widetilde{d} + \overline{d'' \rho D_t c} - \overline{\rho} \widetilde{d''u''_n}\partial_n \widetilde{c} - \overline{\partial_n \rho c''d''u''_n} \\
\erho \fav{D}_t (f_A / \eht{\rho}) = &  \ {\mathcal N_{fA}} -\nabla_r f_A^r - f_A \partial_r \fht{u}_r - \fht{R}_{rr} \partial_r \fht{A} -\eht{A''} \partial_r \eht{P} - \eht{A'' \partial_r P'} - \overline{u''_r \rho A^2\Sigma_\alpha \dot{X}_\alpha^{\rm nuc} / A_\alpha} + {\mathcal G_A}                 \label{eq:rans_fabar} \\
\erho \fav{D}_t (f_Z / \eht{\rho}) = &  \ {\mathcal N_{fZ}} -\nabla_r f_Z^r  - f_Z \partial_r \fht{u}_r - \fht{R}_{rr} \partial_r \fht{Z} -\eht{Z''} \partial_r \eht{P} - \eht{Z'' \partial_r P'} - \overline{u''_r \rho Z A \Sigma_\alpha (\dot{X}_\alpha^{\rm nuc}/ A_\alpha)} - \nonumber \\
& - \overline{u''_r \rho A \Sigma_\alpha (Z_\alpha \dot{X}_\alpha^{\rm nuc} / A_\alpha)}  + {\mathcal G_Z}   \label{eq:rans_fzbar} 
\end{align}

\subsection{Mean angular momentum flux equation}

We can derive the angular momentum flux equation using the general formula for second order moments, where we substitute $c$ with $j_z$ and $d$ with $u_i$.

\begin{align}
\overline{\rho}\widetilde{D}_t \widetilde{c''d''} = & \overline{c'' \rho D_t d} - \overline{\rho} \widetilde{c''u''_n}\partial_n \widetilde{d} + \overline{d'' \rho D_t c} - \overline{\rho} \widetilde{d''u''_n}\partial_n \widetilde{c} - \overline{\partial_n \rho c''d''u''_n} \\
\erho \fav{D}_t (f_{jz} / \rho) = & -\nabla_r f_{jz}^r  - f_{jz} \partial_r \fht{u}_r - \fht{R}_{rr} \partial_r \fht{j_z} -\eht{j''_z} \partial_r \eht{P} - \eht{j''_z \partial_r P'} + {\mathcal G_{jz}} + {\mathcal N_{jz}} \label{eq:rans_fjz}
\end{align}

\subsection{Pressure flux equation in $r$ direction}

We can derive the pressure flux equation using the general formula for second order moments, where we substitute $c$ with $P$ and $d$ with $u_r$.

\begin{align}
\widetilde{D}_t \widetilde{c'd'} = & \ \overline{c' D_t d} - \eht{c'u''_n} \partial_n \eht{d} + \eht{d'D_t c} + \eht{d'u''_n}\partial_n \eht{c} - \partial_n \eht{u''_n c'd'} + \eht{c'd''\partial_n u''_n} \nonumber \\ 
  \fht{D}_t \eht{P'u'_r} = & \ \eht{P'D_t u_r} - \eht{P'u''_n} \partial_n \eht{u}_r + \eht{u'_r D_t P} + \eht{u'_r u''_n} \partial_n \eht{P} - \partial_n \eht{u''_n P' u'_r} + \eht{P'u''_r \partial_n u''_n} \nonumber \\
  \fht{D}_t \eht{P'u'_r} = & \ \eht{P'D_t u_r} - \eht{P'u''_r} \partial_r \eht{u}_r + \eht{u'_r D_t P} + \eht{u'_r u''_r} \partial_r \eht{P} - \nabla_r \eht{P' u''_r u'_r} + \eht{P'u''_r d''} 
\end{align}

\noindent
The evolution equation for $D_t u_r$ (see Sect.\ref{sect:hydro-inst-lag}) is 

\begin{align}
\rho D_{t} \big(u_{r}\big) = & \ +\left( \frac{1}{r^{2}} \partial_{r} \big( r^{2} [\tau_{rr}]\big) + \frac{1}{r\sin{\theta}}\partial_{\theta}(\sin{\theta}[\tau_{r\theta}]\big) + \frac{1}{r\sin{\theta}}\partial_{\phi}\big([\tau_{r\phi}]\big) - G_r^M - \partial_{r} P \right) + \rho g_r 
\end{align}

\noindent
By neglecting viscosity and dividing by $\rho$, it becomes:

\begin{align}
D_{t} \big(u_{r}\big) = & \ - G_r^M/\rho - \partial_{r} P / \rho  + g_r 
\end{align}

\noindent
The evolution equation for $D_t P$ (see Sect.\ref{sect:mean-pressure-eq}) is 

\begin{align}
D_t P = & -(1-\Gamma_3+\Gamma_1)Pd + (\Gamma_3 -1)(-Pd + {\mathcal S} + \nabla \cdot F_T + \tau_{ij}\partial_i u_j)
\end{align}

\noindent
By neglecting viscosity, it becomes:

\begin{align}
D_t P = & -(1-\Gamma_3+\Gamma_1)Pd + (\Gamma_3 -1)(-Pd + {\mathcal S} + \nabla \cdot F_T)
\end{align}

\noindent We continue here by putting everything together and neglecting heat flux due to conduction and radiation $F_T$:

\begin{align}
  \fht{D}_t \eht{P'u'_r} = & \ \eht{P'\left(- G_r^M/\rho - \partial_{r} P / \rho  + g_r \right)} - \eht{P'u''_r} \partial_r \eht{u}_r + \eht{u'_r \left( -(1-\Gamma_3+\Gamma_1)Pd + (\Gamma_3 -1)(-Pd + {\mathcal S} + \nabla \cdot F_T) \right)} + \eht{u'_r u''_r} \partial_r \eht{P} - \nabla_r \eht{P' u''_r u'_r} + \eht{P'u''_r d''} \\
  \fht{D}_t \eht{P'u'_r} = & \ -\eht{P' G_r^M/\rho} - \eht{P'\partial_{r} P / \rho}  + \eht{P'g_r}  - \eht{P'u''_r} \partial_r \eht{u}_r - (1-\Gamma_3-\Gamma_1)\eht{u'_r P d} - (\Gamma_3-1)\eht{u'_rPd} + (\Gamma_3 -1)\eht{u'_r \rho \varepsilon_{nuc}}  + \eht{u'_r u''_r} \partial_r \eht{P} - \nabla_r \eht{P' u''_r u'_r} + \eht{P'u''_r d''} \\
  \fht{D}_t \eht{P'u'_r} = & \ -\nabla_r f_p^r - f_p\partial_r \eht{u}_r + \eht{u'_r u''_r}\partial_r \eht{P} - (1-\Gamma_3-\Gamma_1)\eht{u'_r P d} - (\Gamma_3-1)\eht{u'_r P d} + (\Gamma_3 -1)\eht{u'_r \rho \varepsilon_{nuc}} + \eht{P'u''_r d''} -\eht{P' G_r^M/\rho} - \eht{P'\partial_{r} P / \rho}
\end{align}
  
\noindent With a little more algebraic modifications, we get the final form:

\begin{align}
  \fht{D}_t \eht{P'u'_r} = & \ -\nabla_r f_p^r - f_p\partial_r \eht{u}_r + \eht{u'_r u''_r}\partial_r \eht{P} +\Gamma_1 \eht{u'_r P d} + (\Gamma_3 -1)\eht{u'_r \rho \varepsilon_{nuc}} + \eht{P'u''_r d''} -\eht{P' G_r^M/\rho} - \eht{P'\partial_{r} P / \rho}  
\end{align}

\subsection{Pressure flux equation in $x$ direction (Cartesian geometry)}

We can derive the pressure flux equation using the general formula for second order moments, where we substitute $c$ with $P$ and $d$ with $u_x$ (like in previous section).

\begin{align}
  \fht{D}_t \eht{P'u'_x} = & \ \eht{P'D_t u_x} - \eht{P'u''_x} \partial_x \eht{u}_x + \eht{u'_x D_t P} + \eht{u'_x u''_x} \partial_x \eht{P} - \nabla_x \eht{P' u''_x u'_x} + \eht{P'u''_x d''} 
\end{align}

\noindent
The evolution equation for $D_t u_x$ is \href{https://en.wikipedia.org/wiki/Navier-Stokes_equations}{https://en.wikipedia.org/wiki/Navier-Stokes\_equations}: 

\begin{align}
\rho D_{t} \big(u_{x}\big) = & \ \nabla_x \tau_{xx} + \nabla_y \tau_{xy} + \nabla_z \tau_{xz} - \partial_{x} P + \rho g_x 
\end{align}

\noindent
By neglecting viscosity and dividing by $\rho$, it becomes:

\begin{align}
D_{t} \big(u_{x}\big) = & -\frac{\partial_{x} P}{\rho}  + g_x 
\end{align}

\noindent
The evolution equation for $D_t P$ (see Sect.\ref{sect:mean-pressure-eq}) is 

\begin{align}
D_t P = & -(1-\Gamma_3+\Gamma_1)Pd + (\Gamma_3 -1)(-Pd + {\mathcal S} + \nabla \cdot F_T + \tau_{ij}\partial_i u_j)
\end{align}

\noindent
By neglecting viscosity, it becomes:

\begin{align}
D_t P = & -(1-\Gamma_3+\Gamma_1)Pd + (\Gamma_3 -1)(-Pd + {\mathcal S} + \nabla \cdot F_T)
\end{align}

\noindent We continue here by putting everything together and neglecting heat flux due to conduction and radiation $F_T$:

\begin{align}
  \fht{D}_t \eht{P'u'_x} = & \ \eht{P'\left(- \partial_{x} P / \rho  + g_x \right)} - \eht{P'u''_x} \partial_x \eht{u}_x + \eht{u'_x \left( -(1-\Gamma_3+\Gamma_1)Pd + (\Gamma_3 -1)(-Pd + {\mathcal S} + \nabla \cdot F_T) \right)} + \eht{u'_x u''_x} \partial_x \eht{P} - \nabla_x \eht{P' u''_x u'_x} + \eht{P'u''_x d''} \\
  \fht{D}_t \eht{P'u'_x} = & \ -\eht{P'\partial_{x} P / \rho}  + \eht{P'g_x}  - \eht{P'u''_x} \partial_x \eht{u}_x - (1-\Gamma_3-\Gamma_1)\eht{u'_x P d} - (\Gamma_3-1)\eht{u'_xPd} + (\Gamma_3 -1)\eht{u'_x \rho \varepsilon_{nuc}}  + \eht{u'_x u''_x} \partial_x \eht{P} - \nabla_x \eht{P' u''_x u'_r} + \eht{P'u''_x d''} \\
  \fht{D}_t \eht{P'u'_x} = & \ -\nabla_x f_p^x - f_p\partial_x \eht{u}_x + \eht{u'_x u''_x}\partial_x \eht{P} - (1-\Gamma_3-\Gamma_1)\eht{u'_x P d} - (\Gamma_3-1)\eht{u'_x P d} + (\Gamma_3 -1)\eht{u'_x \rho \varepsilon_{nuc}} + \eht{P'u''_x d''} - \eht{P'\partial_{r} P / \rho}
\end{align}
  
\noindent With a little more algebraic modifications, we get the final form:

\begin{align}
  \fht{D}_t \eht{P'u'_x} = & \ -\nabla_x f_p^x - f_p\partial_x \eht{u}_x + \eht{u'_x u''_x}\partial_x \eht{P} +\Gamma_1 \eht{u'_x P d} + (\Gamma_3 -1)\eht{u'_x \rho \varepsilon_{nuc}} + \eht{P'u''_x d''} - \eht{P'\partial_{x} P / \rho}  
\end{align}


\subsection{Pressure flux equation in $\theta$ direction}

We can derive the pressure flux equation using the general formula for second order moments, where we substitute $c$ with $P$ and $d$ with $u_\theta$.

\begin{align}
\widetilde{D}_t \widetilde{c'd'} = & \ \overline{c' D_t d} - \eht{c'u''_n} \partial_n \eht{d} + \eht{d'D_t c} + \eht{d'u''_n}\partial_n \eht{c} - \partial_n \eht{u''_n c'd'} + \eht{c'd''\partial_n u''_n} \nonumber \\ 
  \fht{D}_t \eht{P'u'_\theta} = & \ \eht{P'D_t u_\theta} - \eht{P'u''_\theta} \partial_n \eht{u}_\theta + \eht{u'_\theta D_t P} + \eht{u'_\theta u''_n} \partial_n \eht{P} - \partial_n \eht{u''_n P' u'_\theta} + \eht{P'u''_\theta \partial_n u''_n} \nonumber \\
  \fht{D}_t \eht{P'u'_\theta} = & \ \eht{P'D_t u_\theta} - \eht{P'u''_\theta} \partial_r \eht{u}_\theta + \eht{u'_\theta D_t P} + \eht{u'_\theta u''_r} \partial_r \eht{P} - \nabla_r \eht{P' u''_r u'_\theta} + \eht{P'u''_\theta d''} 
\end{align}

\noindent
The evolution equation for $D_t u_\theta$ (see Sect.\ref{sect:hydro-inst-lag}) is 

\begin{align}
\rho D_{t} \big(u_{\theta}\big) = & \ +\left( \frac{1}{r^{2}} \partial_{r} \big( r^{2} [\tau_{\theta r}]\big) + \frac{1}{r\sin{\theta}}\partial_{\theta}(\sin{\theta}[\tau_{\theta \theta}]\big) + \frac{1}{r\sin{\theta}}\partial_{\phi}\big([\tau_{\theta \phi}]\big) - G_\theta^M - \frac{1}{r} \partial_{\theta} P \right) + \rho g_\theta 
\end{align}

\noindent
By neglecting viscosity and dividing by $\rho$, it becomes:

\begin{align}
D_{t} \big(u_{\theta}\big) = & \ - \frac{G_\theta^M}{\rho} - \frac{1}{r} \frac{\partial_{\theta} P}{\rho} 
\end{align}

\noindent
The evolution equation for $D_t P$ (see Sect.\ref{sect:mean-pressure-eq}) is 

\begin{align}
D_t P = & -(1-\Gamma_3+\Gamma_1)Pd + (\Gamma_3 -1)(-Pd + {\mathcal S} + \nabla \cdot F_T + \tau_{ij}\partial_i u_j)
\end{align}

\noindent
By neglecting viscosity, it becomes:

\begin{align}
D_t P = & -(1-\Gamma_3+\Gamma_1)Pd + (\Gamma_3 -1)(-Pd + {\mathcal S} + \nabla \cdot F_T)
\end{align}

\noindent We continue here by putting everything together and neglecting heat flux due to conduction and radiation $F_T$:

\begin{align}
  \fht{D}_t \eht{P'u'_\theta} = & \ \eht{P'\left(- G_\theta^M/\rho - \partial_{\theta} P / \rho\right)} - \eht{P'u''_r} \partial_r \eht{u}_\theta + \eht{u'_\theta \left( -(1-\Gamma_3+\Gamma_1)Pd + (\Gamma_3 -1)(-Pd + {\mathcal S} + \nabla \cdot F_T) \right)} + \eht{u'_\theta u''_r} \partial_r \eht{P} - \nabla_r \eht{P' u''_r u'_\theta} + \eht{P'u''_\theta d''} \\
  \fht{D}_t \eht{P'u'_\theta} = & \ -\eht{P' G_\theta^M/\rho} - \eht{P'\partial_{\theta} P / \rho}  - \eht{P'u''_r} \partial_r \eht{u}_r - (1-\Gamma_3-\Gamma_1)\eht{u'_\theta P d} - (\Gamma_3-1)\eht{u'_\theta Pd} + (\Gamma_3 -1)\eht{u'_\theta \rho \varepsilon_{nuc}}  + \eht{u'_\theta u''_r} \partial_r \eht{P} - \nabla_r \eht{P' u''_r u'_\theta} + \eht{P'u''_\theta d''} \\
  \fht{D}_t \eht{P'u'_\theta} = & \ -\nabla_r f_p^\theta - f_p\partial_r \eht{u}_\theta + \eht{u'_\theta u''_r}\partial_r \eht{P} - (1-\Gamma_3-\Gamma_1)\eht{u'_\theta P d} - (\Gamma_3-1)\eht{u'_\theta P d} + (\Gamma_3 -1)\eht{u'_\theta \rho \varepsilon_{nuc}} + \eht{P'u''_\theta d''} - \eht{P'\partial_{\theta} P / (\rho r)}
\end{align}
  
\noindent With a little more algebraic modifications, we get the final form:

\begin{align}
  \fht{D}_t \eht{P'u'_\theta} = & \ -\nabla_r f_p^\theta - f_p\partial_r \eht{u}_\theta + \eht{u'_\theta u''_r}\partial_r \eht{P} +\Gamma_1 \eht{u'_\theta P d} + (\Gamma_3 -1)\eht{u'_\theta \rho \varepsilon_{nuc}} + \eht{P'u''_\theta d''} - \eht{P'\partial_{\theta} P / (\rho r)}  
\end{align}

\subsection{Pressure flux equation in $y$ direction (Cartesian geometry)}

We can derive the pressure flux equation using the general formula for second order moments, where we substitute $c$ with $P$ and $d$ with $u_y$ (like in previous section).

\begin{align}
  \fht{D}_t \eht{P'u'_y} = & \ \eht{P'D_t u_y} - \eht{P'u''_x} \partial_x \eht{u}_y + \eht{u'_y D_t P} + \eht{u'_y u''_x} \partial_x \eht{P} - \nabla_x \eht{P' u''_x u'_y} + \eht{P'u''_y d''} 
\end{align}

\noindent
The evolution equation for $D_t u_y$ is \href{https://en.wikipedia.org/wiki/Navier-Stokes_equations}{https://en.wikipedia.org/wiki/Navier-Stokes\_equations}: 

\begin{align}
\rho D_{t} \big(u_{y}\big) = & \ \nabla_x \tau_{yx} + \nabla_x \tau_{yy} + \nabla_x \tau_{yz} - \partial_{y} P 
\end{align}

\noindent
By neglecting viscosity and dividing by $\rho$, it becomes:

\begin{align}
D_{t} \big(u_{y}\big) = & -\frac{\partial_{y} P}{\rho}
\end{align}

\noindent
The evolution equation for $D_t P$ (see Sect.\ref{sect:mean-pressure-eq}) is 

\begin{align}
D_t P = & -(1-\Gamma_3+\Gamma_1)Pd + (\Gamma_3 -1)(-Pd + {\mathcal S} + \nabla \cdot F_T + \tau_{ij}\partial_i u_j)
\end{align}

\noindent
By neglecting viscosity, it becomes:

\begin{align}
D_t P = & -(1-\Gamma_3+\Gamma_1)Pd + (\Gamma_3 -1)(-Pd + {\mathcal S} + \nabla \cdot F_T)
\end{align}

\noindent We continue here by putting everything together and neglecting heat flux due to conduction and radiation $F_T$:

\begin{align}
  \fht{D}_t \eht{P'u'_y} = & \ \eht{P'\left(- \partial_{y} P / \rho \right)} - \eht{P'u''_x} \partial_x \eht{u}_y + \eht{u'_y \left( -(1-\Gamma_3+\Gamma_1)Pd + (\Gamma_3 -1)(-Pd + {\mathcal S} + \nabla \cdot F_T) \right)} + \eht{u'_y u''_x} \partial_x \eht{P} - \nabla_x \eht{P' u''_x u'_y} + \eht{P'u''_y d''} \\
  \fht{D}_t \eht{P'u'_y} = & \ -\eht{P'\partial_{y} P / \rho}  - \eht{P'u''_x} \partial_x \eht{u}_y - (1-\Gamma_3-\Gamma_1)\eht{u'_y P d} - (\Gamma_3-1)\eht{u'_yPd} + (\Gamma_3 -1)\eht{u'_y \rho \varepsilon_{nuc}}  + \eht{u'_y u''_x} \partial_x \eht{P} - \nabla_x \eht{P' u''_y u'_x} + \eht{P'u''_y d''} \\
  \fht{D}_t \eht{P'u'_y} = & \ -\nabla_x f_p^y - f_p\partial_x \eht{u}_y + \eht{u'_y u''_x}\partial_x \eht{P} - (1-\Gamma_3-\Gamma_1)\eht{u'_y P d} - (\Gamma_3-1)\eht{u'_y P d} + (\Gamma_3 -1)\eht{u'_y \rho \varepsilon_{nuc}} + \eht{P'u''_y d''} - \eht{P'\partial_{y} P / \rho}
\end{align}
  
\noindent With a little more algebraic modifications, we get the final form:

\begin{align}
  \fht{D}_t \eht{P'u'_y} = & \ -\nabla_x f_p^y - f_p\partial_x \eht{u}_y + \eht{u'_y u''_x}\partial_x \eht{P} +\Gamma_1 \eht{u'_y P d} + (\Gamma_3 -1)\eht{u'_y \rho \varepsilon_{nuc}} + \eht{P'u''_y d''} - \eht{P'\partial_{y} P / \rho}  
\end{align}


\subsection{Pressure flux equation in $\phi$ direction}

We can derive the pressure flux equation using the general formula for second order moments, where we substitute $c$ with $P$ and $d$ with $u_\phi$.

\begin{align}
\widetilde{D}_t \widetilde{c'd'} = & \ \overline{c' D_t d} - \eht{c'u''_n} \partial_n \eht{d} + \eht{d'D_t c} + \eht{d'u''_n}\partial_n \eht{c} - \partial_n \eht{u''_n c'd'} + \eht{c'd''\partial_n u''_n} \nonumber \\ 
  \fht{D}_t \eht{P'u'_\phi} = & \ \eht{P'D_t u_\phi} - \eht{P'u''_\phi} \partial_n \eht{u}_\phi + \eht{u'_\phi D_t P} + \eht{u'_\phi u''_n} \partial_n \eht{P} - \partial_n \eht{u''_n P' u'_\phi} + \eht{P'u''_\phi \partial_n u''_n} \nonumber \\
  \fht{D}_t \eht{P'u'_\phi} = & \ \eht{P'D_t u_\phi} - \eht{P'u''_\phi} \partial_r \eht{u}_\phi + \eht{u'_\phi D_t P} + \eht{u'_\phi u''_r} \partial_r \eht{P} - \nabla_r \eht{P' u''_r u'_\phi} + \eht{P'u''_\phi d''} 
\end{align}

\noindent
The evolution equation for $D_t u_\phi$ (see Sect.\ref{sect:hydro-inst-lag}) is 

\begin{align}
\rho D_{t} \big(u_{\phi}\big) = & \ +\left( \frac{1}{r^{2}} \partial_{r} \big( r^{2} [\tau_{\phi r}]\big) + \frac{1}{r\sin{\phi}}\partial_{\phi}(\sin{\phi}[\tau_{\phi \theta}]\big) + \frac{1}{r\sin{\phi}}\partial_{\phi}\big([\tau_{\phi \phi}]\big) - G_\phi^M - \frac{1}{r\sin{\theta}} \partial_{\phi} P \right) + \rho g_\phi 
\end{align}

\noindent
By neglecting viscosity and dividing by $\rho$, it becomes:

\begin{align}
D_{t} \big(u_{\phi}\big) = & \ - \frac{G_\phi^M}{\rho} - \frac{1}{r\sin{\theta}} \frac{\partial_{\phi} P}{\rho} 
\end{align}

\noindent
The evolution equation for $D_t P$ (see Sect.\ref{sect:mean-pressure-eq}) is 

\begin{align}
D_t P = & -(1-\Gamma_3+\Gamma_1)Pd + (\Gamma_3 -1)(-Pd + {\mathcal S} + \nabla \cdot F_T + \tau_{ij}\partial_i u_j)
\end{align}

\noindent
By neglecting viscosity, it becomes:

\begin{align}
D_t P = & -(1-\Gamma_3+\Gamma_1)Pd + (\Gamma_3 -1)(-Pd + {\mathcal S} + \nabla \cdot F_T)
\end{align}

\noindent We continue here by putting everything together and neglecting heat flux due to conduction and radiation $F_T$:

\begin{align}
  \fht{D}_t \eht{P'u'_\phi} = & \ \eht{P'\left(- G_\phi^M/\rho - \partial_{\phi} P / \rho\right)} - \eht{P'u''_r} \partial_r \eht{u}_\phi + \eht{u'_\phi \left( -(1-\Gamma_3+\Gamma_1)Pd + (\Gamma_3 -1)(-Pd + {\mathcal S} + \nabla \cdot F_T) \right)} + \eht{u'_\phi u''_r} \partial_r \eht{P} - \nabla_r \eht{P' u''_r u'_\phi} + \eht{P'u''_\phi d''} \\
  \fht{D}_t \eht{P'u'_\phi} = & \ -\eht{P' G_\phi^M/\rho} - \eht{P'\partial_{\phi} P / \rho}  - \eht{P'u''_r} \partial_r \eht{u}_r - (1-\Gamma_3-\Gamma_1)\eht{u'_\phi P d} - (\Gamma_3-1)\eht{u'_\phi Pd} + (\Gamma_3 -1)\eht{u'_\phi \rho \varepsilon_{nuc}}  + \eht{u'_\phi u''_r} \partial_r \eht{P} - \nabla_r \eht{P' u''_r u'_\phi} + \eht{P'u''_\phi d''} \\
  \fht{D}_t \eht{P'u'_\phi} = & \ -\nabla_r f_p^\phi - f_p\partial_r \eht{u}_\phi + \eht{u'_\phi u''_r}\partial_r \eht{P} - (1-\Gamma_3-\Gamma_1)\eht{u'_\phi P d} - (\Gamma_3-1)\eht{u'_\phi P d} + (\Gamma_3 -1)\eht{u'_\phi \rho \varepsilon_{nuc}} + \eht{P'u''_\phi d''} - \eht{P'\partial_{\phi} P / (\rho r \sin{\theta})}
\end{align}
  
\noindent With a little more algebraic modifications, we get the final form:

\begin{align}
  \fht{D}_t \eht{P'u'_\phi} = & \ -\nabla_r f_p^\phi - f_p\partial_r \eht{u}_\phi + \eht{u'_\phi u''_r}\partial_r \eht{P} +\Gamma_1 \eht{u'_\phi P d} + (\Gamma_3 -1)\eht{u'_\phi \rho \varepsilon_{nuc}} + \eht{P'u''_\phi d''} - \eht{P'\partial_{\phi} P / (\rho r \sin{\theta})}  
\end{align}

\subsection{Pressure flux equation in $z$ direction (Cartesian geometry)}

We can derive the pressure flux equation using the general formula for second order moments, where we substitute $c$ with $P$ and $d$ with $u_z$ (like in previous section).

\begin{align}
  \fht{D}_t \eht{P'u'_z} = & \ \eht{P'D_t u_z} - \eht{P'u''_x} \partial_x \eht{u}_z + \eht{u'_z D_t P} + \eht{u'_z u''_x} \partial_x \eht{P} - \nabla_x \eht{P' u''_x u'_z} + \eht{P'u''_z d''} 
\end{align}

\noindent
The evolution equation for $D_t u_z$ is \href{https://en.wikipedia.org/wiki/Navier-Stokes_equations}{https://en.wikipedia.org/wiki/Navier-Stokes\_equations}: 

\begin{align}
\rho D_{t} \big(u_{z}\big) = & \ \nabla_x \tau_{zx} + \nabla_x \tau_{zy} + \nabla_x \tau_{zz} - \partial_{z} P 
\end{align}

\noindent
By neglecting viscosity and dividing by $\rho$, it becomes:

\begin{align}
D_{t} \big(u_{z}\big) = & -\frac{\partial_{z} P}{\rho}
\end{align}

\noindent
The evolution equation for $D_t P$ (see Sect.\ref{sect:mean-pressure-eq}) is 

\begin{align}
D_t P = & -(1-\Gamma_3+\Gamma_1)Pd + (\Gamma_3 -1)(-Pd + {\mathcal S} + \nabla \cdot F_T + \tau_{ij}\partial_i u_j)
\end{align}

\noindent
By neglecting viscosity, it becomes:

\begin{align}
D_t P = & -(1-\Gamma_3+\Gamma_1)Pd + (\Gamma_3 -1)(-Pd + {\mathcal S} + \nabla \cdot F_T)
\end{align}

\noindent We continue here by putting everything together and neglecting heat flux due to conduction and radiation $F_T$:

\begin{align}
  \fht{D}_t \eht{P'u'_z} = & \ \eht{P'\left(- \partial_{z} P / \rho \right)} - \eht{P'u''_x} \partial_x \eht{u}_z + \eht{u'_z \left( -(1-\Gamma_3+\Gamma_1)Pd + (\Gamma_3 -1)(-Pd + {\mathcal S} + \nabla \cdot F_T) \right)} + \eht{u'_z u''_x} \partial_x \eht{P} - \nabla_x \eht{P' u''_x u'_z} + \eht{P'u''_z d''} \\
  \fht{D}_t \eht{P'u'_z} = & \ -\eht{P'\partial_{z} P / \rho}  - \eht{P'u''_x} \partial_x \eht{u}_z - (1-\Gamma_3-\Gamma_1)\eht{u'_z P d} - (\Gamma_3-1)\eht{u'_zPd} + (\Gamma_3 -1)\eht{u'_z \rho \varepsilon_{nuc}}  + \eht{u'_z u''_x} \partial_x \eht{P} - \nabla_x \eht{P' u''_z u'_x} + \eht{P'u''_z d''} \\
  \fht{D}_t \eht{P'u'_z} = & \ -\nabla_x f_p^z - f_p\partial_x \eht{u}_z + \eht{u'_z u''_x}\partial_x \eht{P} - (1-\Gamma_3-\Gamma_1)\eht{u'_z P d} - (\Gamma_3-1)\eht{u'_z P d} + (\Gamma_3 -1)\eht{u'_z \rho \varepsilon_{nuc}} + \eht{P'u''_z d''} - \eht{P'\partial_{z} P / \rho}
\end{align}
  
\noindent With a little more algebraic modifications, we get the final form:

\begin{align}
  \fht{D}_t \eht{P'u'_z} = & \ -\nabla_x f_p^z - f_p\partial_x \eht{u}_z + \eht{u'_z u''_x}\partial_x \eht{P} +\Gamma_1 \eht{u'_z P d} + (\Gamma_3 -1)\eht{u'_z \rho \varepsilon_{nuc}} + \eht{P'u''_z d''} - \eht{P'\partial_{z} P / \rho}  
\end{align}


\section{Derivation of variance equations}

The derivation of final variance equations can be achieved by utilization of general formula for variances (see Sect.\ref{sect:reynolds-favrian-variance}) and similar algebraic manipulation as shown in previous sections.\\

\noindent
We show derivation of enthalpy variance equation only as an example

\subsection{Enthalpy variance equation}

\noindent
We begin with general formula for Favrian variance equation:

\begin{align}
{\color{red}\eht{\rho}\fht{D}_t \fht{c''c''} =} & {\color{red}+2 \eht{c''\rho D_t c} - 2\eht{\rho}\fht{c''u''_n}\partial_n\fht{c} - \eht{\partial_n{\rho c''c'' u''_n}}} 
\end{align}

\noindent
We substitute specific enthalpy $h = \epsilon_I + P/ \rho$ for the c and get:

\begin{align}
\eht{\rho} \fht{D}_t \fht{h''h''} = & \ +\eht{2 h'' \rho D_t h} - 2 \eht{\rho} \fht{h''u''_n}\partial_r \fht{h} - \nabla_r \eht{\rho h''h''u''_r}
\end{align}  

\noindent
Now, we have to derive evolution equation for specific enthalpy:

\begin{align}
  \rho D_t h = & \ \rho \partial_t h + u_r \partial_r h = \partial_t \rho h + \nabla_r \rho u_r h = \partial_t (\rho \epsilon_I + P) + \nabla_r \rho u_r (\epsilon_I + P/ \rho) \\
  \rho D_t h = & \ \partial_t \rho \epsilon_I + \nabla_r \rho u_r \epsilon_I + \partial_t P + \nabla_r u_r P \\
  \rho D_t h = & \ \rho D_t \epsilon_I + \partial_t P + u_r \partial_r P + P\nabla_r u_r \\
  \rho D_t h = & \ \rho D_t \epsilon_I + D_t P + P\nabla_r u_r   
\end{align}  

\noindent
For the internal energy evolution (see Sect.\ref{sect:mean-internal-energy-eq}) and neglecting viscosity and thermal transport due to conduction and radiation, we have :

\begin{align}
\rho D_t \epsilon_I = -P d + \rho \varepsilon_{nuc}
\end{align}  

\noindent
The evolution equation for $D_t P$ (see Sect.\ref{sect:mean-pressure-eq}) is 

\begin{align}
D_t P = & -(1-\Gamma_3+\Gamma_1)Pd + (\Gamma_3 -1)(-Pd + {\mathcal S} + \nabla \cdot F_T + \tau_{ij}\partial_i u_j)
\end{align}

\noindent
Now, let us combine everything together:

\begin{align}
  \rho D_t h = & \ - P d + \rho \varepsilon_{nuc} - (1-\Gamma_3+\Gamma_1)Pd - (\Gamma_3 -1)P d + (\Gamma_3-1)\rho \varepsilon_{nuc} + P d \\
  \rho D_t h = & \ - (1-\Gamma_3+\Gamma_1)Pd - (\Gamma_3 -1)P d + \Gamma_3 \rho \varepsilon_{nuc} \\
  \rho D_t h = & \ (\Gamma_3 -1)Pd - \Gamma_1 P d - (\Gamma_3-1)Pd + \Gamma_3 \rho \varepsilon_{nuc} \\
  \rho D_t h = & \ (\Gamma_3 -1)Pd - \Gamma_1 P d - (\Gamma_3-1)Pd + \Gamma_3 \rho \varepsilon_{nuc} \\
\rho D_t h = & \ -\Gamma_1 P d + \Gamma_3 \rho \varepsilon_{nuc}  
\end{align}  

\noindent
Substituing $\rho D_t h$ to our general equation for Favrian enthalpy variance, we get

\begin{align}
  \eht{\rho} \fht{D}_t \fht{h''h''} = & \ +\eht{2 h''(-\Gamma_1 P d + \Gamma_3 \rho \varepsilon_{nuc} )} - 2 \eht{\rho} \fht{h''u''_n}\partial_r \fht{h} - \nabla_r \eht{\rho h''h''u''_r} \\
   \eht{\rho} \fht{D}_t \fht{h''h''} = & \ -2\Gamma_1 \eht{h'' P d} + 2\Gamma_3 \eht{h'' \rho \varepsilon_{nuc}} - 2 f_h \partial_r \fht{h} - \nabla_r f_h^r
\end{align}

\noindent
After rearanging of these terms, we get final form or the equation to be:

\begin{align}
 \eht{\rho} \fht{D}_t \fht{h''h''} = & \ - \nabla_r f_h^r - 2 f_h \partial_r \fht{h} - 2\Gamma_1 \eht{h'' P d} + 2\Gamma_3 \eht{h'' \rho \varepsilon_{nuc}}
\end{align}
  

%\newpage

%\section{Derivation of third-order moments equations}

%We can derive the Reynolds stress flux equations using the general formula for third order moments, where we substitute $c$ with $u_i$, $d$ with $u_j$ and $e$ with $u_k$.

%\begin{align}
%\eht{\rho}\fht{D}_t \fht{c''d''e''} & = \ \eht{c''d''\rho D_t e} - \eht{\rho}\fht{c''d''}\fht{D}_t \fht{e} - \eht{\rho}\fht{c''d''u''_n} \partial_n \fht{e}  + \eht{c''e''\rho D_t d} - \eht{\rho}\fht{c''e''}\fht{D}_t \fht{d} - \eht{\rho}\fht{c''e''u''_n} \partial_n \fht{d} + \eht{d''e''\rho D_t c} - \nonumber \\
%& - \eht{\rho}\fht{d''e''}\fht{D}_t \fht{c} - \eht{\rho}\fht{d''e''u''_n} \partial_n \fht{c} - \eht{\partial_n c''d''e'' \rho u''_n}    
%\end{align}

%\begin{align}
%\eht{\rho}\fht{D}_t \left( \fht{F}_{rrr}^R / \eht{\rho} \right) = & -\nabla_r (\fht{F}_{rrrr}^R + 3\eht{u''_r u''_r P'} - 3\eht{u''_r u''_r \tau'_{rr}} ) + 3 \eht{u''_r u''_r \rho g_r} - 3 \fht{F}_{rrr}^R \partial_r \fht{u}_r + 3\fht{u''_r u''_r}\nabla_r \fht{R}_{rr} + \nonumber \\ 
%& + 3 \fht{u''_r u''_r}\partial_r \eht{P} - 3 \eht{u''_r u''_r}\partial_r \eht{P} + 3\eht{P'\nabla_r u''_r u''_r} - 3 \fht{R}_{rr}\fht{g_r} + 3 \fht{u''_ru''_r} \eht{G_r^M} - 3\eht{u''_ru''_r G_r^M} - \eht{G_{rrr}^R} - 3\eht{\rho}\varepsilon_{rrr}^R 
%\end{align}

%\begin{align}
%\eht{\rho}\fht{D}_t \left( \fht{F}_{\theta \theta r}^R / \eht{\rho} \right) = & -\nabla_r (\fht{F}_{\theta \theta rr}^R + \eht{u''_\theta u''_\theta P'} - \eht{u''_\theta u''_\theta \tau'_{rr}} - 2\eht{u''_\theta u''_r \tau'_{\theta r}}) + \eht{u''_\theta u''_\theta \rho g_r} - (\fht{F}_{\theta \theta r}^R \partial_r \fht{u}_r + 2\fht{F}_{\theta rr}\partial_r \fht{u}_\theta ) + \nonumber \\ 
%& + (\fht{u''_\theta u''_\theta}\nabla_r \fht{R}_{rr} + 2\fht{u''_\theta u''_r}\nabla_r \fht{R}_{\theta r} ) - \fht{R}_{\theta \theta} \fht{g_r}+ (\fht{u''_\theta u''_\theta}\partial_r \eht{P} - \eht{u''_\theta u''_\theta}\partial_r \eht{P}) + (\eht{P'\nabla_r u''_\theta u''_\theta} + \nonumber \\ 
%& + 2\eht{P'\nabla_\theta u''_\theta u''_r }) + ( \fht{u''_\theta u''_\theta} \eht{G_r^M} - \eht{u''_\theta u''_\theta G_r^M} + 2\fht{u''_\theta u''_r}\eht{G_\theta^M}- 2\eht{u''_\theta u''_r G_\theta^R}) - \eht{G}_{\theta \theta r}^R - \eht{\rho}\varepsilon_{r\theta \theta}^R - 2\eht{\rho}\varepsilon_{\theta \theta r}^R
%\end{align}

%\begin{align}
%\eht{\rho}\fht{D}_t \left( \fht{F}_{\phi \phi r}^R / \eht{\rho} \right) = &  -\nabla_r (\fht{F}_{\phi \phi rr}^R + \eht{u''_\phi u''_\phi P'} - \eht{u''_\phi u''_\phi \tau'_{rr}} - 2\eht{u''_\phi u''_r \tau'_{\phi r}}) + \eht{u''_\phi u''_\phi \rho g_r} - (\fht{F}_{\phi \phi r}^R \partial_r \fht{u}_r + 2\fht{F}_{\phi rr}\partial_r \fht{u}_\phi ) + \nonumber \\
%& +  (\fht{u''_\phi u''_\phi}\nabla_r \fht{R}_{rr} + 2\fht{u''_\phi u''_r}\nabla_r \fht{R}_{\phi r} ) - \fht{R}_{\phi \phi} \fht{g_r}+ (\fht{u''_\phi u''_\phi}\partial_r \eht{P} - \eht{u''_\phi u''_\phi}\partial_r \eht{P}) + (\eht{P'\nabla_r u''_\phi u''_\phi} + \nonumber \\ 
%& + 2\eht{P'\nabla_\phi u''_\phi u''_r }) + ( \fht{u''_\phi u''_\phi} \eht{G_r^M} - \eht{u''_\phi u''_\phi G_r^M} + 2\fht{u''_\phi u''_r}\eht{G_\phi^M}- 2\eht{u''_\phi u''_r G_\phi^R}) - \eht{G}_{\phi \phi r}^R - \eht{\rho}\varepsilon_{r\phi \phi}^R - 2\eht{\rho}\varepsilon_{\phi \phi r}^R
%\end{align}

%\vspace{0.5cm}

%From the previous three Reynolds stress flux equations we can derive a dynamic equation for the turbulent kinetic energy flux $\fht{F}_{r}^k$ because:

%\begin{align}
%\fht{F}_r^k = \frac{1}{2}\fht{F}_{iir}^R = \frac{1}{2} \eht{\rho}\fht{k u''_r}
%\end{align}

%Therefore:

%\begin{align}
%\eht{\rho}\fht{D}_t \left( \fht{F}_r^k / \eht{\rho} \ \right) = & -\nabla_r (\fht{F}_{rr}^k + \eht{F}_{rr}^P + \eht{F}_{rr}^\tau) - (1/2)(\nabla_r \eht{u''_iu''_iP'} + \nabla_r \eht{u''_iu''_i\tau'_{rr}} + \fht{F}^R_{iir}\partial_r \fht{u}_r - \fht{u''_iu''_r}\nabla_r \fht{R}_{rr} - \fht{u''_iu''_i}\partial_r \eht{P} + \eht{u''_iu''_i}\partial_r \eht{P} - \eht{P'\nabla_r u''_iu''_i}) - \\
%& - (\fht{F}^R_{irr}\partial_r \fht{u}_i - \fht{u''_iu''_r} \nabla_r \fht{R}_{ir} - \fht{u''_ru''_r}\partial_r \eht{P} + \eht{u''_r u''_r} \partial_r \eht{P} - \eht{P'\nabla_iu''_iu''_r}) + rst^R_{geom} - \eht{\rho}\epsilon^R_{iir} - (1/2)\eht{\rho}\epsilon_{rii}^R 
%\end{align}

\newpage

\section{Divergence of tensors in spherical geometry up to third order}

\noindent
{\bf{BACKGROUND READING:}} \\

\vspace{0.3cm}

\noindent
CONTINUUM MECHANICS (Lecture Notes) \\
Zdenek Martinec, Department of Geophysics, Faculty of Mathematics and Physics, Charles University in Prague 

%\vspace{-1.cm}

%\begin{align}
%{\bf{T}} = &  \sum_{ijk} T_{ijk} ({\bf{e_i}} \otimes {\bf{e_j}} \otimes {\bf{e_k}})
%\end{align}

%\begin{align}
%\nabla (.) = \sum_n \frac{{\bf{e_n}}}{h_n}\frac{\partial (.)}{\partial x_n}
%\end{align}

%first parameter is font size and the other line spacing
\fontsize{9pt}{20pt}

\begin{align}
\nabla (.) = \sum_n \frac{{\bf{e_n}}}{h_n}\frac{\partial (.)}{\partial x_n} \ \ \ \ \mbox{: nabla operator} & & {\bf{V}} = &  \sum_i V_i {\bf{{e_i}}} & & \mbox{: tensor of first order (vector)}\\
& & {\bf{S}} = &  \sum_{ij} S_{ij} ({\bf{e_i}} \otimes {\bf{e_j}}) & & {\mbox{: tensor of second order}}\\
& & {\bf{T}} = &  \sum_{ijk} T_{ijk} ({\bf{e_i}} \otimes {\bf{e_j}} \otimes {\bf{e_k}}) & & \mbox{: tensor of third order}
\end{align}

\begin{align}
\nabla \cdot {\bf{V}} = & \sum_i \frac{1}{h_i} \left[ \frac{\partial V_i}{\partial x_i} + \sum_m \Gamma_{mi}^i V_m \right] & & \mbox{: div of first order tensor (vector)} \\
\nabla \cdot {\bf{S}} = & \sum_{ij} \frac{1}{h_i} \left[ \frac{\partial S_{ij}}{\partial x_i} + \sum_m \Gamma_{mi}^i S_{mj} + \sum_m \Gamma_{mi}^j S_{im} \right] {\bf{e_j}} & & \mbox{: div of second order tensor} \\
\nabla \cdot {\bf{T}} = & \sum_{ijk} \frac{1}{h_i} \left[ \frac{\partial T_{ijk}}{\partial x_i} + \sum_m \Gamma_{mi}^i T_{mjk} + \sum_m \Gamma_{mi}^j T_{imk} + \sum_m \Gamma_{mi}^k T_{ijm} \right]  ({\bf{e_j}} \otimes {\bf{e_k}}) & & \mbox{: div of third order tensor}
\end{align}

\begin{align}
& x_1 = r & & x_2 = \theta & & x_3 = \phi & & \mbox{(coordinates)} \\
& {\bf{e_1 = e_r}} & &  {\bf{e_2 = e_\theta}} & & {\bf{e_3 = e_\phi}} && \mbox{(unit base vectors)} \\
& h_1 = h_r = 1 & &  h_2 = h_\theta = r & & h_3 = h_\phi = r \sin{\theta} & & \mbox{(scale factors)}
\end{align}

\begin{align}
\begin{pmatrix}
\Gamma_{r\theta}^\theta = \ 1  &  \Gamma_{r\phi}^\phi = \ \sin{\theta} & \Gamma_{\theta \phi}^{\phi} = \ \cos{\theta}   \\
\Gamma_{\theta \theta}^r = -1   &  \Gamma_{\phi \phi}^r = -\sin{\theta}  & \Gamma_{\phi \phi}^{\theta} = -\cos{\theta}  
\end{pmatrix} &&  \mbox{Christoffel symbols}
\end{align}

\newpage

\noindent
{{\bf{Divergence of first order tensor}} $\nabla \cdot {\bf{V}}$}

\begin{align}
\frac{1}{r^2} \frac{\partial (r^2 V_r) }{\partial_r} + \frac{1}{r \sin{\theta}}\frac{\partial}{\partial \theta}(V_\theta \sin{\theta}) + \frac{1}{r \sin{\theta}} \frac{\partial V_\phi}{\partial \phi}
\end{align}

\noindent
{{\bf{Divergence of second order tensor}} $\nabla \cdot {\bf{S}}$}

\begin{align}
S_r ({\bf{e_r}}): & \hspace{1cm} \frac{1}{r^2}\dr (r^2 S_{rr}) + \frac{1}{r\sin{\theta}}\frac{\partial}{\partial \theta} (\sin{\theta} \ S_{\theta r}) + \frac{1}{r\sin{\theta}}\frac{\partial S_{\phi r}}{\partial \phi} - \frac{S_{\theta\theta}}{r} - \frac{S_{\phi\phi}}{r} \\
S_\theta ({\bf{e_\theta}}): & \hspace{1cm} \frac{1}{r^2}\dr (r^2 S_{r \theta}) + \frac{1}{r\sin{\theta}}\frac{\partial}{\partial \theta} (\sin{\theta} \ S_{\theta\theta}) + \frac{1}{r\sin{\theta}}\frac{\partial S_{\phi \theta}}{\partial \phi} + \frac{S_{\theta r}}{r} - \frac{S_{\phi\phi} \cos{\theta}}{r\sin{\theta}} \\
S_\phi ({\bf{e_\phi}}): & \hspace{1cm} \frac{1}{r^2}\dr (r^2 S_{r \phi}) + \frac{1}{r\sin{\theta}}\frac{\partial}{\partial \theta} (\sin{\theta} \ S_{\theta \phi}) + \frac{1}{r\sin{\theta}}\frac{\partial S_{\phi\phi}}{\partial \phi} + \frac{S_{\phi r}}{r} + \frac{S_{\phi \theta} \cos{\theta}}{r\sin{\theta}}
\end{align}

\noindent
{{\bf{Divergence of third order tensor}} $\nabla \cdot {\bf{T}}$}


\begin{align}
T_{rr} \ ({\bf{e_r \otimes e_r}}): & \hspace{1cm} \frac{1}{r^2}\dr (r^2 T_{rrr}) + \frac{1}{r\sin{\theta}}\frac{\partial}{\partial \theta} (\sin{\theta} \ T_{\theta rr}) + \frac{1}{r\sin{\theta}}\frac{\partial T_{\phi rr}}{\partial \phi} - \frac{T_{\theta\theta r}}{r} - \frac{T_{\theta r \theta}}{r} -\frac{T_{\phi\phi r}}{r} - \frac{T_{\phi r \phi}}{r} \\
T_{r\theta} \ ({\bf{e_r \otimes e_\theta}}): & \hspace{1cm} \frac{1}{r^2}\dr (r^2 T_{rr \theta}) + \frac{1}{r\sin{\theta}}\frac{\partial}{\partial \theta} (\sin{\theta} \ T_{\theta r \theta}) + \frac{1}{r\sin{\theta}}\frac{\partial T_{\phi r \theta}}{\partial \phi} - \frac{T_{\theta \theta \theta}}{r} + \frac{T_{\theta rr}}{r} - \frac{T_{\phi \phi \theta}}{r} - \frac{T_{\phi r \phi} \cos{\theta}}{r\sin{\theta}} \\
T_{r \phi} \ ({\bf{e_r \otimes e_\phi}}): & \hspace{1cm} \frac{1}{r^2}\dr (r^2 T_{rr \phi}) + \frac{1}{r\sin{\theta}}\frac{\partial}{\partial \theta} (\sin{\theta} \ T_{\theta r \phi}) + \frac{1}{r\sin{\theta}}\frac{\partial T_{\phi r \phi}}{\partial \phi} + \frac{T_{\theta \theta \phi}}{r} - \frac{T_{\phi \phi \phi}}{r} + \frac{T_{\phi r \phi} \cos{\theta}}{r\sin{\theta}} \\
T_{\theta r} \ ({\bf{e_\theta \otimes e_r}}): & \hspace{1cm} \frac{1}{r^2}\dr (r^2 T_{r \theta r}) + \frac{1}{r\sin{\theta}}\frac{\partial}{\partial \theta} (\sin{\theta} \ T_{\theta \theta r}) + \frac{1}{r\sin{\theta}}\frac{\partial T_{\phi \theta r}}{\partial \phi} + \frac{T_{\theta rr}}{r} - \frac{T_{\theta \theta \theta}}{r} -\frac{T_{\phi \phi r} \cos{\theta}}{r \sin{\theta}} - \frac{T_{\phi \theta \phi}}{r}  \\
T_{\theta \theta} \ ({\bf{e_\theta \otimes e_\theta}}): & \hspace{1cm} \frac{1}{r^2}\dr (r^2 T_{r \theta \theta}) + \frac{1}{r\sin{\theta}}\frac{\partial}{\partial \theta} (\sin{\theta} \ T_{\theta \theta \theta}) + \frac{1}{r\sin{\theta}}\frac{\partial T_{\phi \theta \theta}}{\partial \phi} + \frac{T_{\theta r \theta}}{r} + \frac{T_{\theta \theta r}}{r} - \frac{T_{\phi \phi \theta} \cos{\theta}}{r\sin{\theta}} - \frac{T_{\phi \theta \phi} \cos{\theta}}{r\sin{\theta}} \\
T_{\theta \phi} \ ({\bf{e_\theta \otimes e_\phi}}): & \hspace{1cm} \frac{1}{r^2}\dr (r^2 T_{r \theta \phi}) + \frac{1}{r\sin{\theta}}\frac{\partial}{\partial \theta} (\sin{\theta} \ T_{\theta \theta \phi}) + \frac{1}{r\sin{\theta}}\frac{\partial T_{\phi \theta \phi}}{\partial \phi} + \frac{T_{\theta r \phi}}{r} + \frac{T_{\phi \theta r}}{r} + \frac{T_{\phi \theta \theta} \cos{\theta}}{r\sin{\theta}} \\
T_{\phi r} \ ({\bf{e_\phi \otimes e_r}}): & \hspace{1cm} \frac{1}{r^2}\dr (r^2 T_{r \phi r}) + \frac{1}{r\sin{\theta}}\frac{\partial}{\partial \theta} (\sin{\theta} \ T_{\theta \phi r}) + \frac{1}{r\sin{\theta}}\frac{\partial T_{\phi \phi r}}{\partial \phi} - \frac{T_{\theta \phi \theta}}{r} + \frac{T_{\phi rr}}{r} + \frac{T_{\phi \theta r} \cos{\theta}}{r\sin{\theta}} - \frac{T_{\phi \phi \phi}}{r} \\
T_{\phi \theta} \ ({\bf{e_\phi \otimes e_\theta}}): & \hspace{1cm} \frac{1}{r^2}\dr (r^2 T_{r \phi \theta}) + \frac{1}{r\sin{\theta}}\frac{\partial}{\partial \theta} (\sin{\theta} \ T_{\theta \phi \theta}) + \frac{1}{r\sin{\theta}}\frac{\partial T_{\phi \phi \theta}}{\partial \phi} + \frac{T_{\theta \phi r}}{r} + \frac{T_{\phi r \theta}}{r} + \frac{T_{\phi \theta \theta} \cos{\theta}}{r\sin{\theta}} - \frac{T_{\phi \theta \phi} \cos{\theta}}{r\sin{\theta}}  \\
T_{\phi \phi} \ ({\bf{e_\phi \otimes e_\phi}}): & \hspace{1cm} \frac{1}{r^2}\dr (r^2 T_{r \phi \phi}) + \frac{1}{r\sin{\theta}}\frac{\partial}{\partial \theta} (\sin{\theta} \ T_{\theta \phi \phi}) + \frac{1}{r\sin{\theta}}\frac{\partial T_{\phi \phi \phi}}{\partial \phi} + \frac{T_{\phi r \phi}}{r} + \frac{T_{\phi \theta \phi} \cos{\theta}}{r\sin{\theta}} + \frac{T_{\phi \phi r}}{r} + \frac{T_{\phi \phi \theta} \cos{\theta}}{r\sin{\theta}}
\end{align}

%\section{The ``zero'' integrals}
%Our simulations are wedges in $\theta$ and $\phi$ having periodic boundary conditions in angular directions. This makes averaged angular components of divergence in spherical geometry always go to zero!

%\begin{align}
%\theta = & \langle\theta_L,\theta_R\rangle = \langle75^{o}, 105^{o}\rangle \\
%\phi = & \langle\phi_L,\phi_r \rangle = \langle-15^{o}, +15^{o}\rangle &
%\end{align}


%\begin{align}
%\int_{\Delta \Omega} \frac{1}{r \sin{\theta}} \partial_\theta (\sin{\theta}[\rho U_\theta]) \ d \Omega = & \int_{\Delta \theta} \int_{\Delta \phi} \frac{1}{r \sin{\theta}} \partial_\theta (\sin{\theta}[\rho U_\theta]) \ \sin{\theta} \ d\theta \ d\phi = \\ 
%\frac{1}{r} \int_{\Delta \theta} \int_{\Delta \phi} \partial_\theta (\sin{\theta}[\rho U_\theta]) \ d\theta \ d\phi = &
%\frac{1}{r} \int_{\Delta \theta} \partial_\theta \langle\sin{\theta}[\rho U_\theta]\rangle_\phi \ d\theta = \frac{1}{r}(\langle\sin{\theta}[\rho U_\theta] \rangle_\phi)_{\theta_R}^{\theta_L} = 0
%\end{align}
 

%\begin{align}
%\int_{\Delta \Omega} \frac{1}{r\sin{\theta}}\partial_\phi [\rho U_\phi] = & \int_{\Delta \theta} \int_{\Delta \phi} \frac{1}{r \sin{\theta}} \partial_\theta [\rho U_\phi] \ \sin{\theta} \ d\theta \ d\phi = \\
%\frac{1}{r} \int_{\Delta \theta} \partial_\phi [\rho U_\phi] \ d\theta d\phi = & \frac{1}{r} \int_{\Delta \theta} \partial_\phi \langle\rho U_\phi\rangle_\theta \ d\phi = \frac{1}{r} (\langle\rho U_\phi\rangle_\theta)_{\phi_R}^{\phi_L} = 0
%\end{align}

%\begin{align}
%\int_{\Delta \Omega} \frac{1}{r\sin{\theta}} \partial_\phi P = & \int_{\Delta \theta} \int_{\Delta \phi}  \frac{1}{r\sin{\theta}} \partial_\phi P \sin{\theta} \ d\theta d\phi = \frac{1}{r}\int_{\Delta \theta} \int_{\Delta \phi} \partial_\phi P \ d\theta d\phi = \\
%\frac{1}{r} \int_{\Delta \phi} \partial_\phi \langle P\rangle_\theta d\phi = & \frac{1}{r} [\langle P\rangle_\theta]_{\phi_R}^{\phi_L} = 0
%\end{align}

%\section{Some properties of averaging}

%\begin{align}
%\eht{A'} = & \ \eht{\rho A''} = 0\\
%\eht{AB} = & \ \eht{A}\eht{B} - \eht{A'B'} \\
%\eht{\rho AB} = & \ \eht{\rho}\bar{A}\bar{B} + \eht{\rho A''B''} \\
%\eht{A''B'} = & \ \eht{A'B'} = \eht{A'B} = \eht{AB'} = \eht{A'B''} \\
%\eht{X} - \fht{X} = & \eht{X''} = -\eht{\rho' X'}/\eht{\rho}
%\end{align}


%\end{landscape}

\bibliography{referenc}

\end{document}


